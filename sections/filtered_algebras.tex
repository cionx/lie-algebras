\subsection{Filtered \texorpdfstring{$\kf$}{k}-Algebras}


\begin{definition}
  A \defemph{filtration}\index{filtration} of a~\algebra{$\kf$}~$A$ is an increasing sequence
  \[
    A_{(0)}
    \subseteq
    A_{(1)}
    \subseteq
    A_{(2)}
    \subseteq
    \dotsb
  \]
  of linear subspaces of~$A$ such that~$A = \bigcup_{i \geq 0} A_{(i)}$ and~$A_{(i)} A_{(\spacing j)} \subseteq A_{(i+j)}$ for all~$i,j \geq 0$, as well as~$1 \in A_{(0)}$.
  A \defemph{filtered~\algebra{$\kf$}} is a~\algebra{$\kf$}~$A$ together with a filtration of~$A$.
\end{definition}


\begin{remark}
  \label{filtration conventions}
  \leavevmode
  \begin{enumerate}
    \item
      We will just say that~\enquote{$A$ is a filtered algebra} without mentioning the filtration explicitely.
      The parts of the filtration will then be denoted by~$A_{(i)}$ with~$i \geq 0$.
    \item
      If~$A$ is a filtered~{\algebra{$\kf$}} then we set~$A_{(-i)} \defined 0$ for all~$i < 0$ for convenience.
      The relation~$A_{(i)} A_{(\spacing j)} \subseteq A_{(i+j)}$ with~$i,j \geq 0$ then extends to all~$i, j \in \Integer$.
  \end{enumerate}
\end{remark}


\begin{definition}
  Let~$A$ and~$B$ be two filtered~\algebras{$\kf$}.
  \begin{enumerate}
    \item
      A~\defemph{homomorphism of filtered~\algebras{$\kf$}}~$A \to B$ is a homomorphism of the underlying~{\algebras{$\kf$}} such that~$f(A_{(i)}) \subseteq B_{(i)}$ for every~$i \geq 0$.
    \item
      If~$f \colon A \to B$ is a homomorphism of filtered~{\algebras{$\kf$}} then for every~$i \geq 0$ the restriction of~$f$ to a linear map~$A_{(i)} \to B_{(i)}$ is denoted by~$f_{(i)}$.
  \end{enumerate}
\end{definition}


\begin{remark}
  Let~$A$,~$B$ and~$C$ be filtered~\algebras{$\kf$}.
  \begin{enumerate}
    \item
      The identity~$\id_A \colon A \to A$ is a homomorphism of filtered~\algebras{$\kf$}.
    \item
      If~$f \colon A \to B$ and~$g \colon B \to C$ are homomorphisms of filtered~\algebras{$\kf$} then their composition~$g \circ f \colon A \to C$ is again a homomorphism of filtered~\algebras{$\kf$}.
  \end{enumerate}
  This shows that filtered~\algebras{$\kf$} together with homomorphisms of filtered~\algebras{$\kf$} between them form a category, which we will denote by~\gls*{filtered algebras}.
  \begin{enumerate}[resume]
    \item
      A homomorphism of filtered~{\algebras{$\kf$}}~$f \colon A \to B$ is hence an isomorphism if and only if there exists a homomorphism of filtered~{\algebras{$\kf$}}~$g \colon B \to A$ with~$f {} g = \id_B$ and~$g \spacing f = \id_A$.
  \end{enumerate}
\end{remark}


\begin{warning}
  Let~$A$ and~$B$ be two filtered~{\algebras{$\kf$}}.
  A bijective homomorphism of filtered~{\algebras{$\kf$}}~$f \colon A \to B$ is not necessarily an isomorphism of filtered~{\algebras{$\kf$}}.
  Indeed, take any~{\algebra{$\kf$}}~$A$ with~$\kf \neq A$.
  We can endow~$A$ with a filtration
  \[
    A_{(0)}
    =
    \kf
    \subseteq
    A
    =
    A
    =
    A
    =
    \dotsb
  \]
  which results in a filtered~{\algebra{$\kf$}}~$B_1$.
  But we can also endow~$A$ with the filtration
  \[
    A_{(0)}
    =
    A
    =
    A
    =
    A
    =
    \dotsb
  \]
  which results in a filtered~{\algebra{$\kf$}}~$B_2$.
  Then the identity~$\id_A \colon A \to A$  results in a bijective homomorphism of filtered~{\algebras{$\kf$}}~$B_1 \to B_2$.
  But this is not an isomorphism of filtered~{\algebras{$\kf$}} because the inverse~$B_2 \to B_1$ does not respect the filtrations.
\end{warning}


\begin{remark}
  \leavevmode
  \begin{enumerate}
    \item
      Any grading~$A = \bigoplus_{i \geq 0} A_i$ of a~{\algebra{$\kf$}}~$A$ results in a filtration~$A = \bigcup_{i \geq 0} A_{(i)}$ given by
      \[
        A_{(i)}
        =
        \bigoplus_{j \geq i} A_j
      \]
      for every~$i \geq 0$.
      We can therefore regard every graded~{\algebra{$\kf$}} as a filtered~{\algebra{$\kf$}}.
      Every homomorphism of graded~{\algebras{$\kf$}}~$f \colon A \to B$ is then also a homomorphism of filtered \algebras{$\kf$}.
      This construction gives us a forgetful functor~$\cgAlg{\kf} \to \cfAlg{\kf}$.
    \item 
      If~$A$ is a filtered algebra and~$I$ is any two-sided ideal in~$A$ then the quotient algebra~$A/I$ inherits from~$A$ a filtration given by~$(A/I)_{(i)} \defined \pi(A_{(i)})$ for every~$i \geq 0$.
      Here~$\pi \colon A \to A/I$ denotes the canonical projection.
  \end{enumerate}
\end{remark}


\begin{examples}
  \leavevmode
  \begin{enumerate}
    \item
      If~$V$ is any vector space then the tensor algebra~$\Tensor(V)$, the symmetric algebra~$\Symm(V)$ and the exterior algebra~$\Exterior(V)$ admit gradings as explained in \cref{examples of graded algebras} and hence also induced filtrations.
    \item
      If~$\glie$ is a Lie algebra then its universal enveloping algebra~$\Univ(\glie)$ inherits from the tensor algebra~$\Tensor(\glie)$ a filtration~$\Univ(\glie) = \bigcup_{i \geq 0} \Univ(\glie)_{(i)}$ with terms
      \[
        \Univ(\glie)_{(i)}
        =
        \gen{
          \class{x_1 \dotsm x_j}
        \suchthat
          j \leq i,
          x_1, \dotsc, x_j \in \glie
        }_{\kf} \,.
      \]
    \item
      Let~$M$ be a multiplicative monoid and let
      \[
        M_{(0)}
        \subseteq
        M_{(1)}
        \subseteq
        M_{(2)}
        \subseteq
        M_{(3)}
        \subseteq
        \dotsb
      \]
      be a filtration of~$M$, i.e.\ it holds that~$M = \bigcup_{i \geq 0} M_{(i)}$ with~$M_{(i)} M_{(\spacing j)} \subseteq M_{(i+j)}$ for all~$i, j \geq 0$ and~$1 \in M_{(0)}$.
      Then the monoid algebra~$A \defined \kf[M]$ inherits a filtration given by
      \[
        A_{(i)}
        =
        \gen{ M_{(i)} }_{\kf}
      \]
      for all~$i \geq 0$.
    \item
      Let us give a special case of the previous example:
      Let~$G$ be a group with generating set~$S$.
      The length of an element~$g \in G$ with respect to the generating set~$S$ is given by
      \[
        l_S(\spacing g)
        =
        \min
        \{
          n \in \Natural
        \suchthat
          \text{there exist~$s_1, \dotsc, s_n \in S$ with~$g = s_1^{\pm 1} \dotsm s_n^{\pm 1}$}
        \}  \,.
      \]
      The length function is subadditive in the sense that
      \[
        l_S(\spacing gh)
        \leq
        l_S(\spacing g) + l_S(h)
      \]
      for all~$g, h \in G$.
      For every~$i \geq 0$ let
      \[
        G_{(i)}
        \defined
        \{
          \spacing g \in G
        \suchthat
          l_S(\spacing g) \leq i
        \}  \,,
      \]
      which is the ball of length~$i$.
      These subsets given a filtration of~$G$.
      That~$G_{(i)} G_{(\spacing j)} \subseteq G_{(i+j)}$ for all~$i, j \geq 0$ follows from the subadditivity of the length function~$l_S$.
      We hence get for the group algebra~$A \defined \kf[G]$ a filtration given by
      \[
        A_{(i)}
        \defined
        \gen{ G_{(i)} }_{\kf}
        =
        \gen{
          g \in G
        \suchthat
          l_S(\spacing g) \leq i
        }_{\kf} \,.
      \]
  \end{enumerate}
\end{examples}


\begin{definition}
  The \defemph{degree}\index{degree!filtration} of an element~$x \in A$ is the minimal~$i \geq 0$ with~$x \in A_{(i)}$.
\end{definition}


\begin{example}
  \leavevmode
  \begin{enumerate}
    \item
      Let~$A \defined \kf[t]$ be the polynomial ring in one variable.
      The standard grading~$\kf[t] = \bigoplus_{i \geq 0} \kf t^i$ gives a filtration
      \[
        A_0
        =
        \kf
        =
        \gen{ 1 }_{\kf}
        \subseteq
        \gen{ 1, t }_{\kf}
        \subseteq
        \gen{ 1, t, t^2 }_{\kf}
        \subseteq
        \gen{ 1, t, t^2, t^3 }_{\kf}
        \subseteq
        \dotsb
      \]
      of~$A$.
      The degree of a nonzero polynomial~$p \in A$ with respect to this filtration is the usual degree of a polynomial.
    \item
      If more generally~$A$ is any graded algebra with associated filtration then the degree of any nonzero element~$x \in A$ with homogeneous decomposition~$x = \sum_{i \geq 0} x_i$ is the largest index~$i$ with~$x_i \neq 0$.
  \end{enumerate}
\end{example}


\subsubsection{The Associated Graded Algebra}


\begin{definition}
  Two elements~$x$ and~$y$ of a filtered~{\algebra{$\kf$}}~$A$ are \defemph{equal up to smaller degree}\index{equal up to smaller degree}\index{up to smaller degree}\index{degree!up to smaller} if~$x = y$ or their difference~$x - y$ is of degree strictly smaller than both~$x$ and~$y$.
\end{definition}


\begin{lemma}
  Let~$A$ be a filtered algebra.
  Two elements~$x, y \in A$ are equal up to smaller degree if and only if they have the same degree~$i \geq 0$ and~$x - y \in A_{(i-1)}$.
\end{lemma}


\begin{proof}
  Suppose first that that~$x$ and~$y$ are equal up to smaller degree.
  Let~$x$ be of degree~$i$ and let~$y$ be of degree~$j$ with~$i \geq j$.
  Then~$x - y \in A_{(\spacing j)}$ and thus~$x = x' + (x- x') \in A_{(\spacing j)}$.
  Hence~$i \leq j$ and thus~$i = j$.  
  If now~$x = y$ then~$x - y = 0 \in A_{(i-1)}$.
  (Recall for the case~$i = 0$ that~$A_{(-1)} = 0$.)
  Otherwise~$x - y \in A_{(k)}$ for some~$0 \leq k < i$ and hence~$x - y \in A_{(k)}$.
  
  If on the other hand~$x$ and~$y$ have the same degree~$i \geq 0$ with~$x - y \in A_{(i-1)}$ then~$x = y$ or the degree of~$x - y$ is at most~$i-1$ and hence strictly smaller than~$i$.
  This means that~$x$ and~$y$ are equal up to smaller degree.
\end{proof}


\begin{corollary}
  If~$A$ is a filtered~{\algebra{$\kf$}} then \enquote{being equal up to smaller degree} is an equivalence relation on~$A$.
  The equivalence class of an element~$x \in A$ of degree~$i$ is given by the coset~$x + A_{(i-1)}$.
  \qed
\end{corollary}


\begin{remark}
  Let~$A$ be a filtered~{\algebra{$\kf$}} and let~$\sim$ be the equivalence relation \enquote{equal up to smaller degree} on~$A$.
  
  When calculating in~$A$ we sometimes want to replace an element~$x \in A$ by another element~$x' \in A$ that is equal to~$x$ up to smaller degree, while hoping that the result ouf our calculation also stays the same up to smaller degree.
  This can be useful if terms of smaller degree are not important in the given situation, or if they can be dealt with by induction.
  
  We would therefore like to have the properties
  \begin{equation}
    \label{wanted compatibility}
    x + y
    \sim
    x' + y' \,,
    \qquad
    x \cdot y
    \sim
    x' \cdot y' \,,
    \qquad
    \lambda x \sim \lambda x'
  \end{equation}
  for all elements~$x, x', y, y' \in A$ with~$x \sim x'$ and~$y \sim y'$ and scalars~$\lambda \in \kf$.
  This would then mean that the quotient set~$A/{\sim}$ inherits from~$A$ the structure of a~{\algebra{$\kf$}}, allowing us do calculations \enquote{up to smaller degree} by switching from~$A$ to~$A/{\sim}$.
  
  But alas the properties~\eqref{wanted compatibility} do not hold in general:%
  \footnote{The equivalence relation~$\sim$ is actually compatible with scalar multiplication:
  Suppose that~$x, x' \in A$ are two element that are equal up to smaller degree, so that~$x$ and~$x'$ are of the same degree~$i$ and~$x - x' \in A_{(i-1)}$.
  If~$\lambda = 0$ then~$\lambda x = 0 = \lambda x'$ and hence~$\lambda x \sim \lambda x'$.
  If~$\lambda \neq 0$ then it follows from~$x \in A_{(i)}$ and~$x \notin A_{(i-1)}$ that also~$\lambda x \in A_{(i)}$ and~$\lambda x' \notin A_{(i-1)}$.
  This means that the element~$\lambda x$ is again of degree~$i$.
  We find in the same way that~$\lambda x'$ is again of degree~$i$.
  We moverover have~$\lambda x - \lambda x' = \lambda (x-x') \in A_{(i-1)}$ because~$x - x' \in A_{(i-1)}$.
  Altogether this shows that~$\lambda x \sim \lambda x'$.}
  \begin{itemize}
    \item
      Suppose that~$x$ and~$x'$ are of degree~$i$ whereas~$y$ and~$y'$ are of degree~$j$.
      We would like to argue that~$x + y$ and~$x' + y'$ have the same degree~$\max(i,j)$ and that their difference
      \[
        (x + y) - (x' + y')
        =
        (x - x') + (y + y')
      \]
      has degree at most~$\max(i-1,j-1)$ and hence strictly smaller than~$i+j$.
      But this argumentation doesn’t work because the sums~$x+y$ and~$x'+y'$ may have smaller degree than~$i+j$, which would mean that the difference~$(x + y) - (x' + y')$ may have too large of a degree.
      
      As a counterexample we can take the polynomial ring~$A = \kf[t]$ wih the filtration induced by the standard grading~$A = \bigoplus_{i \geq 0} \kf t^i$.
      Then~$x = t$ and~$x' = t+1$ are the equal to smaller degree but for~$y = -t$ the sums~$x + y = 0$ and~$x' + y' = 1$ are not equal up to smaller degree.
      
      This shows that the equivalence relation~$\sim$ is not necessarily compatible with addition.
    \item
      The equivalence relation~$\sim$ also hasn’t to be compatible with multiplication.
      
      We take on~$\kf[t]$ the standard grading~$\kf[t] = \bigoplus_{i \geq 0} \kf t^i$.
      The ideal~$(t^2)$ is homogeneous, and the quotient~$A \defined \kf[t]/(t^2)$ therefore inherits a grading given by
      \[
        A
        =
        \kf
        \oplus
        \kf t
        \oplus
        0
        \oplus
        0
        \oplus
        \dotsb
      \]
      The resulting filtration of~$A$ is given by
      \[
        A_0
        =
        \kf
        \subsetneq
        (t)/t^2
        \subsetneq
        A
        =
        A
        =
        A
        =
        \dotsb
      \]
      The elements~$x = t$ and~$x' = t+1$ are equal up to smaller degree but for~$y = y' = t$ the two products~$xy = 0$ and~$x' y' = t$ are not equal up to smaller degree.
  \end{itemize}
  We can also argue more abstractly that this approach to \enquote{working with elements up to smaller degree} cannot work:
  We would have~$A/{\sim} = A/I$ for the then two-sided ideal~$I = \{x \in A \suchthat x \sim 0\}$.
  But the only element that is equal to~$0$ up to smaller degree is~$0$ itself.
  Hence~$I = 0$ and we would have that~$A = A/{\sim}$, meaning that no two elements~$x, y \in A$ with~$x \neq y$ could be equal up to smaller degree.
  
  The following construction gives us a way to circumvent these problems and to calculate \enquote{up to smaller degree} in a rigorous way:
\end{remark}


\begin{construction}[The associated graded algebra]
  \label{construction of associated graded}
  To every filtered~{\algebra{$\kf$}}~$A$ we can associate a graded~{\algebra{$\kf$}} as follows:
  For every~$i \geq 0$ let
  \[
    \gr[i](A)
    \defined
    A_{(i)} / A_{(i-1)}
  \]
  where we use the convention~$A_{(-1)} = 0$ from \cref{filtration conventions}.
  We denote the residue class of~$x \in A_{(i)}$ in~$\gr[i](A)$ by~$[x]_i$.
  
  For all~$i, j \geq 0$ the multiplication~$A \times A \to A$ restricts to a bilinear map~$A_{(i)} \times A_{(\spacing j)} \to A_{(i+j)}$ that in turn induces a well-defined bilinear map
  \[
    \mu_{i,j}
    \colon
    \gr[i](A) \times \gr[j](A)
    \to
    \gr[i+j](A) \,,
    \quad
    ([x]_i, [y]_j)
    \mapsto
    [xy]_{i+j}  \,.
  \]
  We will write~$[x]_i \cdot [y]_j$ or just~$[x]_i [y]_j$ instead of~$\mu_{i,j}([x]_i, [y]_j)$.
  To see that the multipliction~$\mu_{i,j}$ is well-defined let~$x, x' \in A_{(i)}$ and~$y, y' \in A_{(\spacing j)}$ with~$x - x ' \in A_{(i-1)}$ and~$y - y' \in A_{(\spacing j-1)}$.
  Then
  \begin{align*}
    x y
    &=
    (x' + (x - x')) (y' + (y - y'))
    \\
    &=
    x' y' + (x - x') y' + x' (y - y') + (x - x')(y - y')
    \\
    &\in
    x' y' + A_{(i-1)} A_{(\spacing j)} + A_{(i)} A_{(\spacing j-1)} + A_{(i-1)} A_{(\spacing j-1)}
    \\
    &\subseteq
    x' y' + A_{(i+j-1)} + A_{(i+j-1)} + A_{(i+j-2)}
    \\
    &\subseteq
    x' y' + A_{(i+j-1)}
  \end{align*}
  and hence~$[xy]_{i+j} = [x'y']_{i+j}$.
  
  The element~$[1]_0 \in \gr[0](A)$ satisfies~$[1]_0 \cdot [x]_i = [x]_i$ for all~$i \geq 0$ and every~$[x]_i \in \gr[i](A)$, and the multiplications~$\mu_{i,j}$ are relatively associative in the sense that
  \[
    [x]_i \cdot ([y]_j \cdot [z]_k)
    =
    [xyz]_{i+j+k}
    =
    ([x]_i \cdot [y]_j) \cdot [z]_k
  \]
  for all~$i, j, k \geq 0$ and~$[x]_i \in \gr[i](A)$,~$[y]_j \in \gr[j](A)$ and~$[z]_k \in \gr[k](A)$.
  It follows from \cref{external description of graded algebras} that on the direct sum
  \[
    \gr(A)
    \defined
    \bigoplus_{i \geq 0} \gr[i](A)
  \]
  the multiplications~$\mu_{i,j}$ assemble into a single multiplication
  \[
    \mu
    \colon
    \gr(A) \times \gr(A)
    \to
    \gr(A)
  \]
  that makes~$\gr(A)$ into a graded~{\algebra{$\kf$}}.
  The multiplicative neutral element of~$\gr(A)$ is given by~$[1]_0$.
\end{construction}


\begin{definition}
  For a filtered~{\algebra{$\kf$}}~$A$ the graded algebra~\gls*{associated graded} resulting from~$A$ by \cref{construction of associated graded} is the \defemph{associated graded algebra}\index{associated graded algebra} of~$A$.
\end{definition}


\begin{remark}
  Let~$A$ be a filtered algebra.
  \begin{enumerate}
    \item
      To every element~$x \in A$ we can associate an element in~$\gr(A)$:
      We observe that if~$x$ is of degree~$i \geq 0$ then the residue class~$[x]_j$ is not defined for~$j < i$ because~$x \notin A_{(\spacing j)}$, whereas it is defined for all~$j \geq i$.
      But for~$j > i$ we find that~$[x]_j = 0$ because~$x_i \in A_{(i)} \subseteq A_{(\spacing j-1)}$.
      Hence the only interesting residue class associated to~$x$ is~$[x]_i$;
      we note that~$[x]_i \neq 0$ because~$x \notin A_{(i-1)}$.
    \item
      We observe that two elements~$x, y \in A$ give the same associated element in~$\gr(A)$ if and only if~$x$ and~$y$ are equal up to smaller degree.
      To show this we denote by~$\gamma \colon A \to \gr(A)$ the function that maps every~$x \in A$ to the associated element of~$\gr(A)$.
      
      Suppose first that~$x$ and~$y$ are equal up to smaller degree.
      Then~$x$ and~$y$ are of the same degree~$i$ and~$x - y \in A_{(i-1)}$.
      Hence~$\gamma(x) = [x]_i = [y]_i = \gamma(y)$.
      
      Suppose now that~$x$ and~$y$ are not equal up to smaller degree, where~$x$ is of degree~$i$ and that~$y$ is of degree~$j$.
      If~$i \neq j$ then~$\gamma(x) = [x]_i \neq [y]_j = \gamma(y)$ because the terms~$[x]_i$ or~$[y]_j$ live in different homogeneous degrees and at least one of them is nonzero.
      If~$i = j$ then~$x - y \notin A_{(i-1)}$ and hence~$\gamma(x) = [x]_i \neq [y]_i = \gamma(y)$.
    \item
      Observe that the image of~$\gamma$ consists precisely of the homogeneous elements of~$\gr(A)$.
      If~$x \in A$ has degree~$i$ then~$[x]_i$ is homogeneous of degree~$i$ (or at least for~$x$ nonzero).
  \end{enumerate}
\end{remark}


\begin{warning}
  Let~$A$ be a filtered~{\algebra{$\kf$}}
  \begin{enumerate}
    \item
      The map~$\gamma \colon A \to \gr(A)$ that assigns to each element~$x \in A$ the corresponding element of~$\gr(A)$ is in no way a homomorphism.
      It is in general neither multiplicative nor additive.
    \item
      If~$x_i$,~$i \in I$ is a generating set for the algebra~$A$ then the associated elements of~$\gr(A)$ do not have to form a generating set for~$\gr(A)$.
%     TODO: Add a reference to the universal eveloping algebra of heisenberg/sl2 later on.
  \end{enumerate}
\end{warning}


\begin{remark}
  \label{generators of associated graded}
  Let~$A$ be a filtered algebra with finite algebra generating set~$x_1, \dotsc, x_n$ such that the generator~$x_i$ is of degree~$d_i$.
  Then the associated elements in~$\gr(A)$ form a set of algebra generators of~$\gr(A)$ if and only if the generators~$x_1, \dotsc, x_n$ are compatible with the filtration of~$A$ in the sense that for every~$i \geq 0$ the linear subspace~$A_{(i)}$ is spanned by all those monomials~$x_{i_1} \dotsm x_{i_p}$ with~$d_{i_1} + \dotsb + d_{i_p} \leq i$.
  See \cite{associated_generated} for more details on this.
\end{remark}


\begin{example}[Weyl algebra]
  \label{weyl algebra}
  We consider the \defemph{Weyl~algebra}\index{algebra!Weyl}\index{Weyl!algebra}
  \[
    A
    \defined
    \kf\gen{X, Y}/(YX - XY - 1) \,.%
    \footnote{The algebra~$A$ is isomorphic to the subalgebra of~$\End_{\kf}(\kf[t])$ generated by the endomorphisms~$x$ and~$y$ that are given by~$x(p(t)) = t p(t)$ and~$y(p(t)) = p'(t)$.
  These endomorphisms satsify the relations~$yx - xy - \id = 0$ by the product rule.}
  \]
  We denote the residue classes of~$X$ and~$Y$ in~$A$ by~$x$ and~$y$.
  Then~$yx = xy + 1$, which one may think about as a rewriting rule for words in~$x$ and~$y$.
  It follows from induction that
  \[
    y x^n
    =
    x^n y + n x^{n-1}
  \]
  for all~$n \geq 0$ and thus
  \[
    y p(x)
    =
    p(x) y + p'(x)
  \]
  for every polynomial~$p$.
  
  The~{\algebra{$\kf$}}~$T \defined \kf\gen{X, Y}$ carries a unique grading with~$X$ and~$Y$ in degree~$1$.
  The resulting filtration of~$T$ is given by
  \[
    T_{(i)}
    =
    \gen{
      Z_1 \dotsm Z_j
    \suchthat
      1 \leq j \leq i,
      Z_k \in \{X, Y\}
    }_{\kf}
    =
    \gen{
      \text{words in~$X$,~$Y$ of length~$\leq i$}
    }
  \]
  The Weyl~algebra~$A$ inherits a filtration given by
  \begin{align*}
    A_{(i)}
    =
    \gen{
      z_1 \dotsm z_j
    \suchthat
      1 \leq j \leq i,
      z_k \in \{x, y\}
    }_\kf
    =
    \gen{
      \text{words in~$x$,~$y$ of length~$\leq i$}
    }
  \end{align*}
  
  We observe that the two elements~$x, y \in T$ commute up to smaller degree, i.e.\ the two products~$xy$ and~$yx$ are equal up to smaller degree.
  Hence the associated elements~$[x]_1$ and~$[y]_1$ in~$\gr(A)$ do actually commute because
  \[
    [x]_1 [y]_1
    =
    [xy]_2
    =
    [yx - 1]_2
    =
    [yx]_2
    =
    [y]_1 [x]_1 \,.
  \]
  We will in the following show that~$\gr(A) \cong \kf[t,u]$ with~$[x]_1$ and~$[y]_1$ corresponding to~$t$ and~$u$.
  
  \begin{claim}
    \label{subspace spanned by monomials}
    For every~$i \geq 0$ the linear subspace~$A_{(i)}$ is spanned by the monomial~$x^n y^m$ with~$n + m \leq i$.
  \end{claim}
  
  \begin{proof}
    We show the statement by induction over~$i$.
    It holds for~$i = 0$ that
    \[
      A_{(0)}
      =
      \kf
      =
      \gen{1}_{\kf}
      =
      \gen{x^0 y^0}_{\kf}
    \]
    Suppose now that the statement holds for some~$i \geq 0$.
    We have
    \[
      T_{(i+1)}
      =
      x T_{(i)} \oplus y T_{(i)}
    \]
    and in the quotient~$A$ therefore
    \[
      A_{(i+1)}
      =
      x A_{(i)} + y A_{(i)} \,.
    \]
    We find by the induction hypothesis that the summand~$x A_{(i)}$ is spanned by the monomials~$x^{n+1} y^m$ with~$n+m \leq i$ and that the summand~$y A_{(i)}$ is spanned by the monomials~$y x^n y^m$ with~$n + m \leq i$.
    We have for these generators of~$y A_{(i)}$ that
    \[
      y x^n y^m
      =
      (x^n y + n x^{n-1}) y^m
      =
      x^n y^{m+1} + n x^{n-1} y^m \,.
    \]
    This shows that~$A_{(i+1)}$ is spanned by the monomials~$x^n y^m$ with~$n+m \leq i+1$.
   \end{proof}
  
  It follows that the homogeneous component~$\gr[i](A)$ is spanned by the residue classes~$[x^n y^m]_i$ with~$n + m = i$.
  The collection of all these residue classes therefore spans~$\gr(A)$ as a vector space.
  These generators commute:
  
  \begin{claim}
    It holds for all~$i, j \geq 0$ and~$n, m, k, l \geq 0$ with~$i = n+m$ and~$j = k+l$ that
    \[
      [x^n y^m]_i \cdot [x^k y^l]_j
      =
      [x^{n+k} y^{m+l}]_{i+j} \,.
    \]
  \end{claim}
  
  \begin{proof}
    For the case~$i = 0$ we find that~$n = m = 0$ and hence
    \[
      [x^0 y^0]_0 \cdot [x^k y^l]_j
      =
      [1]_0 \cdot [x^k y^l]_j
      =
      [x^k y^l]_j
    \]
    For~$i = 1$ we have the two cases~$n = 1$,~$m = 0$ and~$n = 0$,~$m = 1$.
    We find in these cases that
    \[
      [x^1 y^0]_1 \cdot [x^k y^l]_j
      =
      [x]_1 \cdot [x^k y^l]_j
      =
      [x^{k+1} y^l]_{j+1}
    \]
    and
    \begin{align*}
      [x^0 y^1]_1 \cdot [x^k y^l]_j
      &=
      [y]_1 \cdot [x^k y^l]_j
      \\
      &=
      [y x^k y^l]_{j+1}
      \\
      &=
      [(x^k y + k x^{k-1}) y^l]_{j+1}
      \\
      &=
      [x^k y^{l+1} + k x^{k-1} y^l]_{j+1}
      \\
      &=
      [x^k y^{l+1}]_{j+1} + k [x^{k-1} y^l]_{j+1}
      \\
      &=
      [x^k y^{l+1}]_{j+1}
    \end{align*}
    where we used that~$x y^k = y^k x + k y^{k-1}$ and that~$[z]_{j+1} = 0$ for all~$z \in A_{(\spacing j)}$.
    It follows for the general case that
    \begin{align*}
      {}&
      [x^n y^m]_i \cdot [x^k y^l]_j
      \\
      ={}&
      \underbrace{ [x]_1 \dotsm [x]_1 }_{n}
      \underbrace{ [y]_1 \dotsm [y]_1 }_{m}
      \cdot
      [x^k y^l]_j
      \\
      ={}&
      \underbrace{ [x]_1 \dotsm [x]_1 }_{n}
      \underbrace{ [y]_1 \dotsm [y]_1 }_{m-1}
      \cdot
      [x^{1+k}  y^l]_j
      \\
      ={}&
      \dotsb
      \\
      ={}&
      [x^{n+k} y^{m+l}]_{i+j}
    \end{align*}
    as claimed.
  \end{proof}

  We have now seen that~$\gr(A)$ is spanned by the monomials~$[x^n y^m]_{n+m}$ with~$n, m \geq 0$ and these monomials commute with each other.
  It only remains to show that these monomials~$[x^n y^m]_{n+m}$ are linearly indepependent.
  For this it sufficies to show the following:
  
  \begin{claim}
    \label{linear independence of monomials}
    The monomials~$x^n y^m$ with~$n, m \geq 0$ in~$A$ are linearly independent.
  \end{claim}
  
  We will defer the proof of \cref{linear independence of monomials} to a later occasion, namely to \cref{linear independence for weyl algebra}.
  
  It follows from \cref{subspace spanned by monomials} and \cref{linear independence of monomials} that for every~$i \geq 0$ the quotient~$\gr[i](A) = A_{(i)} / A_{(i-1)}$ has the residue classes~$[x^n y^m]_{n+m}$ with~$n+m = i$ as a basis, and hence overall that the residue classes~$[x^n y^m]_{n+m}$ with~$n, m \geq 0$ form a basis of~$\gr(A)$.
  
  We find altogether that~$\gr(A) \cong \kf[t,u]$ with~$[x]_1$ and~$[y]_1$ corresponding to the variables~$x$ and~$y$.
\end{example}


\begin{remark}
  The Weyl algebra is an example of a \defemph{skew polynomial ring}:
  If~$R$ is a~{\algebra{$\kf$}} and~$\delta$ is a derivation of~$R$ then one can endow the polynomial vector space~$A = R[t]$ uniquely with the structure of a~{\algebra{$\kf$}}, different from the usual one (unless~$\delta = 0$), such that
  \begin{itemize}
    \item
      the inclusion~$R \to A$,~$r \mapsto r t^0$ is a homomorphism of~{\algebras{$\kf$}},
    \item
      $t^i$ is the~{\howmanyth{$i$}} power of~$t$ in~$A$ and
    \item
      the variable~$t$ interacts with the elements~$r \in R$ via~$t r = r t + \delta(r)$.
  \end{itemize}
  The resulting~{\algebra{$\kf$}} is a \defemph{skew polynomial algebra}\index{skew polyomial algebra} over~$R$ and denoted by~\gls*{skew polynomial algebra}.
  Observe that the multiplication in~$R[t;\delta]$ is built precisely so that the derivation~$[t,-]$ of~$R[t;\delta]$ acts on~$R \cong R t^0$ in the same way as~$\delta$.
  The following are two special cases of skew polynomial algebras:
  \begin{enumerate}
    \item
      If~$R$ is any~{\algebra{$\kf$}} and~$\delta = 0$ then~$R[t; \delta] = R[t]$ is the usual polynomial ring.
    \item
      If~$R = \kf[t]$ is the polynomial ring in one variable and~$\delta = \partial_t$ is the derivative with respect to the variable~$t$ than~$R[t;\delta] = R[t;\partial_t]$ is the Weyl~algebra.
  \end{enumerate}
%   Currently not needed and distracting.
%
%   Skew polynomial algebras can be further generalized:
%   If~$\sigma \colon R \to R$ is an algebra homomorphism then a~{\linear{$\kf$}} map~$\delta \colon R \to R$ is a~{\derivation{$\sigma$}}\index{$\sigma$-derivation}\index{derivation!$\sigma$-} if
%   \[
%     \delta(rs)
%     =
%     \delta(r) s + \sigma(r) \delta(s)
%   \]
%   for all~$r, s \in R$.
%   Then the polynomial vector space~$A = R[t]$ can be endowed uniquely with the structure of a~{\algebra{$\kf$}} such that
%   \begin{itemize}
%     \item
%       the inclusion~$R \to A$,~$r \mapsto r t^0$ is a homomorphism of~{\algebras{$\kf$}},
%     \item
%       $t^i$ is the~{\howmanyth{$i$}} power of~$t$ in~$A$ and
%     \item
%       the variable~$t$ interacts with the elements~$r \in R$ via~$t r = \sigma(r) t + \delta(r)$.
%   \end{itemize}
%   The resulting algebra~$A$ is an \defemph{Ore extension}\index{Ore extension} of~$R$ and is denoted by~\gls*{ore extension}.
\end{remark}


\begin{remark}[Weyl and Heisenberg]
  The Weyl~algebra~$A$ is connected to the {\threedimensional} Heisenberg~Lie~algebras~$\hlie$:
  Recall that the Lie~algebra~$\hlie$ has a basis~$p$,~$q$,~$c$ on which the Lie~bracket is given by~$[p,c] = [q,c] = 0$ and~$[p,q] = c$.
  We see that~$c$ is central in~$\hlie$.
  The Weyl~algebra~$A$ results from the universal enveloping algebra~$\Univ(\hlie)$ by~\enquote{setting~$c = 1$}.
  More precisely, we have that
  \[
    \Univ(\hlie)/(\class{c} - 1)
    \cong
    A
  \]
  with~$\class{p}$ and~$\class{q}$ corresponding to the elements~$y$ and~$x$ in the Weyl algebra.
  To formally prove this we abbreviate~$\Univ(\hlie)/(\class{c} - 1)$ as~$B$.
  
  The unique {\linear{$\kf$}} map~$\hlie \to A$ given by
  \[
    p \mapsto y \,,
    \quad
    q \mapsto x \,,
    \quad
    c \mapsto 1
  \]
  is a homomorphism of Lie~algebras and therefore induces an algebra homomorphism~$\phi \colon \Univ(\hlie) \to A$.
  The homomorphism~$\phi$ is given on the algebra generators~$\class{p}$,~$\class{q}$,~$\class{c}$ of~$\Univ(\hlie)$ by~$\phi(\class{p}) = y$,~$\phi(\class{q}) = x$ and~$\phi(\class{c}) \mapsto 1$.
  It holds that~$\phi(\class{c} - 1) = \phi(\class{c}) - \phi(1) = 1 - 1 = 0$ so~$\phi$ descends to a well-defined algebra homomorphism
  \[
    \Phi
    \colon
    B
    \to
    A
  \]
  that is given on the remaining algebra generators~$\class{p}$ and~$\class{q}$ of~$B$ by~$\Phi(\class{p}) = y$ and~$\Phi(\class{q}) = x$.
  
  In the other direction we can start off with the unique algebra homomorphism~$\psi \colon \kf\gen{X,Y} \to B$ given on the free variables~$X$ and~$Y$ by
  \[
    X \mapsto \class{q} \,,
    \quad
    Y \mapsto \class{p} \,.
  \]
  Then
  \[
    \psi(YX)
    =
    \class{p} \, \class{q}
    =
    \class{q} \, \class{p} + \class{[p,q]}
    =
    \class{q} \, \class{p} + \class{c}
    =
    \class{q} \, \class{p} + 1
    =
    \psi(XY + 1)
  \]
  and therefore~$\psi(YX - XY - 1) = 0$.
  It follows that~$\psi$ factors uniquely through an algebra homomorphism
  \[
    \Psi
    \colon
    A
    \to
    B
  \]
  that is given on the algebra generators~$x$ and~$y$ of~$A$ by~$\Psi(x) = \class{q}$ and~$\Psi(y) = \class{p}$.
  
  The two homomorphisms~$\Phi$ are~$\Psi$ are mutually inverse on the algebra generators~$y$ and~$x$ of~$A$ and~$\class{p}$ and~$\class{q}$ of~$B$.
  This means that~$\Phi$ and~$\Psi$ are mutually inverse algebra isomorphisms.
\end{remark}


% Bad example: Already graded
%
% \begin{example}[Quantum plane]
%   Let~$q \in \kf$ wit~$q \neq 0$.
%   The \emph{quantum plane} is the~{\algebra{$\kf$}}~$A \defined \kf\gen{X,Y}/(YX - qXY)$.%
%   \footnote{The usual polynomial ring~$\kf[x,y] \cong \kf\gen{X,Y}/(YX - XY)$ is the algebraic object corresponding to the plane~$\Aff^2 = k^2$ in the sense that~$\kf[x,y]$ is the algebra of (regular) functions on~$\Aff^2$.
%   The given algebra~$A$ should therefore correspond to some (non-existing) plane, the functions on which don’t commute.}
%   Let~$x, y \in A$ be the residue classes of~$X$ and~$Y$.
%   The~{\algebra{$\kf$}}~$A$ inherits from the standard grading of the free algebra~$\kf\gen{X,Y}$ a filtration given by
%   \[
%     A_{(i)}
%     =
%     \gen{
%       z_1 \dotsm z_j
%     \suchthat
%       0 \leq j \leq i,
%       z_1, \dotsc, z_j \in \{x, y\}
%     }_{\kf} \,.
%   \]
%   It can be shown as for the Weyl~algebra (actually easer than for the Weyl~algebra) for all~$i \geq 0$ the vector space~$A_{(i)}$ is spanned by the monomials~$x^n y^m$ with~$n + m \leq i$.
%   The quotient vector space~$\gr[i](A)$ is therefore spanned for every~$i \geq 0$ by the residue classes~$[x^n y^m]_i$ with~$n + m = i$.
%   The algebra~$\gr(A)$ is hence spanned by the collection of all such residue classes~$[x^n y^m]$ with~$n, m \geq 0$.
% \end{example}


\begin{warning}
  For a filtered~{\algebra{$\kf$}} there does not exist a \enquote{canonical} algebra homomorphism~$A \to \gr(A)$.
  There does not even have to exist any algebra homomorphism~$A \to \gr(A)$ as \cref{weyl algebra} shows:
  If~$A \defined \kf\gen{X,Y}/(YX - XY - 1)$ denotes the Weyl algebra with the filtration from \cref{weyl algebra} then we have seen that~$\gr(A) = \kf[t,u]$ is the commutative polynomial ring in two variables.
  Algebra homomorphisms~$A \to \gr(A)$ correspond bijectively to elements~$x, y \in \kf[t,u]$ which satisfy the commutator relation~$yx - xy -1 = 0$, i.e.~$yx = xy + 1$.
  But no such element exist in~$\kf[t,u]$.
  Hence there exist no algebra homorphism~$A \to \gr(A)$.
\end{warning}


\begin{construction}
  If~$f \colon A \to B$ is a homomorphism of filtered~{\algebras{$\kf$}} then for every~$i \geq 0$ the restriction~$f_{(i)}$ induces a linear map
  \[
    \gr[i](\spacing f)
    \colon
    \gr[i](A)
    \to
    \gr[i](B) \,,
    \quad
    [x]_i
    \mapsto
    [\spacing f(x)]_i  \,.
  \]
  These linear maps come together to form a linear map
  \[
    \gr(\spacing f)
    \colon
    \gr(A)
    \to
    \gr(B)  \,.
  \]
  This map is already an algebra homomorphism because~$\gr(\spacing f)([1]_0) = [\spacing f(1)]_0 = [1]_0$ and
  \begin{align*}
    \gr(\spacing f)( [x]_i [y]_j )
    =
    \gr(\spacing f)( [xy]_{i+j} )
    =
    [\spacing f(xy)]_{i+j}
    =
    [\spacing f(x) f(y)]_{i+j}
    &=
    [\spacing f(x)]_i [\spacing f(y)]_j
    \\
    &=
    \gr(\spacing f)([x]_i) \gr(\spacing f)([y]_j) \,.
  \end{align*}
  for all~$i, j \geq 0$ and all~$[x]_i \in \gr[i](A)$ and~$[y]_j \in \gr[j](A)$.  
  It holds for every filtered~{\algebra{$\kf$}}~$A$ that~$\gr(\id_A) = \id_{\gr(A)}$ and for all homomorphisms of filtered~{\algebras{$\kf$}}~$f \colon A \to B$ and~$g \colon B \to C$ that~$\gr(\spacing g \circ f) = \gr(\spacing g) \circ \gr(\spacing f)$.
  
  We have hence constructed a functor~$\gr \colon \cfAlg{\kf} \to \cgAlg{\kf}$.
\end{construction}



% TODO: Move this at the right position.
% 
% \begin{examples}
%  \begin{enumerate}[leftmargin=*]
%   \item
%    Let $\glie$ be a $k$-Lie algebra. The canonical projection
%    \[
%     \pi \colon T(\g) \to \Univ(\glie), \quad x_1 \tensor \dotsb \tensor x_n \to x_1 \dotsm x_n
%     \quad \text{for every $n \in \N$ and all $x_1, \dotsc, x_n \in \g$}
%    \]
%    is a homomorphism of filtered \algebras{$\kf$}, where the filtration of $T(\g)$ is induced by the gradation discussed in Examples \ref{expls: graded algebras}. Hence it induces a homomorphism of graded \algebras{$\kf$}
%    \[
%     \gr(\pi) \colon T(\g) \to \gr(\Univ(\glie)),
%    \]
%    where $T(\g)$ is identified with $\gr(T(\g))$ as above. This homomorphism maps an element $x_1 \tensor \dotsb \tensor x_n$ with $x_1, \dots, x_n \in \g$ to the residue class $[x_1 \dotsm x_n] \in \gr(\Univ(\glie))_n$.
%  \end{enumerate}
% \end{examples}
% 
% 

\begin{lemma}
  Let~$A$ be a graded~{\algebra{$\kf$}}.
  Then the linear map
  \[
    \varphi 
    \colon
    A
    \to
    \gr(A)
  \]
  given by~$\varphi(x) = [x]_i$ for all~$i \geq 0$ and~$x \in A_i$ is an isomorphism of graded~{\algebras{$\kf$}}.
\end{lemma}


\begin{proof}
  We have for all~$i \geq 0$ that~$A_{(i)} = \bigoplus_{j=0}^i A_j$ and hence
  \[
    \gr[i](A)
    =
    A_{(i)} / A_{(i-1)}
    \cong
    A_i
  \]
  as vector spaces.
  It follows that~$\varphi$ is an isomorphism of vector spaces.
  Moreover, we have for all homogeneous elements~$x \in A_i$ and~$y \in A_j$ with~$i, j \geq 0$ that
  \[
    \varphi(x) \varphi(y)
    =
    [x]_i [y]_j
    =
    [xy]_{i+j}
    =
    \varphi(xy)
  \]
  where the last equality holds because~$xy \in A_{i+j}$.
  This shows that~$\varphi$ is multiplicative and hence already an isomorphism of~{\algebras{$\kf$}}.
\end{proof}


\begin{example}
  Let~$A \defined \kf\gen{X, Y}/(YX - XY - 1)$ be the Weyl~algebra with the filtration from \cref{weyl algebra}.
  Then~$\gr(A) \cong \kf[t,u]$ is commutative but~$A$ is not, hence~$\gr(A)$ and~$A$ are not isomorphic as algebras.
  It follows that the filtration on~$A$ cannot come from a grading, as otherwise~$\gr(A) \cong A$ (already as graded~{\algebras{$\kf$}}).
\end{example}


\begin{warning}
  If~$A$ is a filtered~{\algebra{$\kf$}} for which the filtration does not come from a grading then~$A$ and~$\gr(A)$ are not isomorphic as filtered algebras (where the filtration of~$\gr(A)$ is induced by the grading).
\end{warning}


\begin{definition}
  Let~$A$ be a ring.
  \begin{enumerate}
    \item
      An element~$x \in A$ is a \defemph{left zero divisor}\index{zero divisor!left} if there exists some nonzero~$y \in A$ with~$xy = 0$.
    \item
      An element~$x \in A$ is a \defemph{right zero divisor}\index{zero divisor!right} if there exists some nonzero~$y \in A$ with~$yx = 0$.
    \item
      An element~$x \in A$ is a \defemph{zero divisor}\index{zero divisor} if it is a left zero divisor or a right zero divisor.
  \end{enumerate}
\end{definition}


\begin{remark}
  Let~$A$ be a ring.
  \begin{enumerate}
    \item
      An element~$x \in A$ is a left zero divisor if and only if the left multiplication map~$A \to A$,~$y \mapsto xy$ is not injective.
      It is a right zero divisor if and only if the right multiplication map~$A \to A$,~$y \mapsto yx$ is not injective.
    \item
      The zero element~$0 \in A$ is both a left zero divisor and right zero divisor if~$A \neq 0$.
      It is not a zero divisor for~$A = 0$.
  \end{enumerate}
\end{remark}


\begin{proposition}
  \label{associated graded algebra and zero divisors}
  Let~$A$ be a filtered~{\algebra{$\kf$}}.
  \begin{enumerate}
    \item
      \label{associated graded has left zero divisors}
      If~$\gr(A)$ doesn’t contain nonzero left zero divisors then neither does~$A$.
    \item
      \label{associated graded has right zero divisors}
      If~$\gr(A)$ doesn’t contain nonzero right zero divisors then neither does~$A$.
    \item
      If~$\gr(A)$ doesn’t contain nonzero zero divisors then neither does~$A$.
  \end{enumerate}
\end{proposition}


\begin{proof}
  \leavevmode
  \begin{enumerate}
    \item
      Suppose that~$A$ contains a nonzero left zero divisor~$x$.
      Then there exists some nonzero~$y \in A$ with~$xy = 0$.
      If~$x$ is of degree~$i$ and~$y$ is of degree~$j$ then~$[x]_i \neq 0$ and~$[y]_j \neq 0$ because both~$x$ and~$y$ are nonzero.
      But~$[x]_i [y]_j = [xy]_{i+j} = [0]_{i+j} = 0$.
      Hence~$[x]_i$ is again a nonzero left zero divisor.
    \item
      This can be shown in the same way as part~\ref*{associated graded has left zero divisors}.
    \item
      This is a combination of parts~\ref*{associated graded has left zero divisors} and~\ref*{associated graded has right zero divisors}.
    \qedhere
  \end{enumerate}
\end{proof}


\begin{remark}
  The converse of \cref{associated graded algebra and zero divisors} does not hold.
  To see this let~$A$ be any~\algebra{$\kf$} with~$\kf \subsetneq A$ and consider the filtration
  \[
    A_{(0)}
    =
    \kf
    \subseteq
    A
    \subseteq
    A
    \subseteq
    A
    \subseteq
    \dotsb
  \]
  Then~$\gr[0](A) = \kf$,~$\gr[1](A) = A/{\kf} \neq 0$ and~$\gr[i](A) = 0$ for every~$i \geq 2$.
  Hence~$\gr[1](A) \gr[1](A) = 0$ even though~$\gr[1](A) \neq 0$.
  This means that the elements of~$\gr[1](A)$ are both left zero divisors and right zero divisors.
\end{remark}




