\section{The Poincaré--Birkhoff--Witt Theorem, Abstract Version}


\begin{theorem}[Poincaré--Birkhoff--Witt, abstract version]
	\index{Poincaré--Birkhoff--Witt!abstract version}
	\label{pbw abstract}
	Let~$\Pi$ be the canonical quotient homomorphism from~$\Tensor(\glie)$ to~$\Univ(\glie)$, given by
	\begin{alignat*}{2}
		\Pi
		&\colon
		\Tensor(\glie)
		\to
		\Univ(\glie) \,,
		&
		\quad
		x_1 \tensor \dotsb \tensor x_p
		&\mapsto
		x_1 \dotsm x_p
	\intertext{
	for all $x_1, \dotsc, x_p \in \glie$.
	Let~$\Pi'$ be the canonical quotient homomorphism from~$\Tensor(\glie)$ to~$\Symm(\glie)$, given by
	}
		\SwapAboveDisplaySkip
		\Pi'
		&\colon
		\Tensor(\glie)
		\to
		\Symm(\glie) \,,
		&
		\quad
		x_1 \tensor \dotsb \tensor x_p
		&\mapsto
		x_1 \dotsm x_p \,.
	\end{alignat*}
	for all~$x_1, \dotsc, x_p \in \glie$.
	The homomorphism~$\Pi$ is a homomorphism of filtered algebras and thus induces a homomorphism of graded algebras from~$\gr(\Tensor(\glie))$ to~$\gr(\Univ(\glie))$.
	We identify~$\gr(\Tensor(\glie))$ with~$\Tensor(\glie)$ and thus regard~$\gr(\Pi)$ as a homomorphism from~$\Tensor(\glie)$ to~$\gr(\Univ(\glie))$.

	The two homomorphisms~$\gr(\Pi)$ and~$\Pi'$ have the same kernel and thus induce an isomorphism of graded algebras~$\Phi$ from~$\Symm(\glie)$ to~$\Univ(\glie)$, given by
	\begin{equation}
		\label{definition of abstract pbw isomorphism}
		\Phi
		\colon
		\Symm(\glie)
		\to
		\gr(\Univ(\glie)) \,,
		\quad
		x_1 \dotsm x_p
		\mapsto
		\fclass{ x_1 \dotsm x_p }_p
	\end{equation}
	for all~$p \geq 0$ and~$x_1, \dotsc, x_p \in \glie$.
\end{theorem}


\begin{proposition}
	The abstract version of the PBW~theorem and the concrete version of the PBW~theorem are equivalent.
\end{proposition}


\begin{proof}
	Let~$x$ and~$y$ be two elements of~$\glie$.
	The element~$x \otimes y - y \otimes x$ of~$\Tensor(\glie)$ is homogeneous of degree~$2$.
	It is mapped by the homomorphism~$\gr(\Pi)$ to the element~$\fclass{ x y - y x }_2$ of~$\gr(\Univ(\glie))$.
	But we have in~$\Univ(\glie)$ the identity~$xy - yx = [x,y]_{\glie}$, so
	\[
		\fclass{ xy - yx }_2
		=
		\fclass{ [x,y]_{\glie} }_2 \,.
	\]
	The commutator~$[x,y]_{\glie}$ is an element of~$\glie$ and is therefore contained in~$\Univ(\glie)_{(1)}$.
	It follows that~$\fclass{ [x,y]_{\glie} }_2 = 0$.
	We have thus shown that for all~$x, y \in \glie$ the element~$x \otimes y - y \otimes x$ is contained in the kernel of~$\gr(\Pi)$.

	The kernel of~$\Pi'$ is generated by those differences as a two-sided ideal.
	We thus find that there exists a unique homomorphism of algebras~$\Phi$ from~$\Symm(\glie)$ to~$\gr(\Univ(\glie))$ which maps for every element~$t$ of~$\Tensor(\glie)$ the residue class of~$t$ in~$\Symm(\glie)$ to the residue class of~$t$ in~$\gr( \Univ(\glie) )$.
	This is precisely the proposed homomorphism~\eqref{definition of abstract pbw isomorphism}.
	This is a homomorphism of graded algebras because the grading on~$\Symm(\glie)$ is inherited from~$\Tensor(\glie)$ and~$\gr(\Pi)$ is a homomorphism of graded algebras.
	
	\begin{implicationlist}
		\item[concrete~$\implies$~abstract]
			It sufficies to show that for every natural number~$p$ the restriction~$\Phi_p$ of~$\Phi$ to a linear map from~$\Symm^p(\glie)$ to~$\gr[p](\Univ(\glie))$ is an isomorphism of vector spaces.
			The symmetric power~$\Symm^p(\glie)$ has the simple symmetric tensors~$x_{i_1} \dotsm x_{i_p}$ with~$i_1, \dotsc, i_p \in I$ and~$i_1 \leq \dotsb \leq i_p$ as a basis.
			It follows from \cref{pbw concrete basis part filtered part} that~$\gr[p]( \Univ(\glie) )$ has similarly the elements~$\fclass{ x_{i_1} \dotsm x_{i_p} }_p$ with~$i_1, \dotsc, i_p \in $,~$i_1 \leq \dotsb \leq i_p$ as a basis.
			The linear map~$\Phi_p$ restricts to a bijection between these bases and is therefore indeed an isomorphism of vector spaces.
		\item[abstract~$\implies$~concrete]
			The symmetric algebra~$\Symm(V)$ has the elements
			\begin{align*}
				x_{i_1} \dotsm x_{i_p}
				&\qquad
				\text{with~$p \geq 0$ and~$i_1, \dotsc, i_p \in I$ with~$i_1 \leq \dotsb \leq i_p$}
			\intertext{
			as a basis.
			It follows that the elements
			}
				\fclass{ x_{i_1} \dotsm x_{i_p} }_p
				&\qquad
				\text{with~$p \geq 0$ and~$i_1, \dotsc, i_p \in I$ with~$i_1 \leq \dotsb \leq i_p$}
			\intertext{
			are a basis of~$\gr(\Univ(\glie))$ because~$\Phi$ is an isomorphism.
			It follows from \cref{checking basis via associated graded} that the elements
			}
				\SwapAboveDisplaySkip
				x_{i_1} \dotsm x_{i_p}
				&\qquad
				\text{with~$p \geq 0$ and~$i_1, \dotsc, i_p \in I$ with~$i_1 \leq \dotsb \leq i_p$}
			\end{align*}
			are a basis of~$\Univ(\glie)$.
		\qedhere
	\end{implicationlist}
\end{proof}


\begin{remark}
	One may think about the universal enveloping algebra~$\Univ(\glie)$ as a deformation\index{deformation} of the symmetric algebra~$\Symm(\glie)$, in the sense that we have a family of~\algebras{$\kf$}
	\[
		A_t
		\defined
		\Tensor(\glie)
		/
		\ideal{ x \tensor y - y \tensor x - t[x,y] \suchthat x, y \in \glie }
	\]
	that is parametrized by a scalar~$t$ such that~$A_0 \cong \Symm(\glie)$ and~$A_1 \cong \Univ(\glie)$.
	One can actually prove the abstract version of the PBW~theorem by examining this deformation, as explained in~\cite{pbw_deformation}.
\end{remark}





