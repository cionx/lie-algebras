\section{New Representations from Old Ones}


\begin{proposition}
	\label{new representations from old ones}
	Let~$\glie$ be a Lie~algebra.
	\begin{enumerate}
		\item
			Let~$M_\lambda$ with~$\lambda$ in~$\Lambda$ be a collection of representations of~$\glie$.
			Then the product~$\prod_{\lambda \in \Lambda} M_\lambda$\glsadd{product of representations}\index{product of representations} becomes again a representation of~$\glie$ via the action given by
			\[
				x \act (m_\lambda)_\lambda
				\defined
				( x \act m_\lambda )_\lambda
				\qquad
				\text{for all~$x \in \glie$ and~$(m_\lambda)_\lambda \in \prod_{\lambda \in \Lambda} M_\lambda$.}
			\]
		\item
			Let~$M_\lambda$ with~$\lambda$ in~$\Lambda$ be a collection of representations of~$\glie$.
			Then the direct sum~$\bigoplus_{\lambda \in \Lambda} M_\lambda$\glsadd{direct sum of representations}\index{direct sum of representations} becomes again a representation of~$\glie$ via the action given by
			\[
				x \act (m_\lambda)_\lambda
				\defined
				( x \act m_\lambda )_\lambda
				\qquad
				\text{for all~$x \in \glie$ and~$(m_\lambda)_\lambda \in \bigoplus_{\lambda \in \Lambda} M_\lambda$.}
			\]
		\item
			Let~$M$ and~$N$ be two representations of~$\glie$.
			Their tensor product~$M \tensor N$\index{tensor product of representations} becomes again a representation of~$\glie$ via the action
			\[
				x \act (m \tensor n)
				\defined
				(x \act m) \tensor n + m \tensor (x \act n)
				\qquad
				\text{for all~$x \in \glie$,~$m \in M$,~$n \in N$.}
			\]
			Let more generally~$M_1, \dotsc, M_r$ be a finite collection of representations of~$\glie$.
			Then their tensor product~$M_1 \tensor \dotsb \tensor M_r$\glsadd{tensor product of representations} becomes again a representation of~$\glie$ via the action
			\[
				x \act (m_1 \tensor \dotsb \tensor m_r)
				\defined
				\sum_{i=1}^r
								m_1
				\tensor \dotsb
				\tensor m_{i-1}
				\tensor (x \act m_i)
				\tensor m_{i+1}
				\tensor \dotsb
				\tensor m_r
			\]
			for all~$x \in \glie$ and all~$m_i \in M_i$ for~$i = 1, \dotsc, r$.
		\item
			Let~$M$ and~$N$ be two representations of~$\glie$.
			The vector space~$\Hom_{\kf}(M,N)$\glsadd{internal hom representation}\index{Hom-representation@$\Hom$-representation} becomes again a representation of~$\glie$ via the action
			\[
				(x \act f)(m)
				\defined
				x \act f(m) - f(x \act m)
			\]
			for all~$x \in \glie$,~$f \in \Hom_{\kf}(M,N)$,~$m \in M$.
			% TODO: Add abstract Herleitung: Have induced left and right actions, they commute, so have U(g) tensor U(g) module. Now use comult to get U(g) module.
		\item
			Every~{\vectorspace{$\kf$}}~$M$ becomes a representation of~$\glie$ via the \defemph{trivial action}\index{action!trivial}\index{trivial action} given by
			\[
				x \act m
				\defined
				0
				\qquad
				\text{for all~$x \in \glie$ and~$m \in M$.}
			\]
		\item
			Let~$M$ be a representation of~$\glie$.
			The dual space~$M^*$\glsadd{dual space representation}\index{dual space of representation} becomes again a representation of~$\glie$ via the action
			\[
				(x \act m^*)(m)
				\defined
				- m^*(x \act m)
				\qquad
				\text{for all~$x \in \glie$,~$m^* \in M^*$,~$m \in M$.}
			\]
	\end{enumerate}
\end{proposition}


\begin{proof}
	\leavevmode
	\begin{enumerate}
		\item
			We have for all elements~$x$,~$y$ of~$\glie$ and every element~$(m_\lambda)_\lambda$ of~$\prod_{\lambda \in \Lambda} M_\lambda$ that
			\begin{align*}
				x \act y \act (m_\lambda)_\lambda
				- y \act x \act (m_\lambda)_\lambda
				&=
				(x \act y \act m_\lambda)_\lambda
				- (y \act x \act m_\lambda)_\lambda
				\\
				&=
				(x \act y \act m_\lambda - y \act x \act m_\lambda)_\lambda
				\\
				&=
				([x,y] \act m_\lambda)_\lambda
				\\
				&=
				[x,y] \act (m_\lambda)_\lambda \,.
			\end{align*}
		\item
			That the given formula gives an action of~$\glie$ on the direct sum~$\bigoplus_{\lambda \in \Lambda} M_\lambda$ follows from the same calculation as for the product~$\prod_{\lambda \in \Lambda} M_\lambda$.
		\item
			It sufficies by induction to consider the case~$r = 2$, i.e. the binary tensor product~$M \tensor N$.
			We find in this case for every element~$x$ of~$\glie$, every element~$m$ of~$M$ and every element~$n$ of~$N$ that
			\begin{align*}
				x \act y \act (m \tensor n)
				&=
				x \act ( (y \act m) \tensor n
				+ m \tensor (y \act n) )
				\\
				&=
				x \act ( (y \act m) \tensor n)
				+ x \act (m \tensor (y \act n) )
				\\
				&=
				(x \act y \act m) \tensor n
				+ (y \act m) \tensor (x \act n)
				+ (x \act m) \tensor (y \act n)
				+ m \tensor (x \act y \act n)
			\end{align*}
			and therefore
			\begin{align*}
				{}&
				x \act y \act (m \tensor n)
				- y \act x \act (m \tensor n)
				\\
				={}&
				(x \act y \act m) \tensor n
				+ m \tensor (x \act y \act n)
				- (y \act x \act m) \tensor n
				- m \tensor (y \act x \act n)
				\\
				={}&
				(x \act y \act m - y \act x \act m) \tensor n
				+ m \tensor (x \act y \act n - y \act x \act n)
				\\
				={}&
				([x,y] \act m) \tensor n
				+ m \tensor ([x,y] \act n) \,.
			\end{align*}
		\item
			We find all elements~$x$,~$y$ of~$\glie$, every element~$f$ of~$\Hom_{\kf}(M, N)$ and every element~$m$ of~$M$ that
			\begin{align*}
				\SwapAboveDisplaySkip
				(x \act y \act f - y \act x \act f)(m)
				&=
				(x \act y \act f)(m) - (y \act x \act f)(m)
				\\
				&=
				-(y \act f)(x \act m) + (x \act f(y \act m)
				\\
				&=
				f(y \act x \act m) - f(x \act y \act m)
				\\
				&=
				-f(x \act y \act m - y \act x \act m)
				\\
				&=
				-f([x,y] \act m)
				\\
				&=
				([x,y] \act f)(m)  \,,
			\end{align*}
			and therefore
			\[
				x \act y \act f - y \act x \act f
				=
				[x,y] \act f \,.
			\]
		\item
			It holds for all element~$x$,~$y$ of~$\glie$ and every element~$m$ of~$M$ that
			\[
				x \act y \act m - y \act x \act m
				=
				0 - 0
				=
				0
				=
				[x,y] \act m \,.
			\]
		\item
			The action of~$\glie$ on~$M$ together with the trivial action of~$\glie$ on~$\kf$ induce an action of~$\glie$ on~$\Hom_{\kf}(M, \kf) = M^*$.
			This induced action is precisely the desired one.
		\qedhere
	\end{enumerate}
\end{proof}


\begin{remark}
	Let~$M$ be a representation of~$\glie$.
	The action of~$\glie$ on the dual space~$M^*$ can also be explained in a more systematic way, as we will now explain.

	Let~$\rho$ be the homomorphism of Lie~algebras from~$\glie$ to~$\gllie(M)$ corresponding to the given representation~$M$.
	The algebra anti-homomorphism
	\[
		\End_{\kf}(M)
		\to
		\End_{\kf}(M^*) \,,
		\quad
		f
		\mapsto
		f^*
	\]
	is in particular an anti-homomorphism of Lie~algebras from~$\gllie(M)$ to~$\gllie(M^*)$.
	The resulting composition
	\[
		\glie
		\xto{\rho}
		\gllie(M)
		\to
		\gllie(M^*)
	\]
	is again an anti-homorphism of Lie~algebras.
	This anti-homomorphism corresponds to a right action of~$\glie$ on~$M^*$ given by
	\[
		m^* \act x
		\defined
		\rho(x)^*(m^*)
		=
		m^* \circ \rho(x)
	\]
	for all~$x \in \glie$ ~$m^* \in M^*$.
	This action can more explicitely be rewritten as
	\[
		(m^* \act x)(m)
		=
		m^*(x \act m)
	\]
	for all~$x \in \glie$,~$m^* \in M^*$,~$m \in M$.
	This right action of~$\glie$ on~$M^*$ corresponds to a left action of~$\glie$ on~$M$ via the formula
	\[
		x \act m^*
		\defined
		-m^* \act x
	\]
	for all~$x \in \glie$ and~$m^* \in M^*$.
	This left action can more explicitely be rewritten as
	\[
		(x \act m^*)(m)
		=
		-m^*(x \act m)
	\]
	for all~$x \in \glie$,~$m^* \in M^*$,~$m \in M$.
	This is precisely the action of~$\glie$ on~$M^*$ that we have seen in \cref{new representations from old ones}.
\end{remark}


\begin{definition}
	Let~$M$ be a representation of a Lie~algebra~$\glie$.
	A \defemph{subrepresentation}\index{subrepresentation} of~$M$ is a linear subpace~$N$ of~$M$ such that
	\[
		x \act n \in N
		\qquad
		\text{for all~$x \in \glie$ and~$n \in N$.}
	\]
\end{definition}


\begin{remark}
	Let~$(M, \rho)$ be a representation of a Lie~algebra~$\glie$.
	A linear subspace~$N$ of~$M$ is a subrepresentation of~$M$ if and only if~$N$ is~{\invariant{$\rho(x)$}} for every element~$x$ of~$\glie$, in the sense that~$\rho(x)(N) \subseteq N$.
\end{remark}


\begin{examples}
	Let~$\glie$ be a Lie~algebra.
	\begin{enumerate}
		\item
			Let~$M$ be a representation of~$\glie$.
			The linear subspaces~$0$ and~$M$ are subrepresentations of~$M$.%
		\footnote{
			These two subrepresentations are often called the \defemph{trivial} ones.
			We will abstain from doing so because we will introduce the notion of a trivial representation in \cref{trivial representations}.
		}
		\item
			Let~$M$ be a representation of~$\glie$.
			Let~$N_\lambda$ with~$\lambda$ in~$\Lambda$ be a collection of subrepresentations~$N_\lambda$ of~$M$.
			Then both the intersection~$\bigcap_{\lambda \in \Lambda} N_\lambda$\index{intersection!of subrepresentations} and the sum~$\sum_{\lambda \in \Lambda} N_\lambda$\index{sum!of subrepresentations} are again subrepresentations of~$M$.
		\item
			Let~$M_\lambda$ with~$\lambda$ in~$\Lambda$ be a collection of representations of~$\glie$.
			The direct sum~$\bigoplus_{\lambda \in \Lambda} M_\lambda$ is a subrepresentation of the product~$\prod_{\lambda \in \Lambda} M_\lambda$.
		\item
			The subrepresentations of the adjoint representation of~$\glie$ are precisely the ideals of~$\glie$.
		\item
			Let~$M$ be any representation of~$\glie$.
			The the linear subspace~$\glie M$ of~$M$ given by
			\[
				\glie M
				\defined
				\gen{
					x \act m
					\suchthat
					x \in \glie,
					m \in M
				}_{\kf}
				\glsadd{certain subrepresentation}
			\]
			is a subrepresentation of~$\glie$.
			More generally, every linear subspace of~$M$ that contains~$\glie M$ is a subrepresentation of~$M$.
	\end{enumerate}
\end{examples}


\begin{definition}
	\label{trivial representations}
	Let~$M$ be a representation of a Lie~algebra~$\glie$.
	\begin{enumerate}
		\item
			The representation~$M$ is \defemph{trivial}\index{trivial representation} if every element~$x$ of~$\glie$ acts with the zero endomorphism on~$M$, i.e. if~$x \act m = 0$ for all~$x \in \glie$ and~$m \in M$.
		\item
			An element~$m$ of~$M$ is \defemph{\invariant{$\glie$}}\index{invariant}, or simply~\defemph{invariant}, if
			\[
				x \act m = 0
				\qquad
				\text{for all~$x \in \glie$.}
			\]
			The set of invariants of~$M$\index{invariants} is denoted by
			\[
				M^{\glie}
				\defined
				\{
					m \in M
				\suchthat
					\text{$x \act m = 0$ for every~$x \in \glie$}
				\}  \,.
				\glsadd{invariants of representation}
			\]
	\end{enumerate}
\end{definition}


% TODO: Explain connection to representations of groups.


\begin{remark}
	Let~$\glie$ be a~\liealgebra{$\kf$} and let~$M$ be a~\vectorspace{$\kf$}.
	The vector space~$M$ can be made into a trivial~\representation{$\glie$} in precisely one way.
	A trivial~\representation{$\glie$} is therefore \enquote{the same} as a vector space.

	That every~\vectorspace{$\kf$}~$M$ can be made into a representation of~$\glie$ corresponds to the fact that there always exists a homomorphism of Lie~algebras from~$\glie$ to~$\gllie(M)$, namely the zero homomorphism.
\end{remark}



\begin{example}
	Let~$\glie$ be a Lie~algebra.
	The space of invariants of the adjoint representation of~$\glie$ is given by
	\begin{align*}
		\SwapAboveDisplaySkip
		\glie^{\glie}
		&=
		\{
			y \in \glie
		\suchthat
			\text{$x \act y = 0$ for all~$x \in \glie$}
		\}
		\\
		&=
		\{
			y \in \glie
		\suchthat
			\text{$\ad(x)(y) = 0$ for all~$x \in \glie$}
		\}
		\\
		&=
		\{
			y \in \glie
		\suchthat
			\text{$[x,y] = 0$ for all~$x \in \glie$}
		\}
		\\
		&=
		\centerlie(\glie) \,.
	\end{align*}
\end{example}


\begin{proposition}
	Let~$\glie$ be a Lie~algebra and let~$M$ be a representation of~$\glie$.
	The space of invariants~$M^{\glie}$ is the largest trivial subrepresentation of~$M$.
	\qed
\end{proposition}


\begin{example}[Quotient representations]
	\label{quotient representation}
	Let~$M$ be a representation of a Lie~algebra~$\glie$ and let~$N$ be a subrepresentation of~$M$.
	The quotient vector space~$M/N$ inherits from~$M$ the structure of a~{\representation{$\glie$}} via the action of~$\glie$ on~$M/N$ given by
	\[
		x \act \class{m}
		\defined
		\class{x \act m}
		\qquad
		\text{for all~$x \in \glie$ acnd~$m \in M$}.
	\]

	Indeed, if two elements~$m_1$ and~$m_2$ of~$M$ result in the same element of~$M/N$, then the difference~$m_1 - m_2$ is contained in the subrepresentation~$N$.
	The difference
	\[
		x \act m_1 - x \act m_2
		=
		x \act (m_1 - m_2)
	\]
	is again contained in~$N$, whence the element~$x \act m_1$ and~$x \act m_2$ become equal in~$M/N$.
	We have thus a well-defined bilinear map
	\[
		\glie \times (M / N)
		\to
		M / N \,,
		\quad
		(x, \class{m})
		\mapsto
		\class{x \act m} \,.
	\]
	This is an action of~$\glie$ on~$M$ because we have for any two element~$x$,~$y$ of~$\glie$ and every element~$m$ of~$M$ that
	\begin{align*}
		\SwapAboveDisplaySkip
		[x,y] \act \class{m}
		&=
		\class{[x,y] \act m}
		\\
		&=
		\class{x \act (y \act m) - y \act (x \act m)}
		\\
		&=
		\class{x \act (y \act m)} - \class{y \act (x \act m)}
		\\
		&=
		x \act (y \act \class{m}) - y \act (x \act \class{m}) \,.
	\end{align*}
	
	Alternatively, let~$\rho$ be the homomorphism of Lie~algebras from~$\glie$ to~$\gllie(M)$ corresponding to the action of~$\glie$ on~$M$.
	Then the linear subspace~$N$ of~$M$ is~{\invariant{$\rho(x)$}} for all~$x \in \glie$, wence the endomorphism~$\rho(x)$ induces an endomorphism
	\[
		\induced{\rho(x)}
		\colon
		M/N
		\to
		M/N \,,
		\quad
		\class{m}
		\mapsto
		\class{\rho(x)(m)}
	\]
	for every element~$x$ of~$\glie$.
	The resulting map
	\[
		\induced{\rho}
		\colon
		\glie
		\to
		\gllie(M/N) \,,
		\quad
		x
		\mapsto
		\induced{\rho(x)}
	\]
	is again a homomorphism of Lie~algebras because
	\[
		\induced{\rho}([x,y])
		=
		\induced{\rho([x,y])}
		=
		\induced{[\rho(x), \rho(y)]}
		=
		[\induced{\rho(x)}, \induced{\rho(y)}]
		=
		[\induced{\rho}(x), \induced{\rho}(y)]
	\]
	for all~$x, y \in \glie$.
	The homomorphism~$\induced{\rho}$ makes~$M/N$ into a representation of~$\glie$ as desired.
\end{example}


\begin{definition}
	Let~$M$ be a representation of a Lie~algebra~$\glie$ and let~$N$ be a subrepresentation of~$M$.
	The representation~$M/N$\glsadd{quotient representation} from \cref{quotient representation} is the \defemph{quotient representation}\index{quotient!of representation} of~$M$ by~$N$.
\end{definition}


\begin{lemma}
	Let~$\glie$ be a Lie~algebra, let~$M$ be a representation of~$\glie$, and let~$d$ be a natural number.
	\begin{enumerate}
		\item
			The exterior power~$\Exterior^d(M)$\glsadd{exterior power}\index{exterior power} becomes a representation of~$\glie$ via the action
			\[
				x \act (m_1 \wedge \dotsb \wedge m_d)
				\defined
				\sum_{i=1}^d
				m_1 \wedge \dotsb \wedge m_{i-1} \wedge (x \act m_i) \wedge m_{i+1} \wedge \dotsb \wedge m_d
			\]
			for all~$x \in \glie$ and~$m_1, \dotsc, m_d \in M$.
		\item
			The symmetric power~$\Symm^d(M)$\glsadd{symmetric power}\index{symmetric power} becomes a representation of~$\glie$ via the action
			\[
				x \act (m_1 \dotsm m_d)
				\defined
				\sum_{i=1}^d
				m_1 \dotsm m_{i-1} {} (x \act m_i) {} m_{i+1} \dotsm m_d
			\]
			for all~$x \in \glie$ and~$m_1, \dotsc, m_d \in M$.
	\end{enumerate}
\end{lemma}


\begin{proof}
	\leavevmode
	\begin{enumerate}
		\item
			The exterior power~$\Exterior^d(M)$ is given by~$\Exterior^d(M) = M^{\tensor d} / N$ where the linear suspace~$N$ of~$M^{\tensor d}$ is given by
			\[
				N
				=
				\gen{
					m_1 \tensor \dotsb \tensor m_d
				\suchthat
					\text{$m_1, \dotsc, m_d \in M$,~$m_i = m_j$ for some~$i \neq j$}
				}_{\kf} \,.
			\]
			It sufficies to show that this linear subspace~$N$ is already a subrepresentation of~$M^{\tensor d}$.
			To show this, let~$m_1 \tensor \dotsb \tensor m_d$ be a simple tensor in~$M^{\tensor d}$ with~$m_i = m_j$ for some~$i < j$.
			It holds for every element~$x$ of~$\glie$ that
			\[
				x \act (m_1 \tensor \dotsb \tensor m_d)
				=
				\sum_{k=1}^d m_1
				\tensor \dotsb \tensor m_{k-1}
				\tensor (x \act m_k)
				\tensor m_{k+1} \tensor \dotsb \tensor m_d  \,.
			\]
			The summands for~$k \neq i,j$ are again contained in~$N$ as the tensor factors in the~{\howmanyth{$i$}} and~{\howmanyth{$j$}} place remain unchanged, and hence remain equal.
			With the vector~$n \defined m_i = m_j$ the two remaining summands may be rewritten as
			\begin{align*}
				{}&
					m_1 \tensor \dotsb
					\tensor (x \act n)
					\tensor \dotsb
					\tensor n
					\tensor \dotsb \tensor m_d
				\\
				{}&
				+ m_1 \tensor \dotsb
					\tensor n
					\tensor \dotsb
					\tensor (x \act n)
					\tensor \dotsb \tensor m_d
				\\
				={}&
					m_1 \tensor \dotsb
					\tensor ( (x \act n) + n )
					\tensor \dotsb
					\tensor ( (x \act n) + n )
					\tensor \dotsb \tensor m_d
				\\
				{}&
				- m_1 \tensor \dotsb
					\tensor (x \act n)
					\tensor \dotsb
					\tensor (x \act n)
					\tensor \dotsb \tensor m_d
				\\
				{}&
				- m_1 \tensor \dotsb
					\tensor n
					\tensor \dotsb
					\tensor n
					\tensor \dotsb \tensor m_d  \,.
			\end{align*}
			This term is again contained in~$N$.
			This shows altogether that~$N$ is indeed a subrepresentation of~$M^{\tensor d}$.
		\item
			Similarly to before we have to show that the linear subspace~$N$ of~$M^{\tensor d}$ given by
			\[
				N
				\defined
				\gen{
					m_1 \tensor \dotsb \tensor m_d
					- m_{\sigma(1)} \tensor \dotsb \tensor m_{\sigma(d)}
				\suchthat
					m_1, \dotsc, m_d \in M
				}_{\kf}
			\]
			is a subrepresentation of~$M^{\tensor d}$.
			We denote the permutation action of the symmetric group~$\Symm_d$ on the tensor power~$M^{\tensor d}$ from the right by
			\[
				t \act \sigma
			\]
			for all~$\sigma \in \symm_d$ and~$t \in M^{\tensor d}$.
			This action is on simple tensors given by
			\[
				(m_1 \tensor \dotsb \tensor m_d) \act \sigma
				=
				m_{\sigma(1)} \tensor \dotsb \tensor m_{\sigma(d)}
			\]
			for all~$\sigma \in \symm_d$ and~$m_1, \dotsc, m_d \in M$.
			We fix now an element~$x$ of~$\glie$ and denote for all indices~$i = 1, \dotsc, d$ by
			\[
				X_i
				\colon
				M^{\tensor d}
				\to
				M^{\tensor d}
			\]
			the linear map that is given on simple tensors by
			\[
				X_i(m_1 \tensor \dotsb \tensor m_d)
				=
				m_1 \tensor \dotsb \tensor m_{i-1}
				\tensor (x \act m_i)
				\tensor m_{i+1} \tensor \dotsb \tensor m_d
			\]
			for all~$m_1, \dotsc, m_d \in M$.
			
			We observe the relation
			\begin{equation}
				\label{interplay of permutation and multiplication}
				X_i( t \act \sigma )
				=
				X_{\sigma(i)}(t) \act \sigma
			\end{equation}
			for all~$\sigma \in \symm_d$,~$i = 1, \dotsc, d$.
			Indeed, let us consider for~$t$ a simple tensor~$m_1 \tensor \dotsb \tensor m_d$ with~$m_1, \dotsc, m_d \in M$.
			On both sides of~\eqref{interplay of permutation and multiplication} we reorder the tensor factors in the same way (namley via~$\sigma$) and act with~$x$ on the same tensor factor (namely~$m_{\sigma(i)}$).
			This shows the equality~\eqref{interplay of permutation and multiplication} for simple tensors.
			It therefore also holds for arbitrary tensors, i.e. for all elements of~$M^{\tensor d}$.
			
			It follows for every element~$t$ of~$M^{\tensor d}$ that
			\begin{align*}
				x \act ( t - t \act \sigma )
				&=
				x \act t - x \act (t \act \sigma)
				\\
				&=
				\sum_{i=1}^d X_i(t) - \sum_{i=1}^d X_i(t \act \sigma)
				\\
				&=
				\sum_{i=1}^d X_i(t) - \sum_{i=1}^d X_{\sigma(i)}(t) \act \sigma
				\\
				&=
				\sum_{i=1}^d X_i(t) - \sum_{i=1}^d X_i(t) \act \sigma
				\\
				&=
				\sum_{i=1}^d \biggl( X_i(t) - X_i(t) \act \sigma \biggr) \,.
			\end{align*}
			The summands~$X_i(t) - X_i(t) \act \sigma$ are again contained in~$N$.
			We have therefore shown that the linear subspace~$N$ of~$M^{\tensor d}$ is indeed a subrepresentation.
		\qedhere
	\end{enumerate}
\end{proof}





