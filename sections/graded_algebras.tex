\section{Graded \texorpdfstring{$\kf$}{k}-Algebras}


\begin{definition}
  \label{graded algebras}
  Let~$A$ be a~\algebra{$\kf$}.
  A \defemph{grading}\index{grading!of an algebra} or \defemph{gradation} of~$A$ is a direct sum decomposition
  \[
    A
    =
    \bigoplus_{p \geq 0} A_p
  \]
  such that
  \[
    A_p A_q
    \subseteq
    A_{p+q}
  \]
  for all~$p, q \geq 0$.
  A \defemph{graded~\algebra{$\kf$}}\index{graded!algebra} is a~\algebra{$\kf$}~$A$ together with a grading~$A = \bigoplus_{p \geq 0} A_p$.
\end{definition}


\begin{remark}
  We often say that~\enquote{$A$ is a graded algebra} without explicitely mentioning the grading.
\end{remark}


\begin{remark}
  Let~$(S, +)$ be an abelian monoid.
  An~\defemph{\grading{$S$}}\index{grading} of a~\algebra{$\kf$}~$A$ is a direct sum decomposition
  \[
    A
    =
    \bigoplus_{s \in S}
    A_s
  \]
  \[
    A_p A_q
    \subseteq
    A_{p + q}
  \]
  for all~$p, q \in S$.
  An~\defemph{\graded{$S$}}~\algebra{$\kf$} is a~\algebra{$\kf$}~$A$ together with an~\grading{$S$}~$A = \bigoplus_{s \in S} A_s$.

  A graded~\algebra{$\kf$} in the sense of \cref{graded algebras} is the special case of an~{\graded{$\Natural$}}~\algebra{$\kf$}.
  We will restrict our attention troughout these notes to~{\gradings{$\Natural$}}, and refer to \cite[II.{\S}11, III.{\S}3]{bourbaki_algebra} for more general gradings.
\end{remark}


\begin{definition}
  Let~$A$ be a graded~{\algebra{$\kf$}} with grading~$A = \bigoplus_{p \geq 0} A_p$.
  \begin{enumerate}
    \item
      An element~$x$ of~$A$ is \defemph{homogeneous}\index{homogeneous!element} of \defemph{degree}~$p$\index{homogeneous!degree}\index{degree!homogeneous} if~$x$ is contained in the direct summand~$A_p$.
    \item
      Let~$x$ be an element of~$A$ and let~$x = \sum_{p \geq 0} x_p$ be the decomposition of~$x$ with respect to the grading~$A = \bigoplus_{p \geq 0} A_p$.
      The summands~$x_p$ are the~\defemph{homogeneous components}\index{homogeneous!component} of~$x$.
  \end{enumerate}
\end{definition}


\begin{lemma}
  If~$A$ is a graded~\algebra{$\kf$} then the element~$1_A$ is homogeneous of degree~$0$.
\end{lemma}


\begin{proof}
  We consider the decomposition~$1 = \sum_{p \geq 0} x_p$ with respect to the grading~$A = \bigoplus_{p \geq 0} A_p$.
  We have for every degree~$q$ and every element~$a$ of~$A_q$ that
  \[
    a
    =
    1 \cdot a
    =
    \Biggl( \sum_{p \geq 0} e_p \Biggr) \cdot a
    =
    \sum_{p \geq 0} e_p a  \,.
  \]
  Each summand~$e_p a$ is homogeneous of degree~$p + q$.
  It therefore follows from the directness of the decomposition~$A = \bigoplus_{p \geq 0} A_p$ that the element~$a$ equals the summand~$e_0 a$ (and that all other summands vanish).
  It follows that~$e_0 a = a$ for all~$a \in A$, which shows that~$1 = e_0$.
\end{proof}


\begin{corollary}
  If~$A$ is a graded~\algebra{$\kf$} then the homogenous part~$A_0$ is a~{\subalgebra{$\kf$}} of~$A$.
  \qed
\end{corollary}



\begin{examples}
  \label{examples for graded algebras}
  \leavevmode
  \begin{enumerate}
    \item
      Any~\algebra{$\kf$}~$A$ becomes a graded~\algebra{$\kf$} by setting~$A_0 \defined A$ and~$A_p \defined 0$ for all~$p \geq 1$.
    \item
      The polynomial ring~$A = \kf[x_i \suchthat i \in I]$ is a graded~\algebra{$\kf$} by setting
      \[
        A_d
        \defined
        \gen{
          x_{i_1}^{n_1} \dotsm x_{i_r}^{n_r}
        \suchthat
          r \geq 0,
          i_1, \dotsc, i_r \in I,
          n_1 + \dotsb + n_r = d
        }_{\kf}
      \]
      for all~$d \geq 0$.
      The homogeneous part~$A_d$ consists of the homogeneous polynomials of degree~$d$, and each monomial~$x_{i_1}^{n_1} \dotsm x_{i_r}^{n_r}$ is homogeneous of degree~$n_1 + \dotsb + n_r$.
      
      We can more generally put the variable~$x_i$ is any degree~$d_i$, as follows.
      Given any family~$(d_i)_{i \in I}$ of natural numbers~$d_i$ we can define a grading on~$A$ via
      \[
        A_d
        \defined
        \gen{
          x_{i_1}^{n_1} \dotsm x_{i_r}^{n_r}
        \suchthat
          r \geq 0,
          i_1, \dotsc, i_r \in I,
          n_1 d_1 + \dotsb + n_r d_r = d
        }_{\kf}
      \]
      for all~$d \geq 0$.
      Then each monomials~$x_{i_1}^{n_1} \dotsm x_{i_r}^{n_r}$ is homogeneous of degree~$n_1 d_1 + \dotsb + n_r d_r$.
    \item
      Similarly the free~{\algebra{$\kf$}}~$A \defined \kf\gen{x_i \suchthat i \in I}$ can be graded via
      \[
        A_d
        \defined
        \gen{
          x_{i_1}^{n_1} \dotsm x_{i_r}^{n_r}
        \suchthat
          r \geq 0,
          i_1, \dotsc, i_r \in I,
          n_1 + \dotsb + n_r = d
        }_{\kf}
      \]
      for every~$d \geq 0$, and for any family~$(d_i)_{i \in I}$ of natural numbers $d_i$ via
      \[
        A_d
        \defined
        \gen{
          x_{i_1}^{n_1} \dotsm x_{i_r}^{n_r}
        \suchthat
          r \geq 0,
          i_1, \dotsc, i_r \in I,
          n_1 d_1 + \dotsb + n_r d_r = d
        }_{\kf} \,.
      \]
      This grading makes each monomial~$x_{i_1}^{n_1} \dotsm x_{i_r}^{n_r}$ homogeneous of degree~$n_1 d_1 + \dotsb + n_r d_r$.
    \item
      The tensor algebra~$\Tensor(V)$ of a vector space~$V$ has a grading given by~$\Tensor(V)_p \coloneqq V^{\tensor p}$ for all~$p \geq 0$.
      Similarly, both the symmetric algebra~$\Symm(V)$ and the exterior algebra~$\Exterior(V)$ have gradings given by~$\Symm(V)_p = \Symm^p(V)$ and~$\Exterior(V)_p = \Exterior^p(V)$ for all~$p \geq 0$.
    \item
      Let~$(M, {}\cdot{})$ be a multiplicative monoid and let~$M = \coprod_{p \geq 0} M_p$ be a grading of~$M$, i.e.\ a decomposition into subsets~$M_p$ of~$M$ such that~$M_p \cdot M_q \subseteq M_{p+q}$ for all~$p, q \geq 0$.
      Then the monoid algebra~$\kf[M]$ inhericts a grading via
      \[
        \kf[M]_p
        \defined
        \gen{ M_p }_{\kf}
      \]
      for all~$p \geq 0$.
      As special cases of this construction we get the following examples:
      \begin{itemize}
        \item
          Let~$I$ is an index set and let~$M = \Natural^{\oplus I}$.
          A grading on~$M$ is given by
          \[
            M_d
            \defined
            \left\{
              (n_i)_{i \in I} \in M
            \suchthat*
              \sum_{i \in I} n_i = d
            \right\}
          \]
          for all~$d \geq 0$.
          Then~$\kf[M]$ together with the induced grading is the commutative polynomial algebra~$\kf[x_i \suchthat i \in I]$.
        \item
          Let~$I$ be an index set and let let~$\Sigma \defined \{x_i \suchthat i \in I\}$ be an alphabet with letters~$x_i$.
          let~$M \defined \Sigma^*$ be the monoid of words in the alphabet~$\Sigma$ together with concatenation of words as its rule of composition.
          The monoid~$M$ has a grading given by
          \[
            M_d
            \defined
            \{
              w \in M
            \suchthat
              \text{$w$ is a word in~$\Sigma$ of length~$d$}
            \}
          \]
          for all~$d \geq 0$.
          The algebra~$\kf[M]$ together with the induced grading is the noncommutative polynomial algebra~$\kf\gen{x_i \suchthat i \in I}$.
      \end{itemize}
  \end{enumerate}
\end{examples}


\begin{remark}
  \label{external description of graded algebras}
  The grading of the tensor algebra~$\Tensor(V)$, symmetric algebra~$\Symm(V)$ and exterior algebra~$\Exterior(V)$ are basically built into the construction of these algebras.
  This way of constructing graded~{\algebras{$\kf$}} can be generalized as follows:
  
  Suppose that we are given a sequence of vector spaces~$A_p$ with~$p \geq 0$ and bilinear maps
  \[
    \mu_{p,q}
    \colon
    A_p \times A_q
    \to
    A_{p+q} \,,
    \quad
    (x,y)
    \mapsto
    xy
  \]
  for all~$p, q \geq 0$ such that
  \begin{itemize}
    \item
      the maps~$\mu_{p, q}$ are relatively associative in the sense that
      \[
        x(yz)
        =
        (xy)z
      \]
      for all~$p, q, r \geq 0$,~$x \in A_p$,~$y \in A_q$,~$z \in A_r$ and
    \item
      there exists an element~$1 \in A_0$ with
      \[
        1 x
        =
        x
        =
        x 1
      \]
      for all~$p \geq 0$,~$x \in A_p$.
  \end{itemize}
  For the direct sum~$A \defined \bigoplus_{p \geq 0} A_p$ we can fit together the partial multiplication maps~$\mu_{p,q}$ into a single multiplication~ map
  \[
    \mu
    \colon
    A \times A
    \to
    A
  \]
  that is given on elements~$x$,~$y$ of~$A$ with decompositions~$x = (x_p)_{p \geq 0}$ and~$y = (y_p)_{p \geq 0}$ by
  \[
    x y
    =
    \Biggl( \sum_{q=0}^p x_q y_{p-q} \Biggr)_{p \geq 0}  \,.
  \]
  The bilinearity of the map~$\mu$ follows from the bilinearities of the maps~$\mu_{p,q}$.
  It follows from the relative associativity of the multiplication maps~$\mu_{p,q}$ that the multiplication map~$\mu$ is associative.
  And the element~$1$ of~$A_0$ is multiplicative neutral for~$\mu$.
  
  This construction gives an external description of graded~{\algebras{$\kf$}}, in constrast to the previous internal description.
\end{remark}


\begin{definition}
  Let~$A$ and~$B$ be two graded~\algebras{$\kf$}.
  \begin{enumerate}
    \item
      A \defemph{homomorphism of graded~{\algebras{$\kf$}}}\index{homomorphism!of graded $\kf$-algebras} from~$A$ to~$B$ is a homomorphism of~{\algebras{$\kf$}}~$\Phi$ from~$A$ to~$B$ such that~$\Phi(A_p) \subseteq B_p$ for all~$p \geq 0$.
    \item
      Let~$\Phi$ be a homomorphism of graded algebras from~$A$ to~$B$.
      The restriction of~$\Phi$ to a linear map from~$A_p$ to~$B_p$ is for all~$p \geq 0$ denoted by~$\Phi_p$.
  \end{enumerate}
\end{definition}


\begin{remark}
  \leavevmode
  \begin{enumerate}
    \item
      Let~$A$,~$B$ and~$C$ be graded~{\algebra{$\kf$}}.
      \begin{enumerate}
        \item
          The identity map~$\id_A$ is a homorphism of~\algebras{$\kf$}.
        \item
          Suppose that~$\Phi$ is a homomorphism of graded algebras from~$A$ to~$B$ and that~$\Psi$ is a homomorphism of graded algebras from~$B$ to~$C$.
          The composite~$\Psi \circ \Phi$ is a homomorphism of graded algebras from~$A$ to~$C$.
      \end{enumerate}
      Hence the graded~\algebras{$\kf$} together with the homomorphisms of graded~\algebras{$\kf$} between them form a category, which we will denote by~$\cgAlg{\kf}$\glsadd{graded algebras}.
    \item
      For a homomorphism of graded~{\algebras{$\kf$}}~$\Phi$ from~$A$ to~$B$ the following conditions are equivalent.
      \begin{equivalenceslist}
        \item
          The homomorphism~$\Phi$ is an isomorphism of graded~{\algebras{$\kf$}}, i.e.\ there exists a homomorphism of graded algebras~$\Psi$ from~$B$ to~$A$ such that~$\Psi \circ \Phi = \id_A$ and~$\Phi \circ \Psi = \id_B$.
        \item
          The homomorphism~$\Phi$ is bijective.
        \item
          The restrictions~$\Phi_p$ are bijective for all~$p \geq 0$.
      \end{equivalenceslist}
  \end{enumerate}
\end{remark}


\begin{example}
  \leavevmode
  \begin{enumerate}
    \item
      Let~$V$ be a vector space.
      The quotient homomorphisms of algebras
      \begin{alignat*}{2}
        \Tensor(V)
        &\to
        \Symm(V) \,,
        &
        \quad
        x_1 \tensor \dotsb \tensor x_n
        &\mapsto
        x_1 \dotsm x_n
      \shortintertext{and}
        \Tensor(V)
        &\to
        \Exterior(V) \,,
        &
        \quad
        x_1 \tensor \dotsb \tensor x_n
        &\mapsto
        x_1 \wedge \dotsb \wedge x_n
      \end{alignat*}
      are homomorphisms of graded~\algebras{$\kf$}.
    \item
      Let~$V$ be a finite-dimensional vector space with basis~$x_1, \dotsc, x_n$.
      The isomorphism of algebras
      \[
        \kf[X_1, \dotsc, X_n]
        \to
        \Symm(V) \,,
        \quad
        X_i
        \mapsto
        x_i
      \]
      is an isomorphism of graded~\algebras{$\kf$}.
  \end{enumerate}
\end{example}


\begin{lemma}
  \label{characterizations of homogeneous ideals}
  Let~$I$ be some kind of ideal in a graded~{\algebra{$\kf$}}, i.e. a left ideal, right ideal or two-sided ideal.
  \begin{enumerate}
    \item
      The linear subspace~$\bigoplus_{p \geq 0} (I \cap A_p)$ is again an ideal in~$A$ of the same kind.
    \item
      The following conditions on the ideal~$I$ are equivalent.
      \begin{enumerate}
        \item
          \label{direct sum of linear subspaces}
          There exists linear subspaces~$I_p$ of~$A_p$ for all~$p \geq 0$ such that~$I = \bigoplus_{p \geq 0} I_p$.
        \item
          \label{direct sum of intersections}
          It holds that~$I = \bigoplus_{p \geq 0} (I \cap A_p)$.
        \item
          \label{contains all homogeneous components}
          The ideal~$I$ contains the homogeneous components of all its elements.
        \item
          \label{generated by homogeneous}
          The ideal~$I$ is generated by homogeneous elements.
      \end{enumerate}
  \end{enumerate}
\end{lemma}


\begin{proof}
  \leavevmode
  \begin{enumerate}
    \item
      Let~$I' \defined \bigoplus_{p \geq 0} I \cap A_i$.
      We first check that the linear subspace~$I'$ of~$A$ is a left ideal of~$A$ if~$I$ is a left ideal of~$A$.
      Indeed, we have
      \begin{align*}
        A \cdot I'
        &=
              \Biggl( \sum_{p \geq 0} A_p \Biggr)
        \cdot \Biggl( \sum_{q \geq 0} (I \cap A_q) \Biggr)
        \\
        &=
        \sum_{p, q \geq 0} ( A_p \cdot (I \cap A_q) )
        \\
        &\subseteq
        \sum_{p, q \geq 0} ( (A_p \cdot I) \cap (A_p \cdot A_q) )
        \\
        &\subseteq
        \sum_{p, q \geq 0} ( I \cap A_{p+q} )
        \\
        &=
        \sum_{r \geq 0} ( I \cap A_r )
        \\
        &=
        I'  \,.
      \end{align*}
      We find in the same way that the linear subspace~$I'$ of~$A$ is a right ideal of~$A$ if~$I$ is aright ideal of~$A$.
      It follows from these two cases that~$I'$ is a two-sided ideal of~$A$ if~$I$ is a left-sided ideal of~$A$.
    \item
      \begin{implicationlist}
        \item[\ref*{direct sum of linear subspaces}~$\implies$~\ref*{direct sum of intersections}]
          It follows from the assumption that~$I_p = I \cap A_p$ for all~$p \geq 0$.
          It follows from these equalities that altogether~$\bigoplus_{p \geq 0} (I \cap A_p) = \bigoplus_{p \geq 0} I_p = I$.
        \item[\ref*{direct sum of intersections}~$\implies$~\ref*{direct sum of linear subspaces}]
          We may take~$I_p = I \cap A_p$ for all~$p \geq 0$.
        \item[\ref*{direct sum of linear subspaces}~$\implies$~\ref*{contains all homogeneous components}]
          Let~$x$ be an element of~$I$.
          Let~$x = \sum_{p \geq 0} x_p$ be the decomposition of~$x$ into homogeneous components, and let~$x = \sum_{p \geq 0} x'_p$ be the decomposition of~$x$ with respect to the direcm sum decomposition~$I = \bigoplus_{p \geq 0} I_p$.
          Each summand~$x'_p$ is contained in~$I_p$ and thus~$A_p$.
          The summands~$x'_p$ are thus the homogeneous components of~$x_p$.
          It follows conversely that each homogeneous component~$x_p$ is contained in~$I_p$ and thus in~$I$.
        \item[\ref*{contains all homogeneous components}~$\implies$~\ref*{generated by homogeneous}]
          We may start with any generating set for~$I$ and then replace each generator by all of its homogeneous components.
        \item[\ref*{generated by homogeneous}~$\implies$~\ref*{direct sum of intersections}]
          Each homogeneous generater of~$I$ is contained in some intersection~$I \cap A_p$.
          It follows that the linear subspace~$\bigoplus_{p \geq 0} (I \cap A_p)$ contains all homogeneous generators of~$A$.
          This subspace is again a ideal of~$A$ (of the same type as the ideal~$I$), whence~$I = \bigoplus_{p \geq 0} (I \cap A_p)$.
        \qedhere
      \end{implicationlist}
  \end{enumerate}
\end{proof}


\begin{definition}
  Let~$A$ be a graded~\algebra{$\kf$}.
  An ideal~$I$ of~$A$ (of any kind) is a \defemph{homogeneous ideal}\index{homogeneous!ideal} if it satisfies the equivalent conditions from \cref{characterizations of homogeneous ideals}.
  If~$I$ is a homogeneous ideal of~$A$ then the intersection~$I \cap A_p$ is denoted by~$I_p$ for all~$p \geq 0$.
\end{definition}


\begin{example}
  Let~$A$ and~$B$ be two graded~\algebras{$\kf$} and let~$\Phi$ be a homomorphism of graded algebras from~$A$ to~$B$.
  Then the kernel of~$\Phi$ is a homogeneous ideal in~$A$, with~$\ker(\Phi)_p = \ker( \Phi_p )$ for all~$p \geq 0$.
\end{example}


\begin{proposition}
  Let~$A$ be a graded~\algebra{$\kf$} and let~$I$ be a two-sided homogeneous ideal in~$A$.
  Let~$\Pi$ be the quotient algebra homomorphism from~$A$ to~$A/I$.
  \begin{enumerate}
    \item
      The quotient algebra~$A/I$ inherits a grading from~$A$ with homogeneous parts
      \[
        (A/I)_p \defined \Pi(A_p)
      \]
      for all~$p \geq 0$, making~$A/I$ into a graded~\algebra{$\kf$}.
    \item
      This is the unique grading on the algebra~$A/I$ that makes the quotient homomorphism~$\Pi$ into a homomorphism of graded~\algebras{$\kf$}.
  \end{enumerate}
\end{proposition}


\begin{proof}
  \leavevmode
  \begin{enumerate}
    \item
      It follows from the isomorphisms
      \[
        A/I
        =
        \Biggl( \bigoplus_{p \geq 0} A_p \Biggr)
        \bigg/
        \Biggl( \bigoplus_{p \geq 0} I_p \Biggr)
        \cong
        \bigoplus_{p \geq 0} (A_p / I_p)
      \]
      that~$A/I = \bigoplus_{p \geq 0} (A/I)_p$.
      It also holds for all~$p, q \geq 0$ that
      \[
        (A/I)_p (A/I)_q
        =
        \Pi(A_p) \Pi(A_q)
        =
        \Pi(A_p A_q)
        \subseteq
        \Pi( A_{p+q} )
        =
        (A/I)_{p+q} \,.
      \]
    \item
      Any such grading~$A/I = \bigoplus_{p \geq 0} (A/I)_p$ must satisfy the condition~$\Pi(A_p) \subseteq (A/I)_p$ for all~$p \geq 0$.
      It now follows from these inclusions and the decomposition~$A/I = \bigoplus_{p \geq 0} \Pi(A_p)$ and~$A/I = \bigoplus_{p \geq 0} (A/I)_p$ that~$\Pi(A_p) = (A/I)_p$ for all~$p \geq 0$.
    \qedhere
  \end{enumerate}
\end{proof}


\begin{examples}
  Let~$V$ be a vector space.
  The two-sided ideal~$I$ of~$\Tensor(V)$ generated by the elements~$x \tensor y - y \tensor x$ with~$x$,~$y$ in~$V$ is a homogeneous ideal of~$\Tensor(V)$ because it is generated by homogeneous elements.
  Similarly, the two sided ideal~$J$ of~$\Tensor(V)$ generated by the elements~$x \tensor x$ with~$x$ in~$V$ is homogeneous because it generated by homogeneous elements.
  The quotient graded~{\algebras{$\kf$}}~$A/I$ and~$A/J$ are the symmetric algebra~$\Symm(V)$ and the exterior algebra~$\Exterior(V)$.
\end{examples}


% Grading is on the wrong level.
% 
% \begin{remark}
%   One can also consider graded Lie~algebras, and if~$\glie$ is a graded Lie~algebra then~$\Univ(\glie)$ inherits the structure of a graded~{\algebra{$\kf$}}:
%   \begin{enumerate}
%     \item
%       A \defemph{grading}\index{grading!of a vector space} of a vector space~$V$ is a direct sum decomposition~$V = \bigoplus_{i \geq 0} V_i$.
%       A \defemph{graded vector space}\index{graded!vector space} is a vector space~$V$ together with a grading of~$V$.
%     \item
%       A \defemph{grading}\index{grading!of a Lie algebra} of a Lie~algebra~$\glie$ is a direct sum decomposition~$\glie = \bigoplus_{i \geq 0} \glie_i$ such that~$[\glie_i, \glie_j] \subseteq \glie_{i+j}$ for all~$i, j \geq 0$.
%       A \defemph{graded Lie~algebra}\index{graded!Lie~algebra} is a Lie~algebra~$\glie$ together with a grading of~$\glie$.
%     \item
%       If~$V$ is a graded vector space then the tensor algebra~$\Tensor(V)$ inherits a grading from~$V$:
%       We get for every~$d \geq 0$ a decomposition
%       \[
%         V^{\tensor d}
%         =
%         \left(
%           \bigoplus_{i \geq 0} V_i
%         \right)^{\tensor d}
%         =
%         \bigoplus_{i_1, \dotsc, i_d \geq 0} V_{i_1} \tensor \dotsb \tensor V_{i_d}  \,.
%       \]
%       This overall results for the tensor algebra~$\Tensor(V)$ in a decomposition
%       \[
%         \Tensor(V)
%         =
%         \bigoplus_{r \geq 0}
%         V^{\tensor r}
%         =
%         \bigoplus_{\substack{r \geq 0 \\ i_1, \dotsc, i_r \geq 0}}
%         V_{i_1} \tensor \dotsb \tensor V_{i_r}  \,.
%       \]
%       We define for all~$d \geq 0$ the homogeneous component~$\Tensor(V)_d$ as
%       \[
%         \Tensor(V)_d
%         \defined
%         \bigoplus_{
%           \substack{r \geq 0 \\
%                     i_1, \dotsc, i_r \geq 0 \\
%                     i_1 + \dotsb + i_r = d}
%         }
%         V_{i_1} \tensor \dotsb \tensor V_{i_r}  \,.
%       \]
%       This defines a grading on~$\Tensor(V)$ which makes the inclusion~$V \inclusion \Tensor(V)$ into a homomorphism of graded vector spaces.
%     \item
%       If~$\glie$ is a graded Lie~algebra with grading~$\glie = \bigoplus_{i \neq 0} \glie_i$ then we regard the tensor algebra~$\Tensor(\glie)$ as a graded~{\algebra{$\kf$}} in the above way.
%       Let~$I$ be the two-sided ideal in~$\Tensor(\glie)$ generated by all elements~$c_{x,y} \defined x \tensor y - y \tensor x - [x,y]$ with~$x, y \in \glie$.
%       The ideal~$I$ is already generated by all~$c_{x,y}$ with~$x, y \in \glie$ homogeneous because~$c_{x,y}$ is bilinear in~$x$ and~$y$.
%       The ideal~$I$ is hence graded and so the quotient~$\Univ(\glie) = \Tensor(\glie)/I$ inherits a grading from~$\Tensor(\glie)$.
%       This is the unique grading that makes the canonical homomorphism~$\glie \to \Univ(\glie)$ a homomorphism of graded~{\algebras{$\kf$}}.
%   \end{enumerate}
% \end{remark}




