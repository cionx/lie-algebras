\section{Representations of Lie~Algebras}


% TODO: Tensor hom adjunction for internal hom.





\subsection{Definition and Examples}


\begin{definition}
  Let~$\glie$ be a~{\liealgebra{$\kf$}}.
  \begin{enumerate}
    \item
      A \defemph{representation}\index{representation}~$(V, \rho)$ of~$\glie$ is a~{\vectorspace{$\kf$}}~$V$ together with a homomorphism of Lie~algebras~$\rho \colon \glie \to \gllie(V)$.
    \item
      The \defemph{dimension}\index{representation!dimension}\index{dimension of a representation} of a representation~$(V, \rho)$ is the dimension of~$V$.
    \item
      A representation~$(V, \rho)$ is~\defemph{faithful}\index{faithful}\index{representation!faithful} if the homomorphism~$\rho$ is injective.
  \end{enumerate}
\end{definition}


\begin{remark}
  Equivalently, a representation of~$\glie$ is a~{\vectorspace{$\kf$}}~$V$ together with a~{\bilinear{$\kf$}} map~$\glie \times V \to V$,~$(x,v) \mapsto x.v$ --- an action of~$\glie$ on~$V$ --- such that
  \begin{equation}
  \label{representation via action}
    x.(y.v) - y.(x.v)
    =
    [x,y].v
  \end{equation}
  for all~$x, y \in \glie$ and~$v \in V$.
  Such an action results in a homomorphism of Lie~algebras~$\rho \colon \glie \to \gllie(V)$ by setting
  \[
    \rho(x)
    \colon
    V
    \to
    V \,,
    \quad
    v
    \mapsto
    x.v
  \]
  for all~$x \in \glie$ and~$v \in V$.
  On the other hand every Lie~algebra homomorphism~$\rho \colon \glie \to \gllie(V)$ results in an action as above by setting
  \[
    x.v
    \defined
    \rho(x)(v)
  \]
  for all~$x \in \glie$ and~$v \in V$.
  
  The two constructions are mutually inverse.
  We will in the following not distinguish between these two concepts of representations and choose whichever is more useful in the given situation.
\end{remark}


\begin{remark}
  If~$(x_i)_{i \in I}$ is a basis of a Lie~algebra~$\glie$ then for a~{\linear{$\kf$}}~$\rho \colon \glie \to \gllie(V)$ to be a homomorphism of Lie~algebras it sufficies that~$\rho([x_i,x_j])= [\rho(x_i), \rho(x_j)]$ for all~$i, j \in I$.
  It therefore sufficies to check condition~\eqref{representation via action} for basis elements, i.e.\ it sufficies to check that
  \[
    x_i.(x_j.v) - x_j.(x_i.v)
    = 
    [x_i, x_j].v
  \]
  for all~$i, j \in I$ and~$v \in V$.
\end{remark}


\begin{remark}
  Ado’s theorem is equivalent to every finite dimenisonal Lie~algebra having a faithful representation.
  \index{Ado’s theorem}
\end{remark}


\begin{examples}
  \label{examples for representations}
  \leavevmode
  \begin{enumerate}
    \item 
      If~$\glie$ is a Lie~subalgebra of some~$\gllie(V)$ then~$V$ is a representation of~$\glie$ via the inclusion~$\glie \inclusion \gllie(V)$.
      This representation corresponds to the action of~$\glie$ on~$V$ given by
      \[
        x.v
        =
        x(v)
      \]
      for all~$x \in \glie$ and~$v \in V$.
      This representation is the \defemph{natural representation}\index{natural representation}\index{representation!natural} of~$\glie$.
    \item
      If~$\glie$ is a Lie~subalgebra of some~$\gllie_n(\kf)$ then~$\glie$ acts on~$\kf^n$ via
      \[
        x.v
        =
        x \cdot v
      \]
      for all~$x \in \glie$ and~$v \in V$, which correspondings to the Lie~algebra homomorphism
      \[
        \glie
        \to
        \gllie(\kf^n) \,,
        \quad
        x
        \mapsto
        (v \mapsto x \cdot v) \,.
      \]
      This representation is the the \defemph{natural representation}\index{natural representation}\index{representation!natural} of~$\glie$.
    \item
      The Heisenberg Lie~algebra~$\glie$ from \cref{examples for lie algebras} admits a faithful representation (and thus a realization via endomorphisms) on the polynomial ring~$V = \kf[x_1, \dotsc, x_n]$:
      
      Let~$p_1, \dotsc, p_n, q_1, \dotsc, q_n, c \in \gllie(V)$ so that~$p_i = \partial_i$ is the~{\howmanyth{$i$}} partial derivate,~$q_i$ is the multiplication with~$x_i$ and~$c = \id_V$.
      These endomorphisms are linearly independent and satisfy the desired relations.
      It follows that~$\hlie = \gen{p_1, \dotsc, p_n, q_1, \dotsc, q_n, c}_{\kf}$ is a Lie subalgebra of~$\gllie(V)$ that is isomorphic to the Heisenberg Lie~algebra.
    \item
      Let~$\glie \defined \sllie_2(\kf)$ for any field~$\kf$.
      Then the polynomial ring~$\kf[x,y]$ becomes a representation of~$\glie$ via the homomorphism of Lie~algebras~$\rho \colon \glie \to \gllie(\kf[x,y])$ given by
      \[
        \rho(e) = y \dd{x} \,,
        \qquad
        \rho(h) = y \dd{y} - x \dd{x} \,,
        \qquad
        \rho(\spacing f) = x \dd{y}  \,.
      \]
      Here we denote by~$x$ and~$y$ not only the variables of~$\kf[x,y]$ but also the multiplication with these variables, and~$(e,h,f)$ denotes the standard basis of~$\sllie_2(\kf)$.
      To see that this is a homomorphism of Lie~algebra note that
      \begin{align*}
        e.(x^n y^m)
        &=
        n x^{n-1} y^{m+1} \,, \\
        h.(x^n y^m)
        &=
        (m-n) x^n y^m \,, \\
        f.(x^n y^m)
        &=
        m x^{n+1} y^{m-1}
      \end{align*}
      for all~$n, m \geq 0$.
      It follows that
      \begin{gather*}
        \begin{aligned}
        e . \spacing f . (x^n y^m) - f . e . (x^n y^m)
        &=
        (n+1)m x^n y^m - n(m+1) x^n y^m
        \\
        &=
        (m-n) x^n y^m
        \\
        &= h.m = [e,f].(x^n y^m)
        \end{aligned}
      \shortintertext{as well as}
        \begin{aligned}
        h.e.(x^n y^m) - e.h.(x^n y^m)
        &=
        n(m-n+2) x^{n-1} y^{m+1} - n(m-n) x^{n-1} y^{m+1}
        \\
        &=
        2 x^{n-1} y^{m+1}
        \\
        &=
        2e.(x^n y^m)
        \\
        &=
        [h,e].(x^n y^m)
        \end{aligned}
      \shortintertext{and}
        \begin{aligned}
          h . \spacing f . (x^n y^m) - f . h . (x^n y^m)
          &=
          m(m-n-2) x^{n+1} y^{m-1} - m(m-n) x^{n+1} y^{m-1}
          \\
          &=
          -2 x^{n+1} y^{m-1}
          \\
          &=
          -2 \spacing f.(x^n y^{m-1})
          \\
          &=
          [h,f].(x^n y^{m-1})
        \end{aligned}
      \end{gather*}
      for all~$n, m \geq 0$.
    \item
      Let~$\glie \defined \sllie_2(\kf)$ for any field~$\kf$.
      Then the polynomial ring in one variable~$\kf[x]$ is a representation of~$\glie$ via the homomorphism of Lie~algebras~$\rho \colon \glie \to \gllie(\kf[x])$ given by
      \[
        \rho(e)
        =
        \dd{x} \,,
        \qquad
        \rho(h)
        =
        -2x\dd{x} \,,
        \qquad
        \rho(\spacing f)
        =
        -\dd{x} \,.
      \]
      The~$\glie$ acts on~$\kf[x]$ via
      \[
        e.x^n = n x^{n-1} \,,
        \qquad
        h.x^n = -2n x^n \,,
        \qquad
        f.x^n = n x^{n+1}
      \]
      for every~$n \geq 0$.
      This is indeed a representation of~$\glie$ because
      \begin{gather*}
        e . \spacing f . x^n - f.e.x^n
        = -n(n+1) x^n + n(n-1) x^n
        = -2n x^n
        = h.x^n
        = [e,f].x^n
      \shortintertext{as well as}
        h.e.x^n - e.h.x^n
        = -2n(n-1) x^{n-1} + 2n^2 x^{n-1}
        = 2n x^{n-1}
        = 2e.x^n
        = [h,e].x^n
      \shortintertext{and}
        h . \spacing f . x^n - f . h . x^n
        = 2n(n+1) x^{n+1} - 2 n^2 x^{n+1}
        = 2n x^{n+1}
        = -2 \spacing f.x^n
        = [h,f].x^n
      \end{gather*}
      for every~$n \geq 0$.
    \item
      If~$\rho \colon \glie \to \gllie(V)$ is a representation of a Lie~algebra~$\glie$ and~$\phi \colon \hlie \to \glie$ a homomorphism of Lie~algebras then via the composition~$\rho \circ \phi \colon \hlie \to \gllie(V)$ the vector space~$V$ becomes a representation of~$\hlie$.
      This homomorphism corresponds to the action given by
      \[
        x.v = \rho(x).v = \rho(\phi(x))(v)
      \]
      for all~$x \in \hlie$, and~$v \in V$.
    \item
      The map~$\ad \colon \glie \to \gllie(\glie)$,~$x \mapsto \ad(x)$ is a homomorphism of Lie~algebras and hence a representation of~$\glie$.
  \end{enumerate}
\end{examples}


\begin{definition}
  Let~$\glie$ be a Lie~algebra.
  The  Lie~algebra homomorphism
  \[
    \ad
    \colon
    \glie
    \to
    \gllie(\glie) \,,
    \quad
    x
    \mapsto
    \ad(x)
    =
    [x,-]
  \]
  is the \defemph{adjoint representation}\index{adjoint representation} of~$\glie$.
\end{definition}


\begin{remark}
  It follows together with \cref{lie algebras act adjoint by derivations} that every Lie~algebras~$\glie$ acts on itself by derivations of itself via the adjoint representation.
  The author suspects that this is where much of the structure of Lie~algebras comes from and why the Jacobi identity is of interest.
\end{remark}


\begin{remark}
  \label{right representations}
  Let~$\glie$ be a~{\liealgebra{$\kf$}}.
  Instead of left~{\representations{$\glie$}} one could also consider right~{\representations{$\glie$}}:
  A \defemph{right~{\representation{$\glie$}}}\index{representation!right}\index{right representation} of~$\glie$ is a~{\vectorspace{$\kf$}}~$V$ together with a~{\bilinear{$\kf$}} map~$V \times \glie \to V$~$(v,x) \mapsto v.x$ such that
  \begin{equation}
    \label{right representation}
    (v.x).y - (v.y).x
    =
    v.[x,y]
  \end{equation}
  for all~$v \in V$ and~$x, y \in \glie$.
  But it turns out that right representations are \enquote{the same} as left representations:
  
  We note that a bilinear action~$V \times \glie \to V$ gives~$V$ the structure of a right~{\representations{$\glie$}} if and only if the corresponding linear map~$\rho \colon \glie \to \gllie(V)$ given by~$\rho(x)(v) = v.x$ is an anti-homomorphism of Lie~algebras.
  Indeed, for any two elements~$x, y \in \glie$ the identity~\eqref{right representation} holds for every~$v \in V$ if and only if~$\rho(y)\rho(x) - \rho(x) \rho(y) = \rho([x,y])$.
  We know from~\cref{antihomomorphisms correspond to opposite homomorphisms} that anti-homomorphisms of Lie~algebras~$\glie \to \gllie(V)$ are the same as homomorphisms of Lie~algebras~$\glie^\op \to \gllie(V)$.
  Hence we find that the structure of a right~{\representation{$\glie$}} on~$V$ is the same as the structure of a left~{\representation{$\glie^{\op}$}}.
  If we denote the right action of~$\glie$ by~$\cdot$ and the corresponding left action of~$\glie^\op$ by~$*$ then
  \[
    x * v
    =
    v \cdot x
  \]
  for all~$x \in \glie^\op$ and~$v \in V$.
  
  But we know from \cref{lie algebra isomorphic to its opposite} that the Lie~algebras~$\glie$ and~$\glie^{\op}$ are actually isomorphic via the mapping~$x \mapsto -x$.
  Hence left~{\representations{$\glie^{\op}$}} are the same as left~{\representations{$\glie$}}.
  More precisely, if~$V$ is a left~{\representation{$\glie^\op$}} with action denoted by~$*$ then~$V$ becomes a left~{\representation{$\glie$}} with action denoted by~$\bullet$ via
  \[
    x \bullet v
    =
    -x * v
  \]
  for all~$x \in \glie$ and~$v \in V$.
  
  By combining both constructions we see that the structure of a right~{\representation{$\glie$}} on~$V$ corresponds uniquely to the structure of a left~{\representation{$\glie$}} on~$V$.
  If the right action is denoted by~$\cdot$ and the left action is denoted by~$\bullet$ then this correspondence is given by
  \[
    x \bullet v
    =
    -v \cdot x
  \]
  for all~$x \in \glie$ and~$v \in V$.
\end{remark}





\subsection{New Representations from Old Ones}


\begin{proposition}
  \label{new representations from old ones}
  Let~$\glie$ be a Lie~algebra over an arbitrary field~$\kf$.
  \begin{enumerate}
    \item
      If~$(V_i)_{i \in I}$ is a collection of representations of~$\glie$ then the product~$\prod_{i \in I} V_i$ is again a representation of~$\glie$ via
      \[
        x.(v_i)_{i \in I}
        =
        ( x.v_i )_{i \in I}
      \]
      or all~$x \in \glie$ and~$(v_i)_{i \in I} \in \prod_{i \in I} V_i$.
    \item
      If~$(V_i)_{i \in I}$ is a collection of representations of~$\glie$ then the direct sum~$\bigoplus_{i \in I} V_i$ is again a representation of~$\glie$ via
      \[
        x.(v_i)_{i \in I}
        =
        ( x.v_i )_{i \in I}
      \]
      or all~$x \in \glie$ and~$(v_i)_{i \in I} \in \bigoplus_{i \in I} V_i$.
    \item
      If~$V$ and~$W$ are two representations of~$\glie$ then their tensor product~$V \tensor W$ is again a representation of~$\glie$ via
      \[
        x.(v \tensor w)
        =
        (x.v) \tensor w + v \tensor (x.w)
      \]
      for all~$x \in \glie$ and every simple tensor~$v \tensor w \in V \tensor W$.
      More generally, if~$V_1, \dotsc V_n$ are representations of~$\glie$ then their tensor product~$V_1 \tensor \dotsb \tensor V_n$ is again a representation of~$\glie$ via
      \[
        x.(v_1 \tensor \dotsb \tensor v_n)
        = \sum_{i=1}^n
                  v_1
          \tensor \dotsb
          \tensor v_{i-1}
          \tensor (x.v_i)
          \tensor v_{i+1}
          \tensor \dotsb
          \tensor v_n
      \]
      for every~$x \in \glie$ and every simple tensor~$v_1 \tensor \dotsb \tensor v_n \in V_1 \tensor \dotsb \tensor V_n$.
    \item
      If~$V$ and~$W$ are two representations of~$\glie$ then~$\Hom_\kf(V,W)$ is again a representation of~$\glie$ via
      \[
        (x.\spacing f)(v)
        =
        x . \spacing f(v) - f(x.v)
      \]
      for all~$x \in \glie$,\,~$f \in \Hom_\kf(V,W)$ and~$v \in V$.
    \item
      Every~{\vectorspace{$\kf$}}~$V$ becomes a representation of~$\glie$ via the \defemph{trivial action}\index{trivial action}
      \[
        x.v
        =
        0
      \]
      for all~$x \in \glie$ and~$v \in V$.
    \item
      By letting~$\glie$ act trivially on~$\kf$ the dual~$V^* = \Hom_\kf(V, \kf)$ becomes a representation of~$\glie$ in the above way, i.e.\ via
      \[
        (x.\varphi)(v)
        =
        -\varphi(x.v)
      \]
      for all~$x \in \glie$,~$\varphi \in V^*$ and~$v \in V$.
  \end{enumerate}
\end{proposition}


\begin{proof}
  \leavevmode
  \begin{enumerate}
    \item
      Let~$x, y \in \glie$ and~$(v_i)_{i \in I} \in \prod_{i \in I} V_i$, then
      \begin{align*}
        x.y.(v_i)_{i \in I} - y.x.(v_i)_{i \in I}
        &=
        (x.y.v_i)_{i \in I} - (y.x.v_i)_{i \in I}
        \\
        &=
        (x.y.v_i - y.x.v_i)_{i \in I}
        \\
        &=
        ([x,y].v_i)_{i \in I}
        \\
        &=
        [x,y].(v_i)_{i \in I} \,.
      \end{align*}
    \item
      This follows from the same calculation as for the product~$\prod_{i \in I} V_i$.
    \item
      It sufficies by induction to consider the case~$n = 2$, i.e.\ the tensor product~$V \tensor W$.
      Then
      \begin{align*}
        x.y.(v \tensor w)
        &=
        x.((y.v) \tensor w + v \tensor (y.w))
        \\
        &=
        x.((y.v) \tensor w) + x.(v \tensor (y.w))
        \\
        &=
        (x.y.v) \tensor w + (y.v) \tensor (x.w) + (x.v) \tensor (y.w) + v \tensor (x.y.w)
      \end{align*}
      and therefore
      \begin{align*}
        x.y.(v \tensor w) - y.x.(v \tensor w)
        &=
        (x.y.v) \tensor w + v \tensor (x.y.w) - (y.x.v) \tensor w - v \tensor (y.x.w)
        \\
        &=
        (x.y.v - y.x.v) \tensor w + v \tensor (x.y.w - y.x.w)
        \\
        &=
        ([x,y].v) \tensor w + v \tensor ([x,y].w) \,.
      \end{align*}
    \item
      For all~$x,y \in \glie$,~$f \in \Hom(V,W)$ and~$v \in V$ it holds that
      \begin{align*}
        (x.y.\phi)(v) - (y.x.\phi)(v)
        &=
        -(y.\phi)(x.v) + (x.\phi(y.v)
        \\
        &=
        \phi(y.x.v) - \phi(x.y.v)
        \\
        &=
        -\phi(x.y.v - y.x.v)
        \\
        &=
        -\varphi([x,y].v)
        \\
        &=
        ([x.y].\varphi)(v)  \,.
      \end{align*}
    \item
      It holds for all~$x,y \in \glie$ and~$v \in V$ that
      \[
        x.y.v - y.x.v
        =
        0 - 0
        =
        0
        =
        [x,y].v \,.
      \]
    \item
      This is a combination of the previous two constructions.
    \qedhere
  \end{enumerate}
\end{proof}


\begin{remark}
  That the dual~$V^*$ of a representation~$V$ becomes again a representation can also be explained in a more structural way:
  The structure of a left~{\representation{$\glie$}} on~$V$ induces on the dual space~$V^*$ the structure of a right~{\representation{$\glie$}} via
  \[
    (\varphi.x)(v)
    =
    \varphi(x.v)
  \]
  for all~$x \in \glie$,~$\varphi \in V^*$ and~$v \in V$.
  But we have seen in \cref{lie algebra isomorphic to its opposite} that this right~{\representation{$\glie$}} structure is equivalent to the structure of a left~{\representation{$\glie$}} on~$V^*$.
  This left~{\representation{$\glie$}} structure on~$V^*$ is precisely as in \cref{new representations from old ones}.
\end{remark}


\begin{definition}
  Let~$V$ be a representation of a Lie~algebra~$\glie$.
  A \defemph{subrepresentation} of~$V$ is a linear subpace~$U$ of~$V$ such that~$x.u \in U$ for all~$x \in \glie$ and~$u \in U$.
\end{definition}


\begin{remark}
  If~$(V, \rho)$ is a representation of a Lie~algebra~$\glie$ then a linear subspace~$U$ of~$V$ is a subrepresentation if and only if~$U$ is~{\invariant{$\rho(x)$}} for every~$x \in \glie$, in the sense that~$\rho(x)(U) \subseteq U$.
\end{remark}


\begin{examples}
  Let~$\glie$ be a Lie~algebra.
  \begin{enumerate}
    \item
      If~$V$ is any representation of~$\glie$ then the linear subspaces~$0$ and~$V$ itself are subrepresentations.%
    \footnote{These two subrepresentations are often called the \defemph{trivial} ones.
      We will abstain from doing so, as we have already defined the notion of a trivial representation in \cref{trivial representations}.}
    \item
      If~$V$ is a representation of~$\glie$ and~$U_i$ with~$i \in I$ a collection of subrepresentations~$U_i$ of~$V$ then~$\sum_{i \in I} U_i$ is again a subrepresentation of~$V$.
    \item
      If~$(V_i)_{i \in I}$ is any collection of representations of~$\glie$ then the direct sum~$\bigoplus_{i \in I} V_i$ is a subrepresentation of the product~$\prod_{i \in I} V_i$.
    \item
      The subrepresentations of the adjoint representation~$\ad \colon \glie \to \gllie(\glie)$ are precisely the ideals in~$\glie$.
    \item
      If~$V$ is any representation of~$\glie$ then the linear subspace
      \[
        \glie V
        \defined
        \gen{
          x.v
          \suchthat
          x \in \glie,
          v \in V
        }_{\kf}
      \]
      is a subrepresentation of~$\glie$.
    \item
      If~$V$ is any representation of~$\glie$ then the linear subspace of invariants~$V^{\glie}$ is a (trivial) subrepresentation.
    \item
      Let~$f \colon V \to V$ be an endomorphism of a representation~$V$ of~$\glie$.
      Then for any scalar~$\lambda \in \kf$ both the eigenspace
      \[
        V_\lambda
        \defined
        \{
          v \in V
        \suchthat
          \spacing
          f(v)
          =
          \lambda v
        \}
      \]
      and the generalized eigenspace
      \[
        V_{(\lambda)}
        \defined
        \bigcup_{n \geq 0} \ker(\spacing f - \lambda \id_V)^n
        =
        \{
          v \in V
        \suchthat
        \text{$(\spacing f - \lambda \id_V)^n(v) = 0$ for some~$n \geq 0$}
        \}
      \]
      are subrepresentations of~$V$.
  \end{enumerate}
\end{examples}


\begin{definition}
  \label{trivial representations}
  Let~$V$ be a representation of a Lie~algebra~$\glie$.
  \begin{enumerate}
    \item
      The representation~$V$ is \defemph{trivial}\index{trivial representation} if every~$x \in \glie$ acts by multiplication with zero on~$V$.
    \item
      An element~$v \in V$ is \defemph{\invariant{$\glie$}}\index{invariants}, or simply~\defemph{invariant}, if~$x.v = 0$ for every~$x \in \glie$.
      The set of invariants is denoted by
      \[
        \gls*{invariants}
        \defined
        \{
          v \in V
        \suchthat
          \text{$x.v = 0$ for every~$x \in \glie$}
        \}  \,.
      \]
  \end{enumerate}  
\end{definition}


% TODO: Explain connection to representations of groups.


\begin{lemma}
  The space of invariants~$V^{\glie}$ is for every representation~$V$ of~$\glie$ the maximal invariant subrepresentation of~$V$.
  \qed
\end{lemma}


\begin{example}
  The invariants of the adjoint representation (of~$\glie$ on itself) are given by
  \begin{align*}
    \glie^{\glie}
    &=
    \{
      y \in \glie
    \suchthat
      \text{$x.y = 0$ for every~$x \in \glie$}
    \}
    \\
    &=
    \{
      y \in \glie
    \suchthat
      \text{$\ad(x)(y) = 0$ for every~$x \in \glie$}
    \}
    \\
    &=
    \{
      y \in \glie
    \suchthat
      \text{$[x,y] = 0$ for every~$x \in \glie$}
    \}
    \\
    &=
    \centerlie(\glie) \,.
  \end{align*}
\end{example}


\begin{example}[Quotient representations]
  \label{quotient representation}
  Let~$V$ be a representation of a Lie~algebra~$\glie$ and let~$U$ be a subrepresentation of~$V$.
  Then the quotient vector space~$V/U$ inherits from~$V$ the structure of a~{\representation{$\glie$}} via
  \[
    x.\class{v}
    =
    \class{x.v}
  \]
  for all~$x \in \glie$ and~$v \in V$.
  Indeed, we have for all~$x, y \in \glie$ and~$\class{v} \in V$ that
  \[
    [x,y].\class{v}
    =
    \class{[x,y].v}
    =
    \class{x.(y.v) - y.(x.v)}
    =
    \class{x.(y.v)} - \class{y.(x.v)}
    =
    x.(y.\class{v}) - y.(x.\class{v}) \,.
  \]
  
  Alternatively let~$\rho \colon \glie \to \gllie(V)$ be the Lie~algebra homomorphism corresponding to the action of~$\glie$ on~$V$.
  Then the linear subspace~$U$ of~$V$ is~{\invariant{$\rho(x)$}} for every~$x \in \glie$, and hence the endomorphism~$\rho(x)$ induces an endomorphism
  \[
    \induced{\rho(x)}
    \colon
    V/U
    \to
    V/U \,,
    \quad
    \class{v}
    \mapsto
    \class{\rho(x)(v)}
  \]
  for every~$x \in \glie$.
  The resulting map~$\induced{\rho} \colon \glie \to \gllie(V/U)$ given by~$\induced{\rho}(x) = \induced{\rho(x)}$  is a homomorphism of Lie~algebras because
  \[
    \induced{\rho}([x,y])
    =
    \induced{\rho([x,y])}
    =
    \induced{[\rho(x), \rho(y)]}
    =
    [\induced{\rho(x)}, \induced{\rho(y)}]
    =
    [\induced{\rho}(x), \induced{\rho}(y)]
  \]
  for all~$x, y \in \glie$.
\end{example}


\begin{definition}
  Let~$V$ be a representation of a Lie~algebra~$\glie$ and let~$U$ be a subrepresentation of~$V$.
  The representation~\gls*{quotient representation} from \cref{quotient representation} is the \defemph{quotient representation}\index{quotient!representations}\index{representation!quotient} of~$V$ by~$U$.
\end{definition}


\begin{lemma}
  Let~$\glie$ be a Lie~algebra.
  \begin{enumerate}
    \item
      If~$V$ is a representation of~$\glie$ then for every~$d \geq 0$ the exterior power~$\Exterior^d(V)$ is again a representation of~$\glie$ via
      \[
        x.(v_1 \wedge \dotsb \wedge v_d)
        =
        \sum_{i=1}^d
        v_1 \wedge \dotsb \wedge v_{i-1} \wedge (x.v_i) \wedge v_{i+1} \wedge \dotsb \wedge v_d
      \]
      for every simple wedge~$v_1 \wedge \dotsb \wedge v_d \in \Exterior^d(V)$.
    \item
      If~$V$ is a representation of~$\glie$ then for every~$d \geq 0$ the symmetric power~$\Symm^d(V)$ is again a representation of~$\glie$ via
      \[
        x.(v_1 \dotsm v_d)
        =
        \sum_{i=1}^d
        v_1 \dotsm v_{i-1} (x.v_i) v_{i+1} \dotsm v_d
      \]
      for every simple symmetric tensor~$v_1 \dotsm v_d \in \Symm^d(V)$.
  \end{enumerate}
\end{lemma}


\begin{proof}
  \leavevmode
  \begin{enumerate}
    \item
      We have~$\Exterior^d(V) = V^{\tensor d}/U$ for the linear suspace~$U$ of~$V^{\tensor d}$ given by
      \[
        U
        =
        \gen{
          v_1 \tensor \dotsb \tensor v_d
        \suchthat
          \text{$v_1, \dotsc, v_d \in V$,~$v_i = v_j$ for some~$i \neq j$}
        }_{\kf} \,.
      \]
      It sufficies to show that~$U$ is a subrepresentation of~$V^{\tensor d}$.
      To see this let~$v_1 \tensor \dotsb \tensor v_d \in V^{\tensor d}$ be a simple tensor with~$v_i = v_j$ for some~$i < j$.
      Then for every~$x \in \glie$,
      \[
        x.(v_1 \tensor \dotsb \tensor v_d)
        =
        \sum_{k=1}^d v_1
        \tensor \dotsb \tensor v_{k-1}
        \tensor (x.v_k)
        \tensor v_{k+1} \tensor \dotsb \tensor v_d  \,.
      \]
      The summands for~$k \neq i,j$ are again contained in~$U$ as the tensor factors in the~{\howmanyth{$i$}} and~{\howmanyth{$j$}} place remain unchanged.
      With~$w \defined v_i = v_j$ the two remaining summands may be rewritten as
      \begin{align*}
        {}&
          v_1 \tensor \dotsb
          \tensor (x.w)
          \tensor \dotsb
          \tensor w
          \tensor \dotsb \tensor v_d
        \\
        {}&
        + v_1 \tensor \dotsb
          \tensor w
          \tensor \dotsb
          \tensor (x.w)
          \tensor \dotsb \tensor v_d
        \\
        ={}&
          v_1 \tensor \dotsb
          \tensor (x.w + w)
          \tensor \dotsb
          \tensor (x.w + w)
          \tensor \dotsb \tensor v_d
        \\
        {}&
        - v_1 \tensor \dotsb
          \tensor (x.w)
          \tensor \dotsb
          \tensor (x.w)
          \tensor \dotsb \tensor v_d
        \\
        {}&
        - v_1 \tensor \dotsb
          \tensor w
          \tensor \dotsb
          \tensor w
          \tensor \dotsb \tensor v_d  \,.
      \end{align*}
      This term is again contained in~$U$.
      This shows altogether that~$U$ is indeed a subrepresentation of~$V^{\tensor d}$.
    \item
      Similarly to before we have to show that the linear subspace~$U$ of~$V^{\tensor d}$ given by
      \[
        U
        \defined
        \gen{
          v_1 \tensor \dotsb \tensor v_d
          - \sign(\sigma) v_{\sigma(1)} \tensor \dotsb \tensor v_{\sigma(d)}
        }_{\kf}
      \]
      is a subrepresentation of~$V^{\tensor d}$.
      We denote the permutation action of the symmetric group~$S_d$ on the tensor power~$V^{\tensor d}$ by~$\sigma.t$ for all~$\sigma \in S_nd$ and~$t \in V^{\tensor d}$.
      This action is on simple tensors given by
      \[
        \sigma.(v_1 \tensor \dotsb \tensor v_d)
        =
        v_{\sigma(1)} \tensor \dotsb \tensor v_{\sigma(d)}  \,.%
        \footnote{The observant reader will object that this formula does not give a left action but a right action.
        We agree with the reader.
        We have choosen our convention to make the following calculations more readable, and will restrict our choice of convention to this proof only.
        }
      \]
      We fix~$x \in \glie$ and denote for every~$i = 1, \dotsc, n$ by~$X_i \colon V^{\tensor d} \to V^{\tensor d}$ the linear map that is given on simple tensors by
      \[
        X_i(v_1 \tensor \dotsb \tensor v_d)
        =
        v_1 \tensor \dotsb \tensor v_{i-1}
        \tensor (x.v_i)
        \tensor v_{i+1} \tensor \dotsb \tensor v_d  \,.
      \]
      
      We observe the relation
      \[
        X_i( \sigma.t )
        =
        \sigma . X_{\sigma(i)}(t)
      \]
      for all~$\sigma \in S_d$,~$i = 1, \dotsc, n$ and~$t \in V^{\tensor d}$.
      Indeed, if~$t = v_1 \tensor \dotsb \tensor v_d$ is a simple tensor then the~{\howmanyth{$i$}} tensor factor of the right hand side is given by the~{\howmanyth{$\sigma(i)$}} tensor factor of~$X_{\sigma(i)}(t)$, which is~$x.v_{\sigma(i)}$.
      The~{\howmanyth{$i$}} tensor factor of the left hand side~$X_i(\sigma.t)$ is similarly given by~$x.v_{\sigma(i)}$.
      All other tensor factors are on both sides permuted in the same way but otherwise unchanged, and hence also coincide for both sides.
      
      It follows that for every simple tensor~$t \in V^{\tensor d}$,
      \begin{align*}
        x.( t - \sigma.t )
        &=
        x.t - x.(\sigma.t)
        \\
        &=
        \sum_{i=1}^n X_i(t) - \sum_{i=1}^n X_i(\sigma.t)
        \\
        &=
        \sum_{i=1}^n X_i(t) - \sum_{i=1}^n \sigma.X_{\sigma(i)}(t)
        \\
        &=
        \sum_{i=1}^n X_i(t) - \sum_{i=1}^n \sigma.X_i(t)
        \\
        &=
        \sum_{i=1}^n \biggl( X_i(t) - \sigma.X_i(t) \biggr) \,,
      \end{align*}
      The summands~$X_i(t) - \sigma.X_i(t)$ are again contained in~$U$, so we have shown that~$U$ is indeed a subrepresentation of~$V^{\tensor d}$.
    \qedhere
  \end{enumerate}
\end{proof}






\subsection{Homomorphisms of Representations}


\begin{definition}
  Let~$V$ und~$W$ be two representations of a~{\liealgebra{$\kf$}}~$\glie$.
  \begin{enumerate}
    \item
      A~{\linear{$\kf$}} map~$f \colon V \to W$ is a \defemph{homomorphism of representations}\index{homomorphism!of representations} if
      \[
        f(x.v) = x.\spacing f(v)
      \]
      for all~$x \in \glie$ and~$v \in V$.
    \item
      A homomorphism of representations~$f$ is an \defemph{isomorphism of representations}\index{isomorphism!of representations} if it is bijective.
    \item
      The space of homomorphism~$V \to W$ is denoted by~\gls*{rep homo}, and for~$V = W$ by~\gls*{rep endo}.
  \end{enumerate}
\end{definition}

\begin{remark}
  Let~$V$,~$W$ and~$U$ be representations of a Lie~algebra~$\glie$ over~$\kf$.
  \begin{enumerate}
    \item
      The notions of a \defemph{monomorphism}\index{monomorphism of representations}, \defemph{epimorphism}\index{epimorphism of representations}, \defemph{endomorphism}\index{endomorphism of representations} and \defemph{automorphism}\index{automorphism of representations} are defined in the usual way.
    \item
      If the representations~$V$ and~$W$ are given by the Lie~algebra homomorpisms~$\rho_V \colon \glie \to \gllie(V)$ and~$\rho_W \colon \glie \to \gllie(W)$ then a~{\linear{$\kf$}} map~$f \colon V \to W$ is a homomorphism of representations if and only if~$f \circ \rho_V(x) = \rho_W(x) \circ f$ for all~$x \in \glie$, i.e.\ if and only if the square diagram
      \[
        \begin{tikzcd}[column sep = large]
          V
          \arrow{r}[above]{\rho_V(x)}
          \arrow{d}[left]{f}
          &
          V
          \arrow{d}[right]{f}
          \\
          W
          \arrow{r}[above]{\rho_W(x)}
          &
          W
        \end{tikzcd}
      \]
      commutes for every~$x \in \glie$.
    \item
      If~$f, g \colon V \to W$ are homomorphisms of representations then~$f + g$ is again a homomorphism of representations, and~$\lambda f$ is for every~$\lambda \in \kf$ again a homomorphism of representations.
      Hence~$\Hom_{\glie}(V,W)$ is a~{\linear{$\kf$}} subspace of~$\Hom_\kf(V,W)$.
    \item
      If~$f \colon V \to W$ is an isomorphism of representations then its inverse~$f^{-1} \colon W \to V$ is again a homomorphism (and thus isomorphism) of representations.
      Indeed, we find that
      \[
        f^{-1}(x.v)
        =
        f^{-1}(x . \spacing f(\spacing f^{-1}(v)))
        =
        f^{-1}(\spacing f(x . \spacing f^{-1}(v)))
        =
        x . \spacing f^{-1}(v)
      \]
      for all~$x \in \glie$ and~$v \in V$.
    \item
      The identity~$\id_V \colon V \to V$ is always an automorphism of the representation~$V$.
    \item
      If~$f \colon V \to W$ and~$g \colon W \to U$ are homomorphism of representations then their composition~$g \circ f \colon V \to U$ is again a homomorphism of representations.
    \item
      It follows that the representations of~$\glie$ together with the homomorphisms of representations between them form a (\linear{$\kf$}) category.
      We will denote this category by~\gls*{representation category}.
  \end{enumerate}
\end{remark}


\begin{lemma}
  Let~$V$ and~$W$ be two representations of a Lie~algebra~$\glie$ let~$f \colon V \to W$ be a homomorphism of representations.
  Then the kernel of~$f$ is a subrepresentation of~$V$ while the image of~$f$ is a subrepresentation~$W$.
  \qed
\end{lemma}


\begin{remark}
  \label{homomorphisms of representations as invariants}
  Given two representations~$V$ and~$W$ of a Lie~algebra~$\glie$ a linear map~$f \colon V \to W$ is a homomorphism of representations if and only if it is invariant under the induced action of~$\glie$ on~$\Hom(V,W)$:
  Indeed, we see that
  \begin{align*}
        {}& \text{$f$ is a homomorphism}  \\
    \iff{}& \text{$f(x.v) = x.\spacing f(v)$ for all~$x \in \glie$ and~$v \in V$}  \\
    \iff{}& \text{$f(x.v) - x.\spacing f(v) = 0$ for all~$x \in \glie$ and~$v \in V$}  \\
    \iff{}& \text{$(x.\spacing f)(v) = 0$ for all~$x \in \glie$ and~$v \in V$} \\
    \iff{}& \text{$x.\spacing f = 0$ for every~$x \in \glie$}  \\
    \iff{}& \text{$f$ is invariant} \,.
  \end{align*}
  Hence~$\Hom_{\glie}(V,W) = \Hom_\kf(V,W)^{\glie}$.
\end{remark}


\begin{proposition}
  \label{list of homomorphism of representations}
  Let~$\glie$ be a Lie algebra.
 \begin{enumerate}
    \item
      For every representation~$V$ the vector space isomorphism
      \[
        \kf \tensor V
        \to
        V \,,
        \quad
        \lambda \tensor v
        \to
        \lambda v
      \]
      is already an isomorphism of representations.
    \item
      For every homomorphism of representations~$f \colon V \to W$ the dual linear map
      \[
        f^*
        \colon
        W^*
        \to
        V^* \,,
        \quad
        \phi
        \mapsto
        f \circ \phi
      \]
      is again a homomorphism of representations.
    \item
      More generally, for all representations~$U$,~$V$ and~$W$ of~$\glie$ and every homomorphism of representations~$f \colon U \to V$ the two induced linear maps
      \begin{alignat*}{2}
        f_*
        &\colon
        \Hom_{\kf}(W,U)
        \to
        \Hom_{\kf}(W,V) \,,
        &
        \quad
        g 
        &\mapsto
        f \circ g
      \shortintertext{and}
        f^*
        &\colon
        \Hom_{\kf}(V,W)
        \to
        \Hom_{\kf}(U,W) \,,
        &
        \quad
        g
        &\mapsto
        g \circ f
      \end{alignat*}
      are again homomorphisms of representations.
    \item
      For any two representations~$V$ and~$W$ of~$\glie$ the natural linear map
      \[
        \Phi_1
        \colon
        V^* \tensor W
        \to
        \Hom_k(V,W) \,,
        \quad
        \phi \tensor w
        \mapsto
        (v \mapsto \phi(v) w)
      \]
      is a homomorphism of representations.
      If at least one of the two representations~$V$ and~$W$ is finite dimensional then it is an isomorphism of representations.
    \item
      For all representations~$V_1, \dotsc, V_n$ and~$W_1, \dotsc, W_m$ of~$\glie$ the natural isomorphism of vector spaces
      \begin{align*}
        \Phi_2
        \colon
        (V_1 \tensor \dotsb \tensor V_n) \tensor (W_1 \tensor \dotsb \tensor W_m)
        &\longto
        V_1 \tensor \dotsb \tensor V_n \tensor W_1 \tensor \dotsb \tensor W_m \,,
        \\
        (v_1 \tensor \dotsb \tensor v_n) \tensor (w_1 \tensor \dotsb \tensor w_m)
        &\longmapsto
        v_1 \tensor \dotsb \tensor v_n \tensor w_1 \tensor \dotsb \tensor w_m
      \end{align*}
      is already an isomorphism of representations.
    \item
      For any two representations~$V$ and~$W$ of~$\glie$ the natural isomorphism of vector spaces
      \[
        \Phi_3
        \colon
        V \tensor W
        \to
        W \tensor V \,,
        \quad
        v \tensor w
        \mapsto
        w \tensor v
      \]
      is already an isomorphism of representations.
    \item
      For all representations~$V_1$,~$V_2$ and~$W$ of~$\glie$ the natural isomorphism of vector spaces
      \begin{align*}
        \Phi_4
        \colon
        (V_1 \tensor V_2) \tensor W
        &\longto
        (V_1 \tensor W) \oplus (V_2 \tensor W) \,,
        \\
        (v_1, v_2) \tensor w
        &\longmapsto
        (v_1 \tensor w, v_2 \tensor w)
      \end{align*}
      is already an isomorphism of representations.
    \item
      If~$V_1, \dotsc, V_n$ are representations of~$\glie$ and~$\sigma \in S_n$ is any permuation then the natural isomorphism of vector spaces
      \begin{align*}
        \Phi_5
        \colon 
        V_1 \tensor \dotsb \tensor V_n
        &\longto
        V_{\sigma(1)} \tensor \dotsb \tensor V_{\sigma(n)} \,,
        \\
        v_1 \tensor \dotsb \tensor v_n
        &\longmapsto
        v_{\sigma(1)} \tensor \dotsb \tensor v_{\sigma(n)}
      \end{align*}
      is already an isomorphism of representations.
    \item
      If~$V_1, \dotsc, V_n$ and~$W_1, \dotsc, W_n$ are representations of~$\glie$ and~$f_i \colon V_i \to W_i$ with~$i = 1, \dotsc, n$ are homomorphisms of representations then the natural linear map
      \begin{align*}
        f_1 \tensor \dotsb \tensor f_n
        \colon
        V_1 \tensor \dotsb \tensor V_n
        &\longto
        W_1 \tensor \dotsb \tensor W_n
        \\
        v_1 \tensor \dotsb \tensor v_n
        &\longmapsto
        f(v_1) \tensor \dotsb \tensor f(v_n)
      \end{align*}
      is already a homomorphism of representations.
    \qed
  \end{enumerate}
\end{proposition}


\begin{proposition}[Homomorphism theorem]
  \label{homomorphism theorem!for representations}
  Let~$V$ be a representation of a Lie~algebra~$\glie$ and let~$U$ be a subrepresentation of~$V$.
  Let~$W$ be another representation of~$\glie$.
  For every homomorphism of representations~$f \colon V \to W$ with~$U \subseteq \ker f$ there exists a unique homomorphism of representations~$\induced{f} \colon V/U \to W$ that makes the triangular diagram
  \[
    \begin{tikzcd}
      V
      \arrow{r}[above]{f}
      \arrow{d}[left]{\pi}
      &
      W
      \\
      V/U
      \arrow[dashed]{ur}[below right]{\induced{f}}
      &
      {}
    \end{tikzcd}
  \]
  commute.
  It holds that~$\ker \induced{f} = {\ker f}/I$ and~$\im \induced{f} = \im f$.
\end{proposition}


\begin{corollary}[Isomorphism theorems]
  \index{isomorphism theorems!for representations}
  Let~$V$ be a representation of a Lie~algebra~$\glie$.
  \begin{enumerate}
    \item
      If~$W$ is another representation of~$\glie$ and~$f \colon V \to W$ is any homomorphism of representations then~$f$ induces a unique well-defined isomorphism of representations
      \[
        \induced{f}
        \colon
        V/{\ker f}
        \to
        \im f \,,
        \quad
        \class{v}
        \mapsto
        f(v)  \,.
      \]
    \item
      If~$U$ and~$W$ are subrepresentations of~$V$ with~$U \subseteq W$ then~$W/U$ is a subrepresentation of~$V/U$ and the natural isomorphism of vector spaces
      \[
        (V/U) / (W/U)
        \to
        V/W \,,
        \quad
        \class{v}
        \mapsto
        \class{v}
      \]
      is already an isomorphism of representations.
    \item
      If~$U$ and~$W$ are subrepresentations of~$V$ then~$W$ is a subrepresentation of~$U+W$ and~$U \cap W$ is a subrepresentation of~$U$, and the natural isomorphism of vector spaces
      \[
        U/(U \cap W)
        \to
        (U + W)/W  \,,
        \quad
        \class{u}
        \mapsto
        \class{u}
      \]
      is already an isomorphism of representations.
  \end{enumerate}
\end{corollary}


\begin{proposition}[Correspondence theorem]
  \label{correspondence theorem!for representations}
  Let~$V$ be a represenation of a Lie~algebra~$\glie$ and let~$U \subseteq V$ be a subrepresentation.
  Let~$\pi \colon V \to V/U$ the canonical projection.
  
  If~$W$ is a subrepresentation of~$V$ that contains~$U$ then the quotient~$W/U$ is a subrepresentation of~$V/U$.
  This construction results in a {\onetoone} correspondence
  \begin{align*}
    \{ \text{subrepresentations~$W \subseteq V$ containing~$U$} \}
    &\longleftrightarrow
    \{ \text{subrepresentations of~$V/U$} \}  \,,
    \\
    W
    &\longmapsto
    W/U \,,
    \\
    \pi^{-1}(W')
    &\longmapsfrom
    W'  \,.
  \end{align*}
  If~$W$ is a subrepresentation of~$V$ containing~$U$ then it holds for the associated subrepresentation~$W/U$ of~$V/U$ that~$(V/U)/(W/U) \cong V/W$.
\end{proposition}





\subsection{Irreducible and Semisimple Representations}


\begin{definition}
  Let~$V$ be a representation of a Lie~algebra~$\glie$.
  \begin{enumerate}
    \item
      The representation~$V$ is \defemph{irreducible}\index{irreducible representation}\index{representation!irreducible} or \defemph{simple}\index{simple!representation}\index{representation!simple} if it is nonzero and admits only the subrepresentations~$0$ and~$V$ itself.
    \item
      The representation~$V$ is \defemph{indecomposable}\index{indecomposable representation}\index{representation!indecomposable} if there does not exists a decomposition~$V = U_1 \oplus U_2$ into subrepresentations~$U_1$ and~$U_2$ apart from~$V = V \oplus 0$ and~$V = 0 \oplus V$.
      Otherwise~$V$ is \defemph{decomposable}\index{decomposable representation}\index{representation!decomposable}
    \item
      The representation~$V$ is~\defemph{completely reducible}\index{completely reducible representation}\index{representation!completely reducible} or~\defemph{semisimple}\index{semisimple!representation}\index{representation!semisimple} if it has a decomposition~$V = \bigoplus_{i \in I} U_i$ into irreducible subrepresentations~$U_i$.
  \end{enumerate}
\end{definition}


\begin{remark}
  \leavevmode
  \begin{enumerate}
    \item
      Every irreducible representation indecomposable, but the converse does not hold.
    \item
      A representation is irreducible if and only if it is both indecomposable and completely reducible.
    \item
      For any representation~$V$ the following conditions are equivalent:
      \begin{equivalenceslist}
        \item
          $V$ is semisimple, i.e.~$V$ admits a decomposition~$V = \bigoplus_{i \in I} U_i$ into irreducible subrepresentations~$U_i$.
        \item
          $V$ can be written as a (not necessarily direct) sum~$V = \sum_{i \in I} U_i$ for irreducible subrepresentations~$U_i$.
        \item
          Every subrepresentation~$U$ of~$V$ admits a direct complement, i.e.\ there exists a subrepresentation~$W$ of~$V$ with~$V = U \oplus W$.
      \end{equivalenceslist}
  \end{enumerate}
\end{remark}


\begin{example}
  \leavevmode
  \begin{enumerate}
    \item
      Every {\onedimensional} representation is irreducible.
    \item
      The adjoint representation of a Lie~algebra~$\glie$ is irreducible if and only if~$\glie$ is nonzero and contains no ideals beside~$0$ and~$\glie$ itself.
      This is the case if and only if~$\glie$ is either {\onedimensional} and abelian, or simple.
  \end{enumerate}
\end{example}


\begin{remark}
  If~$A$ is a~{\algebra{$\kf$}} (associative and unital) then every simple~{\module{$A$}}~$M$ is cyclic and therefore satisfies~$\dim M \leq \dim A$.
  It follows that if~$A$ is finite dimensional then all simple~{\modules{$A$}} are also finite dimensional, with their dimension uniformly bounded by the dimension of~$A$.
  
  The same does not hold for finite dimensional Lie~algebras:
  If~$\glie$ is a finite dimensional Lie~algebra then the irreducible representations of~$\glie$ do not have to be finite dimensional again, and those irreducible representations that are finite dimensional can have arbitrarily large dimension.
  
  Indeed, we will see in \cref{highest weight irreps for sl2} that the Lie~algebra~$\sllie_2(\kf)$ admits for every~$d \geq 1$ up to isomorphism precisely one irreducible representation of dimension~$d$ (if~$\kf$ is algebraically closed) and that~$\sllie_2(\kf)$ also admits infinite dimensional irreducible representations.
  
  But note that the following statements still hold any Lie~algebra~$\glie$:
  \begin{enumerate}
    \item
      Any nonzero finite dimensional representation of~$\glie$ contains an irreducible subrepresentation.
      We can simply take a subrepresentation of minimal nonzero dimension.
    \item
      There exists a finite dimensional irreducible representation~$V$ for~$\glie$.
      We can simply take~$V = \kf$ with the trivial action.
  \end{enumerate}
\end{remark}


\begin{lemma}[Schur]
  \index{Schur’s Lemma}
  Let~$V$ and~$W$ be representations of a Lie~algebra~$\glie$ and let~$f \colon V \to W$ be a homomorphism of representations.
  \begin{enumerate}
    \item
      If~$V$ is irreducible then either~$f$ is injective or~$f = 0$, but not both.
    \item
      If~$W$ is irreducible then either~$f$ is surjective or~$f = 0$, but not both.
    \item
      If both~$V$ and~$W$ are irreducible then either~$f$ is bijective or~$f = 0$, but not both.
    \item
      If~$V$ is irreducible then the endomorphism algebra~$\End_{\glie}(V)$ is a skew field over~$\kf$.
    \item
      If the field~$\kf$ is algebraically closed and~$V$ finite dimensional and irreducible then every endomorphism~$f \in \End_{\glie}(V)$ is given by multiplication with some scalar~$\lambda \in \kf$.
      In particular~$\End_{\glie}(V) = \kf$.
  \end{enumerate}
\end{lemma}


\begin{proof}
  \leavevmode
  \begin{enumerate}
    \item
      The kernel~$\ker f$ is a subrepresentation of~$V$ and so either~$\ker f = 0$ or~$\ker f = V$, but not both.
    \item
      The image~$\im f$ is a subrepresentation of~$W$ and so either~$\im f = W$ or~$\im f = 0$, but not both.
    \item
      This is a combination of the previous two statements.
    \item
      This is a reformulation of the previous statement;
      note that~$\End_{\glie}(V) \neq 0$ because~$\id_V \neq 0$.
    \item
      The endomorphism~$f$ admits an eigenvalue~$\lambda \in \kf$ because~$\kf$ is algebraically closed.
      The endomorphism~$f - \lambda \id_V$ is non-injective and hence~$f - \lambda \id_V = 0$ as seen above.
    \qedhere
  \end{enumerate}
\end{proof}





\chapter{Extensions}


% TODO: More general notion of equivalence of extensions

% TODO: Classification of semidirect products and central extensions
% (What did I mean by this?)





\section{General Extensions}

\begin{definition}
	A \defemph{short exact sequence of Lie~algebras}\index{short exact sequence of Lie algebras} is a short exact sequence
	\begin{equation}
		\label{general extension}
		0 
		\to
		\hlie
		\xto{\varphi}
		\elie
		\xto{\psi}
		\glie
		\to
		0
	\end{equation}
	of vector spaces where~$\hlie$,~$\elie$, and~$\glie$ are Lie algebras and both~$\varphi$ and~$\psi$ are homomorphisms of Lie algebras.
	The short exact sequence~\eqref{general extension} is an~\defemph{extension}\index{extension of Lie algebras} of~$\elie$ by~$\glie$.
\end{definition}


\begin{remark}
	\leavevmode
	\begin{enumerate}
		\item
			By abuse of notation we often say that \enquote{$\elie$ is an extension of~$\glie$ by~$\hlie$} when talking about an extension
			of~$\glie$ by~$\hlie$ of the form
			\[
				0 
				\to
				\hlie
				\to
				\elie
				\to
				\glie
				\to
				0 \,.
			\]
		\item
			Let
			\[
				0
				\to
				\hlie
				\xto{\varphi}
				\elie
				\xto{\psi}
				\glie
				\to
				0
			\]
			be a short exact sequence of Lie~algebras.
			Then the homomorphism~$\varphi$ is injective and identifies the Lie~algebra~$\hlie$ with the kernel of~$\psi$, which is an ideal of~$\elie$.
		\item
			Let~$\glie$ and~$\hlie$ be two~\liealgebra{$\kf$}.
			An extension~$\elie$ of~$\glie$ by~$\hlie$ encodes the following information:
			the Lie~algebra structure of~$\hlie$, the Lie~algebra structure of~$\glie$, and some kind of interaction between~$\glie$ and~$\hlie$.
			How exactly these informations are encoded into~$\elie$ relies on the specific choice of homomorphism~$\varphi$ and~$\psi$ that are part of the extension.
	\end{enumerate}
\end{remark}


\begin{example}
	\label{generic short exact sequence}
	Let~$I$ is an ideal of a Lie~algebra~$\glie$.
	Let~$\iota$ be the inclusion map from~$I$ to~$\glie$ and let~$\pi$ be the canonical projection map from~$\glie$ to~$\glie/I$.
	Then
	\[
		0
		\to
		I
		\xto{\iota}
		\glie
		\xto{\pi}
		\glie/I
		\to
		0
	\]
	is a short exact sequence of Lie~algebras.
\end{example}


\begin{definition}
	\label{equivalence of extensions}
	Let~$\hlie$ and~$\glie$ be two Lie~algebras and let
	\[
		0 
		\to
		\hlie
		\xto{\varphi}
		\elie
		\xto{\psi}
		\glie
		\to
		0
		\qquad\text{and}\qquad
		0 
		\to
		\hlie
		\xto{\varphi'}
		\elie'
		\xto{\psi'}
		\glie
		\to
		0
	\]
	be two extensions of~$\glie$ by~$\hlie$.
	These extensions are \defemph{equivalent}\index{equivalence of extensions}\index{extension of Lie algebras!equivalence} if there exists an isomorphism of Lie~algebras~$\alpha$ from~$\elie$ to~$\elie'$ that makes the following diagram commute.
	\[
		\begin{tikzcd}[column sep = large]
			0
			\arrow{r}
			&
			\hlie
			\arrow{r}[above]{\varphi}
			\arrow[equal]{d}
			&
			\elie
			\arrow{r}[above]{\psi}
			\arrow[dashed]{d}[right]{\alpha}
			&
			\glie
			\arrow{r}
			\arrow[equal]{d}
			&
			0
			\\
			0
			\arrow{r}
			&
			\hlie
			\arrow{r}[above]{\varphi'}
			&
			\elie'
			\arrow{r}[above]{\psi'}
			&
			\glie
			\arrow{r}
			&
			0
		\end{tikzcd}
	\]
\end{definition}


\begin{proposition}
	Let~$\hlie$ and~$\glie$ be two Lie~algebras.
	Equivalence of extensions is an equivalence relation on the class of extensions of~$\glie$ by~$\hlie$.
	\qed
\end{proposition}


\begin{remark}
	In \cref{equivalence of extensions} it sufficies to require for~$\varphi$ to be a homomorphism of Lie~algebras.
	It then follows from the five lemma that~$\varphi$ is already an isomorphism of vector spaces, and thus an isomorphism of Lie~algebras.
\end{remark}


\begin{fluff}
	In the following we want to classify certain classes of extensions.
	But this needs some preparation.
	We will start by showing that every extensions is, up to equivalence, of a certain standard form.
\end{fluff}


\begin{definition}
	Let~$\glie$ and~$\hlie$ be two~\liealgebras{$\kf$}.
	An extension of~$\glie$ by~$\hlie$ is of \defemph{standard form}\index{standard extension of Lie algebras}\index{extension of Lie algebras!standard}, or simply \defemph{standard}%
	\footnote{
		This is not a standard definition.
		It is made up by the author of these notes to make things more explicit.
	}, if its underlying short exact sequence of vector spaces is given by
	\[
		0
		\to
		\hlie
		\xto{\iota}
		\glie \oplus \hlie
		\xto{\pi}
		\glie
		\to
		0 \,,
	\]
	where~$\iota$ denotes the inclusion map, given by~$c \mapsto (0,c)$, and~$\pi$ denotes the canonical projection map, given by~$(x,c) \mapsto x$.
\end{definition}


\begin{warning}
	In a standard extension of~$\glie$ by~$\hlie$ the middle term~$\glie \oplus \hlie$ carries the structure of a Lie~algebra, which makes the inclusion map~$\iota$ and projection map~$\pi$ into homomorphisms of Lie~algebras.
	But this Lie~algebra structure on the vector space~$\glie \oplus \hlie$ does not have to be given by the componentwise Lie~bracket of the direct sum!
	We will see in \cref{structure of extensions} how such Lie~bracket on~$\glie \oplus \hlie$ has to look like.
\end{warning}


\begin{proposition}
	\label{every extension is equivalent to a standard one}
	Let~$\glie$ and~$\hlie$ be two~\liealgebras{$\kf$}.
	Every extension of~$\glie$ by~$\hlie$ is equivalent to a standard extension.
\end{proposition}


\begin{proof}
	Let
	\[
		0
		\to
		\hlie
		\xto{\varphi}
		\elie
		\xto{\psi}
		\glie
		\to
		0
	\]
	be an extension of~$\glie$ by~$\hlie$.
	This extension is in particular a short exact sequence of vector spaces, and it splits on the level of vector spaces.
	There hence exists an isomorphism of vector spaces~$\alpha$ from~$\elie$ to~$\glie \oplus \hlie$ that makes the diagram
	\begin{equation}
		\label{constructing equivalence with standard extension}
		\begin{tikzcd}[column sep = large]
			0
			\arrow{r}
			&
			\hlie
			\arrow{r}[above]{\varphi}
			\arrow[equal]{d}
			&
			\elie
			\arrow{r}[above]{\psi}
			\arrow[dashed]{d}[right]{\alpha}
			&
			\glie
			\arrow{r}
			\arrow[equal]{d}
			&
			0
			\\
			0
			\arrow{r}
			&
			\hlie
			\arrow{r}[above]{\iota}
			&
			\glie \oplus \hlie
			\arrow{r}[above]{\pi}
			&
			\glie
			\arrow{r}
			&
			0
		\end{tikzcd}
	\end{equation}
	commute.
	We can pushforward the Lie~bracket of~$\elie$ to a Lie~bracket on~$\glie \oplus \hlie$ along~$\alpha$.
	This makes~$\glie \oplus \hlie$ into a Lie~algebra and the map~$\alpha$ into an isomorphism of Lie~algebras.
	It follows from the commutativity of the diagram~\eqref{constructing equivalence with standard extension} that
	\[
		\iota = \alpha \circ \varphi
		\quad\text{and}\quad
		\pi = \psi \circ \alpha^{-1}
	\]
	are again homomorphisms of Lie~algebras.
	We hence find that
	\[
		0
		\to
		\hlie
		\xto{\iota}
		\glie \oplus \hlie
		\xto{\pi}
		\glie
		\to
		0
	\]
	is a standard extension of~$\glie$ by~$\hlie$.
	This standard extension is equivalent to the original extension by the commutativity of the diagram~\eqref{constructing equivalence with standard extension}.
\end{proof}


\begin{fluff}
	\label{general approach to extensions}
	Let~$\hlie$ and~$\glie$ be Lie~algebras.
	We will now investigate how the Lie~bracket on a standard extension of~$\glie$ by~$\hlie$ has to look like.
	We will then also express under what conditions two standard extensions are equivalent.
	The following discussion proceeds in multiple steps.
	\begin{enumerate}
		\item
			We consider a standard extension
			\[
				0
				\to
				\hlie
				\xto{\iota}
				\elie
				\xto{\pi}
				\glie
				\to
				0 \,.
			\]
			We want to understand how the Lie~bracket on~$\elie$ looks like.
			It follows from the bilinearity of the Lie~bracket~$[\ph, \ph]$ on~$\elie$ that
			\begin{align*}
				{}&
				[(x, c), (y,d)]
				\\
				={}&
					[(x,0), (y,0)]
				+ [(x,0), (0,d)]
				+ [(0,c), (y,0)]
				+ [(0,c), (0,d)]
				\\
				={}&
					[(x,0), (y,0)]
				+ [(x,0), (0,d)]
				- [(y,0), (0,c)]
				+ [(0,c), (0,d)]
			\end{align*}
			for all~$(x,c), (y,d) \in \elie$.

			We observe that
			\[
				\pi( [(x,0), (y,0)] )
				=
				[ \pi( (x,0) ), \pi( (y,0) ) ]
				=
				[x, y]
			\]
			because~$\pi$ is a homorphism of Lie~algebras.
			The commutator~$[(x,0), (y,0)]$ is therefore of the form
			\[
				[(x,0), (y,0)]
				=
				( [x,y], \kappa(x,y) )
			\]
			for all~$x, y \in \glie$ for some function
			\[
				\kappa
				\colon
				\glie \times \glie
				\to
				\hlie \,.
			\]
			It follows from the bilinearity of the Lie~bracket~$[\ph, \ph]$ on~$\elie$ that this map~$\kappa$ is again bilinear.

			We note that the two commutators~$[(x,0), (0,c)]$ and~$[(y,0), (0,d)]$ are again contained in~$0 \oplus \hlie$ because this is an ideal of~$\glie \oplus \hlie$, namely the kernel of the homomorphism~$\pi$.
			It follows together with the bilinearity of the Lie~bracket~$[\ph, \ph]$ on~$\elie$ that there exists a unique linear map
			\[
				\theta'
				\colon
				\glie
				\to
				\gllie(\hlie)
			\]
			with
			\[
				[(z,0), (0,e)]
				=
				[0, \theta(z)(e)]
			\]
			for all~$z \in \glie$ and~$e \in \hlie$.
			The map~$[(z,0), \ph] = \ad_{\glie \oplus \hlie}((z,0))$ is a derivation of~$\elie$ by \cref{lie algebras act adjoint by derivations}.
			It follows that the map~$\theta'$ corestricts to a linear map
			\[
				\theta
				\colon
				\glie
				\to
				\Der(\hlie) \,.
			\]
			For this map~$\theta$ we now have
			\[
				[(x,0), (0,d)]
				=
				[0, \theta(x)(d)]
				\quad\text{and}\quad
				[(y,0), (0,e)]
				=
				[0, \theta(y)(e)] \,.
			\]

			The last remaining commutator~$[(0,c), (0,d)]$ can be computed by using that the inclusion map~$\iota$ is a homomorphism of Lie~algebras from~$\hlie$ to~$\elie$, and therefore
			\[
				[(0,c), (0,d)]
				=
				[\iota(c), \iota(d)]
				=
				\iota([c,d])
				=
				(0, [c,d]) \,.
			\]

			We find altogether that the Lie bracket on~$\elie$ can be expressed with the help of the two maps~$\kappa$ and~$\theta$ as
			\begin{align*}
				{}&
				[ (x,c), (y,d) ]
				\\
				={}&
				[(x,0), (y,0)]
				+ [(x,0), (0,d)]
				- [(y,0), (0,c)]
				+ [(0,c), (0,d)]
				\\
				={}&
					( [x,y], \kappa(x,y) )
				+ ( 0, \theta(x)(d) )
				- ( 0, \theta(y)(c) )
				+ ( 0, [c,d] )
				\\
				={}&
				(
					[x,y],
					\kappa(x,y) + \theta(x)(d) - \theta(y)(c) + [c,d]
				) \,.
			\end{align*}
			The Lie~algebra structure of the standard extension~$\elie$ is therefore uniquely described by the two maps~$\kappa$ and~$\theta$.
		\item
			In the second step of this discussion we want to better understand the above map~$\kappa$ and~$\theta$.
			To do so, let
			\[
				\kappa
				\colon
				\glie \times \glie
				\to
				\hlie
			\]
			be any bilinear map, and let
			\[
				\theta
				\colon
				\glie
				\to
				\Der(\hlie)
			\]
			be any linear map.
			We define a bracket~$[\ph, \ph]$ on the vector space~$\glie \oplus \hlie$ via the previously deduced formula, i.e. via
			\begin{equation}
				\label{definition of lie bracket depending on kappa and theta}
				[(x,c), (y,d)]
				\defined
				(
					[x,y] ,
					\kappa(x,y) + \theta(x)(d) - \theta(y)(c) + [c,d]
				)
			\end{equation}
			for all~$(x,c), (y,d) \in \glie \oplus \hlie$.
			This bracket~$[\ph, \ph]$ on~$\glie \oplus \hlie$ is bilinear, and both the inclusion map~$\iota$ from~$\hlie$ to~$\glie \oplus \hlie$ and the canonical projection map~$\pi$ from~$\glie \oplus \hlie$ to~$\glie$ respect this bracket.
			There is only thing missing for
			\[
				0
				\to
				\hlie
				\xto{\iota}
				\glie \oplus \hlie
				\xto{\pi}
				\glie
				\to
				0
			\]
			to be an extension of Lie~algebras, namely that the bracket~$[\ph, \ph]$ on~$\glie \oplus \hlie$ makes it into a Lie~algebra, i.e. that it is actually a Lie~bracket.
			For this we need~$[\ph, \ph]$ to be alternating and to satisfy the Jacobi identity.

			We have for every element~$(x,c)$ of~$\glie \oplus \hlie$ that
			\[
				[(x,c), (x,c)]
				=
				( [x,x], \kappa(x,x) + \theta(x)(c) - \theta(x)(c) + [c,c] )
				=
				(0, \kappa(x,x) ) \,.
			\]
			We therefore find that the bracket~$[\ph, \ph]$ on~$\glie \oplus \hlie$ is alternating if and only if the bilinear map~$\kappa$ is alternating.

			We can also express the Jacobi identity\index{Jacobi identity!for standard extensions} for the bracket~$[\ph, \ph]$ on~$\glie \oplus \hlie$ in terms of conditions on~$\kappa$ ond~$\theta$, as we will now painstakingly demonstrate.
			We first calculate the iterated commutator
			\[
				[(x,c), [(y,d), (z,e)]]
			\]
			for all~$(x,c), (y,d), (z,e) \in \glie \oplus \hlie$ via the formula~\eqref{definition of lie bracket depending on kappa and theta} as
			\begin{align}
				% first term
				{}&
				[(x,c), [(y,d), (z,e)]]
				\notag
				\\
				% second torm
				={}&
				[(x,c), ([y,z], \kappa(y,z) + \theta(y)(e) - \theta(z)(d) + [d,e])]
				\notag
				\\
				% third term
				={}&
				\bigl(
					[x, [y, z]], 
				\notag
				\\
				{}&
					\kappa( x, [y, z] )
					+ \theta(x)
						\bigl(
							\kappa(y,z) + \theta(y)(e) - \theta(z)(d) + [d,e]
						\bigr)
				\notag
				\\
				{}&
				\hphantom{\biggl(}
					- \theta([y,z])(c)
					+ \bigl[
							c,
							\kappa(y,z) + \theta(y)(e) - \theta(z)(d) + [d,e]
						\bigr]
				\bigr)
				\notag
				\\
				% fourth term
				={}&
				\smash{\bigl(}
					[x, [y, z]],
				\notag
				\\
				{}&
				\hphantom{\bigl(}
					\kappa( x, [y, z] )
					+ \theta(x)( \kappa(y,z) )
				\label{ugly row 1}
				\\
				{}&
				\hphantom{\bigl(}
					+ \theta(x)( \theta(y)(e) )
					- \theta(x)( \theta(z)(d) )
				\label{ugly row 2}
				\\
				{}&
				\hphantom{\bigl(}
					+ \theta(x)( [d,e] )
				\label{ugly row 3}
				\\
				{}&
				\hphantom{\bigl(}
					- \theta( [y,z] )(c)
				\label{ugly row 4}
				\\
				{}&
				\hphantom{\bigl(}
					+ [c, \kappa(y,z)]
				\label{ugly row 5}
				\\
				{}&
				\hphantom{\bigl(}
					+ [c, \theta(y)(e)]
					- [c, \theta(z)(d)]
				\label{ugly row 6}
				\\
				{}&
				\hphantom{\bigl(}
					+ [c, [d, c]]
				\bigr) \,.
				\label{ugly row 7}
			\end{align}
			For every expression~$F$ in the three arguments~$(x,c)$,~$(y,d)$,~$(z,e)$ of~$\glie \oplus \hlie$ we will in the following use the abbreviation
			\[
				\sum_{\cyc} F( (x,c), (y,d), (z,e) )
				\glsadd{cyclic sum}
			\]
			to denote the cyclic sum\index{cyclic sum}
			\[
				F( (x,c), (y,d), (z,e) )
				+ F( (y,d), (z,e), (x,c) )
				+ F( (z,e), (x,c), (y,d) ) \,.
			\]
			We want to examine which conditions the two maps~$\kappa$ and~$\theta$ need to satisfy to ensure that
			\[
				\sum_{\cyc}
				[(x,c), [(y,d), (z,e)]]
				=
				0
				\qquad
				\text{for all~$(x,c), (y,d), (z,e) \in \glie \oplus \hlie$.}
			\]
			We see from~\eqref{ugly row 1} that the first entry of~$\sum_{\cyc} [(x,c), [(y,d), (z,e)]]$ is given by the cyclic sum
			\[
				\sum_{\cyc} [x,[y,z]] \,.
			\]
			But this terms vanishes because the Lie~bracket of~$\glie$ satisfies the Jacobi identity.
			We therefore only have to worry about the second entry of~$\sum_{\cyc} [(x,c), [(y,d), (z,e)]]$.
			For this second entry we can make the following observations.
			\begin{itemize*}
				\item
					For the term~\eqref{ugly row 7} we have
					\[
						\sum_{\cyc} [c,[d,e]] = 0
					\]
					because the Lie~bracket on~$\hlie$ satisfies the Jacobi identity.
					We therefore do not have to worry about the term~\eqref{ugly row 7}.
				\item
					The term~\eqref{ugly row 3} can be expanded as
					\[
						\theta(x)([d,e])
						=
						[\theta(x)(d), e] + [d, \theta(x)(e)]
					\]
					because~$\theta(x)$ is a derivation on~$\hlie$.
					We have
					\[
						\sum_{\cyc} [\theta(x)(d), e]
						=
						\sum_{\cyc} [\theta(y)(e), c]
						=
						-\sum_{\cyc} [c, \theta(y)(e)]
					\]
					and similarly
					\[
						\sum_{\cyc} [d, \theta(x)(e)]
						=
						\sum_{\cyc} [c, \theta(z)(d)] \,.
					\]
					With this we find that the cyclic sums of the term~\eqref{ugly row 3} and~\eqref{ugly row 6} cancel out.
					We therefore do not have to worry about these terms from now on.
				\item
					In the special case~$c, d, e = 0$ all terms except~\eqref{ugly row 1} vanish.
					We therefore find that~$\kappa$ and~$\theta$ have to satisfy the condition.
					\[
						\sum_{\cyc}
						(
							\kappa(x, [y,z])
							+ \theta(x)( \kappa(y,z))
						)
						=
						0 \,.
					\]
					By using that~$\kappa$ is alternating and therefore anti-symmetric, we may rewrite this condition as
					\begin{align*}
						0
						&=
						\sum_{\cyc}
						(
							\kappa(x, [y,z])
							+ \theta(x)( \kappa(y,z))
						)
						\\
						&=
						\sum_{\cyc} \kappa(x, [y,z])
						+
						\sum_{\cyc} \theta(x)( \kappa(y,z))
						\\
						&=
						-\sum_{\cyc} \kappa([y,z], x)
						+
						\sum_{\cyc} \theta(x)( \kappa(y,z))
						\\
						&=
						-\sum_{\cyc} \kappa([x,y], z)
						+
						\sum_{\cyc} \theta(x)( \kappa(y,z)) \,,
					\end{align*}
					and therefore as
					\[
						\sum_{\cyc} \kappa([x,y], z)
						=
						\sum_{\cyc} \theta(x)( \kappa(y,z) ) \,.
					\]
					We have thus arrived at the condition
					\begin{equation}
						\label{nonabelian cocycle condition}
						\begin{aligned}
							{}&
							\kappa([x,y], z)
							+ \kappa([y, z], x)
							+ \kappa([z, x], y)
							\\
							={}&
							\theta(x)( \kappa(y,z) )
							+ \theta(y)( \kappa(z,x) )
							+ \theta(z)( \kappa(x,y) )
						\end{aligned}
					\end{equation}
					for all~$x, y, z \in \glie$.

					We have derived this condition on~$\kappa$ and~$\theta$ by considering the special case~$c, d, e = 0$.
					However, as long as the condition~\eqref{nonabelian cocycle condition} is satisfied,  the term~\eqref{ugly row 1} will vanish in the cyclic sum of~$[(x,c), [(y,d), (z,e)]]$ for arbitrary~$c, d, e \in \hlie$.
					So as long as condition~\eqref{nonabelian cocycle condition} is satisfied we do not have to worry about the term~\eqref{ugly row 1}.
				\item
					It remains to consider the terms~\eqref{ugly row 2},~\eqref{ugly row 4} and~\eqref{ugly row 5}.
					For those terms we calculate that
					\begin{align}
						{}&
						\sum_{\cyc}
						\Bigl(
							\theta(x)( \theta(y)(e) )
							-
							\theta(x)( \theta(z)(d) )
							-
							\theta([y,z])(c)
							+
							[c, \kappa(y,z)]
						\Bigr)
						\label{unrotated condition on kappa and theta}
						\\
						={}&
						\sum_{\cyc}
						\theta(x)( \theta(y)(e) )
						-
						\sum_{\cyc}
						\theta(x)( \theta(z)(d) )
						-
						\sum_{\cyc}
						\theta([y,z])(c)
						+
						\sum_{\cyc}
						[c, \kappa(y,z)]
						\notag
						\\
						={}&
						\sum_{\cyc}
						\theta(x)( \theta(y)(e) )
						-
						\sum_{\cyc}
						\theta(y)( \theta(x)(e) )
						-
						\sum_{\cyc}
						\theta([x,y])(e)
						+
						\sum_{\cyc}
						[e, \kappa(x,y)]
						\notag
						\\
						={}&
						\sum_{\cyc}
						\Bigl(
							\theta(x)( \theta(y)(e) )
							-
							\theta(y)( \theta(x)(e)
							-
							\theta([x,y])(e)
							+
							[e, \kappa(x,y)]
						\Bigr) \,.
						\label{rotated conditon on kappa and theta}
					\end{align}
					We note that in the cyclic sum~\eqref{rotated conditon on kappa and theta} each summand depends on only of the three variables~$c$,~$d$,~$e$, und it does so in a linear way.
					It follows that the sum~\eqref{rotated conditon on kappa and theta} vanishes for arbitrary choices of~$(x,c)$,~$(y,d)$,~$(z,e)$ if and only if each summand vanishes.
					(Indeed, we can isolate one of the three summands by considering the special case~$c, d = 0$.
					From this we see that this summand has to vanish on its own.
					But we can also isolate the other two summands by consider the special cases~$c, e = 0$ and~$d, e = 0$.
					So each summand has to vanish on its own.)

					All three summands give the same condition, namely that
					\[
						\theta(x)( \theta(y)(e) )
						-
						\theta(y)( \theta(x)(e)
						-
						\theta([x,y])(e)
						+
						[e, \kappa(x,y)]
						=
						0
					\]
					for all~$x, y \in \glie$ and~$e \in \hlie$.
					We may rewrite this condition as
					\begin{equation}
						\label{rewritten condition on kappa and theta}
						\theta([x,y])
						=
						[\theta(x), \theta(y)]
						-
						\ad_{\hlie}( \kappa(x,y) )
						\qquad
						\text{for all~$x, y \in \glie$.}
					\end{equation}
			\end{itemize*}
			We have altogether shown that the bracket~$[\ph, \ph]$ on~$\glie \oplus \hlie$ satisfies the Jacobi identity if and only if it satisfies the two conditions~\eqref{nonabelian cocycle condition} and~\eqref{rewritten condition on kappa and theta}.

			We have now shown that the bracket~$[\ph, \ph]$ on~$\glie \oplus \hlie$ is a Lie~bracket if and only if the map~$\kappa$ is alternating and the two maps~$\kappa$ and~$\theta$ satisfy the compatibility conditions~\eqref{nonabelian cocycle condition} and~\eqref{rewritten condition on kappa and theta}
	\end{enumerate}

	We have overall constructed a bijection between Lie~brackets on~$\glie \oplus \hlie$ that make the short exact sequence
	\[
		0
		\to
		\hlie
		\xto{\iota}
		\glie \oplus \hlie
		\xto{\pi}
		\glie
		\to
		0
	\]
	into a standard extension of~$\glie$ by~$\hlie$, and certains pairs~$(\kappa, \theta)$ consisting of a map~$\kappa$ from~$\glie \times \glie$ to~$\hlie$ and a map~$\theta$ from~$\glie$ to~$\Der(\hlie)$.

	We now wish to understand under what conditions two such pairs~$(\kappa_1, \theta_1)$ and~$(\kappa_2, \theta_2)$ give equivalent extensions.\index{equivalence of standard extensions}
	For this we proceed again in multiple steps.

	\begin{enumerate}[resume*]
		\item
			Let~$(\kappa_1, \theta_1)$ and~$(\kappa_2, \theta_2)$ be two pairs of maps
			\[
				\kappa_1, \kappa_2
				\colon
				\glie \times \glie
				\to
				\hlie \,,
				\quad
				\theta_1, \theta_2
				\colon
				\glie
				\to
				\Der(\hlie)
			\]
			such that the maps~$\kappa_1$ and~$\kappa_2$ are bilinear and alternating, the maps~$\theta_1$ and~$\theta_2$ are linear, and the compatibility conditions \eqref{nonabelian cocycle condition} and~\eqref{rewritten condition on kappa and theta} are satisfied for both~$\kappa_1$ and~$\theta_1$, as well as for~$\kappa_2$ and~$\theta_2$.
			Let~$[\ph, \ph]_1$ and~$[\ph, \ph]_2$ be the Lie~brackets on~$\glie \oplus \hlie$ corresponding to the pairs~$(\kappa_1, \theta_1)$ and~$(\kappa_2, \theta_2)$.
			More explicitely, we have
			\[
				[(x,c), (y,d)]_i
				=
				(
					[x,y],
					\kappa_i(x,y) + \theta_i(x)(d) - \theta_i(y)(c) + [c,d]
				)
			\]
			for all~$(x,c), (y,d) \in \glie \oplus \hlie$ and~$i = 1, 2$.

			The two resulting extensions of~$\glie$ by~$\hlie$ given by~$[\ph, \ph]_1$ and~$[\ph, \ph]_2$ are equivalent if and only if there exists an homomorphism of Lie~algebras
			\[
				\alpha
				\colon
				( \glie \oplus \hlie, [\ph, \ph]_1 )
				\to
				( \glie \oplus \hlie, [\ph, \ph]_1 )
			\]
			that makes the following diagram commute.
			\begin{equation}
				\label{diagram for equivalence of standard extensions}
				\begin{tikzcd}
					0
					\arrow{r}
					&
					\hlie
					\arrow{r}[above]{\iota}
					\arrow[equal]{d}
					&
					\glie \oplus \hlie
					\arrow{r}[above]{\pi}
					\arrow[dashed]{d}[right]{\alpha}
					&
					\glie
					\arrow{r}
					\arrow[equal]{d}
					&
					0
					\\
					0
					\arrow{r}
					&
					\hlie
					\arrow{r}[above]{\iota}
					&
					\glie \oplus \hlie
					\arrow{r}[above]{\pi}
					&
					\glie
					\arrow{r}
					&
					0
				\end{tikzcd}
			\end{equation}
			We will now explain under what conditions on the pairs~$(\kappa_1, \theta_1)$ and~$(\kappa_2, \theta_2)$ such an isomorphism~$\alpha$ exists, and how it can be constructed.
		\item
			We first want to understand how a linear map
			\[
				\alpha
				\colon
				\glie \oplus \hlie
				\to
				\glie \oplus \hlie
			\]
			has to look like to make the diagram~\eqref{diagram for equivalence of standard extensions} commute.
			The maps~$\iota$,~$\pi$ and~$\alpha$ are linear maps between direct sums, and can therefore be written in the usual matrix calculus as
			\[
				\iota
				\equiv
				\begin{pmatrix}
					0
					\\
					1
				\end{pmatrix} \,,
				\quad
				\alpha
				\equiv
				\begin{pmatrix}
					\alpha_{11} & \alpha_{12} \\
					\alpha_{21} & \alpha_{22}
				\end{pmatrix} \,,
				\quad
				\pi
				\equiv
				\begin{pmatrix}
					1 & 0
				\end{pmatrix}
			\]
			for some unique linear maps
			\[
				\alpha_{11}
				\colon
				\glie
				\to
				\glie \,,
				\quad
				\alpha_{12}
				\colon
				\glie
				\to
				\hlie \,,
				\quad
				\alpha_{21}
				\colon
				\hlie
				\to
				\glie \,,
				\quad
				\alpha_{22}
				\colon
				\hlie
				\to
				\hlie \,.
			\]
			The commutativity of the diagram~\eqref{diagram for equivalence of standard extensions} can be encoded by the two conditions
			\[
				\begin{pmatrix}
					\alpha_{11} & \alpha_{12} \\
					\alpha_{21} & \alpha_{22}
				\end{pmatrix}
				\begin{pmatrix}
					0 \\
					1
				\end{pmatrix}
				=
				\begin{pmatrix}
					0 \\
					1
				\end{pmatrix} \,,
				\quad
				\begin{pmatrix}
					1 & 0
				\end{pmatrix}
				\begin{pmatrix}
					\alpha_{11} & \alpha_{12} \\
					\alpha_{21} & \alpha_{22}
				\end{pmatrix}
				=
				\begin{pmatrix}
					1 & 0
				\end{pmatrix} \,.
			\]
			We can apply the usual rules of matrix multiplication to find that these two conditions are furthermore equivalent to the combination of the three conditions
			\[
				\alpha_{11} = 1 \,,
				\quad
				\alpha_{12} = 0 \,,
				\quad
				\alpha_{22} = 1 \,.
			\]
			We have thus found that the diagram~\eqref{diagram for equivalence of standard extensions} commutes if and only if the linear map~$\alpha$ is of the form
			\[
				\alpha
				=
				\begin{pmatrix}
					1       & 0 \\
					\varphi & 1
				\end{pmatrix}
			\]
			for some linear map
			\[
				\varphi
				\colon
				\glie
				\to
				\hlie \,.
			\]
			This means more explicitely that
			\[
				\alpha(x,c)
				=
				\alpha(x, \varphi(x) + c)
				\qquad
				\text{for all~$(x,c) \in \glie \oplus \hlie$.}
			\]
			We now understand how~$\alpha$ has to look like to make the diagram~\eqref{diagram for equivalence of standard extensions} commute.
		\item
			We will now examine what condition the linear map~$\varphi$ has to satisfy in terms of the two pairs~$(\kappa_1, \theta_1)$ and~$(\kappa_2, \theta_2)$ for~$\alpha$ to be a homomorphism of Lie~algebras.
			The map~$\alpha$ is such a homomorphism if and only if the equality
			\[
				\alpha( [ t, u ]_1 )
				=
				[ t, u ]_2
			\]
			is satisfied for any two elements~$t$ and~$u$ of~$\glie \oplus \hlie$.
			By the bilinearity of the Lie~brackets~$[\ph, \ph]_1$ and~$[\ph, \ph]_2$ and the linearity of~$\alpha$ it sufficies to consider the cases in which~$t$ is of the form~$(x,0)$ or~$(0,c)$, and similarly~$u$ is of the form~$(y,0)$ or~$(0,d)$.
			\begin{itemize}
				\item
					For~$t = (x,0)$ and~$u = (y,0)$ we find on the one hand
					\[
						\alpha( [t,u] )
						=
						\alpha( [ (x,0) , (y,0) ] )
						=
						\alpha( [x,y], \kappa_1(x,y) )
						=
						( [x,y], \varphi( [x,y] ) + \kappa_1(x,y) )
					\]
					and on the other hand
					\begin{align*}
						[ \alpha(t), \alpha(u) ]
						&=
						[ \alpha(x,0), \alpha(y,0) ]
						\\
						&=
						[ (x, \varphi(x)), (y, \varphi(y)) ]
						\\
						&=
						( [x,y], \kappa_2(x,y) + \theta_2(x)(\varphi(y)) - \theta_2(y)(\varphi(x)) + [\varphi(x), \varphi(y)] ) \,.
					\end{align*}
					We thus arrive in this case at the condition
					\[
						\kappa_1(x,y)
						=
						\kappa_2(x,y)
						+ \theta_2(x)(\varphi(y))
						- \theta_2(y)(\varphi(x))
						+ [\varphi(x), \varphi(y)]
						- \varphi( [x,y] )
					\]
					for all~$x, y \in \glie$.
				\item
					For~$t = (x,0)$ and~$u = (0,d)$ we find on the one hand
					\[
						\alpha( [v,w] )
						=
						\alpha( [ (x,0), (0,d) ] )
						=
						\alpha( 0, \theta_1(x)(d) )
						=
						(0, \theta_1(x)(d))
					\]
					and on the other hand
					\[
						[ \alpha(t), \alpha(u) ]
						=
						[ \alpha(x,0), \alpha(0,d) ]
						=
						[ (x, \varphi(x)), (0,d) ]
						=
						( 0, \theta_2(x)(d) + [\varphi(x), d] ) \,.
					\]
					We therefore arrive in this case at the condition
					\begin{equation}
						\label{first occurance of condition}
						\theta_1(x)(d)
						=
						\theta_2(x)(d)
						+ [\varphi(x), d]
					\end{equation}
					for all~$x \in \glie$ and~$d \in \hlie$.
				\item
					In the case of~$t = (0,c)$ and~$u = (y,0)$ we find on the one hand
					\[
						\alpha( [t,u] )
						=
						\alpha( [ (0,c), (y,0) ] )
						=
						\alpha( 0, -\theta_1(y)(c) )
						=
						( 0, -\theta_1(y)(c) )
					\]
					and on the other hand
					\[
						[ \alpha(t), \alpha(u) ]
						=
						[ \alpha(0,c), \alpha(y,0) ]
						=
						[ (0,c), (y, \varphi(y) ]
						=
						( 0, -\theta_2(y)(c) + [c, \varphi(y)] ) \,.
					\]
					We therefore arrive in this case at the condition
					\[
						\theta_1(y)(c)
						=
						\theta_2(y)(c) + [\varphi(y), c]
					\]
					for all~$y \in \glie$ and~$c \in \hlie$.
					But this is the same condition as in the previous case, i.e. condition~\eqref{first occurance of condition}.
				\item 
					In the case of~$t = (0,c)$ and~$u = (0,d)$ we have on the one hand
					\[
						\alpha( [t,u] )
						=
						\alpha( [ (0,c), (0,d) ] )
						=
						\alpha( 0, [c,d] )
						=
						(0, [c,d])
					\]
					and on the other hand
					\[
						[ \alpha(t), \alpha(u) ]
						=
						[ \alpha(0,c), \alpha(0,d) ]
						=
						[ (0,c), (0,d) ]
						=
						( 0, [c,d] ) \,.
					\]
					We don’t get any additional condition from this last case.
			\end{itemize}
			We find altogether that the map~$\alpha$ is a homomorphism of Lie~algebras if and only if the linear map~$\varphi$ satisfies the two conditions
			\begin{gather*} 
				\theta_1(x)
				=
				\theta_2(x)
				+ \ad_{\hlie}(\varphi(x))
			\shortintertext{and}
				\kappa_1(x,y)
				=
				\kappa_2(x,y)
				+ \theta_2(x)(\varphi(y))
				- \theta_2(y)(\varphi(x))
				+ [\varphi(x), \varphi(y)]
				- \varphi( [x,y] )
			\end{gather*}
			for all~$x, y \in \glie$.
	\end{enumerate}
\end{fluff}


\begin{fluff}
	We have now classified standard extensions in terms of certain pairs~$(\kappa, \theta)$, and have examined under what condition two such pairs~$(\kappa_1, \theta_1)$ and~$(\kappa_2, \theta_2)$ give equivalent extensions.
	This will greatly help us in the upcoming classifications of certain classes of extensions because every extension is equivalent to a standard one.

	Let us now finish our discussion of general extensions by summarizing our findings.
\end{fluff}


\begin{theorem}[Structure of extensions]
	\index{structure theorem!for extensions}
	\index{extension of Lie algebras!structure theorem}
	\label{structure of extensions}
	Let~$\glie$ and~$\hlie$ be two~\liealgebras{$\kf$}.
	\begin{enumerate}
		\item
			Every extension of~$\glie$ by~$\hlie$ is equivalent to a standard extension of~$\glie$ by~$\hlie$.
		\item
			Given a standard extension~$\elie$ of~$\glie$ by~$\hlie$, the Lie~bracket~$[\ph, \ph]$ on~$\elie$ is of the form
			\begin{equation}
				\label{formula for lie bracket on standard extension}
				[(x,c), (y,d)]
				=
				( [x,y], \kappa(x,y) + \theta(x)(d) - \theta(y)(c) + [c,d] )
			\end{equation}
			for all~$(x,c), (y,d) \in \elie$, for some bilinear map
			\begin{equation}
				\label{kappa function}
				\kappa
				\colon
				\glie \times \glie
				\to
				\hlie
			\end{equation}
			and some linear map
			\begin{equation}
				\label{theta function}
				\theta
				\colon
				\glie
				\to
				\Der(\hlie) \,.
			\end{equation}
			The maps~$\kappa$ and~$\theta$ are uniquely determined by the formulas
			\begin{alignat*}{2}
				[(x,0), (y,0)]
				&=
				([x,y], \kappa(x,y))
				&
				\qquad
				&\text{for all~$x, y \in \glie$}
			\shortintertext{and}
				[(x,0), (0,c)]
				&=
				(0, \theta(x)(c))
				&
				\qquad
				&\text{for all~$x \in \glie$ and~$c \in \hlie$.}
			\end{alignat*}
			The map~$\kappa$ is alternating, and the maps~$\kappa$ and~$\theta$ satisfy the two compatibility conditions
			\begin{gather}
				\begin{aligned}
					{}&
					\kappa([x,y], z)
					+ \kappa([y,z], x)
					+ \kappa([z,x], y)
					\label{first compatibility condition for extensions}
					\\
					={}&
					\theta(x)( \kappa(y,z) )
					+ \theta(y)( \kappa(z,x) )
					+ \theta(z)( \kappa(x,y) )
				\end{aligned}
			\shortintertext{and}
				\theta([x,y])
				=
				[\theta(x), \theta(y)] - \ad_{\hlie}(\kappa(x,y))
				\label{second compatibility condition for extensions}
			\end{gather}
			for all~$x, y, z \in \glie$.
		\item
			Let on the other hand~$\kappa$ be a bilinear map as in \eqref{kappa function} and let~$\theta$ be a linear map as in~\eqref{theta function}, such that~$\kappa$ is alternating and the two compatibility conditions~\eqref{first compatibility condition for extensions} and~\eqref{second compatibility condition for extensions} are satisfied.
			Then the formula~\eqref{formula for lie bracket on standard extension} defines a Lie~bracket on the vector space~$\glie \oplus \hlie$ which makes it into a standard extension of~$\glie$ by~$\hlie$.
		\item
			The above two constructions are mutually inverse and result in a {\onetoonetext} correspondence
			\[
				\left\{
					\begin{tabular}{c}
						Lie~brackets on~$\glie \oplus \hlie$ \\
						that make it into a \\
						standard extension \\
						of~$\glie$ by~$\hlie$
					\end{tabular}
				\right\}
				\onetoone
				\left\{
					\begin{tabular}{c}
						$(\kappa, \theta)$
					\end{tabular}
				\suchthat*
					\begin{tabular}{c}
						$\kappa \colon \glie \times \glie \to \hlie$ is bilinear, \\
						$\theta \colon \glie \to \Der(\hlie)$ is linear, \\
						$\kappa$ is alternating, and the \\
						conditions~\eqref{first compatibility condition for extensions} and~\eqref{second compatibility condition for extensions} hold
					\end{tabular}
				\right\} \,.
			\]
	\end{enumerate}
	Let now~$(\kappa_1, \theta_1)$ and~$(\kappa_2, \theta_2)$ be two pairs as described above, and let~$\elie_1$ and~$\elie_2$ be the standard extensions corresponding to these pairs.
	\begin{enumerate}[resume*]
		\item
			Let~$\alpha$ be an equivalence of extensions from~$\elie_1$ to~$\elie_2$, i.e. a homomorphism of Lie~algebras
			\[
				\alpha
				\colon
				(\elie_1, [\ph, \ph]_1)
				\to
				(\elie_2, [\ph, \ph]_2)
			\]
			that makes the following diagram commute:
			\[
				\begin{tikzcd}
					0
					\arrow{r}
					&
					\hlie
					\arrow{r}[above]{\iota}
					\arrow[equal]{d}
					&
					\elie_1
					\arrow{r}[above]{\pi}
					\arrow[dashed]{d}[right]{\alpha}
					&
					\glie
					\arrow{r}
					\arrow[equal]{d}
					&
					0
					\\
					0
					\arrow{r}
					&
					\hlie
					\arrow{r}[above]{\iota}
					&
					\elie_2
					\arrow{r}[above]{\pi}
					&
					\glie
					\arrow{r}
					&
					0
				\end{tikzcd}
			\]
			Then the map~$\alpha$ is of the form
			\begin{equation}
				\label{formula for equivalence depending on varphi}
				\alpha(x,c)
				=
				\alpha(x, \varphi(x) + c)
				\qquad
				\text{for every~$(x,c) \in \elie_1$}
			\end{equation}
			for some linear map
			\begin{equation}
				\label{functional definition of varphi}
				\varphi
				\colon
				\glie
				\to
				\hlie \,.
			\end{equation}
			The map~$\varphi$ is unique, and it satisfies the two conditions
			\begin{gather} 
				\label{first condition for equivalence}
				\theta_1(x)
				=
				\theta_2(x)
				+ \ad_{\hlie}(\varphi(x))
			\shortintertext{and}
				\label{second condition for equivalence}
				\kappa_1(x,y)
				=
				\kappa_2(x,y)
				+ \theta_2(x)(\varphi(y))
				- \theta_2(y)(\varphi(x))
				+ [\varphi(x), \varphi(y)]
				- \varphi( [x,y] )
			\end{gather}
			for all~$x, y \in \glie$.
		\item
			Suppose on the other hand that~$\varphi$ is a linear map as in~\eqref{functional definition of varphi} which satisfies the two conditions~\eqref{first condition for equivalence} and~\eqref{second condition for equivalence}.
			Then formula~\eqref{formula for equivalence depending on varphi} defines an equivalence of extensions~$\alpha$ from~$\elie_1$ to~$\elie_2$.
		\item
			The above two constructions are mutually inverse and result in a {\onetoonetext} correspondence
			\[
				\left\{
					\begin{tabular}{c}
						$\alpha$
					\end{tabular}
				\suchthat*
					\begin{tabular}{c}
						$\alpha \colon \elie_1 \to \elie_2$ is \\
						an equivalence \\
						of extensions
					\end{tabular}
				\right\}
				\onetoone
				\left\{
					\begin{tabular}{c}
						$\varphi$
					\end{tabular}
				\suchthat*
					\begin{tabular}{c}
						$\varphi \colon \glie \to \hlie$ is linear \\
						and satisfies the two \\
						conditions~\eqref{first condition for equivalence} and~\eqref{second condition for equivalence}
					\end{tabular}
				\right\}
			\]
			It holds in particular that the extensions~$\elie_1$ and~$\elie_2$ are equivalent if and only if there exists a linear map~$\varphi$ from~$\glie$ to~$\hlie$ satisfying conditions~\eqref{first condition for equivalence} and~\eqref{second condition for equivalence}.
	\end{enumerate}
\end{theorem}


\begin{remark}
	The author of these notes learned about the general form of the compatibility conditions~\eqref{first condition for equivalence} and~\eqref{second condition for equivalence} from~\cite{nlab_extension_of_lie_algebras} and~\cite{nlab_nonabelian_lie_algebra_cohomology}, with further help from~\cite{alekseevsky_michor_ruppert_extension_of_lie_algebras}.
	According to these sources, these two compatiblity conditions express that the pair~$(\kappa, \theta)$ is a~\cocycle{$2$}\index{2-cocycle}\index{coycle@2-cocycle} in nonabelian Lie~algebra cohomology\index{nonabelian Lie algebra cohomology}\index{Lie algebra cohomology!nonabelian}.
\end{remark}


\begin{warning}
	Suppose that we are given any extension of Lie~algebras
	\[
		0 \to \hlie \to \elie \to \glie \to 0 \,.
	\]
	Then according to \cref{structure of extensions} this extension is equivalent to a standard extension~$\elie'$.
	This equivalence is realized by a suitable isomorphism of vector spaces~$\alpha$ from~$\elie$ to~$\elie'$.

	The Lie~bracket of~$\elie'$ can now be parametrized by a pair~$(\kappa, \theta)$ as explained in \cref{structure of extensions}.
	The Lie~bracket of~$\elie$ can therefore be expressed by the pair~$(\kappa, \theta)$.
	However, this pair~$(\kappa, \theta)$ does not only depend on the Lie~bracket of~$\elie$, but also on the choice of isomorphism~$\alpha$.
\end{warning}


\begin{definition}
	In the bijection from \cref{structure of extensions} the standard extension corresponding to a pair~$(\kappa, \theta)$ is \defemph{the standard extension given by~$(\kappa, \theta)$}\index{standard extension of Lie algebras}, or simple \defemph{the standard extension~$(\kappa, \theta)$}.
\end{definition}





\section{Special Kinds of Extensions}


\begin{fluff}
	We will now classify certain kinds of extensions up to equivalence.
	For this we will show that the Lie~algebras in question are equivalent to standard extensions given by~$(\kappa, \theta)$ such that~$\kappa$ and~$\theta$ have to satisfy certain additional conditions.
	We then express under what conditions two such pairs~$(\kappa_1, \theta_1)$ and~$(\kappa_2, \theta_2)$ give equivalent conditions.
\end{fluff}



\subsection{Trivial Extensions}


\begin{example}
	\label{construction of trivial extension}
	Let~$\glie$ and~$\hlie$ be two~\liealgebras{$\kf$}.
	We consider the vector space~$\glie \oplus \hlie$ together with the componentwise Lie~bracket
	\[
		[ (x,c), (y,d) ]
		=
		( [x,y], [c,d] )
	\]
	for all~$(x,c), (y,d) \in \glie \oplus \hlie$.
	This makes the vector space~$\glie \oplus \hlie$ into a standard extension of~$\glie$ by~$\hlie$.
\end{example}


\begin{definition}
	Let~$\glie$ and~$\hlie$ be two~\liealgebras{$\kf$}.
	\begin{enumerate}
		\item
			The extension of~$\glie$ by~$\hlie$ from \cref{construction of trivial extension} is the \defemph{trivial extension}\index{trivial extension of Lie algebras}\index{extension of Lie algebras!trivial} of~$\glie$ by~$\hlie$.
		\item
			An arbitrary extension of~$\glie$ by~$\hlie$ is \defemph{trivial}\index{trivial extension of Lie algebras}\index{extension of Lie algebras!trivial} if it is equivalent to the trivial extension.
	\end{enumerate}
\end{definition}


\begin{remark}
	\leavevmode
	\begin{enumerate}
		\item
			We have choosen our definitions so that \emph{the} trivial extension is one specific extension, whereas \emph{being trivial} is a property of an extension.
		\item
			Let~$\glie$ and~$\hlie$ be two~\liealgebras{$\kf$}.
			Under the correspondence from \cref{structure of extensions} the trivial extension of~$\glie$ by~$\hlie$ corresponds to the pair~$(\kappa, \theta)$ with~$\kappa = 0$ and~$\theta = 0$.
	\end{enumerate}
\end{remark}


\begin{definition}
	Let
	\[
		0
		\to
		\hlie
		\xto{\varphi}
		\elie
		\xto{\psi}
		\glie
		\to
		0
	\]
	be an extension of Lie~algebras.
	A \defemph{retract}\index{retract}\index{short exact sequence of Lie algebras!retract} of this extension is a homomorphism of Lie~algebras~$\rho$ from~$\elie$ to~$\hlie$ such that
	\[
		\rho \circ \varphi
		=
		\id_{\hlie} \,.
	\]
\end{definition}


\begin{lemma}
	\label{having a retract is invariant under equivalence of extensions}
	Let~$\glie$ and~$\hlie$ be two~\liealgebras{$\kf$} and let~$\elie$ be an extension of~$\glie$ by~$\hlie$ that admits a rectract.
	Then every extension equivalent to~$\elie$ also admits a retract.
\end{lemma}


\begin{proof}
	The extension~$\elie$ is given by a short exact sequence of Lie~algebras
	\[
		0
		\to
		\hlie
		\xto{\varphi}
		\elie
		\xto{\psi}
		\glie
		\to
		0 \,.
	\]
	Let
	\[
		0
		\to
		\hlie
		\xto{\varphi'}
		\elie'
		\xto{\psi'}
		\glie
		\to
		0
	\]
	be an equivalent extension.
	This means that there exists a homorphism of Lie~algebras~$\alpha$ from~$\elie$ to~$\elie'$ that makes the diagram
	\[
		\begin{tikzcd}
			0
			\arrow{r}
			&
			\hlie
			\arrow{r}[above]{\varphi}
			\arrow[equal]{d}
			&
			\elie
			\arrow{r}[above]{\psi}
			\arrow[dashed]{d}[right]{\alpha}
			&
			\glie
			\arrow{r}
			\arrow[equal]{d}
			&
			0
			\\
			0
			\arrow{r}
			&
			\hlie
			\arrow{r}[above]{\varphi'}
			&
			\elie'
			\arrow{r}[above]{\psi'}
			&
			\glie
			\arrow{r}
			&
			0
		\end{tikzcd}
	\]
	commute.
	If~$\rho$ is a retract for the second extension then the composite~$\rho \circ \alpha$ will be a retract for the first extension.
	Indeed, we have~$\rho \circ \varphi' = \id_{\hlie}$ and therefore
	\[
		\rho \circ \alpha \circ \varphi
		=
		\rho \circ \varphi'
		=
		\id_{\hlie}
	\]
	by the commutativity of the above diagram.
	This shows that the first extension admits a retract if the second one does.
	The reverse implication follows from this first implication because equivalence of extensions is an equivalence relation.
\end{proof}


\begin{proposition}[Characterizations of trivial extensions]
	\index{characterizationof trivial extensions of Lie algebras}
	\index{trivial extension of Lie algebras}
	\index{extension of Lie algebras!trivial}
	For an extension of Lie algebras
	\[
		0
		\to
		\hlie
		\xto{\varphi}
		\elie
		\xto{\psi}
		\glie
		\to
		0
	\]
	the following conditions are equivalent.
	\begin{equivalenceslist*}
		\item
			\label{extension is trivial}
			The extension is trivial.
		\item
			\label{extension is equivalent to 0 0}
			The extension is equivalent to the standard extension~$(\kappa, \theta)$ with~$\kappa = 0$ and~$\theta = 0$.
		\item
			\label{extension admits a retract}
			The extension admits a retract.
		\item
			\label{image of h admits a direct complement}
			There exists an ideal~$J$ of~$\elie$ such that~$\elie$ is the (internal) direct sum of the two ideals~$\varphi(\hlie)$ and~$J$.
	\end{equivalenceslist*}
%  Suppose furthermore that the underlying vector space of~$\elie$ is given by the direct sum~$\glie \oplus \hlie$.
%  Let~$(\kappa, \theta)$ be the pair correspondung to the Lie~bracket of~$\elie$ under the bijection from \cref{structure of extensions}.
%  Then the following condition is also equivalent to the ones above.
%  \begin{equivalenceslist*}[resume*]
%    \item
%      \label{trivial via kappa and theta}
%      There exists a linear map~$\varphi$ from~$\glie$ to~$\hlie$ such that
%      \[
%        \theta(x) = \ad_{\hlie}(\varphi(x)) \,,
%        \quad
%        \kappa(x,y) = [\varphi(x), \varphi(y)] - \varphi([x,y])
%      \]
%      for all~$x, y \in \glie$.
%      % TODO: Give an explicit explanation of these conditions.
%      % The first condition means that \theta takes values in \ad(\hlie), the Lie algebra of inner derivations
%      % The second condition means that \kappa is the coboundary associated to \varphi.
%      % But what about the interplay between these conditions?
%  \end{equivalenceslist*}
\end{proposition}

\begin{proof}
	We denote the trivial representation of~$\glie$ by~$\hlie$ as~$\glie \oplus \hlie$.
	\begin{implicationlist}
		\item[\ref{extension is trivial}~$\iff$~\ref{extension is equivalent to 0 0}]
			The standard extension given by~$(0,0)$ is precisely the trivial extension.
			This equivalence is therefore just a reformulation of the definition of a trivial extension.
		\item[\ref{extension is trivial}~$\implies$~\ref{extension admits a retract}]
			The trivial extension admits a retract, namly the canonical projection map from~$\glie \oplus \hlie$ to~$\hlie$, given by~$(x,c) \mapsto c$.
			It follows from \cref{having a retract is invariant under equivalence of extensions} that every trivial representation admits a retract.
		\item[\ref{extension admits a retract}~$\implies$~\ref{image of h admits a direct complement}]
			Let~$\rho \colon \elie \to \hlie$ be a retract of the given extension.
			We consider for the ideal~$J$ the kernel of~$\rho$.
			The composite~$\varphi \circ \rho$ is an idempotent endomorphism of~$\elie$, whence~$\elie$ is the direct sum of the kernel and the image of~$\varphi \circ \rho$.
			It follows from the condition~$\rho \circ \varphi = \id_{\hlie}$ that~$\rho$ is surjective and~$\varphi$ is injective, so
			\[
				\ker(\varphi \circ \rho) = \ker(\rho) = J \,,
				\qquad
				\im(\varphi \circ \rho) = \im(\varphi) = \varphi(\hlie) \,.
			\]
			We have thus shown that~$\elie$ is the direct sum of~$\varphi(\hlie)$ and~$J$.
		\item[\ref{image of h admits a direct complement}~$\implies$~\ref{extension is trivial}]
			It also follows from the injectivity of~$\varphi$ that~$\varphi$ restricts to an isomorphism of Lie~algebras from~$\hlie$ to~$\varphi(\hlie)$.
			We denote the restriction by~$\varphi'$.
			We similarly find that the homomorphism~$\psi$ restricts to an isomorphism~$\psi'$ from~$J$ to~$\glie$.
			This follows from the fact that~$\psi$ is surjective and~$J$ is a precisely a direct complement to~$\varphi(\hlie)$, and thus a direct complement to the kernel of~$\psi$.

			It follows from \cref{direct sum of ideals} that the map
			\[
				\alpha
				\colon
				J \oplus \varphi(\hlie)
				\to
				\elie \,,
				\quad
				(x, y)
				\mapsto
				x + y
			\]
			is an isomorphism of Lie~algebras.
			This isomorphism makes the diagram
			\[
				\begin{tikzcd}[sep = large]
					0
					\arrow{r}
					&
					\varphi(\hlie)
					\arrow{r}[above]{\iota}
					\arrow{d}[right]{(\varphi')^{-1}}
					&
					J \oplus \varphi(\hlie)
					\arrow{r}[above]{\pi}
					\arrow[dashed]{d}[right]{\alpha}
					&
					J
					\arrow{r}
					\arrow{d}[right]{\psi'}
					&
					0
					\\
					0
					\arrow{r}
					&
					\hlie
					\arrow{r}[above]{\varphi}
					&
					\elie
					\arrow{r}[above]{\psi}
					&
					\glie
					\arrow{r}
					&
					0
				\end{tikzcd}
			\]
			commute, where~$\iota$ denotes the inclusion map and~$\pi$ denotes the canonical projection map.

			We also find that the map
			\[
				\beta
				\colon
				\glie \oplus \hlie
				\to
				J \oplus \varphi(\hlie) \,,
				\quad
				(x,y)
				\mapsto
				\bigl( (\psi')^{-1}(x) , \varphi'(y) \bigr)
			\]
			is again an isomorphism of Lie~algebras.
			This isomorphism makes the diagram
			\[
				\begin{tikzcd}[sep = large]
					0
					\arrow{r}
					&
					\hlie
					\arrow{r}[above]{\iota}
					\arrow{d}[right]{\varphi'}
					&
					\glie \oplus \hlie
					\arrow{r}[above]{\pi}
					\arrow[dashed]{d}[right]{\beta}
					&
					\glie
					\arrow{r}
					\arrow{d}[right]{(\psi')^{-1}}
					&
					0
					\\
					0
					\arrow{r}
					&
					\varphi(\hlie)
					\arrow{r}[above]{\iota}
					&
					J \oplus \varphi(\hlie)
					\arrow{r}[above]{\pi}
					&
					J
					\arrow{r}
					&
					0
				\end{tikzcd}
			\]
			commute, where~$\iota$ denotes the respective inclusion maps, and~$\pi$ the respective canonical projection maps.

			We have altogether the following commutative diagram:
			\[
				\begin{tikzcd}[sep = large]
					0
					\arrow{r}
					&
					\hlie
					\arrow{r}[above]{\iota}
					\arrow{d}[right]{\varphi'}
					&
					\glie \oplus \hlie
					\arrow{r}[above]{\pi}
					\arrow[dashed]{d}[right]{\beta}
					&
					\glie
					\arrow{r}
					\arrow{d}[right]{(\psi')^{-1}}
					&
					0
					\\
					0
					\arrow{r}
					&
					\varphi(\hlie)
					\arrow{r}[above]{\iota}
					\arrow{d}[right]{(\varphi')^{-1}}
					&
					J \oplus \varphi(\hlie)
					\arrow{r}[above]{\pi}
					\arrow{d}[right]{\alpha}
					&
					J
					\arrow{r}
					\arrow{d}[right]{\psi'}
					&
					0
					\\
					0
					\arrow{r}
					&
					\hlie
					\arrow{r}[above]{\varphi}
					&
					\elie
					\arrow{r}[above]{\psi}
					&
					\glie
					\arrow{r}
					&
					0
				\end{tikzcd}
			\]
			By deleting the middle row of this commutative diagram we arrive at the following commutative diagram:
			\[
				\begin{tikzcd}[sep = large]
					0
					\arrow{r}
					&
					\hlie
					\arrow{r}[above]{\iota}
					\arrow[equal]{d}
					&
					\glie \oplus \hlie
					\arrow{r}[above]{\pi}
					\arrow[dashed]{d}[right]{\alpha \circ \beta}
					&
					\glie
					\arrow{r}
					\arrow[equal]{d}
					&
					0
					\\
					0
					\arrow{r}
					&
					\hlie
					\arrow{r}[above]{\varphi}
					&
					\elie
					\arrow{r}[above]{\psi}
					&
					\glie
					\arrow{r}
					&
					0
				\end{tikzcd}
			\]
			This shows that the homomorphismp~$\alpha \circ \beta$ gives an equivalence between the trivial extension~$\glie \oplus \hlie$ and the given extension.
%    \item[\ref{extension is trivial}~$\iff$~\ref{trivial via kappa and theta}]
%      Under the correspondence from \cref{structure of extensions} the trivial representation of corresponds to the pair~$(0,0)$.
%      The claim thus follows from \cref{structure of extensions}.
		\qedhere
	\end{implicationlist}
\end{proof}

\begin{example}
	We have a short exact sequence of Lie~algebras given by
	\[
		0
		\to
		\sllie(n, \kf)
		\xto{\iota}
		\gllie(n, \kf)
		\xto{\tr}
		\kf
		\to
		0
	\]
	where~$\iota$ denotes the inclusion map.
	The general linear Lie~algebra~$\gllie(n, \kf)$ is thus an extension of the {\onedimensional} Lie~algebra~$\kf$ by the special linear Lie~algebra~$\sllie(n, \kf)$\index{special linear Lie algebra}.
	If the characteristic of~$\kf$ does not divide the size~$n$ then the one-dimensional ideal~$\gen{ \Id }_{\kf}$ of~$\gllie(n, \kf)$ is a direct complement of~$\sllie(n, \kf)$.
	We find in this case that the above extension is trivial.
	% TODO: What happens if the characteristic does divide n?
	% This problem should follow from a classification of ideals in gl, which should be done after root space decompositions have been introduced.
\end{example}

\begin{fluff}
	Let~$\glie$ and~$\hlie$ be two~\liealgebras{$\kf$}.
	It may happen that a standard extension~$(\kappa, \theta)$ is trivial even if neither~~$\kappa$ nor~$\theta$ vanishes.
	Indeed, the standard extension~$(\kappa, \theta)$ is trivial if and only if it is equivalent to the standard extension~$(0,0)$.
	According to \cref{structure of extensions} this happens if and only if there exists a linear map~$\varphi$ from~$\glie$ to~$\Der(\hlie)$ such that
	\[
		\theta(x) = \ad_{\hlie}(\varphi(x)) \,,
		\qquad
		\kappa(x,y) = [\varphi(x), \varphi(y)] - \varphi([x,y])
	\]
	for all~$x, y \in \glie$.

	Indeed, we observe that for any linear map~$\theta$ from~$\hlie$ to~$\Der(\hlie)$ we can define maps~$\kappa$ from~$\glie \times \glie$ to~$\hlie$ and~$\theta$ from~$\glie$ to~$\Der(\hlie)$ via the above two formulas.
	The map~$\kappa$ is then bilinear and the map~$\theta$ is linear.
	Moreover, the maps~$\kappa$ and~$\theta$ satisfy the compatibility conditions from \cref{structure of extensions} because
	\begin{gather*}
		\SwapAboveDisplaySkip
		\begin{aligned}
			{}&
			\kappa([x,y], z) + \kappa([y,z], x) + \kappa([z,x], y)
			\\
			={}&
			[\varphi([x,y]), \varphi(z)] - \varphi([[x,y],z])
			\\
			{}&
			+ [\varphi([y,z]), \varphi(x)] - \varphi([[y,z],x])
			\\
			{}&
			+ [\varphi([z,x]), \varphi(y)] - \varphi([[z,x],y])
			\\
			={}&
			[\varphi([x,y]), \varphi(z)]
			+ [\varphi([y,z]), \varphi(x)]
			+ [\varphi([z,x]), \varphi(y)]
			\\
			{}&
			- \varphi( [[x,y],z] + [[y,z],x] + [[z,x],y] )
			\\
			={}&
			[\varphi([x,y]), \varphi(z)]
			+ [\varphi([y,z]), \varphi(x)]
			+ [\varphi([z,x]), \varphi(y)]
		\end{aligned}
	\shortintertext{and}
		\begin{aligned}
			{}&
			\theta(x)(\kappa(y,z))
			+ \theta(y)(\kappa(z,x))
			+ \theta(z)(\kappa(x,y))
			\\
			={}&
			[ \varphi(x), [\varphi(y), \varphi(z)] - \varphi([y,z]) ]
			\\
			{}&
			+ [ \varphi(y), [\varphi(z), \varphi(x)] - \varphi([z,x]) ]
			\\
			{}&
			+ [ \varphi(z), [\varphi(x), \varphi(y)] - \varphi([x,y]) ]
			\\
			={}&
			[ \varphi(x), [\varphi(y), \varphi(z)] ]
			+ [ \varphi(y), [\varphi(z), \varphi(x)] ]
			+ [ \varphi(z), [\varphi(x), \varphi(y)] ]
			\\
			{}&
			- [ \varphi(x), \varphi([y,z]) ]
			- [ \varphi(y), \varphi([z,x]) ]
			- [ \varphi(z), \varphi([x,y]) ]
			\\
			={}&
			- [ \varphi(x), \varphi([y,z]) ]
			- [ \varphi(y), \varphi([z,x]) ]
			- [ \varphi(z), \varphi([x,y]) ]
			\\
			={}&
			[ \varphi([x,y]), \varphi(z) ]
			+ [ \varphi([y,z]), \varphi(x) ]
			+ [ \varphi([z,x]), \varphi(y) ] \,,
		\end{aligned}
	\end{gather*}
	as well as
	\begin{align*}
		\SwapAboveDisplaySkip
		{}&
		[\theta(x), \theta(y)] - \ad_{\hlie}( \kappa(x,y) )
		\\
		={}&
		[ \ad_{\hlie}(\varphi(x)), \ad_{\hlie}(\varphi(y)) ] - \ad_{\hlie}( [\varphi(x), \varphi(y)] - \varphi([x,y]) )
		\\
		={}&
		\ad_{\hlie}( [\varphi(x), \varphi(y)] ) - \ad_{\hlie}( [\varphi(x), \varphi(y)] ) + \ad_{\hlie}( \varphi([x,y]) )
		\\
		={}&
		\ad_{\hlie}( \varphi([x,y]) )
		\\
		={}&
		\theta([x,y])
	\end{align*}
	for all~$x, y, z \in \glie$.
	Hence every linear map~$\varphi$ from~$\glie$ to~$\hlie$ gives rise to a trivial standard extension.
\end{fluff}



\subsection{Abelian Extensions}

\begin{definition}
	An extension of Lie~algebras
	\[
		0 \to \hlie \to \elie \to \glie \to 0
	\]
	is \defemph{abelian}\index{abelian extension of Lie algebras}\index{extension of Lie algebras!abelian} if the Lie~algebra~$\hlie$ is abelian.
\end{definition}


\begin{fluff}
	\label{discussing abelian extensions}
	Let~$\glie$ be a~\liealgebra{$\kf$}.
	We denote by~$\AbEx(\glie)$\glsadd{abelian extensions} the class of abelian extensions of~$\glie$.
	To better understand the equivalence classes of abelian extensions we will split up the class~$\mathcal{A}$ into pairwise disjoint subclasses, such that no two equivalent abelian extensions are contained in the same subclass.

	We denote for every abelian~\liealgebra{$\kf$}~$\hlie$ by~$\AbEx(\glie, \hlie)$\glsadd{abelian extensions by h} the class of abelian extensions of~$\glie$ by~$\hlie$.
	This gives a disjoint decomposition
	\[
		\AbEx(\glie)
		=
		\coprod_{\hlie}
		\AbEx(\glie, \hlie)
	\]
	where~$\hlie$ runs through the class of abelian~\liealgebras{$\kf$}.
	There exist no equivalences of extensions between an extension in~$\AbEx(\glie, \hlie)$ and an extension in~$\AbEx(\glie, \klie)$ whenever~$\hlie$ and~$\klie$ are distinct.

	Let~$\hlie$ be an abelian~\liealgebra{$\kf$}.
	We want to further decompose the class~$\AbEx(\glie, \hlie)$.
	Let for this
	\[
		0 \to \hlie \to \elie \to \glie \to 0
	\]
	be an extension of Lie~algebras.
	We know that this extension is equivalent to some standard extensions~$(\kappa, \theta)$.
	The parameter~$\theta$ is a linear map from~$\glie$ to~$\Der(\hlie)$ such that
	\[
		\theta( [x,y] )
		=
		[ \theta(x), \theta(y) ] + \ad_{\hlie}( \kappa(x,y) )
	\]
	for all~$x, y \in \elie$.
	But the Lie~algebra~$\hlie$ is abelian, so the term~$\ad_{\hlie}( \kappa(x,y) )$ vanishes.
	We thus find that~$\theta$ needs to be a homomorphism of Lie algebras from~$\glie$ to~$\Der(\hlie)$.
	Moreover, since~$\hlie$ is abelian, we have~$\Der(\hlie) = \gllie(\hlie)$.
	We thus find that~$\theta$ is a homomorphism of Lie~algebras from~$\glie$ to~$\gllie(\hlie)$.

	If the given extension is equivalent to another standard extension~$(\kappa', \theta')$, then it follows from \cref{structure of extensions} that~$\theta = \theta' + \ad_{\hlie}(\varphi(\ph))$ for some linear map~$\varphi$ from~$\glie$ to~$\hlie$.
	But we then have~$\ad_{\hlie}(\varphi(\ph)) = 0$ because~$\hlie$ is abelian, and therefore~$\theta = \theta'$.
	This shows that the homomorphism~$\theta$ is unique.

	We find from the above that we can further decompose the class~$\AbEx(\glie, \hlie)$ as
	\[
		\AbEx(\glie, \hlie)
		=
		\coprod_{\theta}
		\AbEx(\glie, \hlie, \theta)
		\glsadd{abelian extensions by h via theta}
	\]
	where~$\theta$ ranges through the set of Lie~algebra homomorphisms from~$\glie$ to~$\gllie(\hlie)$.
	We thus have the disjoint decomposition
	\[
		\AbEx(\glie)
		=
		\coprod_{\hlie, \theta}
		\AbEx(\glie, \hlie, \theta) \,.
	\]

	We can reparametrize this decomposition.
	For this we note that a pair~$(\hlie, \theta)$ consisting of an abelian Lie~algebra~$\hlie$ and a homomorphism of Lie~algebras~$\theta$ from~$\glie$ to~$\hlie$ is the same as a representation~$M$ of~$\glie$, in the following way.
	\index{representation!associated to an abelian extension}
	Given on the one hand~$\hlie$ and~$\theta$ as above, the pair~$(\hlie, \theta)$ is a representation of~$\glie$, with the action of~$\glie$ on~$\hlie$ given by~$x \act c = \theta(x)(c)$ for all~$x \in \glie$ and~$c \in \hlie$.
	Given on the other hand a representation~$(M, \rho)$ of~$\glie$ we regard~$M$ as an abelian Lie~algebra and let~$\theta$ be~$\rho$.
	We have thus a decomposition
	\[
		\AbEx(\glie)
		=
		\coprod_M \AbEx(\glie, M)
		\glsadd{abelian extensions associated to M}
	\]
	where~$M$ runs through the class of representations of~$\glie$.

	To determine in which subclass~$\AbEx(\glie, M)$ a given abelian extension
	\[
		0
		\to
		\hlie
		\xto{\varphi}
		\elie
		\xto{\psi}
		\glie
		\to
		0
	\]
	is contained we proceed as follows.
	We identify the abelian Lie~algebra~$\hlie$ with an abelian ideal of~$\elie$ via the homomorphism~$\varphi$.
	For an element~$x$ of~$\elie$ let~$y$ be a preimage of~$x$ under~$\psi$.
	We define an action of~$x$ act on~$\hlie$ via~$x \act c \defined [y, c]$.
	This action is well-defined:
	if~$y'$ is another preimage of~$x$ under~$\psi$ then the difference~$y - y'$ is contained in the kernel of~$\psi$, i.e. in~$\hlie$.
	Then
	\[
		[y, c]
		=
		[y' + y - y', c]
		=
		[y', c] + [y - y', c]
		=
		[y', c] + 0
		=
		[y', c]
	\]
	for all~$c \in \hlie$ because~$\hlie$ is abelian.
	This action of~$\glie$ on~$\hlie$ makes~$\hlie$ into a representation of~$\glie$.
	This is precisely the desired representation~$M$ for which the given extension is contained in~$\AbEx(\glie, M)$.

	Let us summarize our results.
	To every abelian extension of~$\glie$ we have associated a representation~$M$ of~$\mathfrak{h}$.
	For an abelian extension
	\[
		0
		\to
		\hlie
		\to
		\elie
		\to
		\glie
		\to
		0
	\]
	the associated representation~$M$ is given by~$\hlie$ as a vector space, and the action of an element~$x$ of~$\glie$ on~$M$ is given by taking a preimage~$y$ of~$x$ in~$\elie$ and acting with~$y$ on~$\hlie$ via the adjoint action of~$\elie$.
	If two abelian extensions of~$\glie$ are equivalent then the resulting representations of~$\glie$ are equal.
\end{fluff}


\begin{theorem}[Structure of abelian extensions]
	\index{structure theorem!for abelian extensions}
	\index{abelian extension of Lie algebras!structure theorem}
	\label{structure of abelian extensions}
	Let~$\glie$ be a Lie algebra.
	\begin{enumerate}
		\item
			Suppose that
			\[
				0
				\to
				\hlie
				\to
				\elie
				\to
				\glie
				\to
				0
			\]
			is an abelian extension of~$\glie$.
			For every element~$x$ of~$\glie$ let~$y$ be a preimage of~$x$ in~$\elie$, and define an action of~$x$ on~$\hlie$ via~$x \act c = [y,c]$ for all~$c \in \hlie$.
			This action makes~$\hlie$ into a representation of~$\glie$.
		\item
			If two abelian extensions of~$\glie$ are equivalent then their associated representations are equal.
		\item
			Let~$\AbEx(\glie)$ be the class of abelian extensions of~$\glie$.
			For every representation~$M$ of~$\glie$ let~$\AbEx(\glie, M)$ be the subclass of~$\AbEx(\glie)$ consisting of all those abelian extensions of~$\elie$ whose associated representation is~$M$.
			Then
			\[
				\AbEx(\glie)
				=
				\coprod_M \AbEx(\glie, M)
			\]
			where~$M$ ranges through the class of representations of~$\glie$.
	\end{enumerate}
	Let now~$M$ be a representation of~$\glie$.
	Let~$\theta$ be the corresponding homomorphism of Lie~algebras from~$\glie$ to~$\gllie(M)$.
	We regard~$M$ as an abelian Lie~algebra, which we denote by~$\hlie$, and regard~$\theta$ as a homomorphism of Lie~algebras from~$\glie$ to~$\Der(\hlie)$.
	\begin{enumerate}[resume*]
		\item
			The standard extensions contained in~$\AbEx(\glie, M)$ are precisely those standard extensions of~$\glie$ by~$\hlie$ of the form~$(\kappa, \theta)$ with~$\theta$ as above.
		\item
			An alternating, bilinear map~$\kappa$ from~$\glie \times \glie$ to~$\hlie$ defines a standard extension~$(\kappa, \theta)$ of~$\glie$ by~$\hlie$, which is then necessarily contained in~$\AbEx(\glie, M)$, if and only if it satisfies the condition
			\[
				\kappa( [x,y], z ) + \kappa( [y,z], x ) + \kappa( [z,x], y )
				=
				x \act \kappa(y,z) + y \act \kappa(z,x) + z \act \kappa(x,y)
			\]
			for all~$x, y, z \in \glie$.%
			\footnote{
				We want to point out that the right hand side of this equation depends on~$\theta$.
			}
		\item
			Two such maps~$\kappa_1$ and~$\kappa_2$ result in equivalent standard extensions~$(\kappa_1, \theta)$ and~$(\kappa_2, \theta)$ of~$\glie$ by~$\hlie$ if and only if there exists a linear map~$\varphi$ from~$\glie$ to~$\hlie$ such that
			\[
				\kappa_1(x,y)
				- \kappa_2(x,y)
				=
				x \act \varphi(y)
				- y \act \varphi(x)
				- \varphi( [x,y] )
			\]
			for all~$x, y \in \glie$.
	\end{enumerate}
\end{theorem}


\begin{proof}
	This follows from \cref{structure of extensions} and the discussion in \cref{discussing abelian extensions}.
\end{proof}



\subsection{Central Extensions}

\begin{definition}
	An extension of Lie~algebras
	\[
		0 \to \hlie \to \elie \to \glie \to 0
	\]
	is \defemph{central}\index{central extension of Lie algebras}\index{extension of Lie algebras!central} if the image of~$\hlie$ in~$\elie$ is contained in the center of~$\elie$.
\end{definition}


\begin{remark}
	Suppose that
	\[
		0 \to \hlie \to \elie \to \glie \to 0
	\]
	is a central extensions of Lie~algebras.
	Then~$\hlie$ is isomorphic to its image in~$\elie$, and thus abelian.
	This means that every central extension is in particular an abelian extension.
	It therefore only makes sense to talk about central extensions of a Lie~algebras~$\glie$ by a Lie~algebra~$\hlie$ if the Lie~algebra~$\hlie$ is abelian, as otherwise there are now such extensions.
\end{remark}


\begin{proposition}
	\label{central iff theta is zero}
	Let~$\glie$ and~$\hlie$ be two~\liealgebras{$\kf$} with~$\hlie$ being abelian.
	A standard extension~$(\kappa, \theta)$ of~$\glie$ by~$\hlie$ is central if and only if~$\theta$ vanishes.
\end{proposition}


\begin{proof}
	The Lie~bracket~$[\ph, \ph]$ of the standard extension~$(\kappa, \theta)$ is given by
	\[
		[ (x,c), (y,d) ]
		=
		( [x,y], \kappa(x,y) + \theta(x)(d) - \theta(y)(c) + [c,d] )
	\]
	for all~$(x,c), (y,d) \in \glie \oplus \hlie$.
	We have~$[c,d] = 0$ for all~$c, d \in \hlie$ because the Lie~algebra~$\hlie$ is abelian.
	Thus
	\[
		[ (x,c), (0,d) ]
		=
		( 0, \theta(x)(d) )
	\]
	for all~$(x,c) \in \glie \oplus \hlie$ and~$d \in \hlie$.
	It follows that the image of~$\hlie$ in~$\glie \oplus \hlie$ is central if and only if
	\[
		\theta(x)(d) = 0
		\qquad
		\text{for all~$x \in \glie$ and~$d \in \hlie$,}
	\]
	i.e. if and only if~$\theta = 0$.
\end{proof}



\begin{theorem}[Structure of central extensions]
	\index{structure theorem!for central extensions}
	\index{central extension of Lie algebras!structure theorem}
	\label{structure of central extensions}
	Let~$\glie$ and~$\hlie$ be two~\liealgebras{$\kf$} where~$\hlie$ is abelian.
	\begin{enumerate}
		\item
			Every central extension of~$\glie$ by~$\hlie$ is equivalent to a central, standard extension of~$\glie$ by~$\hlie$.
		\item
			We have a {\onetoonetext} correspondence
			\[
				\left\{
					\begin{tabular}{c}
						central standard \\
						extensions of~$\glie$ by~$\hlie$
					\end{tabular}
				\right\}
				\onetoone
				\left\{
					\begin{tabular}{c}
						bilinear, alternating maps \\
						$\kappa \colon \glie \times \glie \to \hlie$ such that \\
						$\kappa([x,y], z) + \kappa([y,z], x) + \kappa([z, x], y) = 0$
					\end{tabular}
				\right\}
			\]
			where the extension corresponding to a map~$\kappa$ is the standard extension~$(\kappa, 0)$.
		\item
			The Lie~bracket on the extension~$\elie$ corresponding to~$\kappa$ is given by
			\[
				[ (x,c), (y,d) ]
				=
				( [x,y], \kappa(x,y) )
			\]
			for all~$(x,c), (y,d) \in \elie$.
		\item
			Let
			\[
				\kappa_1, \kappa_2
				\colon
				\glie \times \glie
				\to
				\hlie
			\]
			be two maps as above, and let~$\elie_1$ and~$\elie_2$ be the correspond central extensions.
			Then we have a {\onetoonetext} correspondence given by
			\[
				\left\{
					\begin{tabular}{c}
						equivalences \\ of extensions \\
						$\alpha \colon \elie_1 \to \elie_2$
					\end{tabular}
				\right\}
				\onetoone
				\left\{
					\begin{tabular}{c}
						linear maps~$\varphi \colon \glie \to \hlie$ with \\
						$\kappa_1(x,y) = \kappa_2(x,y) - \varphi([x,y])$ \\
						for all~$x, y \in \glie$
					\end{tabular}
				\right\} \,.
			\]
			It follows in particular that the two extensions~$\elie_1$ and~$\elie_2$ are equivalent if and only if there exists a linear map~$\varphi$ from~$\glie$ to~$\hlie$ with
			\[
				\kappa_1(x,y)
				=
				\kappa_2(x,y)
				-
				\varphi([x,y])
			\]
			for all~$x, y \in \glie$.
	\end{enumerate}
\end{theorem}


\begin{proof}
	The \lcnamecref{structure of central extensions} follows from \cref{structure of extensions} thanks to \cref{central iff theta is zero}.
\end{proof}


\begin{remark}
	Let~$\glie$ be a Lie~algebra.
	Under the decomposition~$\AbEx(\glie) = \coprod_M \AbEx(\glie, M)$ from \cref{structure of abelian extensions}, the subclass of central extensions on the left hand side corresponds on the right hand side to~$\coprod_{M'} \AbEx(\glie, M')$ where~$M'$ ranges through the class of trivial representations of~$\glie$.
\end{remark}


%\begin{remark}
%  One can also think about central extenions, and our above structure theorem about central extensions, in a homological way.
%  We do so as follows:
%  \begin{enumerate}
%    \item
%      Let~$\elie$ be a Lie~algebra.
%      We can form a chain complex~$\Exterior^\bullet(\elie)$ whose~\howmanyth{$n$} term is given by the exterior power~$\Exterior^n(\elie)$ for every~$n \geq 0$, and which vanishes in negative degrees.
%      The differential of this chain complex is given by the maps
%      \begin{align*}
%        d_n
%        \colon
%        \bigwedge^n(\elie)
%        &\to
%        \bigwedge^{n-1}(\elie) \,,
%        \\
%        x_1 \wedge \dotsb \wedge x_n
%        &\mapsto
%        \sum_{1 \leq i < j \leq n}
%        (-1)^{i+j-1}
%        [x_i, x_j] \wedge x_1 \wedge \dotsb \wedge \widehat{x_i} \wedge \dotsb \wedge \widehat{x_j} \wedge \dotsb \wedge x_n
%      \end{align*}
%      for every~$n \geq 1$, and~$d_n \defined 0$ for every~$n \leq 0$.
%  \end{enumerate}
%\end{remark}

%\begin{example}[Central extensions]
%  In the notation of \cref{general approach to extensions} we find that an extension
%  \[
%    0
%    \to
%    I
%    \xto{\iota}
%    \glie \oplus \hlie
%    \xto{\pi}
%    \glie
%    \to
%    0
%  \]
%  is central if and only
%  \begin{equation}
%    \label{central extension condition on elements}
%    [(x,c), (0,d)] = 0
%    \qquad
%    \text{for all~$x \in \glie$ and~$c, d \in \hlie$.}
%  \end{equation}
%  We have
%  \[
%    [(x,c), (0,d)]
%    =
%    (0, \theta(x)(d) + [c,d])
%    \qquad
%    \text{for all~$x \in \glie$,~$c, d \in \hlie$,}
%  \]
%  so we find that the condition~\eqref{central extension condition on elements} is equivalent to the combination of the two conditions~$\theta = 0$ and also~$[c,d] = 0$ for all~$c, d \in \hlie$.
%  Indeed, by choosing~$d = c$ we see that~$\theta(x)(c) = 0$ for all~$x \in \glie$ and~$c \in \hlie$, and thus~$\theta = 0$.
%  And from this we then find that~$[c,d] = 0$ for all~$c, d \in \hlie$.
%
%  We now find from the general formula
%  \begin{equation}
%    \label{central extension formula}
%    [ (x,c), (y,d) ]
%    =
%    ( [x,y], \kappa(x,y) )
%  \end{equation}
%  for all~$(x,c), (y,d) \in \glie \oplus \hlie$.
%  
%  Given any bilinear form~$\kappa \colon \glie \times \glie \to \hlie$ we have already seen that \eqref{central extension formula} defines a bilinear bracket on~$\glie \oplus \hlie$ that is alternating if and only if~$\kappa$ is alternating.
%  We find for the Jacobi identity that
%  \[
%    [ (x,c), [ (y,d), (z,e) ] ]
%    =
%    [ (x,c), ([y,z], \kappa(y,z)) ]
%    =
%    ( [x,[y,z]], \kappa(x, [y,z]) )
%  \]
%  for all~$(x,c), (y,d), (z,e) \in \glie \oplus \hlie$ and therefore
%  \begin{align*}
%    {}&
%      [(x,c), [(y,d), (z,e)]]
%    + [(y,d), [(z,e), (x,c)]]
%    + [(z,e), [(x,c), (y,d)]]
%    \\
%    ={}&
%    (
%      [x,[y,z]] + [y,[z,x]] + [z,[x,y]],
%      \kappa(x, [y,z]) + \kappa(y, [z,x]) + \kappa(z, [x,y])
%    )
%    \\
%    ={}&
%    ( 0, \kappa(x, [y,z]) + \kappa(y, [z,x]) + \kappa(z, [x,y]) ) \,.
%  \end{align*}
%  We hence find that the bracket~$[\ph, \ph]$ on~$\glie \oplus \hlie$ satisfies the Jacobi identity if and only if the bilinear map~$\kappa$ satisfies the similar looking identity
%  \[
%    \kappa(x, [y,z]) + \kappa(y, [z,x]) + \kappa(z, [x,y])
%    =
%    0 \,.
%  \]
%  This condition is the \defemph{{\twococycle} condition}, and a bilinear map~$\kappa \colon \glie \to \glie \to \hlie$ that is both alternating is satisfies the {\twococycle} condition is a {\twococycle}.
%  
%  We have overall constructed for all Lie~algebras~$\glie$ and~$\hlie$ a {\onetoonetext} correspondence between
%  \begin{itemize}
%    \item
%      Lie~brackets~$[\ph, \ph]$ on the vector space~$\glie \oplus \hlie$ that make the standard short exact sequence
%      \[
%        0
%        \to
%        I
%        \to
%        \glie
%        \oplus
%        I
%        \to
%        \glie
%        \to
%        0
%      \]
%      into a central extension of~$\glie$ by~$\hlie$ and
%    \item
%      {\twococycles}~$\kappa \colon \glie \times \glie \to \hlie$,
%  \end{itemize}
%  and this correspondence is given by~$[(x,c), (y,d)] = ([x,y], \kappa(x,y))$.
%\end{example}



\subsection{Split Extensions and Semidirect Products}

\begin{fluff}
	Another important kind of extensions are so called semidirect products.
	We will give three characterizations of these extensions:
	internal semidirect product, external semidirect products, and split extensions.
	We will also explain the equivalence of these characterizations.
\end{fluff}

\subsubsection{Internal Semidirect Products}


\begin{definition}
	Let~$\elie$ be a Lie~algebra.
	Let~$\glie$ be a Lie~subalgebra of~$\elie$ and let~$\hlie$ be an ideal of~$\elie$.
	If~$\elie = \glie \oplus \hlie$ as vector spaces, then~$\elie$ is the \defemph{internal semidirect product} of~$\glie$ by~$\hlie$, or of~$\glie$ and~$\hlie$.
	This is then denoted by~$\elie = \glie \ltimes \hlie$\glsadd{internal semidirect product}\index{internal semidirect product}\index{semidirect product!index}.
\end{definition}


\begin{fluff}
	\label{calculating structure of internal semidirect product}
	Let~$\elie$ be a Lie~algebra and suppose that~$\elie$ is the semidirect product of a Lie~subalgebra~$\glie$ by an ideal~$\hlie$.
	The adjoint action of~$\elie$ on itself restricts to an action of~$\glie$ on~$\hlie$ because~$\hlie$ is an ideal of~$\elie$.
	The action of~$\elie$ on itself happens by derivatations, whence the action of~$\glie$ on~$\hlie$ is also by derivations.
	This action is given by a homomorphism of Lie~algebras
	\[
		\theta
		\colon
		\glie
		\to
		\Der(\hlie) \,.
	\]
	Any two elements of~$\elie$ may be written uniquely as sums~$x + c$ and~$y + d$ with~$x$,~$y$ in~$\glie$ and~$c$,~$d$ in~$\hlie$.
	The Lie~bracket of~$\elie$ can be expressed in terms of the homomorphisms~$\theta$ as
	\begin{align*}
		[x + c, y + d]_{\elie}
		&=
		[x, y]_{\elie}
		+ [x, d]_{\elie}
		+ [c, y]_{\elie}
		+ [c, d]_{\elie}
		\\
		&=
		[x, y]_{\elie}
		+ [x, d]_{\elie}
		- [y, c]_{\elie}
		+ [c, d]_{\elie}
		\\
		&=
		[x, y]_{\glie}
		+ \theta(x)(d)
		- \theta(y)(c)
		+ [c, d]_{\hlie} \,.
	\end{align*}
	The term~$[x,y]_{\glie}$ is again contained in the Lie~subalgebra~$\glie$, and the term~$\theta(x)(d) - \theta(y)(c) + [c,d]_{\hlie}$ is again contained in the ideal~$\hlie$.
\end{fluff}


\begin{example}
	The Lie~algebra~$\trianglie(n, \kf)$\index{upper triangular matrices} is the direct sum of its Lie~subalgebra of diagonal matrices~$\diaglie(n, \kf)$\index{diagonal matrices} and its ideal of strictly upper triangular matrices~$\upperlie(n, \kf)$\index{strictly upper triangular matrices}.
	The Lie~algebra~$\trianglie(n, \kf)$ is thus the internal semidirect product of~$\diaglie(n, \kf)$ by~$\upperlie(n, \kf)$.
\end{example}


\begin{example}
	\label{affine lie algebra is an internal semidirect product}
	The Lie~algebra of affine transformations~$\afflie(n, \kf)$\index{affine transformations} contains the Lie~subalgebra
	\[
		\glie
		\defined
		\begin{pmatrix}
			\gllie(n, \kf)  & 0 \\
			0               & 0
		\end{pmatrix}
	\]
	and the ideal
	\[
		\hlie
		\defined
		\begin{pmatrix}
			0 & \kf^n \\
			0 & 0
		\end{pmatrix} \,.
	\]
	We have~$\afflie(n, \kf) = \glie \oplus \hlie$ as vector spaces.
	The Lie~algebra~$\afflie(n, \kf)$ is therefore the internal semidirect product of~$\glie$ by~$\hlie$.
\end{example}


% TODO: Express aff(n,k) as an external semidirect product.


\begin{example}
	Let~$\heisenberglie$ be the Heisenberg Lie~algebra\index{Heisenberg Lie algebra} from \cref{examples for lie algebras}.
	The linear subspace~$\hlie$ of~$\heisenberglie$ spanned by the elements~$p_1, \dotsc, p_n, c$ is an ideal of~$\heisenberglie$, and the linear subspace~$\glie$ of~$\heisenberglie$ spanned by the elements~$q_1, \dotsc, q_n$ is a Lie~subalgebra of~$\elie$.
	The Lie~algebra~$\heisenberglie$ is thus the internal semidirect product of~$\glie$ by~$\hlie$.
\end{example}

\subsubsection{External Semidirect Products}

\begin{definition}
	Let~$\glie$ and~$\hlie$ be two~\liealgebras{$\kf$} and let~$\theta$ be a homomorphism of Lie~algebras from~$\glie$ to~$\Der(\hlie)$.
	The standard extension~$(\theta, 0)$ of~$\glie$ by~$\hlie$ is the \defemph{external semidirect product}\index{external semidirect product}\index{semidirect product!external} of~$\glie$ by~$\hlie$ along~$\theta$.
	It is denoted by~$\glie \ltimes_\theta \hlie$\glsadd{external semidirect product}.
\end{definition}


\begin{remark}
	An external semidirect product~$\glie \ltimes_\theta \hlie$ is given by the external direct sum~$\glie \oplus \hlie$ on the level of vector spaces.
	Its Lie~bracket is given by
	\[
		[ (x,c), (y,d) ]
		=
		( [x,y], \theta(x)(d) - \theta(y)(c) + [c,d] )
	\]
	for all~$(x,c), (y,d) \in \glie \ltimes_\theta \hlie$.
	We have in particular
	\[
		[(x,0), (0,c)]
		=
		( 0, \theta(x)(c) )
	\]
	for all~$x \in \glie$ and~$c \in \hlie$.
\end{remark}


\begin{fluff}
	\label{internal semidirect product is also external}
	Let~$\elie$ be a Lie~algebra, let~$\glie$ be a Lie~subalgebra of~$\elie$ and let~$\hlie$ be an ideal of~$\elie$.
	Suppose that~$\elie$ is the internal semidirect product of~$\glie$ by~$\hlie$.
	We have seen in~\cref{calculating structure of internal semidirect product} that the Lie~bracket on~$\elie$ is uniquely determined by the Lie~brackets of~$\glie$ and~$\hlie$ and a certain homomorphism of Lie~algebras~$\theta$ from~$\glie$ to~$\Der(\hlie)$.
	Let~$\elie_{\ext}$ be the external semidirect product of~$\glie$ by~$\hlie$ along~$\theta$.
	Then the linear map
	\[
		\varphi
		\colon
		\elie_{\ext}
		\to
		\elie \,,
		\quad
		(x, c)
		\mapsto
		x + c
	\]
	is an isomorphism of vector spaces, and according to the calculations in \cref{calculating structure of internal semidirect product} it is an isomorphism of Lie~algebras.
	In this way, we see that every internal semidirect product is isomorphic to an external semidirect product.
\end{fluff}


\begin{fluff}
	Let on the other hand~$\glie$ and~$\hlie$ be two~\liealgebras{$\kf$} and let~$\theta$ be a homomorphism of Lie~algebras from~$\glie$ to~$\Der(\hlie)$.
	Let~$\elie$ be the external semidirect product of~$\glie$ by~$\hlie$ along~$\theta$.
	Then~$\elie = \glie \oplus \hlie$ as vector spaces, where the right hand side denotes an external direct sum.
	The linear subspace~$\glie'$ of~$\elie$ given by~$\glie' \defined \glie \oplus 0$ is a Lie~subalgebra of~$\elie$, and the linear subspace~$\hlie'$ given by~$\hlie' \defined 0 \oplus \hlie$ is an ideal of~$\elie$.
	We also see that~$\elie = \elie' \oplus \hlie'$ on the level of vector spaces, where the right hand side denotes an internal direct sum.
	We see from this that~$\elie$ is the internal direct sum of~$\glie'$ by~$\hlie'$.

	We know from \cref{calculating structure of internal semidirect product} that the Lie~bracket of~$\elie$ can be expressed by the Lie~brackets of~$\glie'$ and~$\hlie'$ together with a certain homomorphism of Lie~algebras~$\theta'$ from~$\glie'$ to~$\Der(\hlie')$.
	We can identify~$\glie'$ and~$\hlie'$ with~$\glie$ and~$\hlie$ via the isomorphism of Lie~algebras
	\begin{alignat*}{2}
		\glie
		&\to
		\glie' \,,
		&\quad
		x
		&\mapsto
		(x, 0) \,,
		\\
		\hlie
		&\to
		\hlie' \,,
		&\quad
		c
		&\mapsto
		(0, c) \,.
	\end{alignat*}
	The homomorphism~$\theta'$ corresponds to the original homomorphism~$\theta$ under these isomorphism.

	We have now explained how every external semidirect product can be regarded as an internal semidirect product.
\end{fluff}


\begin{example}
	Let~$\glie$ be a Lie~algebra and let~$\hlie$ be a representation of~$\glie$.
	We may regard~$\hlie$ as an abelian Lie~algebra.
	Then~$\Der(\hlie) = \gllie(\hlie)$.
	We can therefore regard the homomorphism of Lie~algebras~$\rho$ that corresponds to the action of~$\glie$ on~$\hlie$ as a homomorphism of Lie~algebras from~$\glie$ to~$\Der(\hlie)$.
	This allows us to form the external semidirect product~$\glie \ltimes_\rho \hlie$.
	It is given by the vector space~$\glie \oplus \hlie$ together with the Lie~bracket given by
	\[
		[ (x, c), (y, d) ]
		=
		( [x,y], x \act d - y \act c )
	\]
	for all~$x, y \in \glie$ and~$c, d \in \hlie$.
\end{example}


\begin{example}
	Let~$\hlie$ be the abelian two-dimensional Lie~algebra with basis~$x$,~$y$ and let~$\glie$ be the one-dimensional Lie~algebra with basis~$t$.
	Then~$\Der(\hlie) = \gllie(\hlie)$ as seen in \cref{derivations of abelian lie algebra}, and every element~$f$ of~$\gllie(\hlie)$ determines a unique homomorphism of Lie algebras from~$\glie$ to~$\Der(\hlie)$ that maps~$t$ to~$f$.
	We denote for every scalar~$\tau$ in~$\kf$ by~$f_\tau$ the endomorphism of~$\hlie$ given by
	\[
		f_\tau(x) = x \,,
		\quad
		f_\tau(y) = \tau y \,,
	\]
	and by~$\theta_\tau$ the resulting homomorphism of Lie~algebras from~$\glie$ to~$\Der(\hlie)$ that maps~$t$ to~$f_\tau$.
	The external semidirect product~${\kf} \ltimes_{\theta_t} \hlie$ has the elements~$x$,~$y$,~$t$ as a basis, and its Lie~bracket is given on this basis by
	\[
		[x,y] = 0 \,,
		\quad
		[t,x] = x \,,
		\quad
		[t,y] =  \tau y \,.
	\]
	We have thus constructed the Lie~algebras~$\glie_\tau$ from \cref{infinitely many three-dimensional lie algebras} as semidirect products.
\end{example}

\subsubsection{Split Extensions}

\begin{definition}
	Let
	\[
		0
		\to
		\hlie
		\xto{\varphi}
		\elie
		\xto{\psi}
		\glie
		\to
		0
	\]
	be an extension of Lie~algebras.
	A \defemph{section}\index{section}\index{short exact sequence of Lie algebras!section} or \defemph{split}\index{split}\index{short exact sequence of Lie algebras!split} of this extension is a homomorphism of Lie~algebras~$\sigma$ from~$\glie$ to~$\elie$ such that
	\[
		\psi \circ \sigma
		=
		\id_{\glie} \,.
	\]
	The extension \defemph{splits} if it admits a split.
\end{definition}


\begin{lemma}
	\label{having a section is invariant under equivalence of extensions}
	Let~$\glie$ and~$\hlie$ be two~\liealgebras{$\kf$}.
	If an extension~$\elie$ of~$\glie$ by~$\hlie$ admits a split, then every extension of~$\glie$ by~$\hlie$ that is equivalent to~$\elie$ also admits a split.
\end{lemma}


\begin{proof}
	Let the extension~$\elie$ be given by
	\[
		0
		\to
		\hlie
		\xto{\varphi}
		\elie
		\xto{\psi}
		\glie
		\to
		0 \,2
	\]
	and let
	\[
		0
		\to
		\hlie
		\xto{\varphi'}
		\elie'
		\xto{\psi'}
		\glie
		\to
		0
	\]
	be an extensions of~$\glie$ by~$\hlie$ that is equivalent to~$\elie$.
	Let~$\alpha$ be an isomorphism of Lie~algebras from~$\elie$ to~$\elie'$ which makes the diagram
	\[
		\begin{tikzcd}
			0
			\arrow{r}
			&
			\hlie
			\arrow{r}[above]{\varphi}
			\arrow[equal]{d}
			&
			\elie
			\arrow{r}[above]{\psi}
			\arrow[dashed]{d}[right]{\alpha}
			&
			\glie
			\arrow{r}
			\arrow[equal]{d}
			&
			0
			\\
			0
			\arrow{r}
			&
			\hlie
			\arrow{r}[above]{\varphi'}
			&
			\elie'
			\arrow{r}[above]{\psi'}
			&
			\glie
			\arrow{r}
			&
			0
		\end{tikzcd}
	\]
	commute.
	If~$\sigma$ is a split for the first extension, then the composite~$\alpha \circ \sigma$ is a split for the second extension.
	This shows that the second extension admits a split if the first one does.
	The converse implication follows from this first implication because equivalence of extensions is an equivalence relation.
\end{proof}


\begin{proposition}
	Let~$\glie$ and~$\hlie$ be two Lie~algebras and let
	\begin{equation}
		\label{extension that may split}
		0
		\to
		\hlie
		\xto{\varphi}
		\elie
		\xto{\psi}
		\glie
		\to
		0
	\end{equation}
	be an extension of~$\glie$ by~$\hlie$.
	The following conditions on this extension are equivalent.
	\begin{equivalenceslist}
		\item
			\label{the extension splits}
			The extension~\eqref{extension that may split} splits.
		\item
			\label{is an internal semidirect product}
			There exists a Lie~subalgebra~$\glie'$ of~$\elie$ such that~$\elie$ is the internal semidirect product of~$\glie'$ by~$\varphi(\hlie)$.
		\item
			\label{is an external semidirect product}
			The extension~\eqref{extension that may split} is isomorphic (as extensions) to an external semidirect product of~$\glie$ by~$\hlie$.
	\end{equivalenceslist}
\end{proposition}


\begin{proof}
	\leavevmode
	\begin{implicationlist}
		\item[\ref{the extension splits}~$\implies$~\ref{is an internal semidirect product}]
			There exists by assumption a split~$\sigma$ for the extension~\eqref{extension that may split}, i.e. a homomorphism of Lie~algebras from~$\glie$ to~$\elie$ with~$\sigma \circ \psi = \id_{\glie}$.
			Let~$\glie'$ be the image of~$\sigma$.
			The composite~$\varepsilon \defined \sigma \circ \psi$ is an idempotent endomorphism of Lie~algebras of~$\elie$.
			It follows that
			\[
				\elie
				=
				\im(\varepsilon) \oplus \ker(\varepsilon)
			\]
			as vector spaces, with~$\im(\varepsilon)$ being a Lie~subalgebra of~$\elie$ and~$\ker(\varepsilon)$ beinan ideal of~$\elie$.
			The homomorphism~$\psi$ is surjective, so
			\[
				\im(\varepsilon)
				=
				\im(\sigma \circ \psi)
				=
				\im(\sigma)
				=
				\glie' \,.
			\]
			The homomorphism~$\sigma$ is injective (because it admits a left inverse, namely~$\psi$), so
			\[
				\ker(\varepsilon)
				=
				\ker(\sigma \circ \psi)
				=
				\ker(\psi)
				=
				\im(\varphi)
				=
				\varphi(\hlie) \,.
			\]
			We have overall shown that~$\elie$ is the internal semidirect product of~$\glie'$ by~$\varphi(\hlie)$.
		\item[\ref{is an internal semidirect product}~$\implies$~\ref{is an external semidirect product}]
			This follows from the discussion in \cref{internal semidirect product is also external} by noting that the constructed isomorphism of Lie~algebras~$\varphi$ is already an isomorphism of extensions.
		\item[\ref{is an external semidirect product}~$\implies$~\ref{the extension splits}]
			According to \cref{having a section is invariant under equivalence of extensions} we may assume that~$\elie$ is an external semidirect product of~$\glie$ by~$\hlie$ along a homomorphism of Lie~algebras~$\theta$ from~$\glie$ to~$\Der(\hlie)$.
			The linear map
			\[
				\glie
				\to
				\elie \,,
				\quad
				x
				\mapsto
				(x,0)
			\]
			is then a homomorphism of Lie~algebras, and the desired split.
		\qedhere
	\end{implicationlist}
\end{proof}


\begin{remark}
	\label{each class of abelian extensions contains precisely one split extension}
	Let~$\glie$ be a Lie~algebra and let~$M$ be a representation of~$\glie$.
	We regard the underlying vector space of~$M$ as an abelian Lie~algebra~$\hlie$ and denote by~$\theta$ the homomorphism of Lie~algebras from~$\hlie$ to~$\Der(\hlie) = \gllie(\hlie) = \gllie(M)$ that corresponds to the action of~$\elie$ on~$M$.
	We consider the resulting class of abelian extensions~$\AbEx(\glie, M)$ from~\cref{structure of abelian extensions}.

	The external semidirect product~$(0,\theta)$ is an abelian extension of~$\glie$ by~$\hlie$.
	We have seen in~\cref{structure of abelian extensions} that this extension is contained in~$\AbEx(\glie, M)$.

	Suppose on the other hand that~$\elie$ is some split extension of~$\glie$ by~$\hlie$ contained in~$\AbEx(\glie, M)$.
	The extension~$\elie$ is then isomorphic to a standard extension of the form~$(0, \theta')$.
	It follows from \cref{structure of abelian extensions} that~$\theta' = \theta$.

	We have thus shown that the class of extensions~$\AbEx(\glie, M)$ contains precisely one split extension up to equivalence.
\end{remark}










