\section{Definition and Basic Properties}


\begin{definition}
  A Lie~algebra~$\glie$ is \defemph{semisimple}\index{semisimple!Lie algebra}\index{Lie algebra!semisimple} if it is the direct sum of finitely many simple ideals, i.e.\ if there exist ideals~$I_1, \dotsc, I_n$ in~$\glie$ with~$\glie = I_1 \oplus \dotsb \oplus I_n$ such that every~$I_j$ is simple (as a Lie~algebra).
\end{definition}


\begin{recall}
  If a Lie~algebra~$\glie$ can be decomposed into a direct sum~$\glie = I_1 \oplus \dotsb \oplus I_n$ where~$I_1, \dotsc, I_n$ are ideals of~$\glie$ then it follows (see \cref{direct sum of ideals}) that
  \[
    [x_1 + \dotsb + x_n, y_1 + \dotsb + y_n]
    =
    [x_1, y_1] + \dotsb + [x_n, y_n]
  \]
  for all~$x_j, y_j \in I_j$ where~$j = 1, \dotsc, n$.
  In particular
  \[
    [\glie, \glie]
    =
    [I_1, I_1] \oplus \dotsb \oplus [I_n, I_n]
  \]
  and
  \[
    \centerlie(\glie)
    =
    \centerlie(I_1) \oplus \dotsb \oplus \centerlie(I_n)  \,.
  \]
\end{recall}


\begin{lemma}
  Let~$\glie$ be a semisimple Lie~algebra.
  Then~$[\glie, \glie] = \glie$ and~$\centerlie(\glie) = 0$.
\end{lemma}


\begin{proof}
 Let~$I_1, \dotsc, I_n$ be simple ideals in~$\glie$ with~$\glie = I_1 \oplus \dotsb \oplus I_n$.
 Then
 \begin{gather*}
  [\glie, \glie]
  =
  [I_1, I_1] \oplus \dotsb \oplus [I_n, I_n]
  =
  I_1 \oplus \dotsb \oplus I_n
  =
  \glie
 \shortintertext{as well as}
  \centerlie(\glie)
  =
  Z(I_1) \oplus \dotsb \oplus Z(I_n)
  =
  0 \oplus \dotsb \oplus 0
  =
  0
 \end{gather*}
 because~$I_1, \dotsc, I_n$ are simple.
\end{proof}


\begin{corollary}
  \label{representation of semisimple lie algebra are traceless}
    Let~$\glie$ be a semisimple Lie~algebra and let~$\rho \colon \glie \to \gllie(V)$ be a finite dimensional representation of~$\glie$.
    Then~$\rho(\glie)$ is contained in~$\sllie(V)$, i.e.~$\tr(\rho(x)) = 0$ for every~$x \in \glie$.
\end{corollary}


\begin{proof}
  We observe that~$\rho(\glie) = \rho([\glie,\glie]) = [\rho(\glie), \rho(\glie)] \subseteq [\gllie(V), \gllie(V)] = \sllie(V)$.
\end{proof}


\begin{lemma}
  \label{properties of simple decompositions}
  Let~$\glie$ be a Lie~algebra.
  \begin{enumerate}
    \item
      \label{intersection of simples}
      If~$I$ and~$J$ are two simple ideals in a Lie~algebra~$\glie$ then~$I \cap J = 0$ or~$I = J$.
    \item
      If~$\glie = I_1 \oplus \dotsb \oplus I_n$ is a decomposition into ideals and~$J$ is any other simple ideal in~$\glie$ then~$J = I_i$ for exactly one~$i$.
  \end{enumerate}
\end{lemma}


\begin{proof}
  \leavevmode
  \begin{enumerate}
    \item
      If then~$I \cap J \neq 0$ then this intersection is a nonzero ideal in~$\glie$, and hence a nonzero ideal in both~$I$ and~$J$.
      Then~$I = I \cap J = J$ because~$I$ and~$J$ are simple.
    \item
      We have~$J = [\glie, J]$ because on the one hand~$[\glie, J] \subseteq J$ because~$J$ is an ideal and on the other hand~$J = [J,J] \subseteq [\glie, J]$ because~$J$ is simple.
      It follows that
      \[
        J
        =
        [J,J]
        =
        \left[ \bigoplus_{i=1}^n I_i, J \right]
        =
        \sum_{i=1}^n [I_i, J] \,.
      \]
      The sum~$\sum_{i=1}^n [I_i, J]$ is direct because the summand~$[I_i, J]$ is contained in~$I_i$, the sum of which is direct.
      Hence
      \begin{equation}
        \label{decomposing into direct sum}
        J
        =
        \bigoplus_{i=1}^n [I_i, J]  \,.
      \end{equation}
      It follows from~$J \neq 0$ that~$[I_i, J] \neq 0$ for some~$i$.
      It follows from part~\ref*{intersection of simples} that already~$I_i = J$.
      It follows from the directness of the sum~\eqref{decomposing into direct sum} that~$[J, I_j] = 0$ for every~$j \neq i$, and hence~$J \neq I_j$ for every~$j \neq i$.
    \qedhere
  \end{enumerate}
\end{proof}


\begin{corollary}
  \label{uniqueness of semisimple decomposition}
  Let~$\glie$ be a Lie~algebra and let~$I_1, \dotsc, I_n$ be simple ideals in~$\glie$ with~$\glie = I_1 \oplus \dotsb \oplus I_n$.
  Then the ideals~$I_1, \dotsc, I_n$ are unique up to reordering.
\end{corollary}


\begin{proof}
  Let~$J_1, \dotsc, J_m$ be simple ideals in~$\glie$ with~$\glie = J_1 \oplus \dotsb \oplus J_m$.
  Then by \cref{properties of simple decompositions} there exists for every~$i \in \{1, \dotsc, n\}$ a unique index~$\sigma(i) \in \{1, \dotsc, m\}$ with~$I_i = J_{\sigma(i)}$.
  In the same way we find that there exists for every index~$j \in \{1, \dotsc, m\}$ some index~$\tau(j) \in \{1, \dotsc, n\}$ with~$J_j = I_{\tau(j)}$.
  Then~$I_{\tau(\sigma(i))} = I_i$ and hence~$\tau(\sigma(i)) = i$ for every~$i \in \{1, \dotsc, n\}$.
  Similarly~$\sigma(\tau(j)) = j$ for every~$j \in \{1, \dotsc, m\}$.
  This shows together that the maps~$\sigma$ and~$\tau$ are mutually inverse bijections, which give the claimed reordering.
\end{proof}
 
 
\begin{remark}
  If~$\glie$ is a semisimple Lie~algebra then we will often just talk about \emph{the} decomposition of~$\glie$ into  direct sum of simple ideals, as this decomposition is unique up to reordering of the summands.
\end{remark}


\begin{proposition}
  \label{characterisation of zero radical}
  For any finite dimensional Lie algebra~$\glie$ the following conditions are equivalent:
  \begin{equivalenceslist}
    \item
      \label{killing form is nondegenerate}
      The Killing~form~$\kappa$ of~$\glie$ is non-degenerate (i.e.~$\rad \kappa = 0$).
    \item
      \label{contains no solvable ideal}
      $\glie$ contains no nonzero solvable ideals (i.e.~$\rad \glie = 0$).
    \item
      \label{contains no abelian ideal}
      $\glie$ contains no nonzero abelian ideals.
  \end{equivalenceslist}
\end{proposition}


\begin{proof}
  \leavevmode
  \begin{implicationlist}
    \item[\ref*{killing form is nondegenerate}~$\implies$~\ref*{contains no abelian ideal}:]
      Let~$I$ be an abelian ideal in~$\glie$.
      We show that~$I$ is contained in the radical of~$\kappa$, from which it then follows that~$I = 0$ because~$\rad \kappa = 0$.
      We need to show that~$\kappa(x,y) = 0$ for all~$x \in I$ and~$y \in \glie$.
      We have~$(\ad(x) \ad(y))^2 = 0$ because
      \begin{align*}
        (\ad(x) \ad(y))^2(\glie)
        &=
        \ad(x)\ad(y)\ad(x)\ad(y)(\glie)
        \\
        &\subseteq
        \ad(x)\ad(y)\ad(x)(\glie)
        \\
        &\subseteq
        \ad(x)\ad(y)(I)
        \\
        &\subseteq
        \ad(x)(I)
        \\
        &\subseteq
        [I,I] \,,
      \end{align*}
      where we use that~$\ad(x)(\glie) \subseteq I$ because~$I$ is an ideal.
      Now~$[I,I] = 0$ because~$I$ is abelian, and thus~$(\ad(x)\ad(y))^2 = 0$.
      This shows that~$\ad(x)\ad(y)$ is nilpotent, which gives
      \[
        \kappa(x,y)
        =
        \tr(\ad(x) \ad(y))
        =
        0 \,.
      \]
%     \item[\ref*{contains no solvable ideal}~$\implies$~\ref*{contains no abelian ideal}:]
%       Every abelian ideal is solvable.
    \item[\ref*{contains no abelian ideal}~$\implies$~\ref*{contains no solvable ideal}:]
      Suppose that~$I$ is a nonzero solvable ideal in~$I$.
      Then~$I^{(n)} \neq 0$ but~$I^{(n+1)} = 0$ for some~$n \geq 1$.
      This means that~$I^{(n)}$ is a nonzero abelian ideal in~$\glie$.
    \item[\ref*{contains no solvable ideal}~$\implies$~\ref*{killing form is nondegenerate}:]
      The radical~$\rad \kappa$ is a solvable ideal by \cref{rad kappa is a solvable ideal}.
%     \item[\ref*{product of simple lie algebras}~$\implies$~\ref*{sum of simple ideals}:]
%       The factors~$\glie_i$ correspond to such simple ideals~$I_i$.
    \qedhere
  \end{implicationlist}
\end{proof}


\begin{corollary}
  \label{decomposition into orthogonals for semisimple}
  Let~$\glie$ be a finite dimensional Lie~algebra whose Killing~form~$\kappa$ is non-degenerate.
  Then~$\glie = I \oplus I^\perp$ for every ideal~$I$ in~$\glie$.
\end{corollary}


\begin{proof}
  The orthogonal~$I^\perp$ is again in ideal in~$\glie$ by \cref{orthogonal complement of an ideal is again an ideal}.
  The intersection~$I \cap I^\perp$ is again an ideal in~$\glie$ with~$\restrict{\kappa}{I \cap I^\perp} = 0$.
  This restriction is the Killing~form of~$I \cap I^\perp$ by \cref{restriction of the killing form to an ideal}.
  It follows from Cartan’s~criterion that the ideal~$I \cap I^\perp$ is solvable.
  Hence~$I \cap I^\perp = 0$ by \cref{characterisation of zero radical}.
  It also holds that~$\dim \glie = \dim I + \dim I^\perp$ because~$\kappa$ is non-degenerate.
  Together this means that~$\glie = I \oplus I^\perp$.
\end{proof}


\begin{theorem}[Characterizations of finite dimensional semisimple Lie~algebras]
  \label{characterizations of fd semisimple lie algebras}
  For any finite dimensional Lie algebra~$\glie$ the following conditions are equivalent:
  \begin{equivalenceslist}
    \item
    \label{product of simple lie algebras general}
      $\glie \cong \glie_1 \times \dotsb \times \glie_r$ for some~$n \geq 0$ and simple Lie~algebras~$\glie_1, \dotsc, \glie_n$.
    \item
      \label{sum of simple ideals general}
      $\glie = I_1 \oplus \dotsb \oplus I_m$ for some~$m \geq 0$ and simple ideals~$I_1, \dotsc, I_m$ of~$\glie$.
    \item
      \label{killing form is nondegenerate general}
      The Killing~form~$\kappa$ of~$\glie$ is non-degenerate (i.e.~$\rad \kappa = 0$).
    \item
      \label{contains no solvable ideal general}
      $\glie$ contains no nonzero solvable ideals (i.e.~$\rad \glie = 0$).
    \item
      \label{contains no abelian ideal general}
      $\glie$ contains no nonzero abelian ideals.
  \end{equivalenceslist}
\end{theorem}


\begin{proof}
  The equivalence of the conditions~\ref*{killing form is nondegenerate general},~\ref*{contains no solvable ideal general} and~\ref*{contains no abelian ideal general} is \cref{characterisation of zero radical}.
  \begin{implicationlist}
    \item[\ref*{killing form is nondegenerate general}~$\implies$~\ref*{sum of simple ideals general}:]
      We show the implication by induction over the dimension~$n$ of~$\glie$.
      If~$n = 0$ then~$\glie = 0$ is the empty sum over all its simple ideals.
      Suppose that~$n \geq 1$ and that the implication holds for all strictly smaller dimensions.
      If~$\glie$ is simple then we are already finished.
      Otherwise~$\glie$ contains some nonzero proper ideal~$I$.
      Then~$\glie = I \oplus I^\perp$ by \cref{decomposition into orthogonals for semisimple}.
      The Killing~form~$\kappa$ of~$\glie$ is by \cref{orthogonal ideals with respect to the killing form} given by the orthogonal sum of the Killing~forms of~$I$ and~$I^\perp$.
      That~$\kappa$ is non-degenerate therefore means that both~$\kappa_{I}$ and~$\kappa_{I^\perp}$ are non-degenerate. 
      By induction hypothesis both~$I$ and~$I^\perp$ can be decomposed into direct sums of simple ideals.
      By combining these two decompositions we get a decomposition of~$\glie$ into simple ideals.
      (Here we use that if~$\glie = I \oplus J$ is a decomposition into ideals then any ideal of~$I$ (resp.~$J$) is again an ideal in~$\glie$.)
    \item[\ref*{sum of simple ideals general}~$\implies$~\ref*{product of simple lie algebras general}:]
      We can take~$\glie_i = I_i$ for every~$i$.
    \item[\ref*{product of simple lie algebras general}~$\implies$~\ref*{contains no solvable ideal general}:]
      For every~$i = 1, \dotsc, n$ let~$\pi_i \colon \glie \to \glie_i$ be the canonical projection.
      If~$I$ is a nonzero ideal in~$\glie$ then~$\pi_i(I) \neq 0$ for some~$i$, and~$\pi_i(I)$ is an ideal in~$\glie$ because~$\pi_i$ is surjective.
      Hence~$\pi_i(I) = \glie_i$ because~$\glie_i$ is simple.
      If~$I$ is solvable then so is~$\pi(I) = \glie_i$, but this contradicts~$\glie_i$ being simple.
    \qedhere
  \end{implicationlist}
\end{proof}


\begin{corollary}
  \label{ideals are again sum of simples}
  Let~$\glie$ be a finite dimensional semisimple Lie~algebra with decomposition into simple ideals~$\glie = I_1 \oplus \dotsb \oplus I_n$.
  Then any ideal of~$\glie$ is of the form~$I_{j_1} \oplus \dotsb \oplus I_{j_m}$ for some~$j_1, \dotsc, j_m \in \{1, \dotsc, n\}$.
\end{corollary}


\begin{proof}
  Let~$I$ be any ideal of~$\glie$.
  We proceed as in the proof of \cref{characterizations of fd semisimple lie algebras}:
  
  We have~$\glie = I \oplus I^\perp$ by \cref{decomposition into orthogonals for semisimple} and the Killing form of~$\glie$ is the orthogonal sum of the Killing forms of~$I$ and~$I^\perp$ by \cref{orthogonal ideals with respect to the killing form}.
  Both~$\restrict{\kappa}{I}$ and~$\restrict{\kappa}{I^\perp}$ are therefore non-degenerate.
  We can by \cref{characterizations of fd semisimple lie algebras} decompose both~$I$ and~$I^\perp$ into simple ideals
  \[
    I
    =
    J_1 \oplus \dotsb \oplus J_m \,,
    \qquad
    I^\perp
    =
    J_{m+1} \oplus \dotsb \oplus J_k  \,.
  \]
  Then
  \[
    \glie
    =
    I \oplus I^\perp
    =
    J_1 \oplus \dotsb \oplus J_m \oplus J_{m+1} \oplus \dotsb \oplus J_k
  \]
  is a decomposition of~$\glie$ into simple ideals.
  We know from \cref{uniqueness of semisimple decomposition} that this decomposition coincides with the given decomposition~$\glie = I_1 \oplus \dotsb \oplus I_n$ up to permutation of the direct summands~$I_j$.
  The direct summands~$J_1, \dotsc, J_m$ do therefore occur in the list of simple ideals~$I_1, \dotsc, I_n$.
\end{proof}


\begin{corollary}
  \label{ideals and quotients of semisimple again semisimple}
  If~$\glie$ is a semisimple Lie algebra and~$I$ is any ideal in~$\glie$ then both~$I$ and~$\glie/I$ are again semisimple.
  \qed
\end{corollary}


\begin{proof}
  Let~$\glie = I_1 \oplus \dotsb \oplus I_n$ be a decomposition into simple ideals.
  Then by \cref{ideals are again sum of simples} we may assume that~$I = I_1 \oplus \dotsb \oplus I_m$ for some~$m$.
  We see that~$I$ is semisimple, and it also follows that~$\glie/I \cong I_{m+1} \oplus \dotsb \oplus I_n$ is semisimple.
\end{proof}


\begin{warning}
  Lie~subalgebras of semisimple Lie~algebras need not be semisimple again.
  Indeed, the Lie~algebra~$\sllie_2(\kf)$ is simple and hence semisimple.
  But the Lie~subalgebra~$\blie$ of~$\sllie_2(\kf)$ that consist of upper triangular matrices of trace zero is solvable and can therefore not be semisimple.
\end{warning}



\begin{lemma}
  \label{semisimple lie algebra identification with dual space}
  Let~$\glie$ be a finite dimensional semisimple Lie~algebra.
  Then the map
  \[
    \glie
    \to
    \glie^* \,,
    \quad
    x
    \mapsto
    \kappa(x,-)
  \]
  is an isomorphism of~{\representations{$\glie$}}, where~$\kappa$ denotes the Killing~form of~$\glie$.
\end{lemma}


\begin{proof}
  This follows from \cref{associative non-degenerate bilinear forms induce isomorphism to the dual} because~$\kappa$ is non-degenerate.
\end{proof}


\begin{corollary}
  Let~$\glie$ be a finite dimensional simple Lie~algebra.
  Then any associative bilinear form~$\beta \colon \glie \times \glie \to \kf$ is a scalar multiple of the killing form~$\kappa$ of~$\glie$.
\end{corollary}


\begin{proof}
 The map
 \[
  \varphi
  \colon
  \glie
  \to
  \glie^* \,,
  \quad
  x
  \mapsto
  \beta(x, -)
 \]
 is a homomorphism of~{\representations{$\glie$}} by \cref{associative bilinear form induces homomorphism of representations}.
 The Killing~form~$\kappa$ is non-degenerate because~$\glie$ is simple (and thus semisimple).
 The map
 \[
  \psi
  \colon
  \glie
  \to
  \glie^* \,,
  \quad
  x
  \mapsto
  \kappa(x, -)
 \]
 is therefore an isomorphism of representations by \cref{associative non-degenerate bilinear forms induce isomorphism to the dual}.
 The composition
 \[
  \psi^{-1}
  \circ
  \varphi
  \colon
  \glie
  \to
  \glie 
 \]
 is again a homomorphism of representations, where~$\glie$ is irreducible as a representation because~$\glie$ is simple as a Lie~algebra.
 It follows from Schur’s~lemma that~$\psi^{-1} \circ \varphi = \lambda \id_{\glie}$ for some scalar~$\lambda \in \kf$.
 So~$\varphi = \lambda \psi$ and equivalently~$\beta = \lambda \kappa$.
\end{proof}




