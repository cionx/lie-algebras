\chapter{Schur’s Lemma}
Unless otherwise noted $k$ always is some arbitrary field. Whenever we talk about a ring (resp.\ $k$-algebra) we always mean an associative and unitary one, and homomorphisms of rings (resp.\ $k$-algebras) are required to respect the unit. We assume that the reader is familiar with the definition of a module over a ring notion of a submodules. By an (left) $R$-module $M$ over a ring $R$ we always mean an unitial module, i.e.\ $1 \cdot m = m$ for every $m \in M$.





\section{Classic version}


\begin{definition}
 Let $M$ be a module over a ring $R$. Then $M$ is called \emph{simple} or \emph{irreducible} if $M$ contains precisely two sumbodules. Equivalently $M$ in nonzero and its only submodules are the \emph{trivial} ones, namely $0$ and $M$ itself.
\end{definition}


\begin{lemma}[Schur] \label{lem: Schur general part about skew field}
 Let $R$ be a ring and $M$ a simple module over $R$. Then any endomorphism of modules $f \colon M \to M$ is either zero or an isomorphism. In particular $\End_R(M)$ is a skew field.
\end{lemma}
\begin{proof}
 As $M$ is nonzero $f$ cannot be zero and an isomorphism at the same time. If $f \neq 0$ then $\ker f$ is a proper submodule of $M$ and $\im f$ is a nonzero submodule of $M$, so $\ker f = 0$ and $\im f = M$ because $M$ is simple.
\end{proof}


\begin{corollary}
 Let $M$ be an $A$-module over a $k$-algebra $A$. Then $\End_A(M)$ is a division algebra over $k$.
\end{corollary}


\begin{lemma}\label{lem: algebraic elements over algebraically closed fields}
 Let $D$ be a division algebra over an algebraically closed field $k$. If $x \in D$ is algebraic over $k$ then already $x \in k$.
\end{lemma}
\begin{proof}
 Let $P \in k[T]$ be nonzero with $P(x) = 0$. W.l.o.g.\ $P$ can assumed to be monic. Because $k$ is algebraically closed there exist $\alpha_1, \dotsc, \alpha_r \in k$ with $P = \prod_{i=1}^r (x-\alpha_i)$. Because $0 = P(x) = c \prod_{i=1}^n (x-\alpha_i)$ and $D$ is a skew field it follows that $x = \alpha_i$ for some $i$ and therefore $x \in k$.
\end{proof}


\begin{corollary}
 Let $k$ be an algebraically closed field and $L$ a finite-dimensional division algebra over $k$. Then $L = k$.
\end{corollary}
\begin{proof}
 Let $x \in L$. Because $L$ is finite-dimenisonal over $k$ there exists some $n \geq 1$ such that $1, x, x^2, \dotsc, x^n$ are linearly dependent over $k$. Therefore there exist some $a_0, a_1, \dotsc, a_n \in k$ such that $a_0 + a_1 x + \dotsb + a_n x^n = 0$ is a non-trivial linear combination. Then $P = \sum_{i=0}^n a_i T^i \in k[T]$ is nonzero with $P(x) = 0$, so $x$ is algebraic over $k$. From Lemma~\ref{lem: algebraic elements over algebraically closed fields} it follows that $x \in k$.
\end{proof}


\begin{corollary}[Schur, classic Version] \label{cor: classic Schur}
 Let $k$ be an algebraically closed field and $M$ a simple $A$-module for a $k$-algebra $A$. If $M$ is finite-dimensional over $k$ then $\End_A(M) = k$, i.e.\ every module endomorphism of $M$ is given by multiplication with a scalar.
\end{corollary}


\begin{corollary}
 Let $\g$ be a Lie algebra over an algebraically closed field $k$ and $V$ a irreducible and finite-dimensional representation of $\g$. Then $\End_\g(V) = k$, i.e.\ every endomorpism of $V$ as a representation of $\g$ is given by an multiplication with some scalar.
\end{corollary}
\begin{proof}
 Take $V$ as a simple module over the universal enveloping algebra $\Ue(\g)$ and apply Corollary~\ref{cor: classic Schur}.
\end{proof}





\section{Generalization by Dixmier}


\begin{definition}
 Let $V$ be a vector space over a field $k$. An endomorphism $\varphi \in \End_k(V)$ is called \emph{algebraic} over $k$ there exists some nonzero polynomial $P \in k[T]$ mit $P(\varphi) = 0$.
\end{definition}


\begin{lemma}\label{lem: algebraic endomorphisms over algebraically closed fields}
 Let $k$ be an algebraically closed field, $V$ a vector space over $k$ and $D \subseteq \End_k(V)$ a division algebra over $k$. If $\varphi \in D$ is algebraic over $k$ then $\varphi = \alpha \id_V$ for some $\alpha \in k$.
\end{lemma}
\begin{proof}
 This follows directly from Lemma~\ref{lem: algebraic elements over algebraically closed fields}.
\end{proof}


\begin{corollary}\label{cor: Schur generally needs only algebraic endomorphisms}
 Let $k$ be an algebraically closed field, $A$ a $k$-algebra and $M$ a simple $A$-module. If $\varphi \in \End_A(M)$ is algebraic then $\varphi = \alpha \id_M$ for some $\alpha \in k$.
\end{corollary}
\begin{proof}
 This follows directly from Lemma~\ref{lem: algebraic endomorphisms over algebraically closed fields} because $\End_A(M) \subseteq \End_k(M)$ is a division algebra over $k$ by Lemma~\ref{lem: Schur general part about skew field}.
\end{proof}


The following Proposition traces back to \cite{Dixmier}. (At least this is what I found on the web --- I could not find the original article, nor would I be able to read it (as it was apparently written in French)).


\begin{proposition}[Dixmier]\label{prop: Dixmier}
 Let $M$ be a simple $A$-module for a $k$-algebra $A$, such that $\dim_k M > \card k$. Then every $\varphi \in \End_A(M)$ is algebraic over $k$.
\end{proposition}
\begin{proof}
 Suppose that there exists some $\varphi \in \End_A(M)$ which is not algebraic over $k$. Then the kernel of the map
 \[
  \iota \colon k[T] \to \End_A(M), \quad P \mapsto P(\varphi)
 \]
 is zero, hence $\iota$ is an inclusion of $k[T]$ into $\End_A(M)$, which is a skew field by Lemma \ref{lem: Schur general part about skew field}. It follows That $\iota$ can be extended to a well-defined inclusion
 \[
  \theta \colon k(T) \to \End_A(M), \quad \frac{P}{Q} \mapsto P(\varphi) Q(\varphi)^{-1}.
 \]
 Hence $M$ carries the structure of a $k(T)$-vector space with
 \[
  \frac{P}{Q} \cdot m = P(\varphi)Q(\varphi)^{-1}(m)
  \quad \text{for every $\frac{P}{Q} \in k(T)$ and $m \in M$}.
 \]
 
 As $M$ is a nonzero $k(T)$-vector space it follows that $\dim_k M \geq \dim_k k(T)$. To see this notice that if $L/k$ is any field extension and $V$ a nonzero $L$-vector space then there exists an inclusion $L \inc V$ of $L$-vector spaces. This is then also an inclusion of $k$-vector spaces, which is why $\dim_k V \geq \dim_k L$. The statement follows with $L = k(T)$ and $V = M$. (This is a straightforward generalization of the fact that every complex nonzero vector space is at least twodimensional as a real vector space.) Since $(1/(T-a))_{a \in k}$ is a familiy of elements of $k(T)$ which is linearly independent over $k$ it also follows that $\dim_k k(T) \geq \card k$.

 Putting the above observations together it follows that
 \[
  \dim_k M \geq \dim_k k(T) \geq \card k,
 \]
 contradicting the assumption that $\card k > \dim_k M$.
\end{proof}


\begin{corollary}\label{cor: Dixmier algebraically closed}
 Let $k$ be an algebraically closed field, $A$ a $k$-algebra and $M$ a simple $A$-module. If $\card k > \dim_k M$ then $\End_A(M) = k$.
\end{corollary}
\begin{proof}
 This is a combination of Corollary \ref{cor: Schur generally needs only algebraic endomorphisms} and Proposition \ref{prop: Dixmier}.
\end{proof}


\begin{corollary}
 Let $\g$ be a Lie algebra over an algebraically closed field $k$ and $V$ an irreducible representation of $\g$ with $\card k > \dim_k V$. Then $\End_\g(V) = k$.
\end{corollary}
\begin{proof}
 Take $V$ as a simple module over the universal enveloping algebra $\Ue(\g)$ and apply Corollary \ref{cor: Dixmier algebraically closed}.
\end{proof}


\begin{example}
 Let $\g$ be complex Lie algebra and $V$ an irreducible representation of $\g$ of countable dimension. Then $\End_\g(V) = \Cbb$.
\end{example}


\begin{remark}
 The requirement that $\card k > \dim_k M$ in Corollary \ref{cor: Dixmier algebraically closed} can not be dropped without adding some other restraints. To see this take $k \coloneqq \overline{\Q}$ as well as $A = M = \overline{\Q}(T)$. Then $\dim_k M = \card k$ and $\End_A(M) = \End_{\overline{\Q}(T)}(\overline{\Q}(T)) = \overline{\Q}(T)$.
\end{remark}





\section{Generalizaton by Quillen}


The following Proposition is due to \cite{Quillen}.


\begin{proposition}[Quillen] \label{prop: Quillen}
 Let $k$ be a field and $A$ a filtered $k$-algebra, such that $\gr A$ is finitely generated and commutative as a $k$-algebra. If $M$ is a simple $A$-module then every $\varphi \in \End_A(M)$ is algebraic over $k$.
\end{proposition}


\begin{corollary}
 Let $\g$ be finite-dimensional Lie algebra over an algebraically closed field $k$ and $V$ as irreducible representation of $\g$. Then $\End_\g(V) = k$.
\end{corollary}
\begin{proof}
 Take $V$ as a simple module over the universal enveloping algebra $\Ue(\g)$. If $x_1, \dotsc, x_n$ is a $k$-basis of $\g$ then by the abstract version of the PBW theorem
 \[
  \gr \Ue(\g) \cong S(\g) \cong k[x_1, \dotsc, x_n].
 \]
 Applying Proposition~\ref{prop: Quillen} to $\Ue(\g)$ and $V$ the statement follows from Corollary~\ref{cor: Schur generally needs only algebraic endomorphisms}.
\end{proof}

































