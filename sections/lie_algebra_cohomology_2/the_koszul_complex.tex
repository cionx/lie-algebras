\section{The Koszul Resolution}
\label{koszul resolution}


\begin{convention}
	We fix for this \lcnamecref{koszul resolution} a Lie~algebra~$\glie$.
	For every natural number~$n$ we regard the tensor product~$\Univ(\glie) \otimes_{\kf} \Exterior^n(\glie)$ as an~\module{$\Univ(\glie)$} via extension of scalars, i.e. via the multiplication given by
	\[
		x \cdot (y \otimes t)
		=
		(xy) \otimes t
	\]
	for all~$x, y \in \Univ(\glie)$ and~$t \in \Exterior^n(\glie)$.
\end{convention}


\begin{lemma}
	\label{extension of scalars is free}
	The~\module{$\Univ(\glie)$}~$\Univ(\glie) \otimes_{\kf} \Exterior^n(\glie)$ is free for every~$n \geq 0$.
\end{lemma}


\begin{proof}
	The exterior power~$\Exterior^n(\glie)$ is free as a~\module{$\kf$}.
	The extension of scalars~$\Univ(\glie) \otimes_{\kf} \Exterior^n(\glie)$ is therefore free as a~\module{$\Univ(\glie)$}.
\end{proof}


\begin{recall}
	Let~$A$ be a~\algebra{$\kf$} and let~$M$ be an~\module{$A$}.
	A \defemph{resolution}\index{resolution} of~$M$ is an exact sequence of the form
	\[
		\dotsb
		\to
		M_2
		\xto{d_2}
		M_1
		\xto{d_1}
		M_0
		\xto{\varepsilon}
		M
		\to
		0 \,.
	\]
	Such a resolution is \defemph{free}\index{free resolution}\index{resolution!free} if the modules~$M_i$ are free for every~$i \geq 0$.
\end{recall}


\begin{proposition}
	\label{construction of koszul differential}
	\leavevmode
	\begin{enumerate}
		\item
			There exists for every natural number~$n$ with~$n \geq 1$ a unique linear map
			\[
				d_n
				\colon
				\Univ(\glie) \otimes_{\kf} \Exterior^n(\glie)
				\to
				\Univ(\glie) \otimes_{\kf} \Exterior^{n-1}(\glie)
			\]
			given by
			\begin{align*}
				d_n( y \otimes x_1 \wedge \dotsb \wedge x_n )   
				={}&
				\sum_{1 \leq i < j \leq n}
				(-1)^{i+j}
				y \otimes [x_i, x_j] \wedge x_1 \wedge \dotsb \wedge \widehat{x_i} \wedge \dotsb \wedge \widehat{x_j} \wedge \dotsb \wedge x_n
				\\
				{}&
				+
				\sum_{i=1}^n
				(-1)^{i+1}
				(y x_i) \otimes x_1 \wedge \dotsb \wedge \widehat{ x_i } \wedge \dotsb \wedge x_n
			\end{align*}
			for all~$y \in \Univ(\glie)$ and~$x_1, \dotsc, x_n \in \glie$.
		\item
			These maps are homomorphism of~\modules{$\Univ(\glie)$}.
		\item
			These maps satisfy the condition~$d_{n-1} \circ d_n = 0$ for every~$n \geq 2$.
	\end{enumerate}
\end{proposition}


\begin{proof}
	We have a right~\module{$\Univ(\glie)$} structure on~$\Univ(\glie)$ given by right multiplication.
	This module structure corresponds to a right action of~$\glie$ on~$\Univ(\glie)$ given by
	\[
		y \act x
		=
		yx
	\]
	for all~$x \in \glie$ and~$y \in \Univ(\glie)$.
	This right action corresponds to a left action given by
	\begin{equation}
		\label{technical action of uea}
		x \act y
		=
		-yx
	\end{equation}
	for all~$x \in \glie$ and~$y \in \Univ(\glie)$.
	We have in this way made~$\Univ(\glie)$ into a representation of~$\glie$.

	We can thus form the Lie algebra chain complex~$\Chain_\bullet(\glie, \Univ(\glie))$, which is given for every~$n \geq 0$ by the vector space
	\[
		\Chain_n(\glie, \Univ(\glie))
		=
		\Exterior^n(\glie) \otimes_{\kf} \Univ(\glie)
	\]
	and the differential
	\[
		\widetilde{d_n}
		\colon
		\Exterior^n(\glie) \otimes_{\kf} \Univ(\glie)
		\to
		\Exterior^{n-1}(\glie) \otimes_{\kf} \Univ(\glie)
	\]
	given by
	\begin{align*}
		\widetilde{d_n}( x_1 \wedge \dotsb \wedge x_n \otimes y )
		={}&
		\sum_{1 \leq i < j \leq n}
		(-1)^{i+j}
		[x_i, x_j] \wedge x_1 \wedge \dotsb \wedge \widehat{x_i} \wedge \dotsb \wedge \widehat{x_j} \wedge \dotsb \wedge x_n \otimes y
		\\
		{}&
		+
		\sum_{i=1}^n (-1)^i x_1 \wedge \dotsb \wedge \widehat{x_i} \wedge \dotsb \wedge x_n \otimes (x_i \act y)
	\end{align*}
	for all~$x_1, \dotsc x_n \in \glie$ and~$y \in \Univ(\glie)$.
	By using the isomorphism of vector spaces
	\[
		\Exterior^n(\glie) \otimes_{\kf} \Univ(\glie)
		\cong
		\Univ(\glie) \otimes_{\kf} \Exterior^n(\glie)
	\]
	and the formula~\eqref{technical action of uea} we arrive at a chain complex as claimed.

	It remains to check that the differential maps~$d_n$ are homomorphisms of~\modules{$\Univ(\glie)$}.
	This holds because
	\begin{align*}
		\SwapAboveDisplaySkip
		{}&
		d_n(x \cdot (y \otimes x_1 \wedge \dotsb \wedge x_n) )
		\\
		={}&
		d_n( (xy) \otimes x_1 \wedge \dotsb \wedge x_n )
		\\
		={}&
		\sum_{1 \leq i < j \leq n}
		(-1)^{i+j}
		(xy) \otimes [x_i, x_j] \wedge x_1 \wedge \dotsb \wedge \widehat{x_i} \wedge \dotsb \wedge \widehat{x_j} \wedge \dotsb \wedge x_n
		\\
		{}&
		+
		\sum_{i=1}^n
		(-1)^{i+1}
		(x y x_i) \otimes x_1 \wedge \dotsb \wedge \widehat{ x_i } \wedge \dotsb \wedge x_n
		\\
		={}&
		x
		\cdot
		\Biggl(
		\sum_{1 \leq i  j \leq n}
			(-1)^{i+j}
			y \otimes [x_i, x_j] \wedge x_1 \wedge \dotsb \wedge \widehat{x_i} \wedge \dotsb \wedge \widehat{x_j} \wedge \dotsb \wedge x_n
		\\
		{}&
		\phantom{
			x \cdot
			\Biggl(
		}
			+
			\sum_{i=1}^n
			(-1)^{i+1}
			(y x_i) \otimes x_1 \wedge \dotsb \wedge \widehat{ x_i } \wedge \dotsb \wedge x_n
		\Biggr)
		\\
		={}&
		x \cdot d_n(y \otimes x_1 \wedge \dotsb \wedge x_n)
	\end{align*}
	for all~$x \in \glie$ and~$y \in \Univ(\glie)$,~$x_1, \dotsc, x_n \in \glie$.
\end{proof}


\begin{fluff}
	We use in the following the identifications
	\[
		\Univ(\glie) \otimes_{\kf} \Exterior^0(\glie)
		\cong
		\Univ(\glie) \otimes_{\kf} \kf
		\cong
		\Univ(\glie)
	\]
	and
	\[
		\Univ(\glie) \otimes_{\kf} \Exterior^1(\glie)
		\cong
		\Univ(\glie) \otimes \glie \,.
	\]
	We regard the {\onedimensional} vector space~$\kf$ as the trivial representation of~$\glie$ and thus a~\module{$\Univ(\glie)$}.
\end{fluff}


\begin{theorem}
	\label{koszul is a free resolution}
	\leavevmode
	The sequence
	\begin{equation}
		\label{koszul complex written out}
		\dotsb
		\to
		\Univ(\glie) \otimes_{\kf} \Exterior^2(\glie)
		\xto{d_2}
		\Univ(\glie) \otimes_{\kf} \glie
		\xto{d_1}
		\Univ(\glie)
		\xto{\varepsilon}
		\kf
		\to
		0
	\end{equation}
	is a free resolution of~$\kf$ as an~\module{$\Univ(\glie)$}, where~$\varepsilon$ denotes the counit of~$\Univ(\glie)$.
\end{theorem}


\begin{definition}
	The free resolution of~$\kf$ as a~\module{$\Univ(\glie)$} from \cref{koszul is a free resolution} is the \defemph{Koszul resolution}\index{Koszul resolution}\index{resolution!Koszul}, or \defemph{Chevalley--Eilenberg-complex}\index{Chevalley--Eilenberg complex} of~$\glie$.%
	\footnote{
		Named after Jean-Louis Koszul (1921--2018), Claude Chevalley (1909--1984) and Samuel Eilenberg (1913--1998).
	}
\end{definition}


\begin{fluff}
	We will spend the rest of this \lcnamecref{koszul resolution} on proving \cref{koszul is a free resolution}.
	We have taken the following proof from \cite[IV.6]{knapp_lie_and_cohomology}.
	The reader who is familiar with spectral sequences may also want check out the proof given in \cite[Theorem~7.2.2]{weibel_homological_algebra}.

	The proof will proceed in four steps, with each step showing the \lcnamecref{koszul is a free resolution} for an increasingly larger class of Lie~algebras.
	We first prove \cref{koszul is a free resolution} under the assumption that the Lie~algebra~$\glie$ is both abelian and finite-dimensional.
	We then prove \cref{koszul is a free resolution} under the assumption that~$\glie$ is abelian, but possibly infinite-dimensional.
	Finally, we prove \cref{koszul is a free resolution} without restriction on~$\glie$.

	We begin the proof by making some general observations that affect each of the next three steps.
\end{fluff}


\begin{proof}[Proof of \cref{koszul is a free resolution}, general part]
	It follows from~\cref{construction of koszul differential} that~$d_{n-1} \circ d_n = 0$ for every~$n \geq 2$.
	The map~$d_1$ is given by
	\[
		d_1( y \otimes x )
		=
		yx
	\]
	for all~$y \in \Univ(\glie)$ an~$x \in \glie$.
	It follows from~\cref{augumentation ideal is spanned by monomials} that the image of~$d_1$ is the augumentation ideal of~$\Univ(\glie)$ and thus the kernel of~$\varepsilon$.
	This shows the exactness of~\eqref{koszul complex written out} at~$\Univ(\glie)$.
	The given sequence is moreover exact at~$\kf$ because the map~$d_0$ is surjective, since~$\kf$ is one-dimensional and the image element~$d_0(1)$ is nonzero.

	This shows that the given sequence is a chain complexe, which is moreover exact at~$\Univ(\glie)$ and at~$\kf$.
	We also know \cref{extension of scalars is free} that the terms~$\Univ(\glie) \otimes_{\kf} \Exterior^n(\glie)$ with~$n \geq 0$ are free~\modules{$\Univ(\glie)$}.

	The maps~$d_n$ for~$n \geq 1$ are homomorphism of~\modules{$\Univ(\glie)$} according to \cref{construction of koszul differential}.
	The counit~$\varepsilon$ is also a homomorphism of~\modules{$\Univ(\glie)$}.
	Indeed, every element~$x$ of~$\Univ(\glie)$ acts on~$\kf$ by multiplication with the scalar~$\varepsilon(x)$, whence
	\[
		\varepsilon(x \cdot y)
		=
		\varepsilon(x) \cdot \varepsilon(y)
		=
		x \cdot \varepsilon(y)
	\]
	for all~$x \in \glie$ and~$y \in \Univ(\glie)$.

	It remains to check the exactness of the given complex at~$\Univ(\glie) \otimes \Exterior^n(\glie)$ for every~$n \geq 1$.
\end{proof}


\begin{recall}
	\label{exactness in ses of chain complex}
	Let
	\[
		0 \to X_\bullet \to Y_\bullet \to Z_\bullet \to 0
	\]
	be a short exact sequence of chain complexes.
	If the chain complexes~$X_\bullet$ and~$Z_\bullet$ are both exact (i.e. they are exact sequences), then the chain complex~$Y_\bullet$ is also a exact.
	Indeed, in the resulting long exact homology sequence
	\begin{equation}
		\label{long exact sequence in homology}
		\dotsb
		\to
		\Homology_{n+1}(Z_\bullet)
		\to
		\Homology_n(X_\bullet)
		\to
		\Homology_n(Y_\bullet)
		\to
		\Homology_n(Z_\bullet)
		\to
		\Homology_{n-1}(X_\bullet)
		\to
		\dotsb
	\end{equation}
	both~$\Homology_n(X_\bullet)$ and~$\Homology_n(Z_\bullet)$ vanish for every~$n \in \Integer$ because both~$X_\bullet$ and~$Z_\bullet$ are exact.
	It now follows from the exactness of the sequence~\eqref{long exact sequence in homology} that also~$\Homology_n(Y_\bullet)$ vanishes for every~$n \in \Integer$.
\end{recall}


\begin{recall}
	Let~$X_\bullet$ be a chain complex and let~$Y_\bullet$ be a subcomplex of~$X_\bullet$.
	We can form the \defemph{quotient chain complex}~$X_\bullet / Y_\bullet$\glsadd{quotient chain complex}\index{quotient of chain complexes}.
	This chain complex is given by the vector spaces~$X_n / Y_n$ for every~$n \in \Integer$.
	The differentials of~$X_\bullet / Y_\bullet$ are given by the unique linear maps that make the diagrams
	\[
		\begin{tikzcd}
			X_n
			\arrow{d}
			\arrow{r}[above]{d_n}
			&
			X_{n-1}
			\arrow{d}
			\\
			Y_n
			\arrow{r}[above]{d_n}
			\arrow{d}
			&
			Y_{n-1}
			\arrow{d}
			\\
			X_n / Y_n
			\arrow[dashed]{r}
			&
			X_{n-1} / Y_{n-1}
		\end{tikzcd}
	\]
	commute for every~$n \in \Integer$.
	In other words, the differential of~$X_\bullet / Y_\bullet$ is given on representatives by the differential of~$X_\bullet$.
	The chain complexes~$X_\bullet$,~$Y_\bullet$ and~$X_\bullet / Y_\bullet$ fit into a short exact sequence
	\[
		0
		\to
		Y_\bullet
		\to
		X_\bullet
		\to
		X_\bullet / Y_\bullet
		\to
		0 \,.
	\]
\end{recall}


\begin{proof}[Proof of \cref{koszul is a free resolution}, first part]
	Suppose first that the Lie~algebra~$\glie$ is both abelian and finite-dimensional.
	The differential maps~$d_n$ are then given by
	\[
		d_n( y \otimes x_1 \wedge \dotsb \wedge x_n )
		=
		\sum_{i=1}^n
		(-1)^{i+1}
		(y x_i) \otimes x_1 \wedge \dotsb \wedge \widehat{x_i} \wedge \dotsb \wedge x_n
	\]
	for all~$n \geq 1$ and~$y \in \Univ(\glie)$,~$x_1, \dotsc, x_n \in \glie$ because~$\glie$ is abelian.

	Let~$u_1, \dotsc, u_N$ be a basis of~$\glie$.
	The universal enveloping algebra~$\Univ(\glie)$ isomorphic to the symmetric algebra~$\Symm(\glie)$ because~$\glie$ is abelian.
	The algebra~$\Univ(\glie)$ thus isomorphic to the polynomial algebra~$\kf[u_1, \dotsc, u_N]$.

	We show now the exactness of the given chain complex~\eqref{koszul complex written out} by induction over the dimension~$N$ of~$\glie$.
	For~$N = 0$ we have~$\Univ(\glie) = 0$ and is chain complex~\eqref{koszul complex written out} is given by
	\[
		\dotsb
		\to
		0
		\to
		0
		\to
		\kf
		\xto{\id}
		\kf
		\to
		0 \,.
	\]
	This sequence is exact, which proves the base case.

	For~$N \geq 1$ we use that we have already shown~\eqref{koszul complex written out} to be a chain complex.
	To porve that this chain complex is exact we consider for every number of variables~$k = 0, \dotsc, N$ the ideal
	\[
		J_k
		\defined
		\ideal{u_1, \dotsc, u_k}
	\]
	of~$\Univ(\glie)$, and the linear subspaces~$V(n,k)$ of~$\Univ(\glie) \otimes \Exterior^n(\glie)$ given by
	\[
		V(n,k)
		\defined
		\Univ(\glie) \otimes \Exterior^n(u_1, \dotsc, u_k)  \,,
	\]
	where we use the abbreviation~$\Exterior^n(u_1, \dotsc, u_k)$ for~$\Exterior^n( \gen{u_1, \dotsc, u_k}_{\kf} )$.
	We have
	\[
		d_n( V(n, k) )
		\subseteq
		V(n-1, k)
	\]
	for all~$n \geq 1$ and~$k = 0, \dotsc, N$.
	For every number of variables~$k = 0, \dotsc, N$ we have therefore a sequence~$X^{(k)}_\bullet$ given by
	\[
		\dotsb
		\to
		V(3,k)
		\xto{d_{3,k}}
		V(2,k)
		\xto{d_{2,k}}
		V(1,k)
		\xto{d_{1,k}}
		\Univ(\glie)
		\xto{\varepsilon_k}
		\Univ(\glie) / J_k
		\to
		0 \,,
	\]
	where~$\Univ(\glie) = V(0,k)$ and~$\varepsilon$ denotes the quotient homomorphism.
	We will show that this sequence~$X^{(k)}_\bullet$ is exact for all~$k = 0, \dotsc, N$.
	For~$k = N$ this then shows the exactness of~\eqref{koszul complex written out} because~$X^{(N)}_\bullet$ is precisely~\eqref{koszul complex written out}.

	We already know that~\eqref{koszul complex written out} is a chain complex.
	It follows for the sequence~$X^{(k)}_\bullet$ by restriction that~$d_{n-1,k} \circ d_{n,k} = 0$ for every~$n \geq 2$.
	Thi image of~$V(1,k) = \Univ(\glie) \otimes \gen{u_1, \dotsc, u_k}_{\kf}$ under~$d_{1,k}$ is precisely the ideal of~$\Univ(\glie)$ generated by~$u_1, \dotsc, u_k$, i.e. the ideal~$J_k$.
	But this is precisely the kernel of~$\varepsilon_k$.
	This shows that the sequence~$X^{(k)}_\bullet$ is exact at~$\Univ(\glie)$.
	We observe lastly that the projection~$\varepsilon$ is surjective, which shows the exactness of~$X^{(k)}_\bullet$ at~$\Univ(\glie) / J_k$.

	We have overall shown that~$X^{(k)}_\bullet$ is a chain complex that is exact at both~$\Univ(\glie)$ and~$\Univ(\glie)/ / J_k$.
	For the exactness of~$X^{(k)}_\bullet$ it hence sufficies to show the exactness of the subcomplex~$Y^{(k)}_\bullet$ given by
	\[
		\dotsb
		\to
		V(3,k)
		\xto{d_{3,k}}
		V(2,k)
		\xto{d_{2,k}}
		V(1,k)
		\xto{d_{1,k}}
		J_k
		\to
		0 \,.
	\]
	We show this exactness by induction over the number of variables~$k$.

	The complex~$Y^{(0)}_\bullet$ is the zero chain complex, which is exact.
	Suppose now that~$k \geq 1$ and that the chain complex~$Y^{(k-1)}_\bullet$ is exact by induction hypothesis.
	We note that the chain complex~$Y^{(k-1)}_\bullet$ is a subcomplex of~$Y^{(k)}_\bullet$.
	We will show in the following that the quotient complex~$Y^{(k)}_\bullet / Y^{(k-1)}_\bullet$ is exact.
	It then follows from~\cref{exactness in ses of chain complex} that the complex~$Y^{(k)}_\bullet$ is also exact.

	The quotient complex~$Y^{(k)}_\bullet / Y^{(k-1)}_\bullet$ is of the form
	\[
		\dotsb
		\to
		V(2,k) / V(2, k-1)
		\xto{\induced{d_{2,k}}}
		V(1,k) / V(1, k-1)
		\xto{\induced{d_{1,k}}}
		J_k / J_{k-1}
		\to
		0
	\]

	We have
	\begin{align*}
		V(n,k) / V(n, k-1)
		&=
		\Bigl( \Univ(\glie) \otimes \Exterior^n( u_1, \dotsc, u_k ) \Bigr)
		\mathbin{\big/}
		\Bigl( \Univ(\glie) \otimes \Exterior^n( u_1, \dotsc, u_{k-1} ) \Bigr)
		\\
		&\cong
		\Univ(\glie)
		\otimes
		\Bigl(
			\Exterior^n( u_1, \dotsc, u_k )
			\mathbin{\big/}
			\Exterior^n( u_1, \dotsc, u_{k-1} )
		\Bigr) \,.
	\end{align*}
	The exterior power~$\Exterior^n( u_1, \dotsc, u_k )$ has the simple wedges
	\[
		u_{i_1} \wedge \dotsb \wedge u_{i_n}
		\qquad
		\text{with~$1 \leq i_1 < \dotsb < i_n \leq k$}
	\]
	as a basis, and those simple wedges with~$1 \leq i_1 < \dotsb < i_n \leq k-1$ form a basis of the linear subspace~$\Exterior^n( u_1, \dotsc, u_{k-1} )$.
	The quotient~$\Exterior^n(u_1, \dotsc, u_k) \mathbin{/} \Exterior^n(u_1, \dotsc, u_{k-1})$ does therefore have the residue classes
	\[
		\class{ u_{i_1} \wedge \dotsb \wedge u_{i_{n-1}} \wedge u_k }
		\qquad
		\text{with~$1 \leq i_1 < \dotsb < i_{n-1} \leq k-1$}
	\]
	as a basis.
	We can therefore identify the quotient~$\Exterior^n(u_1, \dotsc, u_k) \mathbin{/} \Exterior^n(u_1, \dotsc, u_{k-1})$ with the exterior power~$\Exterior^{n-1}(u_1, \dotsc, u_{k-1})$ via the isomorphism
	\begin{align*}
		\Exterior^{n-1}(u_1, \dotsc, u_{k-1})
		&\to
		\Exterior^n(u_1, \dotsc, u_k)
		\mathbin{\big/}
		\Exterior^n(u_1, \dotsc, u_{k-1}) \,,
		\\
		t
		&\mapsto
		\class{ t \wedge u_k } \,.
	\end{align*}
	We thus have altogether
	\[
		V(n, k) / V(n, k-1)
		\cong
		V(n-1, k-1)
	\]
	for every~$n \geq 1$, and such an isomorphism~$f_n$ is given by
	\begin{align*}
		f_{n-1}
		\colon
		V(n-1, k-1)
		&\to
		V(n,k) / V(n,k-1) \,,
		\\
		y \otimes t
		&\mapsto
		\class{ y \otimes (t \wedge u_k) }
	\end{align*}
	for all~$y \in \Univ(\glie)$ and~$t \in \Exterior^{n-1}(u_1, \dotsc, u_{k-1})$.

	We note for the quotient~$J_k / J_{k-1}$ that the ideal~$J_k$ has as a basis all those monomials in the variables~$u_1, \dotsc, u_N$ that contain at least one of the variables~$u_1, \dotsc, u_k$, and that~$J_{k-1}$ has as a basis all those monomials that contain at least on of the variables~$u_1, \dotsc, u_{k-1}$.
	The quotient~$J_k / J_{k-1}$ does therefore have a basis given by all those monomials which contain the variable~$u_k$ but not the variables~$u_1, \dotsc, u_{k-1}$.
	We can thus identify the quotient~$J_k / J_{k-1}$ on the level of vector spaces with the ideal~$\ideal{u_k}_{\kf[u_k, \dotsc, u_N]}$.
	(We want to emphasize here that we take this ideal in~$\kf[u_k, \dotsc, u_N]$.)
	This ideal we can then identify as a vector space with~$\kf[u_k, \dotsc, u_N]$ via the multiplicaiton with~$u_k$.
	But the algebra~$\kf[u_k, \dotsc, u_N]$ is the quotient~$\Univ(\glie) / J_{k-1}$.
	We have thus overall constructed an isomorphism of vector spaces
	\[
		f_{-1}
		\colon
		\Univ(\glie) / J_{k-1}
		\to
		J_k / J_{k-1} \,,
		\quad
		\class{y}
		\mapsto
		\class{y u_k} \,.
	\]

	We have now the following diagram.
	\[
		\begin{tikzcd}[row sep = large]
			\dotsb
			\arrow{r}
			&
			V(2, k) / V(2, k-1)
			\arrow{r}[above]{\induced{d_{2,k}}}
			&
			V(1, k) / V(1, k-1)
			\arrow{r}[above]{\induced{d_{1,k}}}
			&
			J_k / J_{k-1}
			\arrow{r}
			&
			0
			\\
			\dotsb
			\arrow{r}
			&
			V(1,k-1)
			\arrow{u}[right]{f_1}
			\arrow{r}[above]{d_{1,k-1}}
			&
			V(0,k-1)
			\arrow{u}[right]{f_0}
			\arrow{r}[above]{\varepsilon_{k-1}}
			&
			\Univ(\glie) / J_{k-1}
			\arrow{u}[right]{f_{-1}}
			\arrow{r}
			&
			0
		\end{tikzcd}
	\]
	The lower row is the chain complex~$X^{(k-1)}_\bullet$, which is exact by induction hypothesis.
	The vertical arrows are isomorphisms.
	To conclude the exactness of the upper row it sufficies to show that this diagram commutes.
	We have one the one hand
	\begin{align*}
		{}&
		\induced{d_{n,k}}( f_{n-1}( y \otimes x_1 \wedge \dotsb \wedge x_{n-1} ) )
		\\
		={}&
		\induced{d_{n,k}}\bigl( \class{ y \otimes x_1 \wedge \dotsb \wedge x_{n-1} \wedge u_k} \bigr)
		\\
		={}&
		\sum_{i=1}^{n-1}
		(-1)^{i+1} \class{ (y x_i) \otimes x_1 \wedge \dotsb \wedge \widehat{x_i} \wedge \dotsb \wedge x_{n-1} \wedge u_k }
		+ (-1)^{n+1} \underbrace{ \class{ (y u_k) \otimes x_1 \wedge \dotsb \wedge x_{n-1} } }_{=0}
		\\
		={}&
		\sum_{i=1}^{n-1}
		(-1)^{i+1} \class{ (y x_i) \otimes x_1 \wedge \dotsb \wedge \widehat{x_i} \wedge \dotsb \wedge x_{n-1} \wedge u_k }
		\\
		={}&
		f_{n-2}
		\Biggl(
			\sum_{i=1}^{n-1}
			(-1)^{i+1}
			(y x_i) \otimes x_1 \wedge \dotsb \wedge \widehat{x_i} \wedge \dotsb \wedge x_{n-1}
		\Biggr)
		\\
		={}&
		f_{n-2}( d_{n-1,k-1}( y \otimes x_1 \wedge \dotsb \wedge x_{n-1} ) )
	\end{align*}
	for all~$y \in \Univ(\glie)$ and~$x_1, \dotsc, x_{n-1} \in \gen{u_1, \dotsc, u_{k-1}}$.
	This shows the commutativity of all squares except for the rightmost one.
	For the commutativity of this last square we calculate on the other hand that
	\[
		\induced{d_{1,k}}( f_0( y ) )
		=
		\induced{d_{1,k}}\bigl( \class{ y \otimes u_k } \bigr)
		=
		\class{ y u_k }
		=
		f_{-1}( \class{y} )
		=
		f_{-1}( \varepsilon_{k-1}( y ) )
	\]
	for every~$y \in \Univ(\glie)$.
\end{proof}


\begin{fluff}
	In the next step of the proof of \cref{koszul is a free resolution} we will allow~$\glie$ to be infinite-dimensional, albeit still abelian.
	We observe for this that every abelian Lie~algebra is a the directed union of all its finite-dimensional Lie~subalgebra, each of which is both abelian and finite-dimensional.
	We want to extend this observation about the Lie~subalgebras of~$\glie$ to the associated Koszul resolutions.
\end{fluff}


\begin{recall}
	Let~$X^{(i)}_\bullet$ with~$i \in I$ be a directed family of chain complexes.
	More explicitely, each~$X^{(i)}_\bullet$ is a chain complex, and for any two indices~$i$ and~$j$ in~$I$ there exists another index~$k$ in~$I$ such that both~$X^{(i)}_\bullet$ and~$X^{(j)}_\bullet$ are subcomplexes of~$X^{(k)}_\bullet$.
	\begin{enumerate}
		\item
			It follows for every integer~$n$ that the family of vector spaces~$X^{(i)}_n$ with~$i \in I$ is directed.
			The union~$X_n \defined \bigcup_{i \in I} X^{(i)}_n$ does therefore inherits the structure of a vector space from the~$X^{(i)}_n$, such that exach~$X_n^{(i)}$ is a linear subspace of~$X_n$.
			Moreover, the differentials
			\[
				d^{(i)}_n
				\colon
				X^{(i)}_n
				\to
				X^{(i)}_{n-1}
			\]
			with~$i \in I$
			assemble into a unique linear map~$d_n$ from~$Y_n$ to~$X_{n-1}$, such that the restriction of~$d_n$ to~$X^{(i)}_n$ is precisely~$d^{(i)}_n$ for every~$i \in I$.
			
			These linear maps~$d_n$ satisfy the condition~$d_{n-1} \circ d_n = 0$ for every~$n \in \Integer$.
			Indeed, there exists for every element~$x$ of~$X_n$ some index~$i$ such that~$x$ is contained in~$X^{(i)}_n$.
			Then~$d_n(x) = d^{(i)}_n(x)$, and this element is contained in~$X^{(i)}_{n-1}$.
			Thus
			\[
				d_{n-1}( d_n( x ) )
				=
				d_{n-1}( d^{(i)}_n( x ) )
				=
				d^{(i)}_{n-1}( d^{(i)}_n( x ) )
				=
				0 \,.
			\]

			We have constructed a chain complex~$X_{\bullet}$ such that each~$X^{(i)}_\bullet$ is a subcomplex of~$X_\bullet$.
			We denote this complex by~$\bigcup_{i \in I} X^{(i)}_\bullet$\glsadd{directed union of chain complexes}\index{directed union of chain complexes}.
		\item
			If each complex~$X^{(i)}_\bullet$ is exact then the complex~$X_\bullet$ is again exact.
			Indeed, let~$x$ be a~\cycle{$n$} of~$X_\bullet$.
			The element~$x$ is contained in~$X^{(i)}_\bullet$ for some index~$i$.
			Then
			\[
				0
				=
				d_n(x)
				=
				d^{(i)}_n(x) \,,
			\]
			whence the element~$x$ is an~\cycle{$n$} of~$X^{(i)}_\bullet$.
			It follows from the exactness of the chain complex~$X^{(i)}_\bullet$ that there exists an element~$y$ of~$X^{(i)}_{n+1}$ with~$x = d^{(i)}_{n+1}(y)$.
			We thus have
			\[
				x
				=
				d^{(i)}_{n+1}( y )
				=
				d_{n+1}(y) \,.
			\]
			This shows the exactness of~$X_\bullet$ it~$X_n$.
	\end{enumerate}
\end{recall}


\begin{proof}[Proof of \cref{koszul is a free resolution}, second part]
	Suppose now that~$\glie$ is an abelian Lie~algebra of arbitrary dimension.
	Let~$(u_\lambda)_{\lambda \in \Lambda}$ be a basis of~$\glie$.
	For every subset~$I$ of~$\Lambda$ let~$\glie_I$ be linear subspace of~$\glie$ with basis~$(u_i)_{i \in I}$.
	This subspace inherts from~$\glie$ the structure of an abelian Lie~algebra.
	Let~$X^{(I)}_\bullet$ denote the Koszul complex of~$\glie_I$.

	Let~$\mathcal{I}$ denote the set of finite subsets of~$\Lambda$.
	We have already seen that for every~$I \in \mathcal{I}$ the complex~$X^{(I)}_\bullet$ is exact.
	If~$I$ and~$J$ are two elements of~$\mathcal{I}$ then there exists a third element~$K$ of~$\mathcal{I}$ which contains both~$I$ and~$J$ (e.g.~$K = I \cup J$).
	Then the complexes~$X^{(I)}_\bullet$ and~$X^{(J)}_\bullet$ are both subcomplexes of~$X^{(K)}_\bullet$.
	This shows that the family of complexes~$X^{(I)}_\bullet$ with~$I \in \mathcal{I}$ is directed.

	The directed union~$\bigcup_{I \in \mathcal{I}} X^{(I)}_\bullet$ is thus again an exact complex.
	But this union is precisely the Koszul complex of~$\glie$.
\end{proof}


\begin{fluff}
	In the last step of the proof of \cref{koszul is a free resolution} we will allow the Lie~algebra~$\glie$ to be both non-abelian and of arbitrary dimension.
	But for this last step we will need some more preparation.
\end{fluff}


\begin{recall}
	\label{restricting decompositions}
	Let~$V$ be a vector space.
	Let~$U$ and~$W$ be two linear subspaces of~$V$ such that~$U$ is contained in~$W$.
	If~$C$ is a direct complement for~$U$ in~$V$, then~$C \cap W$ is a direct complement vor~$U$ in~$W$.
\end{recall}


\begin{lemma}
	\label{decomposition for restricted counit}
	Let~$\varepsilon$ be the counit of~$\Univ(\glie)$ and let~$\varepsilon_p$ be the restriction of~$\varepsilon$ to~$\Univ(\glie)_{(p)}$.
	Then
	\[
		\Univ(\glie)_{(p)}
		=
		\kf \oplus \ker(\varepsilon_p) \,.
	\]
\end{lemma}


\begin{proof}
	We have the decomposition~$\Univ(\glie) = \kf \oplus \ker(\varepsilon)$ by \cref{decomposition for augumented algebra}.
	The direct summand~$\kf$ is contained in~$\Univ(\glie)_{(p)}$.
	It therefore follows~\cref{restricting decompositions} that~$\Univ(\glie)_{(p)} = \kf \oplus \ker(\varepsilon_p)$ because~$\ker(\varepsilon_p) = \ker(\varepsilon) \cap \Univ(\glie)_{(p)}$.
\end{proof}


\begin{corollary}
	\label{kernel of restricted counit}
	Let~$\varepsilon$ be the counit of~$\glie$ and let~$\varepsilon_p$ be the restriction of~$\varepsilon$ to~$\Univ(\glie)_{(p)}$.
	The kernel of~$\varepsilon_{(p)}$ is spanned as a vector space by all those monomials~$x_1 \dotsm x_n$ with~$n = 1, \dotsc, p$ and~$x_1, \dotsc, x_n \in \glie$ for every~$p \geq 0$.
\end{corollary}


\begin{proof}
	Let~$\varepsilon_p$ be the restriction of~$\varepsilon$ to~$\Univ(\glie)_{(p)}$.
	We know that the vector space~$\Univ(\glie)_{(p)}$ is spanned by the monomials
	\[
		x_1 \dotsm x_n
		\qquad
		\text{with~$n = 0, \dotsc, p$ and~$x_1, \dotsc, x_n \in \glie$.}
	\]
	With respect to the decomposition~$\Univ(\glie)_{(p)} = \kf \oplus \ker(\varepsilon_p)$ from \cref{decomposition for restricted counit} we see that the single generator for~$n = 0$ is contained in the direct summand~$\kf$, while all other generators are contained in the direct summand~$\ker(\varepsilon)_p$.
	The assertion now follows.
\end{proof}


\begin{recall}
	Let~$X^{(i)}_\bullet$ with~$i \in I$ be a family of chain complexes.
	For every integer~$n$ let
	\[
		X_n \defined \bigoplus_{i \in I} X^{(i)}_n
	\]
	and let~$d_n \defined \bigoplus_{i \in I} d^{(i)}_n$.
	This defines a chain complex~$X_\bullet$ since we have
	\[
		d_{n-1} \circ d_n
		=
		\bigoplus_{i \in I} d^{(i)}_{n-1}
		\circ
		\bigoplus_{i \in I} d^{(i)}_n
		=
		\bigoplus_{i \in I} \Bigl( d^{(i)}_{n-1} \circ d^{(i)}_n \Bigr)
		=
		\bigoplus_{i \in I} 0
		=
		0
	\]
	for every~$n \in \Integer$.\glsadd{direct sum of chain complexes}
	For this chain complex~$X_\bullet$\index{direct sum of chain complexes} we have
	\[
		\Cycle_n( X_\bullet )
		=
		\ker\biggl( \bigoplus_{i \in I} d^{(i)}_n \biggr)
		=
		\bigoplus_{i \in I} \ker\Bigl( d^{(i)}_n \Bigr)
		=
		\bigoplus_{i \in I} \Cycle_n\Bigl( X^{(i)}_\bullet \Bigr)
	\]
	and similarly
	\[
		\Boundary_n( X_\bullet )
		=
		\im\biggl( \bigoplus_{i \in I} d^{(i)}_{n+1} \biggr)
		=
		\bigoplus_{i \in I} \im\Bigl( d^{(i)}_{n+1} \Bigr)
		=
		\bigoplus_{i \in I} \Boundary_n\Bigl( X^{(i)}_\bullet \Bigr) \,.
	\]
	It follows for the homology of~$X_\bullet$ that
	\begin{align*}
		\Homology_n( X_\bullet )
		&=
		\Cycle_n( X_\bullet ) / \Boundary_n( X_\bullet )
		\\
		&=
		\biggl(
			\bigoplus_{i \in I} \Cycle_n\Bigl( X^{(i)}_\bullet \Bigr)
		\biggr)
		\Big/
		\biggl(
			\bigoplus_{i \in I} \Boundary_n\Bigl( X^{(i)}_\bullet \Bigr)
		\biggr)
		\\
		&\cong
		\bigoplus_{i \in I}
		\bigl(
			\Cycle_n( X_\bullet ) / \Boundary_n( X_\bullet )
		\bigr)
		\\
		&=
		\bigoplus_{i \in I} \Homology_n( X_\bullet ) \,.
	\end{align*}
	It follows from this isomorphism that the chain complex~$X_\bullet$ is exact if and only if the chain complex~$X^{(i)}_\bullet$ is exact for every~$i \in I$.
\end{recall}


\begin{proof}[Proof of \cref{koszul is a free resolution}, third part]
	Suppose now that~$\glie$ is an arbitrary Lie~algebra.
	For every natural number~$p$ let
	\[
		V(n,p)
		\defined
		\Univ(\glie)_{(p-n)} \otimes \Exterior^n(\glie)
	\]
	for every~$n \geq 0$.
	The differentials of the Kozsul complex satisfy the condition
	\[
		d_n( V(n,p) )
		\subseteq
		V(n-1,p)
	\]
	for every~$n \geq 1$.
	The Koszul complex for~$\glie$ restricts therefore for every~$p \geq 0$ to a complex~$X^{(p)}_\bullet$ given by
	\[
		\dotsb
		\to
		V(3,p)
		\xto{d_{3,p}}
		V(2,p)
		\xto{d_{2,p}}
		V(1,p)
		\xto{d_{1,p}}
		\Univ(\glie)_{(p)}
		\xto{\varepsilon_p}
		\kf
		\to
		0 \,,
	\]
	with~$V(0,p) = \Univ(\glie)_{(p)}$.
	If~$p \leq q$, then~$X^{(p)}_\bullet$ is a subcomplex of~$X^{(q)}_\bullet$.
	It follows that the family of complexes~$X^{(p)}_\bullet$ with~$p \geq 0$ is directed, and the union~$\bigcup_{p \geq 0} X^{(p)}_\bullet$ is the Koszul complex of~$\glie$.
	(One may think about the complexes~$X^{(p)}_\bullet$ as a filtration of the Koszul complex.)
	We show in the following that the restricted complex~$X^{(p)}_\bullet$ is exact for every~$p \geq 0$.
	It then follows that the original Koszul complex is also exact, since it is the directed union~$\bigcup_{p \geq 0} X^{(p)}_\bullet$.

	We note that the restriction~$\varepsilon_p$ from~$\Univ(\glie)_{(p)}$ to~$\kf$ is surjective for very~$p \geq 0$.
	The complex~$X^{(p)}_\bullet$ is therefore exact at~$\kf$ for every~$p \geq 0$.
	We also know from \cref{kernel of restricted counit} that the kernel of~$\varepsilon_p$ is spanned as a vector space by all the monomials~$x_1 \dotsm x_n$ with~$n = 1, \dotsc, p$ and~$x_1, \dotsc, x_n \in \glie$.
	This kernel is therefore precisely the image of~$V(1,p) = \Univ(\glie)_{(p-1)} \otimes \glie$ under the differential~$d_{1,p}$, since the map~$d_{1,p}$ is given by
	\[
		d_{1,p}( y \otimes x )
		=
		y x
	\]
	for all~$y \in \Univ(\glie)_{(p-1)}$ and~$x \in \glie$.
	The complex~$X^{(p)}_\bullet$ is thus also exact at~$\Univ(\glie)_{(p)}$ for every~$p \geq 0$.

	To prove the exactness of~$X^{(p)}_\bullet$ it now sufficies to show the exactness of the its subcomplex~$Y^{(p)}_\bullet$ given by
	\[
		\dotsb
		\to
		V(3,p)
		\xto{d_{3,p}}
		V(2,p)
		\xto{d_{2,p}}
		V(1,p)
		\xto{d_{1,p}}
		\ker( \varepsilon_p )
		\to
		0 \,.
	\]
	We show this exactness by induction over~$p$.

	The chain complex~$Y^{(0)}_\bullet$ is the zero complex, which is exact.
	This takes care of the base case.
	Let now~$p \geq 1$ and suppose that~$Y^{(p-1)}_\bullet$ is exact by induction hypothesis.
	We note that the chain complex~$Y^{(p-1)}_\bullet$ is a subcomplex of~$Y^{(p)}_\bullet$.
	The chain complex~$Y^{(p-1)}_\bullet$ is exact by induction hypothesis.
	We show in the following that the quotient chain complex~$Y^{(p)}_\bullet / Y^{(p-1)}_\bullet$ is also exact.
	It then follows from \cref{exactness in ses of chain complex} that~$Y^{(p)}_\bullet$ is exact, as desired.

	The quotient~$Y^{(p)}_\bullet / Y^{(p-1)}_\bullet$ is given by
	\[
		\dotsb
		\to
		V(2,p) / V(2, p-1)
		\xto{\induced{d_{2,p}}}
		V(1,p) / V(1, p-1)
		\xto{\induced{d_{1,p}}}
		\ker(\varepsilon_p) / {\ker(\varepsilon_{p-1})}
		\to
		0 \,.
	\]
	The differentials of this chain complex are given by
	\begin{align*}
		{}&
		\induced{d_{n,p}}( \class{y \otimes x_1 \wedge \dotsb \wedge x_n} )
		\\
		={}&
		\sum_{1 \leq i < j \leq n}
		(-1)^{i+j}
		\,
		\underbrace{
			\class{
				y \otimes [x_i, x_j] \wedge x_1 \wedge \dotsb \wedge \widehat{x_i} \wedge \dotsb \wedge \widehat{x_j} \wedge \dotsb \wedge x_n
			}
		}_{=0}
		\\
		{}&
		+
		\sum_{i=1}^n
		(-1)^{i+1}
		\,
		\class{ (y x_i) \otimes x_1 \wedge \dotsb \wedge \widehat{x_i} \wedge \dotsb \wedge x_n }
		\\
		={}&
		\sum_{i=1}^n
		(-1)^{i+1}
		\,
		\class{ (y x_i) \otimes x_1 \wedge \dotsb \wedge \widehat{x_i} \wedge \dotsb \wedge x_n }
	\end{align*}
	for all~$y \in \Univ(\glie)_{(p-n)}$ and~$x_1, \dotsc, x_n \in \glie$.
	We used above that the term
	\[
		y \otimes [x_i,x_j] \wedge x_1 \wedge \dotsb \wedge \widehat{x_i} \wedge \dotsb \wedge \widehat{x_j} \wedge \dotsb \wedge x_n
	\]
	is contained in~$\Univ(\glie)_{(p-n)} \otimes \Exterior^{n-1}(\glie) = V(n-1,p-1)$ and thus vanishes in~$V(n-1,p) / V(n-1, p-1)$.
	We can also compute the terms of~$Y^{(p)}_\bullet / Y^{(p-1)}_\bullet$.
	Indeed, we have
	\begin{align*}
		V(n, p) / V(n, p-1)
		&=
		\Bigl(
			\Univ(\glie)_{(p-n)} \otimes \Exterior^n(\glie)
		\Bigr)
		\big/
		\Bigl(
			\Univ(\glie)_{(p-n-1)} \otimes \Exterior^n(\glie)
		\Bigr)
		\\
		&\cong
		\bigl(
			\Univ(\glie)_{(p-n)} / \Univ(\glie)_{(p-n-1)}
		\bigr)
		\otimes
		\Exterior^n(\glie)
		\\
		&=
		\gr[p-n]( \Univ(\glie) )
		\otimes
		\Exterior^n(\glie)
	\end{align*}
	for every~$n \geq 1$.
	Such an isomorphism of vector spaces is explicitely given by
	\begin{align*}
		f_n
		\colon
		V(n, p) / V(n, p-1)
		&\to
		\gr[p-n]( \Univ(\glie) )
		\otimes
		\Exterior^n(\glie) \,,
		\\
		\class{ y \otimes x_1 \wedge \dotsb \wedge x_n }
		&\mapsto
		\fclass{ y }_p \otimes x_1 \wedge \dotsb \wedge x_n
	\end{align*}
	for all~$y \in \Univ(\glie)_{(p-n)}$ and~$x_1, \dotsc, x_n \in \glie$.
	We also have
	\begin{align*}
		\ker(\varepsilon_p)
		/
		{\ker(\varepsilon_{p-1})}
		&\cong
		(\ker(\varepsilon_p) \oplus \kf)
		/
		(\ker(\varepsilon_{p-1}) \oplus \kf)
		\\
		&=
		\Univ(\glie)_{(p)} / \Univ(\glie)_{(p-1)}
		\\
		&=
		\gr[p]( \Univ(\glie) ) \,.
	\end{align*}
	Such an isomorphism of vector spaces is explicitely given by
	\[
		f_0
		\colon
		\ker(\varepsilon_p) / {\ker(\varepsilon_{p-1})}
		\to
		\gr[p]( \Univ(\glie) ) \,,
		\quad
		\class{y}
		\mapsto
		\fclass{ y }_p
	\]
	for every~$y \in \ker(\varepsilon_p)$.
	Under the above isomorphism~$\ker(\varepsilon_p) / \ker(\varepsilon_{p-1}) \cong \Univ(\glie)_{(p)} / \Univ(\glie)_{(p-1)}$, the formula for~$f_0$ is the same as for~$f_n$ with~$n \geq 1$.
	In other words, we have
	\[
		f_n( \class{ y \otimes x_1 \wedge \dotsb \wedge x_n } )
		=
		\fclass{ y }_{p-n} \otimes x_1 \wedge \dotsb \wedge x_n
	\]
	for all~$n \geq 0$ and~$y \in \Univ(\glie)_{(p-n)}$,~$x_1, \dotsc, x_n \in \glie$.

	We have now a commutative diagram of vector spaces
	\[
		\begin{tikzcd}
			\dotsb
			\arrow{r}
			&[-1em]
			V(2,p) / V(2, p-1)
			\arrow{r}[above]{\induced{d_{2,p}}}
			\arrow{d}[right]{f_2}
			&
			V(1,p) / V(1, p-1)
			\arrow{r}[above]{\induced{d_{1,p}}}
			\arrow{d}[right]{f_1}
			&
			\ker(\varepsilon_p) / {\ker(\varepsilon_{p-1})}
			\arrow{r}
			\arrow{d}[right]{f_0}
			&[-1em]
			0
			\\
			\dotsb
			\arrow{r}
			&
			\gr[p-2]( \Univ(\glie) ) \otimes \Exterior^2(\glie)
			\arrow{r}[above]{\del_{2,p}}
			&
			\gr[p-1]( \Univ(\glie) ) \otimes \glie
			\arrow{r}[above]{\del_{1,p}}
			&
			\gr[p]( \Univ(\glie) )
			\arrow{r}
			&
			0
		\end{tikzcd}
	\]
	for suitable linear maps~$\del_{n,p}$ with~$n \geq 1$.
	The vertical maps are isomorphism of vector spaces.
	The lower row is thus again a chain complex, and both rows are isomorphic as chain complexes.
	We denote the lower row by~$Z^{(p)}_\bullet$.
	We calculate the differentials~$\del_{n,p}$ of~$Z^{(p)}_\bullet$ as
	\begin{align*}
		{}&
		\del_{n,p}( \fclass{ y }_{p-n} \otimes x_1 \wedge \dotsb \wedge x_n )
		\\
		={}&
		\del_{n,p}\Bigl( f_n\Bigl( \class{ y \otimes x_1 \wedge \dotsb \wedge x_n } \Bigr) \Bigr) 
		\\
		={}&
		f_{n-1}\Bigl( \induced{d_{n,p}}\Bigl( \class{ y \otimes x_1 \wedge \dotsb \wedge x_n } \Bigr) \Bigr)
		\\
		={}&
		f_{n-1}
		\Biggl(
			\sum_{i=1}^n
			(-1)^{i+1}
			\,
			\class{ (y x_i) \otimes x_1 \wedge \dotsb \wedge \widehat{x_i} \wedge \dotsb \wedge x_n }
		\Biggr)
		\\
		={}&
		\sum_{i=1}^n
		(-1)^{i+1}
		\fclass{ y x_i }_{p-n+1} \otimes x_1 \wedge \dotsb \wedge \widehat{x_i} \wedge \dotsb \wedge x_n
	\end{align*}
	for all~$y \in \Univ(\glie)_{(p-n)}$ and~$x_1, \dotsc, x_n \in \glie$.
	This means in the space case of~$n = 1$ that
	\[
		\del_{1,p}( \fclass{ y }_{p-1} \otimes x )
		=
		\fclass{ yx }_p
	\]
	for all~$y \in \Univ(\glie)_{(p-1)}$ and~$x \in \glie$.
	To show the desired exactness of~$Y^{(p)}_\bullet / Y^{(p-1)}_\bullet$ we show the equivalent exactness of~$Z^{(p)}_\bullet$.

	Let~$\hlie$ be the abelian Lie~algebra that has the same underlying vector space as~$\glie$.
	According to the abstract version of the PBW theorem we may identify the associated graded algebra of~$\Univ(\glie)$ with the symmetric algebra~$\Symm(\glie)$ as graded algebras.
	Then
	\[
		\gr(\Univ(\glie))
		=
		\Symm(\glie)
		=
		\Symm(\hlie)
%		=
%		\Univ(\hlie)
	\]
	as graded algebra because~$\hlie$ is abelian, with~$\Symm(\hlie) = \Univ(\hlie)$.
	We have already seen that the Koszul complex~$K_\bullet$ of~$\hlie$, given by
	\[
		\dotsb
		\to
		\Symm(\hlie) \otimes \Exterior^2(\hlie)
		\xto{d_2}
		\Symm(\hlie) \otimes \hlie
		\xto{d_1}
		\Symm(\hlie)
		\xto{\varepsilon}
		\kf
		\to
		0 \,,
	\]
	is exact, where the differentials of this chain complex are given by
	\[
		d_n( y \otimes x_1 \wedge \dotsb \wedge x_n)
		=
		\sum_{i=1}^n
		(-1)^{i+1} (y x_i) \otimes x_1 \wedge \dotsb \wedge \widehat{x_i} \wedge \dotsb \wedge x_n \,.
	\]
	The grading~$\Symm(\hlie) = \bigoplus_{d \geq 0} \Symm^d(\hlie)$ induces for every~$n \geq 0$ a grading on~$\Symm(\hlie) \otimes \Exterior^n(\hlie)$ such that the degree~$q$ part of~$\Symm(\hlie) \otimes \Exterior^n(\hlie)$ is given by~$\Symm^{q-n}(\hlie) \otimes \Exterior^n(\hlie)$ (where we have~$S^d(\hlie) = 0$ whenever~$d < 0$).
	We also have a grading on~$\kf$ given by~$\kf_0 = \kf$ and~$\kf_q = 0$ for every~$q > 1$.
	The differentials of the Koszul complex respect these gradings, in these sense that we have for every~$q$ a subcomplex~$K^{(q)}_\bullet$ of~$K_\bullet$ given by
	\[
		\dotsb
		\to
		\Symm^{q-2}(\hlie) \otimes \Exterior^2(\hlie)
		\xto{d_{2,q}}
		\Symm^{q-1}(\hlie) \otimes \hlie
		\xto{d_{1,q}}
		\Symm^q(\hlie)
		\xto{\varepsilon_q}
		\kf_q
		\to
		0 \,,
	\]
	We have conversely~$K_\bullet = \bigoplus_{q \geq 0} K^{(q)}_\bullet$.
	The exactness of the Koszul complex~$K_\bullet$ shows that each subcomplex~$K^{(q)}_\bullet$ is again exact.
	We find for~$q = p$ that the exact chain complex~$K^{(p)}_\bullet$ is given by
	\[
		\dotsb
		\to
		\Symm^{p-2}(\hlie) \otimes \Exterior^2(\hlie)
		\xto{d_{2,p}}
		\Symm^{p-1}(\hlie) \otimes \hlie
		\xto{d_{1,p}}
		\Symm^p(\hlie)
		\xto{\varepsilon_p}
		0
		\to
		0 \,,
	\]
	because~$p \geq 0$ and thus~$\kf_p = 0$.

	We have now
	\[
		\gr[d](\Univ(\glie))
		=
		\gr(\Univ(\glie))_d
		=
		\Symm(\glie)_d
		=
		\Symm(\hlie)_d
		=
		\Symm^d(\hlie)
	\]
	for every~$d \geq 0$ and thus
	\[
		\gr[d]( \Univ(\glie) ) \otimes \Exterior^n(\glie)
		=
		\Symm^d( \hlie ) \otimes \Exterior^n(\hlie)
	\]
	for all~$d \geq 0$ and~$n \geq 0$.
	We have therefore the following diagram.
	\[
		\begin{tikzcd}
			\dotsb
			\arrow{r}
			&
			\gr[p-2]( \Univ(\glie) ) \otimes \Exterior^2(\glie)
			\arrow{r}[above]{\del_{2,p}}
			\arrow[equal]{d}
			&
			\gr[p-1]( \Univ(\glie) ) \otimes \glie
			\arrow{r}[above]{\del_{1,p}}
			\arrow[equal]{d}
			&
			\gr[p]( \Univ(\glie) )
			\arrow{r}
			\arrow[equal]{d}
			&
			0
			\\
			\dotsb
			\arrow{r}
			&
			\Symm^{p-2}( \Univ(\hlie) ) \otimes \Exterior^2(\hlie)
			\arrow{r}[above]{d_{2,p}}
			&
			\Symm^{p-1}( \Univ(\hlie) ) \otimes \hlie
			\arrow{r}[above]{d_{1,p}}
			&
			\Symm^p( \Univ(\hlie) )
			\arrow{r}
			&
			0
		\end{tikzcd}
	\]
	We know from the explicit formulas for~$\del_{n,p}$ and~$d_{n,p}$ that this diagram commutes.
	The lower row is the exact complex~$K^{(p)}_\bullet$.
	It thus follows that the upper row is also exact.
\end{proof}


\begin{fluff}
	We also have a Koszul resolution of right modules.
	For this we regard~$\Exterior(\glie) \otimes_{\kf} \Univ(\glie)$ as a right~\module{$\Univ(\glie)$} via extension of scalars, i.e. via
	\[
		(t \otimes y) \cdot x
		=
		t \otimes (yx)
	\]
	for all~$t \in \Exterior^n(\glie)$ and~$y, x \in \Univ(\glie)$.
\end{fluff}


\begin{corollary}[Koszul resolution for right modules]
	\index{Koszul resolution}\index{resolution!Koszul}
	\label{right koszul is a free resolution}
	\leavevmode
	\begin{enumerate}
		\item
			There exists for every~$n \geq 1$ a unique linear map
			\[
				\del_n
				\colon
				\Exterior^n(\glie) \otimes_{\kf} \Univ(\glie)
				\to
				\Exterior^{n-1}(\glie) \otimes_{\kf} \Univ(\glie)
			\]
			given by
			\begin{align*}
				\del_n( x_1 \wedge \dotsb \wedge x_n \otimes y )
				={}&
				\sum_{1 \leq i < j \leq n}
				(-1)^{i+j}
				[x_i, x_j] \wedge x_1 \wedge \dotsb \wedge \widehat{x_i} \wedge \dotsb \wedge \widehat{x_j} \wedge \dotsb \wedge x_n
				\otimes y
				\\
				{}&
				+
				\sum_{i=1}^n
				(-1)^i
				x_1 \wedge \dotsb \wedge \widehat{x_i} \wedge \dotsb \wedge x_n \otimes (x_i y)
			\end{align*}
			for all~$x_1, \dotsc, x_n \in \glie$ and~$y \in \Univ(\glie)$.
		\item
			The maps~$\del_n$ for~$n \geq 1$ are homomorphisms of right~\modules{$\Univ(\glie)$}.
		\item
			The sequence
			\[
				\dotsb
				\to
				\Exterior^2(\glie) \otimes \Univ(\glie)
				\xto{\del_2}
				\glie \otimes_{\kf} \Univ(\glie)
				\xto{\del_1}
				\Univ(\glie)
				\xto{\varepsilon}
				\kf
				\to
				0
			\]
			is a free resolution of~$\kf$ as a right~\module{$\Univ(\glie)$}, where~$\varepsilon$ denotes the counit of~$\Univ(\glie)$.
	\end{enumerate}
\end{corollary}


\begin{proof}
	The existence and uniqueness of the maps~$d_n$ can be shown as in \cref{construction of koszul differential}.
	That these maps are homomorphisms of~\modules{$\Univ(\glie)$} can also be shown as in \cref{construction of koszul differential}.

	To show the exactness of the given sequence we consider the Koszul complex of~$\glie$, given by
	\[
		\dotsb
		\to
		\Univ(\glie) \otimes \Exterior^2(\glie)
		\xto{d_2}
		\Univ(\glie) \otimes \glie
		\xto{d_1}
		\Univ(\glie)
		\xto{\varepsilon}
		\kf
		\to
		0 \,.
	\]
	Let~$S$ be the antipode of~$\Univ(\glie)$.
	We have for every~$n \geq 0$ an isomorphism of vector spaces
	\[
		f_n
		\coloneqq
		S \otimes \id
		\colon
		\Univ(\glie) \otimes \Exterior^n(\glie)
		\to
		\Univ(\glie) \otimes \Exterior^n(\glie) \,.
	\]
	We get from these isomorphisms a commutative diagram
	\[
		\begin{tikzcd}[row sep = large]
			\dotsb
			\arrow{r}
			&
			\Univ(\glie) \otimes \Exterior^2(\glie)
			\arrow{r}[above]{\del'_2}
			\arrow{d}[right]{f_2}
			&
			\Univ(\glie) \otimes \glie
			\arrow{r}[above]{\del'_1}
			\arrow{d}[right]{f_1}
			&
			\Univ(\glie)
			\arrow{r}[above]{\varepsilon'}
			\arrow{d}[right]{f_0}
			&
			\kf
			\arrow{r}
			\arrow[equal]{d}
			&
			0
			\\
			\dotsb
			\arrow{r}
			&
			\Univ(\glie) \otimes \Exterior^2(\glie)
			\arrow{r}[above]{d_2}
			&
			\Univ(\glie) \otimes \glie
			\arrow{r}[above]{d_1}
			&
			\Univ(\glie)
			\arrow{r}[above]{\varepsilon}
			&
			\kf
			\arrow{r}
			&
			0
		\end{tikzcd}
	\]
	for suitable linear maps~$\del'_n$ and~$\varepsilon'$.
	The lower row is exact and the vertical linear maps are isomorphisms, so the upper row is exact.
	We can explicitely calculate the maps in the upper row.
	We note that
	\[
		S(y) \cdot x
		=
		- S(y) \cdot (-x)
		=
		- S(y) \cdot S(x)
		=
		- S(x y)
	\]
	for all~$y \in \Univ(\glie)$ and~$x \in \glie$, and thus
	\begin{align*}
		{}&
		f_{n-1}( \del'_n( y \otimes x_1 \wedge \dotsb \wedge x_n ) )
		\\
		={}&
		d_n( f_n( y \otimes x_1 \wedge \dotsb \wedge x_n ) )
		\\
		={}&
		d_n( S(y) \otimes x_1 \wedge \dotsb \wedge x_n )
		\\
		={}&
		\sum_{1 \leq i < j \leq n}
		(-1)^{i+j}
		S(y) \otimes
		[x_i, x_j] \wedge
		x_1 \wedge \dotsb \wedge \widehat{x_i} \wedge \dotsb \wedge \widehat{x_j} \wedge \dotsb \wedge x_n
		\\
		{}&
		+
		\sum_{i=1}^n
		(-1)^{i+1}
		( S(y) \cdot x_i )
		\otimes
		x_1 \wedge \dotsb \wedge \widehat{x_i} \wedge \dotsb \wedge x_n
		\\
		={}&
		\sum_{1 \leq i < j \leq n}
		(-1)^{i+j}
		S(y) \otimes
		[x_i, x_j] \wedge
		x_1 \wedge \dotsb \wedge \widehat{x_i} \wedge \dotsb \wedge \widehat{x_j} \wedge \dotsb \wedge x_n
		\\
		{}&
		+
		\sum_{i=1}^n
		(-1)^i
		S(x_i y)
		\otimes
		x_1 \wedge \dotsb \wedge \widehat{x_i} \wedge \dotsb \wedge x_n
		\\
		={}&
		f_{n-1}
		\Biggl(
			\sum_{1 \leq i < j \leq n}
			(-1)^{i+j}
			y \otimes
			[x_i, x_j] \wedge
			x_1 \wedge \dotsb \wedge \widehat{x_i} \wedge \dotsb \wedge \widehat{x_j} \wedge \dotsb \wedge x_n
		\\
			{}&
			\phantom{ f_{n-1} \Biggl( }
			+
			\sum_{i=1}^n
			(-1)^i
			(x_i y)
			\otimes
			x_1 \wedge \dotsb \wedge \widehat{x_i} \wedge \dotsb \wedge x_n
		\Biggr)
	\end{align*}
	for all~$n \geq 1$ and~$y \in \Univ(\glie)$,~$x_1, \dotsc, x_n \in \glie$.
	This shows that
	\begin{align*}
		\del'_n( y \otimes x_1 \wedge \dotsb \wedge x_n )
		={}&
		\sum_{1 \leq i < j \leq n}
		(-1)^{i+j}
		y \otimes
		[x_i, x_j] \wedge
		x_1 \wedge \dotsb \wedge \widehat{x_i} \wedge \dotsb \wedge \widehat{x_j} \wedge \dotsb \wedge x_n
		\\
		{}&
		+
		\sum_{i=1}^n
		(-1)^i
		(x_i y)
		\otimes
		x_1 \wedge \dotsb \wedge \widehat{x_i} \wedge \dotsb \wedge x_n
	\end{align*}
	for all~$n \geq 1$ and~$y \in \Univ(\glie)$,~$x_1, \dotsc, x_n \in \glie$.
	The map~$\varepsilon'$ is the composite of the homomorphism of algebras~$\varepsilon$ and the anti-homomorphism of algebras~$f_0 = S$.
	It follows that~$\varepsilon'$ is an anti-homomorphism of algebras from~$\Univ(\glie)$ to~$\kf$.
	But the algebra~$\kf$ is commutative, whence~$\varepsilon'$ is a homomorphism of algebras.
	Moreover, we have
	\[
		\varepsilon'(x)
		=
		\varepsilon( S(x) )
		=
		\varepsilon( -x )
		=
		- \varepsilon(x)
		=
		0
	\]
	for every~$x \in \glie$.
	This shows that~$\varepsilon'$ is the antipode of~$\Univ(\glie)$, i.e. that~$\varepsilon' = \varepsilon$.
	
	It follows from these calculations that the diagram
	\[
		\begin{tikzcd}[row sep = large]
			\dotsb
			\arrow{r}
			&
			\Univ(\glie) \otimes \Exterior^2(\glie)
			\arrow{r}[above]{\del'_2}
			\arrow{d}[right]{t_2}
			&
			\Univ(\glie) \otimes \glie
			\arrow{r}[above]{\del'_1}
			\arrow{d}[right]{t_1}
			&
			\Univ(\glie)
			\arrow{r}[above]{\varepsilon'}
			\arrow{d}[right]{t_0}
			&
			\kf
			\arrow{r}
			\arrow[equal]{d}
			&
			0
			\\
			\dotsb
			\arrow{r}
			&
			\Exterior^2(\glie) \otimes \Univ(\glie)
			\arrow{r}[above]{\del_2}
			&
			\glie \otimes \Univ(\glie)
			\arrow{r}[above]{\del_1}
			&
			\Univ(\glie)
			\arrow{r}[above]{\varepsilon}
			&
			\kf
			\arrow{r}
			&
			0
		\end{tikzcd}
	\]
	commutes, where~$t_n$ denotes the twist map for all~$n \geq 0$.
	It follows from the exactness of the upper row that the lower row is also exact.
\end{proof}





