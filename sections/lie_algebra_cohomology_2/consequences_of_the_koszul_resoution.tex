\section{Consequences of the Koszul Resolution}


\begin{fluff}
	With help of the Koszul complex we can now give an alternative description of Lie algebra homology and Lie algebra cohomology.
\end{fluff}


\begin{construction}
	\label{lie algebra homology and cohomology via koszul complex}
	Let~$\glie$ be a Lie~algebra and let~$M$ be a representation of~$\glie$.
	\begin{enumerate}
		\item
			Let
			\[
				\dotsb
				\to
				\Exterior^2(\glie) \otimes_{\kf} \Univ(\glie)
				\xto{d_2}
				\glie \otimes_{\kf} \Univ(\glie)
				\xto{d_1}
				\Univ(\glie)
				\xto{\varepsilon}
				\kf
				\to
				0
			\]
			be the right Koszul resolution of~$\glie$.
			By removing the term~$\kf$ from this resolution we get the chain complex
			\begin{equation}
				\label{shortened right koszul resolution}
				\dotsb
				\to
				\Exterior^2(\glie) \otimes_{\kf} \Univ(\glie)
				\xto{d_2}
				\glie \otimes_{\kf} \Univ(\glie)
				\xto{d_1}
				\Univ(\glie)
				\to
				0 \,.
			\end{equation}
			We can then apply the functor~$(\ph) \otimes_{\Univ(\glie)} M$ to~\eqref{shortened right koszul resolution} to get a new chain complex given by
			\[
				\dotsb
				\to
				\Exterior^2(\glie) \otimes_{\kf} \Univ(\glie) \otimes_{\Univ(\glie)} M
				\xto{\! d_2 \otimes \id \!}
				\glie \otimes_{\kf} \Univ(\glie) \otimes_{\Univ(\glie)} M
				\xto{\! d_1 \otimes \id\! }
				\Univ(\glie) \otimes_{\Univ(\glie)} M
				\to
				0
			\]
			The differentials of this chain complex are given by
			\begin{align*}
				{}&
				(d_n \otimes \id)(x_1 \wedge \dotsb \wedge x_n \otimes y \otimes m)
				\\
				={}&
				\sum_{1 \leq i < j \leq n}
				(-1)^{i+j}
				[x_i, x_j] \wedge x_1 \wedge \dotsb \wedge \widehat{x_i} \wedge \dotsb \wedge \widehat{x_j} \wedge \dotsb \wedge x_n
				\otimes y \otimes m
				\\
				{}&
				+
				\sum_{i=1}^n
				(-1)^i
				x_1 \wedge \dotsb \wedge \widehat{x_i} \wedge \dotsb \wedge x_n \otimes (x_i y) \otimes m \,.
			\end{align*}
			We can simplify this chain complex by using the isomorphism
			\begin{equation}
				\label{isomorphism for triple tensor product}
				\Exterior^n(\glie) \otimes_{\kf} \Univ(\glie) \otimes_{\Univ(\glie)} M
				\cong
				\Exterior^n(\glie) \otimes_{\kf} M
			\end{equation}
			that is given by
			\[
				t \otimes y \otimes m
				\mapsto
				t \otimes (y \cdot m)
			\]
			for all~$t \in \Exterior^n(\glie)$,~$y \in \Univ(\glie)$,~$m \in M$.
			We thus arrive at a chain complex
			\[
				\dotsb
				\to
				\Exterior^2(\glie) \otimes_{\kf} M
				\xto{\del_2}
				\glie \otimes_{\kf} M
				\xto{\del_1}
				M
				\to
				0 \,.
			\]
			To compute the differentials for this chain complex we take the above formula for~$d_n \otimes \id$, apply the isomorphism~\eqref{isomorphism for triple tensor product}, and then set~$y  = 1$.
			We find that
			\begin{align*}
				{}&
				\del_n(x_1 \wedge \dotsb \wedge x_n \otimes m)
				\\
				={}&
				\sum_{1 \leq i < j \leq n}
				(-1)^{i+j}
				[x_i, x_j] \wedge x_1 \wedge \dotsb \wedge \widehat{x_i} \wedge \dotsb \wedge \widehat{x_j} \wedge \dotsb \wedge x_n
				\otimes m
				\\
				{}&
				+
				\sum_{i=1}^n
				(-1)^i
				x_1 \wedge \dotsb \wedge \widehat{x_i} \wedge \dotsb \wedge x_n \otimes (x_i \cdot m)
			\end{align*}
			for all~$m \in M$ and~$x_1, \dotsc, x_n \in \glie$.
			We have thus constructed the Lie algebra chain complex~$\Chain_\bullet(\glie, M)$\index{Lie algebra chain complex} from the Koszul resolution.
		\item
			Let
			\[
				\dotsb
				\to
				\Univ(\glie) \otimes_{\kf} \Exterior^2(\glie)
				\xto{d_2}
				\Univ(\glie) \otimes_{\kf} \glie
				\xto{d_1}
				\Univ(\glie)
				\xto{\varepsilon}
				\kf
				\to
				0
			\]
			be the Koszul resolution of~$\glie$.
			By removing the term~$\kf$ from this resolution we get the chain complex
			\begin{equation}
				\label{shortened koszul resolution}
				\dotsb
				\to
				\Univ(\glie) \otimes_{\kf} \Exterior^2(\glie)
				\xto{d_2}
				\Univ(\glie) \otimes_{\kf} \glie
				\xto{d_1}
				\Univ(\glie)
				\to
				0
			\end{equation}
			We can also apply the functor~$\Hom_{\Univ(\glie)}(-, M)$ to the chain complex~\eqref{shortened koszul resolution} to arrive at a cochain complex
			\begin{align*}
				\SwapAboveDisplaySkip
				0
				\to
				\Hom_{\Univ(\glie)}( \Univ(\glie), M )
				&\xto{d_1^*}
				\Hom_{\Univ(\glie)}( \Univ(\glie) \otimes_{\kf} \glie, M )
				\\
				&\xto{d_2^*}
				\Hom_{\Univ(\glie)}\Bigl( \Univ(\glie) \otimes_{\kf} \Exterior^2(\glie), M \Bigr)
				\\
				&\xto{d_3^*}
				\Hom_{\Univ(\glie)}\Bigl( \Univ(\glie) \otimes_{\kf} \Exterior^3(\glie), M \Bigr)
				\to
				\dotsb
			\end{align*}
			The differential of this cochain complex is given by
			\begin{align*}
				d_n^*(f)( y \otimes x_1 \wedge \dotsb \wedge x_n )
				={}&
				f( d_n(y \otimes x_1 \wedge \dotsb \wedge x_n) )
				\\
				={}&
				\sum_{1 \leq i < j \leq n}
				(-1)^{i+1}
				f
				(
					y \otimes
					[x_i, x_j] \wedge x_1 \wedge \dotsb \wedge \widehat{x_i} \wedge \dotsb \wedge \widehat{x_j} \wedge \dotsb \wedge x_n
				)
				\\
				{}&
				+
				\sum_{i=1}^n
				(-1)^{i+1}
				f( (y x_i) \otimes x_1 \wedge \dotsb \wedge \widehat{x_i} \wedge \dotsb \wedge x_n )
			\end{align*}
			for all~$f \in \Hom( \Univ(\glie) \otimes_{\kf} \Exterior^{n-1}(\glie), M )$ and~$y \in \Univ(\glie)$,~$x_1, \dotsc, x_n \in \glie$.
			We have by the universal property of the extension of scalars an isomorphism of vector spaces
			\[
				\Hom_{\Univ(\glie)}\Bigl( \Univ(\glie) \otimes \Exterior^n( \glie, M ) \Bigr)
				\cong
				\Hom_{\kf}\Bigl( \Exterior^n(\glie), M \Bigr)
			\]
			for every~$n \geq 0$, given by
			\begin{align*}
				\SwapAboveDisplaySkip
				f
				&\mapsto
				\bigl( t \mapsto f(1 \otimes t) \bigr) \,, \\
				\bigl( y \otimes t \mapsto y \cdot g(t) \bigr)
				&\mapsfrom
				g \,.
			\end{align*}
			The above cochain complex is thus isomorphic to the cochain complex given by
			\[
				0
				\to
				\Hom_{\kf}( \kf, M )
				\xto{\del_0}
				\Hom_{\kf}( \glie, M )
				\xto{\del_1}
				\Hom_{\kf}\Bigl( \Exterior^2(\glie), M \Bigr)
				\to
				\dotsb
			\]
			and differentials
			\begin{align*}
				\del_n(f)(x_1 \wedge \dotsb \wedge x_{n+1})
				={}&
				\sum_{1 \leq i < j \leq n+1}
				(-1)^{i+1}
				f
				(
					[x_i, x_j] \wedge
					x_1 \wedge \dotsb \wedge \widehat{x_i} \wedge \dotsb \wedge \widehat{x_j} \wedge \dotsb \wedge x_{n+1}
				)
				\\
				{}&
				+
				\sum_{i=1}^{n+1}
				(-1)^{i+1}
				x_i \cdot f( \otimes x_1 \wedge \dotsb \wedge \widehat{x_i} \wedge \dotsb \wedge x_{n+1} )
			\end{align*}
			for all~$f \in \Hom_{\kf}( \Exterior^n(\glie), M )$ and~$x_1, \dotsc, x_{n+1} \in \glie$.

			By further identifying for every~$n \geq 0$ the vector space~$\Hom_{\kf}( \Exterior^n(\glie), M )$ with~$\Alt^n(\glie, M)$ we arrive at the Lie algebra cochain complex~$\Chain^\bullet(\glie, M)$\index{Lie algebra cochain complex}.
	\end{enumerate}

	We have thus seen that the Lie~algebra chain complex~$\Chain_\bullet(\glie, M)$ and Lie~algebra cochain complex~$\Chain^\bullet(\glie, M)$ arise from the Koszul resolutions of~$\glie$ by applying the functors~$(\ph) \tensor_{\Univ(\glie)} M$ and~$\Hom_{\Univ(\glie)}(M, \ph)$.
\end{construction}


\begin{proposition}
	\label{invariant and coinvariants via hom and tensor}
	Let~$\glie$ be a Lie~algebra and let~$M$ be a representation of~$\glie$.
	\begin{enumerate}
		\item
			The map
			\[
				\Hom_{\Univ(\glie)}(\kf, M)
				\to
				M^{\glie} \,,
				\quad
				f
				\mapsto
				f(1)
			\]
			is a well-defined isomorphism of vector spaces.
		\item
			The map
			\[
				\kf \otimes_{\Univ(\glie)} M
				\to
				M_{\glie} \,,
				\quad
				1 \otimes m
				\mapsto
				\class{ m }
			\]
			is a well-defined isomorphism of vector spaces.
	\end{enumerate}
\end{proposition}


\begin{proof}
	\leavevmode
	\begin{enumerate}
		\item
			We have already seen this in \cref{invariants via internal hom}.
		\item
			The counit~$\varepsilon$ of~$\Univ(\glie)$ induces an isomorphism of~\module{$\Univ(\glie)$}
			\[
				\Univ(\glie) / I
				\to
				\kf \,,
				\quad
				\class{x}
				\mapsto
				\varepsilon(x)
			\]
			where~$I$ is the augumentation ideal of~$\Univ(\glie)$.
			It follows that
			\[
				\kf \otimes_{\Univ(\glie)} M
				\cong
				( \Univ(\glie) / I ) \otimes_{\Univ(\glie)} M
				\cong
				M / I M \,,
			\]
			and this isomorphism is given by
			\[
				1 \otimes m
				=
				\varepsilon(1) \otimes m
				\mapsto
				\class{1} \otimes m
				\mapsto
				\class{ 1 \cdot m }
				=
				\class{ m }
			\]
			for every~$m \in M$.
			The augumentation ideal~$I$ of~$\Univ(\glie)$ is generated by~$\glie$ as a left ideal of~$\Univ(\glie)$, whence
			\[
				M / I M
				=
				M / \ideal{\glie} M
				=
				M / \glie M
				=
				M_{\glie} \,.
			\]
			We have altogether shown the claimed isomorphism.
		\qedhere
	\end{enumerate}
\end{proof}


\begin{remark}
	\label{invariants and coinvariants via hom and tensor as functors}
	The isomorphisms from \cref{invariant and coinvariants via hom and tensor} are natural in~$M$ and thus assemble into isomorphisms of functors
	\[
		(\ph)^{\glie}
		\cong
		\Hom_{\Univ(\glie)}( \kf, - )
		\quad\text{and}\quad
		(\ph)_{\glie}
		\cong
		\kf \otimes_{\Univ(\glie)} (\ph) \,.
	\]
\end{remark}


\begin{fluff}
	Let~$\glie$ be a Lie~algebra.
	The Koszul complex of~$\glie$ is a free, and thus projective, resolution of~$\kf$ as a left~\module{$\Univ(\glie)$}.
	We can therefore use this resolution to compute the right derived functors~$\Ext_{\Univ(\glie)}^n(\kf, \ph)$\index{Ext@$\Ext$} with~$n \geq 0$.
	We can similarly use the right Koszul resolution to compute the left derived functors~$\cramped{\Tor^{\Univ(\glie)}_n(\kf, \ph)}$\index{Tor@$\Tor$} with~$n \geq 0$.
	We see from \cref{lie algebra homology and cohomology via koszul complex} that the result of these computations are precisely the Lie algebra cohomology\index{Lie algebra cohomology} of~$\glie$ and Lie algebra homology\index{Lie algebra homology} of~$\Univ(\glie)$.
	We thus have
	\[
		\Homology_n(\glie, \ph)
		\cong
		\Tor^{\Univ(\glie)}_n(\kf, \ph)
		\quad\text{and}\quad
		\Homology^n(\glie, \ph)
		\cong
		\Ext^n_{\Univ(\glie)}(\kf, \ph)
	\]
	for all~$n \geq 0$.
	In other words, the Lie algebra cohomology of~$\glie$ is given by the right derived functors of~$\Hom_{\glie}(\kf, \ph)$, and the Lie algebra homology of~$\glie$ is given by the left derived functors of~$\kf \otimes_{\Univ(\glie)} (\ph)$.
	It follows from \cref{invariants and coinvariants via hom and tensor as functors} that the Lie algebra cohomology of~$\glie$ is given by the right derived functors of~$(\ph)^{\glie}$, and the Lie~algebra homology of~$\glie$ is given by the left derived functors of~$(\ph)_{\glie}$.
\end{fluff}


\begin{proposition}
	Let~$\glie$ be a Lie~algebra and let~$I$ be the augumentation ideal\index{augumentation ideal} of~$\glie$.
	Let~$M$ be a representation of~$\glie$.
	The restriction map
	\[
		R
		\colon
		\Hom_{\Univ(\glie)}(I, M)
		\to
		\Der(\glie, M) \,,
		\index{derivation!of a representation}
		\quad
		d
		\mapsto
		\restrict{d}{\glie}
	\]
	is well-defined isomorphism of vector spaces.
\end{proposition}


\begin{proof}[First proof]
	We first note that the augumentation ideal~$I$ is a~\submodule{$\Univ(\glie)$} of~$\Univ(\glie)$.
	It therefore makes sense to consider~$\Hom_{\Univ(\glie)}(I, M)$.

	We know from the exactness of the Koszul complex that we have an exact sequence
	\[
		\Univ(\glie) \otimes \Exterior^2(\glie)
		\xto{d_2}
		\Univ(\glie) \otimes \glie
		\xto{d_1}
		I
		\to
		0 \,.
	\]
	This sequence is right-exact, so we may apply the contravariant~\functor{$\Hom$}~$\Hom_{\Univ(\glie)}(\ph, M)$ to get the left-exact sequence of vector spaces
	\[
		0
		\to
		\Hom_{\Univ(\glie)}(I, M)
		\xto{d_1^*}
		\Hom_{\Univ(\glie)}( \Univ(\glie) \otimes \glie, M )
		\xto{d_2^*}
		\Hom_{\Univ(\glie)}\Bigl( \Univ(\glie) \otimes \Exterior^2(\glie), M \Bigr) \,.
	\]
	We have by the universal property of the extension of scalars the linear {\onetoonetext} correspondence
	\begin{align*}
		\SwapAboveDisplaySkip
		\Hom_{\Univ(\glie)}( \Univ(\glie) \otimes \glie, M )
		&\onetoone
		\Hom_{\kf}( \glie, M ) \,,
		\\
		f
		&\mapsto
		\bigl( x \mapsto f(1 \otimes x) \bigr) \,,
		\\
		\bigl( \induced{g} \colon y \otimes x \mapsto y \cdot g(x) \bigr)
		&\mapsfrom
		g \,.
	\end{align*}
	Under this isomorphism, we have now the exact sequence
	\begin{equation}
		\label{left exact sequence for derivations}
		0
		\to
		\Hom_{\Univ(\glie)}(I, M)
		\xto{\del_1}
		\Hom_{\kf}( \glie, M )
		\xto{\del_2}
		\Hom_{\Univ(\glie)}\Bigl( \Univ(\glie) \otimes \Exterior^2(\glie), M \Bigr) \,.
	\end{equation}
	
	We have
	\[
		\del_1(f)(x)
		=
		d_1^*(f)(1 \otimes x)
		=
		f( d_1(1 \otimes x) )
		=
		f(1 \cdot x)
		=
		f(x) \,.
	\]
	In other words~$\del_1$ is precisely the restriction map~$R$.
	It follows from the left-exactness of the sequence~\eqref{left exact sequence for derivations} that the restriction map~$R$ identifies~$\Hom_{\Univ(\glie)}(I, M)$ with a certain linear subspace of~$\Hom_{\kf}(\glie, M)$, namely the kernel of~$\del_2$.

	We also have
	\begin{align*}
		\del_2(g)(y \otimes x_1 \wedge x_2)
		&=
		d_2^*(\induced{g})(y \otimes x_1 \wedge x_2)
		\\
		&=
		\induced{g}( d_2( y \otimes x_1 \wedge x_2 ) )
		\\
		&=
		\induced{g}
		(
			- y \otimes [x_1, x_2]
			+ (y x_1) \otimes x_2
			- (y x_2) \otimes x_1
		)
		\\
		&=
		- y \cdot g([x_1, x_2])
		+ (y x_1) \cdot g(x_2)
		+ (y x_2) \cdot g(x_1)
		\\
		&=
		- y \cdot g([x_1, x_2])
		+ y \cdot x_1 \cdot g(x_2)
		- y \cdot x_2 \cdot g(x_1)
		\\
		&=
		y \cdot
		(
			x_1 \cdot g(x_2) - x_2 \cdot g(x_1) - g([x_1, x_2])
		) 
	\end{align*}
	for all~$g \in \Hom_{\kf}(\glie, M)$ and~$y \in \Univ(\glie)$,~$x_1, x_2 \in \glie$.
	It follows that an element~$g$ of~$\Hom_{\kf}(\glie, M)$ is contained in the kernel of~$\del_2$ if and only if it satisfies the condition
	\[
		x_1 \cdot g(x_2) - x_2 \cdot g(x_1)
		=
		g([x_1, x_2])
	\]
	for all~$x_1, x_2 \in \glie$, i.e. if and only if~$g$ is a derivation from~$\glie$ to~$M$.
	We thus find that the kernel of~$\del_2$ is the space of derivations from~$\glie$ to~$M$.

	We have altogether shown that the restriction map~$R$ identifies the space of homomorphisms~$\Hom_{\Univ(\glie)}(I, M)$ with the space of derivations~$\Der(\glie, M)$.
\end{proof}


\begin{proof}[Second proof]
	We first note that the augumentation ideal~$I$ is a~\submodule{$\Univ(\glie)$} of~$\Univ(\glie)$.
	It therefore makes sense to consider~$\Hom_{\Univ(\glie)}(I, M)$.
	Now we calculate for every homomorphisms of~\modules{$\Univ(\glie)$}~$d$ from~$I$ to~$M$ that
	\[
		d([x,y])
		=
		d(xy - yx)
		=
		d(xy) - d(yx)
		=
		x \cdot d(y) - y \cdot d(x)
	\]
	for all~$x, y \in \glie$.
	The restriction~$\restrict{d}{\glie}$ is therefore a derivation from~$\glie$ to~$M$, which shows that the map~$R$ is well-defined.
	It is also linear.

	We begin by showing that every derivation~$\delta$ from~$\glie$ to~$M$ extends to a homomorphism of~\modules{$\Univ(\glie)$} from~$I$ to~$M$.
	We know from the exactness of the Koszul complex that we have an exact sequence
	\[
		\Univ(\glie) \otimes \Exterior^2(\glie)
		\xto{d_2}
		\Univ(\glie) \otimes \glie
		\xto{d_1}
		I
		\to
		0 \,.
	\]
	It follows from the universal property of the extension of scalars that the derivation~$\delta$ -- which is in particular a linear map -- extends to a homomorphism of~\modules{$\Univ(\glie)$}~$d'$ from~$\Univ(\glie) \otimes \glie$ to~$M$ given by
	\[
		d'(y \otimes x)
		=
		y \cdot \delta(x)
	\]
	for all~$y \in \Univ(\glie)$ and~$x \in \glie$.
	This homomorphism~$d'$ vanishes on the image of~$d_2$ because
	\begin{align*}
		d'( d_2(y \otimes x_1 \wedge x_2) )
		&=
		d'
		(
			- y \otimes [x_1, x_2]
			+ (y x_1) \otimes x_2
			- (y x_2) \otimes x_1
		)
		\\
		&=
		- y \cdot \delta([x_1, x_2])
		+ (y x_1) \cdot \delta(x_2)
		- (y x_2) \cdot \delta(x_1)
		\\
		&=
		- y \cdot \delta([x_1, x_2])
		+ y \cdot x_1 \cdot \delta(x_2)
		- y \cdot x_2 \cdot \delta(x_1)
		\\
		&=
		y \cdot
		(
			x_1 \cdot \delta(x_2)
			- x_2 \cdot \delta(x_1)
			- \delta([x_1, x_2])
		)
		\\
		&=
		y \cdot 0
		\\
		&=
		0 \,,
	\end{align*}
	where we have used that~$\delta([x_1, x_2]) = x_1 \cdot \delta(x_2) - x_2 \cdot \delta(x_1)$ because~$\delta$ is a derivation.
	It follows that~$d'$ factors through a homomorphism of~\modules{$\Univ(\glie)$} from~$I$ to~$M$.
	This homomorphism satisfies the equalities
	\[
		d(x)
		=
		d( d_1(1 \otimes x) )
		=
		d'( 1 \otimes x )
		=
		1 \cdot \delta(x)
		=
		\delta(x)
	\]
	for every~$x \in \glie$.
	It is therefore an extension of~$\delta$.
	We have thus shown the surjectivity of~$R$.

	It follows from \cref{augumentation ideal is spanned by monomials} that the augumentation ideal~$I$ is generated by~$\glie$ as an ideal, and thus as a~\module{$\Univ(\glie)$}.
	Every homomorphism of~\modules{$\Univ(\glie)$}~$d$ from~$I$ to~$M$ is therefore uniquely determined by its restriction~$\restrict{d}{\glie}$.
	This shows the injectivity of~$R$.
\end{proof}


\begin{recall}
	\label{homological algebra for pulling out an exact functor}
	Let~$\Acat$,~$\Bcat$ and~$\Ccat$ be three abelian categories such that the categories~$\Acat$ and~$\Bcat$ admit enough injectives.
	Let~$F$ be an exact functor from~$\Acat$ to~$\Bcat$ that maps injectives to injectives.
	Let~$G$ be a left exact functor from~$\Bcat$ to~$\Ccat$.
	Then the composite~$G \circ F$ is again left exact, and we have for every~$n \geq 0$ an isomorphism
	\[
		\Right^n (G \circ F)
		\cong
		(\Right^n G) \circ F \,.
		\glsadd{right derived functor}
		\index{right derived functor}
	\]
	
	Indeed, by precomposing the~\functor{$\delta$}~$\Right^\bullet G$ with the exact functor~$F$ yields a~\functor{$\delta$}, which one might denote by~$(\Right^\bullet G) \circ F$.
	Every injective object of~$\Acat$ is mapped by~$F$ to an injective object of~$\Bcat$, which is then annihilated by~$\Right^n G$ for every~$n \geq 1$.
	This shows that the functors~$(\Right^n G) \circ F$ for~$n \geq 1$ annihilate all injective objects of~$\Acat$.
	The~\functor{$\delta$}~$(\Right^\bullet G) \circ F$ is therefore universal.
	We also have~$(\Right^0 G) \circ F \cong G \circ F$ because~$\Right^0 G \cong G$.
	We thus find that the universal~\functor{$\delta$}~$(\Right^\bullet G) \circ F$ is the right derived functor of~$G \circ F$.
\end{recall}


\begin{proposition}
	Let~$\glie$ be a Lie~algebra and let~$M$,~$N$ be two representations of~$\glie$.
	Then
	\[
		\Ext_{\Univ(\glie)}^n(M, N)
		\cong
		\Homology^n(\glie, \Hom_{\kf}(M, N)) \,.
	\]
\end{proposition}


\begin{proof}
	The functor~$\Hom_{\kf}(M, \ph)$ exact.
	It also maps injectives to injectives because it is right adjoint to the exact functor~$(\ph) \otimes_{\kf} M$ by \cref{enriched tensor hom adjunction}.
	It follows from \cref{homological algebra for pulling out an exact functor} that
	\begin{align*}
		\Ext_{\Univ(\glie)}^n(M, \ph)
		&\cong
		\Right^n \Hom_{\Univ(\glie)}(M, \ph)
		\\
		&=
		\Right^n \Hom_{\kf}(M, \ph)^{\glie}
		\\
		&=
		\Right^n \bigl( (\ph)^{\glie} \circ \Hom_{\kf}(M, \ph) \bigr)
		\\
		&\cong
		\Right^n\bigl( (\ph)^{\glie} \bigr) \circ \Hom_{\kf}(M, \ph)
		\\
		&\cong
		\Homology^n(\glie, \ph) \circ \Hom_{\kf}(M, \ph)
		\\
		&=
		\Homology^n(\glie, \Hom_{\kf}(M, \ph)) \,,
	\end{align*}
	as claimed.
\end{proof}
