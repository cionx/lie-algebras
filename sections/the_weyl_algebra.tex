\section{Example: The Weyl Algebra}


%\begin{fluff}
%	To motivate the upcoming theorem of Poincaré--Birkhoff--Witt and its proof we consider first another example.
%\end{fluff}


\begin{definition}
	The \defemph{Weyl algebra} is the~\algebra{$\kf$}~$\kf\gen{X, Y} / \ideal{YX - XY - 1}$.%
	\footnote{The algebra~$A$ is isomorphic to the subalgebra of~$\End_{\kf}(\kf[t])$ generated by the endomorphisms~$x$ and~$y$ that are given by~$x(p(t)) = t p(t)$ and~$y(p(t)) = p'(t)$.
	These endomorphisms satsify the relations~$yx - xy - \id = 0$ by the product rule.}%
	\multiplefootnoteseparator%
	\footnote{Named after Hermann Weyl (1885--1955).}
\end{definition}


\begin{fluff}
	\label{filtration on the weyl algebra}
	Let~$A = \kf\gen{X,Y} / \ideal{YX - XY - 1}$ be the Weyl algebra.
	We denote the residue classes of~$X$ and~$Y$ in~$A$ by~$x$ and~$y$.
	Then~$yx = xy + 1$, which one may think about as a rewriting rule for words in~$x$ and~$y$.
	It follows by induction that
	\[
		y x^n
		=
		x^n y + n x^{n-1}
	\]
	for every~$n \geq 0$, and thus
	\[
		y p(x)
		=
		p(x) y + p'(x)
	\]
	for every polynomial~$p$ in~$\kf[t]$.
	
	The free~{\algebra{$\kf$}}~$T \defined \kf\gen{X, Y}$ carries a unique grading with~$X$ and~$Y$ in degree~$1$.
	The resulting filtration of~$T$ is given by
	\[
		T_{(p)}
		=
		\gen{
			Z_1 \dotsm Z_q
		\suchthat
			0 \leq q \leq p,
			Z_q \in \{X, Y\}
		}_{\kf}
		=
		\gen{
			\text{words in~$X$,~$Y$ of length~$\leq p$}
		}_{\kf}
	\]
	for every~$p \geq 0$.
	The Weyl~algebra~$A$ inherits a filtration from~$T$ given by
	\begin{align*}
		A_{(p)}
		=
		\gen{
			z_1 \dotsm z_q
		\suchthat
			1 \leq q \leq p,
			z_q \in \{x, y\}
		}_{\kf}
		=
		\gen{
			\text{words in~$x$,~$y$ of length~$\leq p$}
		}_{\kf}
	\end{align*}
	for every~$p \geq 0$.

	We may rewrite the relation~$yx = xy - 1$ as~$xy - yx = 1$.
	We can think about this relation as saying that the two elements~$x$ and~$y$ of~$A$ \enquote{commute up to smaller degree}.
	The elements~$\fclass{x}_1$ and~$\fclass{y}_1$ of~$\gr(A)$ therefore commute, because
	\[
		\fclass{x}_1 \fclass{y}_1
		=
		\fclass{xy}_2
		=
		\fclass{yx - 1}_2
		=
		\fclass{yx}_2 - \underbrace{ \fclass{1}_2 }_{=0}
		=
		\fclass{ yx }_2
		=
		\fclass{y}_1 \fclass{x}_1 \,.
	\]
	We will in the following show that~$\gr(A) \cong \kf[t,u]$ with the elements~$\fclass{x}_1$ and~$\fclass{y}_1$ corresponding to the commutative variables~$t$ and~$u$.
\end{fluff}


\begin{lemma}
	\label{subspace spanned by monomials}
	The linear subspace~$A_{(p)}$ of~$A$ is spanned by the monomial~$x^n y^m$ with~$n + m \leq p$, for every~$p \geq 0$.
\end{lemma}
	

\begin{proof}
	We show the statement by induction over~$p$.
	It holds for~$p = 0$ because
	\[
		A_{(0)}
		=
		\gen{1}_{\kf}
		=
		\gen{x^0 y^0}_{\kf} \,.
	\]
	Suppose now that the \lcnamecref{subspace spanned by monomials} holds for some~$p \geq 0$.
	We have
	\[
		T_{(p+1)}
		=
		x T_{(p)} \oplus y T_{(p)}
	\]
	and therefore
	\[
		A_{(p+1)}
		=
		x A_{(p)} + y A_{(p)} \,.
	\]
	We find by the induction hypothesis that the summand~$x A_{(p)}$ is spanned by all those monomials~$x^{n+1} y^m$ with~$n+m \leq p$ and that the summand~$y A_{(p)}$ is spanned by all those monomials~$y x^n y^m$ with~$n + m \leq p$.
	We have for these kind of monomials that
	\[
		y x^n y^m
		=
		(x^n y + n x^{n-1}) y^m
		=
		x^n y^{m+1} + n x^{n-1} y^m \,.
	\]
	Together this shows that~$A_{(p+1)}$ is spanned by all those monomials~$x^n y^m$ with~$n+m \leq p+1$.
\end{proof}
 
\begin{fluff}
	It follows from \cref{subspace spanned by monomials} for every natural number~$p$ that the homogeneous component~$\gr[p](A)$ of~$\gr(A)$ is spanned by all those residue classes~$\fclass{ x^n y^m }_p$ with~$p = n + m$.
	It further follows that the algebra~$\gr(A)$ is spanned by the monomials~$\fclass{ x^n y^m }_p$ with~$p \geq 0$ and~$n + m = p$ as a vector space.

	We will now see that the multiplication of these elementgenerators works in the naive way.
\end{fluff}
	
\begin{lemma}
	\label{generators commute for weyl}
	It holds for all~$p, q \geq 0$ and~$n, m, k, l \geq 0$ with~$p = n+m$ and~$q = k+l$ that
	\[
		\fclass{ x^n y^m }_p \cdot \fclass{ x^k y^l }_q
		=
		\fclass{ x^{n+k} y^{m+l} }_{p+q} \,.
	\]
\end{lemma}
	
\begin{proof}
	For the case~$p = 0$ we find that~$n = m = 0$ and hence
	\[
		\fclass{ x^n y^m }_p \cdot \fclass{ x^k y^l }_q
		=
		\fclass{ x^0 y^0 }_0 \cdot \fclass{ x^k y^l }_q
		=
		\fclass{ 1 }_0 \cdot \fclass{ x^k y^l }_q
		=
		\fclass{ x^k y^l }_q
		=
		\fclass{ x^{n+k} y^{m+l} }_{p+q} \,.
	\]
	For~$p = 1$ we have the two cases~$n = 1$ and~$m = 0$, as well as~$n = 0$ and~$m = 1$ to consider.
	We find in the first case that
	\[
		\fclass{ x^1 y^0 }_1 \cdot \fclass{ x^k y^l}_q
		=
		\fclass{x}_1 \cdot \fclass{ x^k y^l }_q
		=
		\fclass{x \cdot x^k y^l}_{q+1}
		=
		\fclass{ x^{k+1} y^l }_{q+1} \,.
	\]
	In the second case we find that
	\begin{align*}
		\fclass{ x^0 y^1 }_1 \cdot \fclass{ x^k y^l }_q
		&=
		\fclass{y}_1 \cdot \fclass{ x^k y^l }_q
		\\
		&=
		\fclass{ y x^k y^l }_{q+1}
		\\
		&=
		\fclass{ (x^k y + k x^{k-1}) y^l }_{q+1}
		\\
		&=
		\fclass{ x^k y^{l+1} + k x^{k-1} y^l }_{q+1}
		\\
		&=
		\fclass{ x^k y^{l+1} }_{q+1} + k \underbrace{ \fclass{ x^{k-1} y^l }{q+1} }_{=0}
		\\
		&=
		\fclass{ x^k y^{l+1} }_{q+1} \,,
	\end{align*}
	where we used that~$x y^k = y^k x + k y^{k-1}$ and that~$\fclass{ z }_{q+1} = 0$ for all~$z \in A_{(q)}$.
	It now follows for the general case that
	\begin{align*}
		\SwapAboveDisplaySkip
		{}&
		\fclass{ x^n y^m }_p \cdot \fclass{ x^k y^l }_q
		\\
		={}&
		\underbrace{ \fclass{x}_1 \dotsm \fclass{x}_1 }_{n}
		\underbrace{ \fclass{y}_1 \dotsm \fclass{y}_1 }_{m}
		\cdot
		\fclass{ x^k y^l }_q
		\\
		={}&
		\underbrace{ \fclass{x}_1 \dotsm \fclass{x}_1 }_{n}
		\underbrace{ \fclass{y}_1 \dotsm \fclass{y}_1 }_{m-1}
		\cdot
		\fclass{ x^k  y^{l+1} }_{1+q}
		\\
		={}&
		\dotsb
		\\
		={}&
		\fclass{ x^{n+k} y^{m+l} }_{p+q} \,,
	\end{align*}
	as claimed.
\end{proof}


\begin{fluff}
	To show the desired isomorphism~$\gr(A) \cong \kf[t,u]$ we are still missing one key ingredient, namely the linear independence of the monomials~$[x^n y^m]_p$ with~$p \geq 0$ and~$p = n+m$.
	This will follow from the next \lcnamecref{linear independence of monomials}
\end{fluff}

	
\begin{lemma}
	\label{linear independence of monomials}
	The monomials~$x^n y^m$ with~$n, m \geq 0$ in~$A$ are linearly independent.
\end{lemma}


\begin{fluff}
	To prove \cref{linear independence of monomials} we will use a general approach from representation theory, which we will now explain.
\end{fluff}


\begin{construction}
	\label{representation theory trick to construct a basis}
	Let~$A$ be a~\algebra{$\kf$} and let~$(b_j)_{j \in J}$ be a vector space generating set of~$A$, which we would like to show to be a basis of~$A$.

	Suppose that we are miraculously given an~\module{$A$}~$M$ together with an element~$m_0$ of~$M$ such that the family~$(b_j m_0)_{j \in J}$ is linearly independent in~$M$.
	It then follows that the original family~$(b_j)_{j \in J}$ is also linearly independent.
	Indeed, if we have a linear combination
	\[
		\sum_{j \in J} \lambda_j b_j = 0
	\]
	then by acting upon the element~$m_0$ it follows that
	\[
		\sum_{j \in J} \lambda_j b_j m_0 = 0
	\]
	and thus~$\lambda_j = 0$ for all~$j \in J$.

	We therefore would like to construct such an~\module{$A$}~$M$.
	We make for its construction a basic observation:
	if~$(b_j)_{j \in J}$ really turns out to be a basis of~$A$, then we could -- a posteriori -- have choosen~$M = A$ and~$m_0 = 1$.
	In other words, we could have chosen~$M$ as a vector space with a basis indexed by~$J$, say~$(B_j)_{j \in J}$, and could have choosen the action of~$A$ on these basis elements \enquote{in the same way} as the action of~$A$ on the basis elements~$b_j$ of~$A$.

	To construct the desired module~$M$ we therefore let~$M$ be a vector space with basis~$(B_j)_{j \in J}$.
	We want the algebra~$A$ to act on the vector space~$M$ in the same way as~$A$ acts on itself.
	How this action can be formally constructed depends on the given algebra~$A$ and family~$(b_j)_{j \in J}$.
	One possible way is as follows.

	Suppose that the algebra~$A$ is given by a family of algebra generators~$(x_i)_{i \in I}$ together with certain relations between these generators.
	The action of~$A$ on itself can be encoded as linear combinations
	\[
		x_i b_j = \sum_{k \in J} c_{ij}^k b_k
	\]
	with~$i \in I$ and~$j \in J$.
	We then want to define an action of~$A$ on the vector space~$M$ in such a way that~$x_i B_j = \sum_{k \in J} c_{ij}^k B_k$ for all~$i \in I$ and~$j \in J$.
	This can be done in two steps.
	\begin{enumerate}
		\item
			We first define a~\module{$\kf\gen{X_i \suchthat i \in I}$} structure on~$M$ via~$X_i \cdot B_j = \sum_{k \in I} c_{ij}^k B_k$ for all~$i \in I$ and~$j \in J$.
			We can do this because a~\module{$\kf\gen{X_i \suchthat i \in I}$} structure on~$M$ is the same a homomorphism of algebras from~$\kf\gen{X_i \suchthat i \in I}$ to~$\End_{\kf}(M)$, which can be constructed by using the universal property of the free algebra~$\kf\gen{X_i \suchthat i \in I}$.
		\item
			We then check that this action descends to an~\module{$A$} structure on~$M$ by checking that it is compatible with the given relations between the algebra generators~$x_i$.
	\end{enumerate} 
	Suppose now that~$1 = b_k$ for some index~$k$.
	We then set~$m_0 \defined B_k$.
	Then~$b_j \act m_0 = B_j$ for all~$j \in J$, and these elements are linearly independent in~$M$.
\end{construction}


\begin{fluff}
	The general approach from \cref{representation theory trick to construct a basis} can be best understood by considering a specific example, as we will now do.
\end{fluff}


\begin{proof}[Proof of  \cref{linear independence of monomials}]
	\label{linear independence for weyl algebra}
	Let~$M$ be the free vector space with basis~$T^n U^m$ where~$n, m \geq 0$.
	We have
	\begin{align*}
		x \cdot x^n y^m
		&=
		x^{n+1} y^m \,,
		\\
		y \cdot x^n y^m
		&=
		x^n y^{m+1} + n x^{n-1} y^m
	\end{align*}
	for all~$n, m \geq 0$ in~$A$.
	We therefore define a~{\module{$\kf\gen{X,Y}$}} structure on~$M$ by
	\begin{align*}
		X \cdot T^n U^m
		&=
		T^{n+1} U^m \,,
		\\
		Y \cdot T^n U^m
		&=
		T^n U^{m+1} + n T^{n-1} U^m
	\end{align*}
	for all~$n, m \geq 0$.
	We have
	\begin{gather*}
		YX \cdot T^n U^m
		=
		Y \cdot T^{n+1} U^m
		=
		T^{n+1} U^{m+1} + (n+1) T^n U^m
	\shortintertext{and}
		XY \cdot T^n U^m
		=
		X \cdot ( T^n U^{m+1} + n T^{n-1} U^m )
		=
		T^{n+1} U^{m+1} + n T^n U^m
	\end{gather*}
	for all~$n, m \geq 0$ and thus
	\[
		(YX - XY - 1) \cdot T^n U^m
		=
		0
	\]
	for all~$n, m \geq 0$.
	The~\module{$\kf\gen{X,Y}$} structure on~$M$ therefore descends to an~\module{$A$} structure on~$M$.
	We have
	\[
		y \cdot T^0 U^0
		=
		Y \cdot T^0 U^0
		=
		T^0 U^1
	\]
	and more generally inductively
	\[
		y^m \cdot T^0 U^0
		=
		T^0 U^m
	\]
	for all~$m \geq 0$.
	It further follows that
	\[
		x^n y^m \cdot T^0 U^0
		=
		x^n \cdot T^0 U^m
		=
		T^n U^m
	\]
	for all~$n, m \geq 0$.

	It now follows from the linear independence of the basis elements~$T^n U^m$ of~$M$ that the elements~$x^n y^m$ of~$A$ are also linearly independent.
\end{proof}


\begin{fluff}
	We can summarize our findings as follows.
\end{fluff}


\begin{theorem}[Structure of the Weyl algebra]
	\label{results about the weyl algerba}
	Let~$A = \kf\gen{X,Y} / \ideal{YX - XY - 1}$ be the Weyl algebra.
	Let~$x$ be the residue class of~$X$ in~$A$ and let~$y$ be the residue class of~$Y$ in~$A$.
	\begin{enumerate}
		\item
			The algebra~$A$ admits a filtration given by
			\[
				A_{(p)}
				=
				\gen{
					z_1 \dotsm z_q
					\suchthat
					q \leq p,
					z_i \in \{ x, y \}
				}_{\kf}
				=
				\gen{
					\text{words in~$x$,~$y$ of length~$\leq p$}
				}_{\kf}
			\]
			for every~$p \geq 0$.
		\item
			\label{basis for filtration of weyl algebra}
			The linear subspace~$A_{(p)}$ of~$A$ has for every~$p \geq 0$ the monomials~$x^n y^m$ with~$n, m \geq 0$ and~$n + m \leq p$ as a basis.
		\item
			The algebra~$A$ has the monomials~$x^n y^m$ with~$n, m \geq 0$ as a basis.
		\item
			\label{homogeneous basis for associated graded of weyl algebra}
			The homogeneous part~$\gr[p](A)$ has for every~$p \geq 0$ the monomials~$\fclass{ x^n y^m }_p$ with~$n, m \geq 0$ and~$p = n+m$ as a basis.
		\item
			\label{basis for associated graded of weyl algebra}
			The algebra~$\gr(A)$ has the monomials~$\fclass{ x^n y^m }_p$ with~$n, m \geq 0$ and~$p = n+m$ as a basis.
		\item
			\label{multiplication on basis of associated graded of weyl algebra}
			The multiplication of~$\gr(A)$ is given on these basis elements by
			\[
				\fclass{ x^n y^m }_p \cdot \fclass{ x^k y^l }_q
				=
				\fclass{ x^{n+k} y^{m+l} }_{p+q}
			\]
			for all~$n, m, k, l \geq 0$ and~$p = n + m$,~$q = k + l$.
		\item
			The algebra~$\gr(A)$ is isomorphic to the polynomial algebra~$\kf[t,u]$ as graded algebras, such that~$x$ corresponds to~$t$ and~$y$ corresponds to~$u$.
	\end{enumerate}
\end{theorem}


\begin{proof}
	\leavevmode
	\begin{enumerate}
		\item
			This follows from \cref{filtration on the weyl algebra}.
		\item
			This follows from \cref{subspace spanned by monomials} and \cref{linear independence of monomials}.
		\item
			This follows from part~\ref{basis for filtration of weyl algebra}.
		\item
			This follows from part~\ref{basis for filtration of weyl algebra}.
		\item
			This follows from the previous part.
		\item
			This is \cref{generators commute for weyl}. 
		\item
			It follows from part~\ref{basis for associated graded of weyl algebra} that there exists an isomorphism of vector spaces between~$\gr(A)$ and~$\kf[t,u]$ such that~$\fclass{ x^n y^m }_p$ corresponds to~$t^n u^m$ for all~$n, m \geq 0$ with~$p = n + m$.
			This is an isomorphism of algebras by part~\ref{multiplication on basis of associated graded of weyl algebra}, and finally an isomorphism of graded algebras by part~\ref{homogeneous basis for associated graded of weyl algebra}.
		\qedhere
	\end{enumerate}
\end{proof}


\begin{remark}
	The Weyl algebra is an example of a \defemph{skew polynomial ring}:
	Let~$R$ be a~\algebra{$\kf$} and let~$\delta$ be a derivation of~$R$.
	One can endow the polynomial vector space~$A \defined R[t]$ uniquely with the structure of a~\algebra{$\kf$} -- different from the usual one unless~$\delta = 0$ -- such that
	\begin{itemize*}
		\item
			the inclusion from~$R$ to~$A$ given by~$r \mapsto r t^0$ is a homomorphism of~\algebras{$\kf$},
		\item
			$t^i$ is the~\howmanyth{$i$} power of~$t$ in~$A$, and
		\item
			the variable~$t$ interacts with the elements~$r$ of~$R$ via~$tr = rt + \delta(r)$.
	\end{itemize*}
	The resulting~\algebra{$\kf$} is a \defemph{skew polynomial algebra} over~$R$, and this algebra is denoted by~$R[t;\delta]$.
	The multiplication of~$R[t;\delta]$ is built precisely so that the derivation~$[t,-]$ of~$R[t;\delta]$ acts on~$R t^0 \cong R$ by~$\delta$.
	More explicitely,
	\[
		[t, r t^0]
		=
		\delta(r) t^0
	\]
	for all~$r \in R$.
	The following two constructions are special cases of skew polynomial algebras.
	\begin{enumerate*}
		\item
			Let~$R$ be any~\algebra{$\kf$} and let~$\delta = 0$.
			Then~$R[t; \delta]$ is the usual polynomial algebra~$R[t]$.
		\item
			Let~$R$ be the polynomial algebra~$\kf[t]$ in a single variable~$t$ and let~$\partial_t$ is the partial derivative with respect to this variable.
			The resulting algebra~$R[t;\partial_t]$ is the Weyl~algebra.
	\end{enumerate*}

%   Currently not needed and distracting.
%
%   Skew polynomial algebras can be further generalized:
%   If~$\sigma \colon R \to R$ is an algebra homomorphism then a~{\linear{$\kf$}} map~$\delta \colon R \to R$ is a~{\derivation{$\sigma$}} if
%   \[
%     \delta(rs)
%     =
%     \delta(r) s + \sigma(r) \delta(s)
%   \]
%   for all~$r, s \in R$.
%   Then the polynomial vector space~$A = R[t]$ can be endowed uniquely with the structure of a~{\algebra{$\kf$}} such that
%   \begin{itemize}
%     \item
%       the inclusion~$R \to A$,~$r \mapsto r t^0$ is a homomorphism of~{\algebras{$\kf$}},
%     \item
%       $t^i$ is the~{\howmanyth{$i$}} power of~$t$ in~$A$ and
%     \item
%       the variable~$t$ interacts with the elements~$r \in R$ via~$t r = \sigma(r) t + \delta(r)$.
%   \end{itemize}
%   The resulting algebra~$A$ is an \defemph{Ore extension} of~$R$ and is denoted by~\gls*{ore extension}.
\end{remark}


\begin{remark}[Weyl and Heisenberg]
	The Weyl~algebra~$A$ is related to the {\threedimensional} Heisenberg~Lie~algebras~$\heisenberglie$ as follows.

	The Lie~algebra~$\heisenberglie$ has a basis~$p$,~$q$,~$c$ on which the Lie~bracket is given by
	\[
		[p,c] = 0 \,,
		\quad
		[q,c] = 0 \,,
		\quad
		[p,q] = c \,.
	\]
	The Weyl~algebra~$A$ results from the universal enveloping algebra~$\Univ(\heisenberglie)$ by~\enquote{setting~$c$ equal to~$1$}.
	More precisely, we have that
	\[
		\Univ(\heisenberglie) / \ideal{ \class{c} - 1 }
		\cong
		A
	\]
	with the elements~$\class{p}$ and~$\class{q}$ of~$\Univ(\heisenberglie)$ corresponding to the elements~$y$ and~$x$ of~$A$.

	To formally prove this we abbreviate~$\Univ(\heisenberglie)/(\class{c} - 1)$ as~$B$.	
	Let~$\varphi$ be the unique linear map from~$\heisenberglie$ to~$A$ given by
	\[
		\varphi(p) = y \,,
		\quad
		\varphi(q) = x \,,
		\quad
		\varphi(c)  =1 \,.
	\]
	This map is a homomorphism of Lie~algebras and thus induces a homomorphism of algebras~$\Phi$ from~$\Univ(\heisenberglie)$ to~$A$.
	This homomorphism~$\Phi'$ is given on the algebra generators~$\class{p}$,~$\class{q}$,~$\class{c}$ of~$\Univ(\heisenberglie)$ by
	\[
		\Phi'( \class{p} ) = y \,,
		\quad
		\Phi'( \class{q} ) = x \,,
		\quad
		\Phi'( \class{c} ) = 1 \,.
	\]
	It holds that
	\[
		\Phi'(\class{c} - 1)
		=
		\Phi'(\class{c}) - \Phi'(1)
		=
		1 - 1
		=
		0 \,,
	\]
	whence~$\Phi'$ descends to a well-defined algebra homomorphism~$\Phi$ from~$B$ to~$A$.
	The homomorphism~$\Phi$ is given on the remaining algebra generators~$\class{p}$ and~$\class{q}$ of~$B$ given by
	\[
		\Phi(\class{p}) = y \,,
		\quad
		\Phi(\class{q}) = x \,.
	\]
	
	To construct the inverse homomorphism~$\Psi$ of~$\Phi$ we can start off with the unique algebra homomorphism~$\Psi'$ from the free~\algebra{$\kf$}~$\kf\gen{X,Y}$ to~$B$ that is given on the variables~$X$ and~$Y$ by
	\[
		\Psi'(X) = \class{q} \,,
		\quad
		\Psi'(Y) = \class{p} \,.
	\]
	Then
	\[
		\Psi'(YX)
		=
		\class{p} \cdot \class{q}
		=
		\class{q} \cdot \class{p}
		+
		[ \class{p}, \class{q} ]
		=
		\class{q} \cdot \class{p} + \class{[p,q]}
		=
		\class{q} \cdot \class{p} + \class{c}
		=
		\class{q} \cdot \class{p} + 1
		=
		\Psi'(XY + 1)
	\]
	and therefore~$\Psi'(YX - XY - 1) = 0$.
	It follows that the homomorphism~$\Psi'$ factors uniquely through a homomorphism of algebras~$\Psi$ from~$A$ to~$B$ that is given on the algebra generators~$x$ and~$y$ of~$A$ by
	\[
		\Psi(x) = \class{q} \,,
		\quad
		\Psi(y) = \class{p} \,.
	\]
	
	The two homomorphisms~$\Phi$ are~$\Psi$ are mutually inverse on the algebra generators~$y$,~$x$ of~$A$ and the algebra generators~$\class{p}$,~$\class{q}$ of~$B$.
	This shows that~$\Phi$ and~$\Psi$ are mutually inverse isomorphisms.
\end{remark}


\begin{remark}
	The approach from \cref{representation theory trick to construct a basis} can be generalized to other algebraic structures, for example to groups.
	\begin{enumerate}
		\item
			Let~$X$ be a set and let~$F$ be the free group on~$X$.
			Every element of~$F$ can be uniquely written as a reduced word in the alphabet~$X^{\pm 1}$.
			The uniqueness of such a reduced expression can be shown by letting~$F$ act on the set of all reduced words.
		\item
			Let~$G_i$ with~$i \in I$ be a collection of groups and let~$H$ be another group.
			For every index~$i$ in~$I$ let~$f_i$ be a homomorphism of groups from~$H$ to~$G_i$.
			Then elements of the resulting amalgamated product~$A = \coprod_{H,i \in I} G_i$ can be represented in certain a canonical form, the uniqueness of which can be shown by letting~$A$ act on the set of all possible canonical forms.
			We refer to \cite[\S 1.2]{serre_trees} for more information on this example.
	\end{enumerate}
\end{remark}


% Bad example: Already graded
%
% \begin{example}[Quantum plane]
%   Let~$q \in \kf$ wit~$q \neq 0$.
%   The \emph{quantum plane} is the~{\algebra{$\kf$}}~$A \defined \kf\gen{X,Y}/(YX - qXY)$.%
%   \footnote{The usual polynomial ring~$\kf[x,y] \cong \kf\gen{X,Y}/(YX - XY)$ is the algebraic object corresponding to the plane~$\Aff^2 = k^2$ in the sense that~$\kf[x,y]$ is the algebra of (regular) functions on~$\Aff^2$.
%   The given algebra~$A$ should therefore correspond to some (non-existing) plane, the functions on which don’t commute.}
%   Let~$x, y \in A$ be the residue classes of~$X$ and~$Y$.
%   The~{\algebra{$\kf$}}~$A$ inherits from the standard grading of the free algebra~$\kf\gen{X,Y}$ a filtration given by
%   \[
%     A_{(p)}
%     =
%     \gen{
%       z_1 \dotsm z_q
%     \suchthat
%       0 \leq j \leq i,
%       z_1, \dotsc, z_q \in \{x, y\}
%     }_{\kf} \,.
%   \]
%   It can be shown as for the Weyl~algebra (actually easer than for the Weyl~algebra) for all~$i \geq 0$ the vector space~$A_{(p)}$ is spanned by the monomials~$x^n y^m$ with~$n + m \leq i$.
%   The quotient vector space~$\gr[p](A)$ is therefore spanned for every~$i \geq 0$ by the residue classes~$[x^n y^m]_p$ with~$n + m = i$.
%   The algebra~$\gr(A)$ is hence spanned by the collection of all such residue classes~$[x^n y^m]$ with~$n, m \geq 0$.
% \end{example}


\begin{warning}
	Let~$A$ be a filtered~\algebra{$\kf$}.
	There does not exist a \enquote{canonical} algebra homomorphism from~$A$ to~$\gr(A)$.
	There does not even have to exist any homomorphism of algebras from~$A$ to~$\gr(A)$.

	As an example we consider for~$A$ the Weyl algebra with the filtration from \cref{results about the weyl algerba}.
	The associated graded algebra~$\gr(A)$ is isomorphic to the polynomial algebra~$\kf[t,u]$, and it is therefore in particular commutative.
	The homomorphisms of algebras from~$A$ to~$\gr(A)$ correspond bijectively to pairs~$(x, y)$ consisting of elements of $\gr(A)$ which satisfy the commutator relation~$yx - xy - 1 = 0$.
	But no such pair exists because the algebra~$\gr(A)$ is commutative.
	There hence exist no homomorphism of algebras from~$A$ to~$\gr(A)$.
\end{warning}





