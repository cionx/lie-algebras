\section{Definition and Construction of the UEA}





\subsection{Definition and Functoriality}


\begin{definition}
  A \defemph{universal enveloping algebra}\index{universal enveloping algebra} of a Lie~algebra~$\glie$ is a~\algebra{$\kf$}~\gls*{universal enveloping algebra} together with a homomorphism of Lie~algebras~$\iota \colon \glie \to \Univ(\glie)$ such that for every other~{\algebra{$\kf$}}~$A$ and homomorphism of Lie~algebras~$\phi \colon \glie \to A$ there exists a unique homomorphism of~\algebras{$\kf$}~$\Phi \colon \Univ(\glie) \to A$ that makes the triangular diagram
  \[
    \begin{tikzcd}
      \glie
      \arrow{r}[above]{\phi}
      \arrow{d}[left]{\iota}
      &
      A
      \\
      \Univ(\glie)
      \arrow[dashed]{ur}[below right]{\Phi}
      &
      {}
    \end{tikzcd}
  \]
  commute, i.e.\ such that~$\phi = \Phi \circ \iota$.
\end{definition}


\begin{remark}[Uniqueness of universal enveloping algebras]
  \label{uniqueness of universal enveloping algebras}
  Suppose that a Lie~algebra~$\glie$ admits two {\uas}~$(\Univ(\glie)_1, \iota_1)$ and~$(\Univ(\glie)_2, \iota_2)$.
  Then there exists unique algebra homomorphisms~$\phi \colon \Univ(\glie)_1 \to \Univ(\glie)_2$ and~$\psi \colon \Univ(\glie)_2 \to \Univ(\glie)_1$ that make the triangular diagrams
  \[
    \begin{tikzcd}[column sep = small]
      {}
      &
      \glie
      \arrow{dl}[above left]{\iota_1}
      \arrow{dr}[above right]{\iota_2}
      &
      {}
      \\
      \Univ(\glie)_1
      \arrow[dashed]{rr}[below]{\phi}
      &
      {}
      &
      \Univ(\glie)_2
    \end{tikzcd}
    \qquad\text{and}\qquad
    \begin{tikzcd}[column sep = small]
      {}
      &
      \glie
      \arrow{dl}[above left]{\iota_1}
      \arrow{dr}[above right]{\iota_2}
      &
      {}
      \\
      \Univ(\glie)_2
      \arrow[dashed]{rr}[below]{\phi}
      &
      {}
      &
      \Univ(\glie)_1
    \end{tikzcd}
  \]
  commute.
  It follows that the compositions~$\phi \circ \psi \colon \Univ(\glie)_1 \to \Univ(\glie)_1$ and~$\psi \circ \phi \colon \Univ(\glie)_2 \to \Univ(\glie)_2$ make the triangular diagrams
  \[
    \begin{tikzcd}[column sep = small]
      {}
      &
      \glie
      \arrow{dl}[above left]{\iota_1}
      \arrow{dr}[above right]{\iota_1}
      &
      {}
      \\
      \Univ(\glie)_1
      \arrow[dashed]{rr}[below]{\psi \circ \phi}
      &
      {}
      &
      \Univ(\glie)_1
    \end{tikzcd}
    \qquad\text{and}\qquad
    \begin{tikzcd}[column sep = small]
      {}
      &
      \glie
      \arrow{dl}[above left]{\iota_2}
      \arrow{dr}[above right]{\iota_2}
      &
      {}
      \\
      \Univ(\glie)_2
      \arrow[dashed]{rr}[below]{\phi \circ \psi}
      &
      {}
      &
      \Univ(\glie)_2
    \end{tikzcd}
  \]
  commutes.
  The algebra homomorphisms~$\phi \circ \psi$ and~$\psi \circ \phi$ are unique with this property by the universal property of the {\uas}~$(\Univ(\glie)_1, \iota_1)$ and~$(\Univ(\glie)_2, \iota_2)$.
  But the identities~$\id_{\Univ(\glie)_1}$ and~$\id_{\Univ(\glie)_2}$ also makes these diagrams commute.
  We thus find that~$\psi \circ \phi = \id_{\Univ(\glie)_1}$ and~$\phi \circ \psi = \id_{\Univ(\glie)_2}$, so that~$\phi$ and~$\psi$ are mutually inverse isomorphisms of~{\algebras{$\kf$}}.
  
  This shows that the {\ua} (if it exists) is \enquote{unique up to unique isomorphisms}.
  We will therefore talk about \emph{the} {\ua} of~$\glie$.
  We will often also surpress the algebra homorphism~$\iota \colon \glie \to \Univ(\glie)$ from our notation.
\end{remark}


% \begin{remark}
%   One can also formulate the above argument is a more categorical way:
%   Consider the category~$\catC$ where
%   \begin{itemize}
%     \item
%       objects of~$\catC$ is a pairs~$(A, i)$ consisting of a~{\algebra{$\kf$}}~$A$ and a Lie~algebra homomorphism~$i \colon \glie \to A$,
%     \item
%       a morphism~$\phi \colon (A, i) \to (B, j)$ is an algebra homomorphism~$\phi \colon A \to B$ that makes the triangular diagram
%       \[
%         \begin{tikzcd}[column sep = small]
%         {}
%         &
%         \glie
%         \arrow{dl}[above left]{i}
%         \arrow{dr}[above right]{j}
%         &
%         {}
%         \\
%         A
%         \arrow[dashed]{rr}[below]{\phi}
%         &
%         {}
%         &
%         B
%       \end{tikzcd}
%     \]
%       commute, and
%     \item
%       the composition of two morphisms is just their usual set-theoretic composition.
%   \end{itemize}
%   A {\ua} of~$\glie$ is nothing but an inital object in this category~$\catC$.
%   The argumentation from \cref{uniqueness of universal enveloping algebras} is then the usual argument for the uniqueness of inital objects up to unique isomorphism.
% \end{remark}


\begin{proposition}
  \label{representations are modules}
  Let~$V$ be a~{\vectorspace{$\kf$}}.
  Let~$\glie$ be a Lie~algebra and let~$\iota \colon \glie \to \Univ(\glie)$ be the canonical Lie~algebra homomorphism.
  Then the assignments
  \begin{align*}
    \left\{
    \begin{tabular}{@{}c@{}}
      representations of~$\glie$, \\
      $\rho \colon \glie \to \gllie(V)$
    \end{tabular}
    \right\}
    &\onetoone
    \left\{
    \begin{tabular}{@{}c@{}}
      $\Univ(\glie)$-module structures \\
      $\theta \colon \Univ(\glie) \to \End_{\kf}(V)$
    \end{tabular}
    \right\}  \,,
    \\
    \rho
    &\mapsto
    \hat{\rho} \,,
    \\
    \theta \circ \iota
    &\mapsfrom
    \theta  \,,
  \end{align*}
  constitute a {\onetoonetext} correspondence,~where $\hat{\rho} \colon \Univ(\glie) \to \End_{\kf}(V)$ is the unique~\algebra{$\kf$} homomorphism induced by the homomorphism of Lie~algebras~$\rho \colon \glie \to \gllie(V)$ via the universal property of the~{\ua}~$\Univ(\glie)$.
  \qed
\end{proposition}


\begin{remark}
  \leavevmode
  \begin{enumerate}
    \item
      \Cref{representations are modules} shows that representations of~$\glie$ are the same as~{\modules{$\Univ(\glie)$}}.
      The categories~$\cRep{\glie}$ and~$\cMod{\Univ(\glie)}$ are hence isomorphic.
    \item
      A representation~$V$ of~$\glie$ is completely reducible if and only if it is semisimple as an~{\module{$\Univ(\glie)$}}.
  \end{enumerate}
\end{remark}


\begin{lemma}[Functoriality of the universal enveloping algebra]
  \label{functoriality of universal enveloping algebra}
  Let~$\glie$,~$\hlie$ and~$\klie$ be Lie~algebras.
  \begin{enumerate}
    \item
      For every homomorphism of Lie~algebras~$\phi \colon \glie \to \hlie$ there exists a unique induced homomorphism of~\algebras{$\kf$}~$\phi^* \colon \Univ(\glie) \to \Univ(\hlie)$ that makes the following square diagram commute:
      \[
        \begin{tikzcd}[column sep = large]
          \glie
          \arrow{r}[above]{\phi}
          \arrow{d}
          &
          \hlie
          \arrow{d}
          \\
          \Univ(\glie)
          \arrow[dashed]{r}[below]{\phi_*}
          &
          \Univ(\hlie)
        \end{tikzcd}
      \]
    \item
      It holds that~$(\id_{\glie})_* = \id_{\Univ(\glie)}$.
    \item
      It holds for all composable homomorphisms of Lie~algebras~$\phi \colon \glie \to \hlie$ and~$\psi \colon \hlie \to \klie$ that
      \[
        (\psi \circ \phi)_*
        =
        \psi_* \circ \phi_* \,.
      \]
  \end{enumerate}
\end{lemma}


\begin{proof}
  \leavevmode
  \begin{enumerate}
    \item
      This follows from the universal property of the universal enveloping algebra~$\Univ(\glie)$ by applying it to the composition~$\glie \to \hlie \to \Univ(\hlie)$.
    \item
      The square diagram
      \[
        \begin{tikzcd}[column sep = huge]
          \glie
          \arrow{r}[above]{\id_{\glie}}
          \arrow{d}
          &
          \glie
          \arrow{d}
          \\
          \Univ(\glie)
          \arrow[dashed]{r}[below]{\id_{\Univ(\glie)}}
          &
          \Univ(\glie)
        \end{tikzcd}
      \]
      commutes, which shows that~$\id_{\Univ(\glie)}$ satisfies the defining property of the induced algebra homomorphism~$(\id_{\glie})_*$.
    \item
      We have the following commutative diagram:
      \[
        \begin{tikzcd}[column sep = large]
          \glie
          \arrow[dashed, bend left = 40]{rr}[above]{\psi \circ \phi}
          \arrow{r}[above]{\phi}
          \arrow{d}
          &
          \hlie
          \arrow{r}[above]{\psi}
          \arrow{d}
          &
          \klie
          \arrow{d}
          \\
          \Univ(\glie)
          \arrow{r}[below]{\phi_*}
          \arrow[dashed, bend right = 40]{rr}[below]{\psi_* \circ \phi_*}
          &
          \Univ(\hlie)
          \arrow{r}[below]{\psi_*}
          &
          \Univ(\klie)
        \end{tikzcd}
      \]
      The commutativity of the outer square diagram
      \[
        \begin{tikzcd}[column sep = huge]
          \glie
          \arrow{r}[above]{\psi \circ \phi}
          \arrow{d}
          &
          \klie
          \arrow{d}
          \\
          \Univ(\glie)
          \arrow[dashed]{r}[below]{\psi_* \circ \phi_*}
          &
          \Univ(\klie)
        \end{tikzcd}
      \]
      shows that~$\psi_* \circ \phi_*$ satisfies the defining property of the induced algebra homomorphism~$(\psi \circ \phi)_*$.
    \qedhere
  \end{enumerate}
\end{proof}





\subsection{Construction of the Universal Enveloping Algebra}


\begin{remark}
  \Cref{functoriality of universal enveloping algebra} shows that the assignment~$\glie \mapsto \Univ(\glie)$ of a Lie~algebra~$\glie$ to its universal eveloping algebra~$\Univ(\glie)$ can be extended to a (covariant) functor~$\Univ \colon \cLie{\kf} \to \cAlg{\kf}$.
  The universal property of the {\ua} states that the functor~$\Univ$ is left adjoint to the forgetful functor~$\cAlg{\kf} \to \cLie{\kf}$ that assigns to each~{\algebra{$\kf$}} its underlying Lie~algebra.
\end{remark}


\begin{remark}
  Let~$\glie$ be a Lie~algebra.
  It follows from the universal propery of the {\ua} that~$\Univ(\glie)$ is generated by the image of the canonical homomorphism~$\iota \colon \glie \to \Univ(\glie)$:
  
  Indeed, let~$U$ be the subalgebra of~$\Univ(\glie)$ that is generated by the image of~$\iota$, and let~$i \colon \glie \to U$ be the restriction of~$\iota$.
  Then for every~{\algebra{$\kf$}}~$A$ and every Lie~algebra homomorphism~$\phi \colon \glie \to A$ the induced algebra homomorphism~$\Phi' \colon \Univ(\glie) \to A$ restricts to an algebra homomorphism~$\Phi \colon U \to A$ that makes the triangular diagram
  \[
    \begin{tikzcd}
      \glie
      \arrow{r}[above]{\phi}
      \arrow{d}[left]{j}
      &
      A
      \\
      U
      \arrow[dashed]{ur}[below right]{\Phi}
      &
      {}
    \end{tikzcd}
  \]
  commute.
  The homomorphism~$\Phi$ is unique with this property because it is uniquely determined by the restriction~$\Phi \circ j = \phi$, since~$U$ is generated by the image of~$j$.
  This shows that~$(U, j)$ is again a {\ua} for~$\glie$.
  
  It follows from the uniqueness of the {\ua}, as discussed in \cref{uniqueness of universal enveloping algebras}, that the unique algebra homomorphism~$i \colon U \to \Univ(\glie)$ that makes the triangular diagram
  \[
    \begin{tikzcd}[column sep = small]
      {}
      &
      \glie
      \arrow{dl}[above left]{j}
      \arrow{dr}[above right]{\iota}
      &
      {}
      \\
      U
      \arrow[dashed]{rr}[below]{i}
      &
      {}
      &
      \Univ(\glie)
    \end{tikzcd}
  \]
  commute is already an isomorphism.
  This homomorphism is necessarily the inclusion of~$U$ into~$\Univ(\glie)$, and that it is an isomorphism means that already~$U = \Univ(\glie)$.
  
  The canonical homomorphism~$\iota \colon \glie \to \Univ(\glie)$ is in particular~{\linear{$\kf$}} and hence induces an algebra homomorphism~$\phi \colon \Tensor(\glie) \to \Univ(\glie)$ that makes the triangular diagram
  \[
    \begin{tikzcd}[column sep = small]
      {}
      &
      \glie
      \arrow{dl}
      \arrow{dr}[above right]{\iota}
      &
      {}
      \\
      \Tensor(\glie)
      \arrow[dashed]{rr}[below]{\phi}
      &
      {}
      &
      \Univ(\glie)
    \end{tikzcd}
  \]
  commute.
  That~$\Univ(\glie)$ is generated by the image of~$\iota$ means that~$\phi$ is surjective.
  Therefore~$\phi$ induces an algebra isomorphism
  \[
    \Phi
    \colon
    \Tensor(\glie)/I
    \to
    \Univ(\glie)
  \]
  for~$I = \ker \phi$, that makes the resulting diagram
   \[
    \begin{tikzcd}[column sep = small]
      {}
      &
      \glie
      \arrow[bend right]{dl}
      \arrow[bend left]{ddr}[above right]{\iota}
      &
      {}
      \\
      \Tensor(\glie)
      \arrow[bend left]{drr}[below left]{\phi}
      \arrow{d}
      &
      {}
      &
      {}
      \\
      \Tensor(\glie)/I
      \arrow[dashed]{rr}[below]{\Phi}
      &
      {}
      &
      \Univ(\glie)
    \end{tikzcd}
  \]
  commute.
  
  If~$A$ is any other~{\algebra{$\kf$}} then we know on the one hand that algebra homorphisms~$\Univ(\glie) \to A$ correspond to Lie~algebra homomorphisms~$\glie \to A$.
  We know find on the other hand that~{\algebra{$\kf$}} homomorphisms~$\Tensor(V)/I \to A$ correspond to algebra homomorphisms~$\Tensor(V) \to A$ that annihilate~$I$, with the algebra homorphisms~$\Tensor(V) \to A$ corresponding to~{\linear{$\kf$}} maps~$\glie \to A$.
  A linear map~$f \colon \glie \to A$ is a homorphism of Lie~algebras if and only if
  \[
      f(x) f(y)
    - f(y) f(x)
    - f([x,y])
    =
    0 \,,
  \]
  so it seems reasonable to assume that the corresponding ideal~$I$ of~$\Tensor(V)$ ought to be given by
  \[
    I
    =
    (x \tensor y - y \tensor x - [x,y] \suchthat x, y \in \glie)
  \]
  We will now show that this is indeed the case.
\end{remark}


\begin{proposition}[Existence of the universal enveloping algebra]
  Let~$\glie$ be a Lie~algebra.
  Let~$\Tensor(\glie)$ be the tensor algebra of the underlying vector space of~$\glie$ and let~$I$ the two-sided ideal in~$\Tensor(\glie)$ generated by the elements $x \tensor y - y \tensor x - [x,y]$ with~$x,y \in \glie$.
  The the quotient algebra~$U \defined T(\glie)/I$ together with the~{\linear{$\kf$}} map
  \[
    i
    \colon
    \glie
    \to
    \Univ(\glie) \,,
    \quad
    x
    \mapsto
    \class{x}
  \]
  is a {\ua} for~$\glie$.
\end{proposition}


\begin{proof}
  The map~$i$ is~{\linear{$\kf$}} and it compatible with the Lie brackets because
  \[
    [i(x), i(y)]
    =
    [\class{x}, \class{y}]
    =
    \class{x} \, \class{y} - \class{y} \, \class{x}
    =
    \class{x \tensor y - y \tensor x}
    =
    \class{[x,y]}
    =
    i([x,y]) \,.
  \]
  Given any~{\algebra{$\kf$}}~$A$ and Lie algebra homomorphism~$\phi \colon \glie \to A$ there exists a unique homorphism of~{\algebras{$\kf$}}~$\Phi' \colon \Tensor(\glie) \to A$ that makes the triangular diagram
  \[
    \begin{tikzcd}
      \glie
      \arrow{r}[above]{\phi}
      \arrow{d}[left]{\iota}
      &
      A
      \\
      \Tensor(V)
      \arrow[dashed]{ur}[below right]{\Phi'}
      &
      {}
    \end{tikzcd}
  \]
  commute, where~$\iota \colon \glie \to \Tensor(\glie)$ is the inclusion.
  The homomorphism~$\Phi'$ is given by~$\Phi'(x) = \phi(x)$ for every~$x \in \glie$.
  It follows that
  \begin{align*}
    \Phi'(x \tensor y - y \tensor x)
    &=
    \Phi'(x \tensor y) - \Phi'(y \tensor x)
    \\
    &=
    \Phi'(x) \Phi'(y) - \Phi'(y) \Phi'(x)
    \\
    &=
    \phi(x) \phi(y) - \phi(y) \phi(x)
    \\
    &=
    [\phi(x), \phi(y)]
    \\
    &=
    \phi([x,y])
    \\
    &=
    \Phi'([x,y])
  \end{align*}
  for all~$x, y \in \glie$, so that the ideal~$I$ is contained in the kernel of~$\Phi'$.
  It follows that there exists a unique algebra homomorphism~$\Phi \colon \Tensor(\glie)/I \to A$ that makes the triangular diagram
  \[
    \begin{tikzcd}
      \glie
      \arrow{r}[above]{\phi}
      \arrow{d}[left]{\iota}
      \arrow[bend right = 55]{dd}[left]{i}
      &
      A
      \\
      \Tensor(\glie)
      \arrow[bend right= 20]{ur}[above left]{\Phi'}
      \arrow{d}[left]{\pi}
      &
      {}
      \\
      \Tensor(\glie)/I
      \arrow[dashed, bend right = 30]{uur}[below right]{\Phi}
      &
      {}
    \end{tikzcd}
  \]
  commute, where~$\pi \colon \Tensor(V) \to \Tensor(V)/I$ denotes the canonical projection.
  Then the subdiagram
  \[
    \begin{tikzcd}
      \glie
      \arrow{r}[above]{\phi}
      \arrow{d}[left]{i}
      &
      A
      \\
      \Tensor(\glie)/I
      \arrow{ur}[below right]{\Phi}
      &
      {}
    \end{tikzcd}
  \]
  commutes.
  That~$\Phi$ is unique with this property follows from the uniqueness of~$\Phi'$.
\end{proof}


\begin{remark}
  The above proof may be reorganized by observing that we have bijections
  \begin{align*}
    {}&
    \{ \text{algebra homomorphisms~$\Phi \colon \Tensor(\glie)/I \to A$} \}
    \\
    \cong{}&
    \{ \text{algebra homomorphisms~$\Phi' \colon \Tensor(\glie) \to A$ with~$\Phi'(I) = 0$} \}
    \\
    \cong{}&
    \left\{
      \begin{tabular}{@{}c@{}}
        algebra homomorphisms~$\Phi' \colon \Tensor(\glie) \to A$ with  \\
        $\Phi'(x \tensor y - y \tensor x - [x,y]) = 0$ for all~$x, y \in \glie$
      \end{tabular}
    \right\}
    \\
    \cong{}&
    \left\{
      \begin{tabular}{@{}c@{}}
        algebra homomorphisms~$\Phi' \colon \Tensor(\glie) \to A$ with  \\
        $\Phi'(x) \Phi'(y) - \Phi'(y) \Phi'(x) - \Phi'([x,y]) = 0$ for all~$x, y \in \glie$
      \end{tabular}
    \right\}
    \\
    \cong{}&
    \left\{
      \begin{tabular}{@{}c@{}}
        algebra homomorphisms~$\Phi' \colon \Tensor(\glie) \to A$ with  \\
        $\Phi'(x) \Phi'(y) - \Phi'(y) \Phi'(x) = \Phi'([x,y])$ for all~$x, y \in \glie$
      \end{tabular}
    \right\}
    \\
    \cong{}&
    \left\{
      \begin{tabular}{@{}c@{}}
        {\linear{$\kf$}} maps~$\phi \colon \glie \to A$ with  \\
        $\phi(x) \phi(y) - \phi(y) \phi(x) = \phi([x,y])$ for all~$x, y \in \glie$
      \end{tabular}
    \right\}
    \\
    \cong{}&
    \left\{
      \begin{tabular}{@{}c@{}}
        {\linear{$\kf$}} maps~$\phi \colon \glie \to A$ with  \\
        $[\phi(x), \phi(y)] = \phi([x,y])$ for all~$x, y \in \glie$ 
      \end{tabular}
    \right\}
    \\
    ={}&
    \{ \text{Lie~algebra homomorphisms~$\phi \colon \glie \to A$} \}
  \end{align*}
  that are natural in~$A$.
  This shows that the~{\algebra{$\kf$}}~$\Tensor(\glie)/I$ represented the right kind of functor.
  We can also see that the identity~$\Tensor(\glie)/I \to \Tensor(\glie)/I$ corresponds under the above bijections to the map~$\iota \colon \glie \to \Tensor(\glie)/I$ as desired.
\end{remark}





\section{Examples}


% TODO: UEA of semidirect product.


\begin{convention}
  Let~$\glie$ be a Lie~algebra.
  For every element~$x$ of~$\glie$ the image of~$x$ of~$\Univ(\glie)$ is denoted by~$\class{x}$.
\end{convention}


\subsection{Abelian Lie~Algebras}


\begin{examples}
  Let~$\glie$ be an abelian Lie~algebra.
  It follows from the explicit construction of the universal enveloping algebra~$\Univ(\glie)$ that
  \[
    \Univ(\glie)
    \cong
    \Tensor(\glie)/(x \tensor y - y \tensor x \suchthat x, y \in \glie)
    \cong
    \Symm(\glie)
  \]
  with the canonical homomorphism of Lie~algebras from~$\glie$ to~$\Univ(\glie)$ corresponding to the inclusion map from~$\glie$ to~$\Symm(\glie)$.

  This can also be seen more abstractly, as follows.
  
  We observe that if~$V$ is any~\vectorspace{$\kf$} and~$A$ is any~\algebra{$\kf$} then a linear map~$f$ from~$V$ to~$A$ extends to a homomorphism of algebras from~$\Symm(V)$ to~$A$ (necessarily uniquely) if and only if the image of~$f$ is contained in a commutative subalgebra of~$A$, if and only if the image of~$f$ is commutative in~$A$.
  It follows from this observation that we have for any~{\algebra{$\kf$}} bijections
  \begin{align*}
    {}&
    \{ \textstyle \text{Lie~algebra homomorphisms~$\glie \to A$} \}
    \\
    \cong{}&
    \{ \textstyle \text{{\linear{$\kf$}} maps~$\glie \to A$ with commutative image} \}
    \\
    \cong{}&
    \{ \textstyle \text{algebra homomorphisms~$\Symm(\glie) \to A$} \} \,.
  \end{align*}
  These bijections are natural in~$A$.
  This shows that the symmetric algebra~$\Symm(\glie)$ together with the inclusion from~$\glie$ to~$\Symm(\glie)$ satisfies the universal property of the universal enveloping algebra of~$\glie$.
\end{examples}


\begin{example}
  We find for~$\glie = 0$ that~$\Univ(\glie) = \kf$.
\end{example}


\begin{definition}
  \leavevmode
  \begin{enumerate}
    \item
      Let~$A$ be an algebra.
      An \defemph{augumentation} of~$A$ is a homomorphism of algebras~$\varepsilon$ from~$A$ to~$\kf$.
    \item
      An \defemph{augumented algebra} is a~\algebra{$\kf$}~$A$ together with an augumentation of~$A$.
    \item
      Let~$(A, \varepsilon)$ be an augumented algebra.
      The kerrnel of~$\varepsilon$ is the \defemph{augumentation ideal} of~$A$.
  \end{enumerate}
\end{definition}


\begin{remark}
  Augumented algebras are always nonzero.
\end{remark}


\begin{proposition}
  \label{decomposition for augumented algebra}
  Let~$(A, \varepsilon) $ be an augumented algebra.
  Then~$A = \kf \oplus \ker(\varepsilon)$ as vector spaces.
\end{proposition}


\begin{proof}
  Let~$\eta$ be the inclusion from~$\kf$ to~$A$.
  The composito~$\varepsilon \circ \eta$ is the identity on~$\kf$ whence the composite~$\eta \circ \varepsilon$ is idempotent.
  It follows that
  \[
    A
    =
    \im(\eta \circ \varepsilon)
    \oplus \ker(\eta \circ \varepsilon) \,.
  \]
  It follows from the surjectivity of~$\varepsilon$ that~$\im(\eta \circ \varepsilon) = \im(\eta) = \kf$ and from the injectivity of~$\eta$ that~$\ker(\eta \circ \varepsilon) = \ker(\varepsilon)$.
\end{proof}


\begin{construction}
  \label{construction of counit}
  Let~$\glie$ be a Lie~algebra.
  We have a unique homomorphism of Lie~algebras from~$\glie$ to~$0$.
  This homomorphism of Lie~algebras induces a homomorphism of algebras
  \[
    \varepsilon
    \colon
    \Univ(\glie)
    \to
    \kf \,.
  \]
  This homomorphism of algebras is uniquely determined by the condition~$\varepsilon( \class{x} ) = 0$ for all~$x \in \glie$ because~$\Univ(\glie)$ is generated as an algebra by the image of~$\glie$ of~$\Univ(\glie)$.
\end{construction}


\begin{definition}
  Let~$\glie$ be a Lie~algebra.
  The homomorphism of algebras~$\varepsilon$ from~$\Univ(\glie)$ to~$\kf$ from \cref{construction of counit} is the \defemph{counit} of~$\glie$.
  Its makes~$\Univ(\glie)$ into an augumented algebra.
\end{definition}


\begin{remark}
  Let~$\glie$ be a Lie~algebra.
  We can regard~$\kf$ as the trivial representation of~$\glie$, and thus as a~\module{$\Univ(\glie)$}.
  This~\module{$\Univ(\glie)$} structure can also be explained with help of the counit~$\varepsilon$, by regarding~$\varepsilon$ as a homomorphism of algebras from~$\Univ(\glie)$ to~$\End_{\kf}(\kf) = \kf$.
\end{remark}


\begin{proposition}
  \label{augumentation ideal is spanned by monomials}
  The augumentation ideal of~$\Univ(\glie)$ is spanned as a vector space by all the monomials~$\class{x_1} \dotsm \class{x_n}$ with~$n \geq 1$,~$x_1, \dotsc, x_n \in \glie$.
\end{proposition}


\begin{proof}
  Let~$\varepsilon$ be the counit of~$\Univ(\glie)$, let~$U$ be the linear subspace of~$\Univ(\glie)$ spanned by all monomials~$\class{x_1} \dotsm \class{x_n}$ with~$n \geq 0$,~$x_1, \dotsc, x_n \in \glie$.
  We have
  \[
    \varepsilon( \class{x_1} \dotsm \class{x_n} )
    =
    \varepsilon( \class{x_1} ) \dotsm \varepsilon( \class{x_n} )
    =
    0 \dotsm 0
    =
    0
  \]
  for each such monomial, whence the linear space~$U$ is contained in~$\ker(\varepsilon)$.
  We know on the other hand that~$\Univ(\glie)$ is generated by~$\class{\glie}$ as an algebra.
  This means that the monomials
  \[
    \class{x_1}  \dotsm \class{x_n}
    \qquad
    \text{with~$n \geq 0$,~$x_1, \dotsc, x_n \in \glie$}
  \]
  span the algebra~$\Univ(\glie)$ as a vector space.
  We thus have~$\Univ(\glie) = \kf + U$.
  Together with the decomposition~$\Univ(\glie) = \kf \oplus \ker(\varepsilon)$ and the inclusion~$U \subseteq \ker(\varepsilon)$ we find that~$U = \ker(\varepsilon)$.
\end{proof}



\subsection{Opposite Lie~Algebra}

\begin{example}
  \label{uea of opposite by first principles}
  Let~$\glie$ be a Lie~algebra.
  We have for every~{\algebra{$\kf$}}~$A$ bijections
  \begin{align*}
    {}&
    \{ \text{algebra homomorphisms~$\Univ(\glie^\op) \to A$   } \}
    \\
    \cong{}&
    \{ \text{Lie~algebra homomorphisms~$\glie^\op \to A$} \}
    \\
    ={}&
    \{ \text{Lie~algebra homomorphisms~$\glie \to A^\op$} \}
    \\
    \cong{}&
    \{ \text{algebra homomorphisms~$\Univ(\glie) \to A^\op$} \}
    \\
    ={}&
    \{ \text{algebra homomorphisms~$\Univ(\glie)^\op \to A$} \}
  \end{align*}
  that are natural in~$A$.
  (We used implicitely that taking the underlying Lie~algebra of a~{\algebra{$\kf$}} commutes with taking opposites.)
  It follows from Yoneda’s~lemma that~$\Univ(\glie^\op) \cong \Univ(\glie)^\op$.
  The canonical homomorphism of Lie~algebras from~$\glie^{\op}$ to~$\Univ(\glie^{\op})$ corresponds to the homomorphism from~$\glie^{\op}$ to~$\Univ(\glie)^{\op}$ given by~$\class{x^{\op}} \mapsto \class{x}^{\op}$ for all~$x \in \glie$.

  We can also derive the above isomorphism in a more explicit way, as we will now explain.

  The map from~$\glie$ to~$\glie^{\op}$ given by$x \mapsto x^{\op}$ for all~$x \in \glie$ is an anti-isomorphism of Lie~algebras.
  Similarly, the map from~$\Univ(\glie^{\op})$ to~$\Univ(\glie^{\op})^{\op}$ given by~$y \mapsto y^{\op}$ for all~$y \in \Univ(\glie^{\op})$ is an anti-isomorphism of algebras, and thus an anti-isomorphism of Lie~algebras. 
  The composite
  \[
    \glie
    \xto{x \mapsto x^{\op}}
    \glie^{\op}
    \to
    \Univ(\glie^{\op})
    \xto{y \mapsto y^{\op}}
    \Univ(\glie^{\op})^{\op}
  \]
  is therefore a homomorphism of Lie~algebras.
  This composite hence induces a homomorphism of algebras~$\Phi$ from~$\Univ(\glie)$ to~$\Univ(\glie^{\op})^{\op}$.
  This is the unique algebra homomorphism that makes the square diagram
  \[
    \begin{tikzcd}
      \glie
      \arrow{r}[above]{x \mapsto x^{\op}}
      \arrow{d}
      &
      \glie^{\op}
      \arrow{d}
      \\
      \Univ(\glie)
      \arrow[dashed]{r}[below]{\varphi}
      &
      \Univ(\glie^{\op})^{\op}
    \end{tikzcd}
  \]
  commute.
  We can regard~$\Phi$ is an algebra homomorphism from~$\Univ(\glie)^{\op}$ to~$\Univ(\glie^{\op})$.
  This homomorphism~$\Phi$ is uniquely determined by the identity
  \[
    \Phi\Bigl( \class{x}^{\,\op} \Bigr)
    =
    \class{ x^{\op} } 
  \]
  for all~$x \in \glie$.

  By switching the roles of~$\glie$ and~$\glie^{\op}$ we also have a homomorphism of algebras
  \[
    \Psi''
    \colon
    \Univ(\glie^{\op})^{\op}
    \to
    \Univ( (\glie^{\op})^{\op} )
  \]
  which is given by
  \[
    \Psi''\Bigl( \class{y}^{\,\op} \Bigr)
    =
    \class{ y^{\op} }
  \]
  for all~$y \in \glie^{\op}$
  We have~$(\glie^{\op})^{\op}$ whence~$\Psi$ is a homomorphism of algebras~$\Phi'$ from~$\Univ(\glie^{\op})^{\op}$ to~$\Univ(\glie)$, that is given by
  \[
    \Psi'\biggl( \class{ x^{\op} }^{\,\op} \biggr)
    =
    \class{ (x^{\op})^{\op} }
    =
    \class{ x }
  \]
  for all~$x \in \glie$.
  We can now regard~$\Psi'$ as a homomorphism of algebras~$\Phi$ from~$\Univ(\glie^{\op})$ to~$\Univ(\glie)^{\op}$, that is given by
  \[
    \Psi\Bigl( \class{ x }^{\,\op} \Bigr)
    =
    \class{ x }^{\op}
  \]
  for all~$x \in \glie$.

  The composite~$\Phi \circ \Psi$ as a homomorphism of algebras from~$\Univ( \glie^{\op} )$ to~$\Univ( \glie^{\op} )$, given by
  \[
    (\Phi \circ \Psi)\Bigl( \class{ x^{\op} } \Bigr)
    =
    \Phi\Bigl( \Psi\Bigl( \class{ x^{\op} }  \Bigr) \Bigr)
    =
    \Phi\Bigl( \class{ x }^{\,\op} \Bigr)
    =
    \class{ x^{\op} }
  \]
  for all~$x \in \glie$, and thus
  \[
    (\Phi \circ \Psi)\bigl( \class{y} \bigr)
    =
    \class{ y }
  \]
  for all~$y \in \glie^{\op}$.
  This shows that the composite~$\Phi \circ \Psi$ is the identity of~$\Univ( \glie^{\op} )$.

  The composite~$\Psi \circ \Phi$ is a homomorphism of algebras from~$\Univ( \glie )^{\op}$ to~$\Univ( \glie )^{\op}$, given by
  \[
    (\Psi \circ \Phi)\Bigl( \class{x}^{\,\op} \Bigr)
    =
    \Psi\Bigl( \Phi\Bigl( \class{x}^{\,\op} \Bigr) \Bigr)
    =
    \Psi\Bigl( \class{ x^{\op} } \Bigr)
    =
    \class{x}^{\op} 
  \]
  for all~$x \in \glie$.
  The algebra~$\Univ(\glie)$ is generated by the elements~$\class{x}$ with~$x$ in~$\glie$, whence the opposite algebra~$\Univ(\glie)^{\op}$ is generated by the elements~$\class{x}^{\,\op}$ with~$\glie$.
  It therefore follows from the above calculation that the composite~$\Psi \circ \Phi$ is the identity of~$\Univ( \glie )^{\op}$.

  We have altogether shows that the two homomorphisms of algebras~$\Phi$ and~$\Psi$ are mutually inverse isomorphisms.
\end{example}


\begin{construction}
  \label{construction of antipode}
  Let~$\glie$ be a Lie~algebra.
  We have an isomorphism of Lie~algebras
  \[
    \glie
    \to
    \glie^{\op} \,,
    \quad
    x
    \mapsto
    - x^{\op} \,.
  \]
  This isomorphism of Lie~algebras induces an isomorphism of algebras
  \[
    S'
    \colon
    \Univ( \glie )
    \to
    \Univ( \glie^{\op} )
  \]
  which is uniquely determined by
  \[
    S'( \class{x} )
    =
    - \class{ x^{\op} }
  \]
  for all~$x \in \glie$.
  Under the above isomorphism~$\Univ( \glie^{\op} ) \cong \Univ(\glie)^{\op}$ we can regard~$S'$ as an isomorphism of algebras
  \[
    S
    \colon
    \Univ( \glie )
    \to
    \Univ( \glie )^{\op}
  \]
  which is uniquely determined by
  \[
    S( \class{x} )
    =
    - \class{x}^{\,\op}
  \]
  for all~$x \in \glie$.
\end{construction}


\begin{definition}
  Let~$\glie$ be a Lie~algebra.
  The isomorphism of algebras~$S$ from~$\Univ(\glie)$ to~$\Univ(\glie)^{\op}$ from \cref{construction of antipode} is the \defemph{antipode} of~$\Univ(\glie)$.
\end{definition}


\begin{remark}
  Let~$\glie$ be a Lie~algebra.
  For every representation~$M$ of~$\glie$ its dual~$M^*$ becomes again a representation of~$\glie$ via the action
  \[
    (x \act \varphi)(m)
    =
    - \varphi(m)
  \]
  for all~$x \in \glie$,~$\varphi \in M^*$,~$m \in M$.
  In other words, for every~\module{$\Univ(\glie)$}~$M$ its dual~$M^*$ becomes again a~\module{$\Univ(\glie)$}.

  This can also be explained via the antipode.
  Let~$M$ be a~\module{$\Univ(\glie)$}.
  The the dual~$M^*$ becomes a right~\module{$\Univ(\glie)$} via the multiplication
  \[
    (\varphi \cdot y)(m)
    =
    \varphi(ym)
  \]
  for all~$y \in \Univ(\glie)$,~$\varphi \in M^*$,~$m \in M$.
  This right~\module{$\Univ(\glie)$} structure on~$M^*$ corresponds to a left~\module{$\Univ(\glie)^{\op}$} structure on~$M^*$ given by
  \[
    y^{\op} \cdot \varphi
    =
    \varphi \cdot y
  \]
  for all~$y \in \Univ(\glie)$,~$\varphi \in M^*$.
  By using the isomorphism of algebras~$S$ from~$\Univ(\glie)$ to~$\Univ(\glie)^{\op}$ we can pull back this~\module{$\Univ(\glie)^{\op}$} structure to a~\module{$\Univ(\glie)$} structure given by
  \[
    y \cdot \varphi
    =
    S(y) \cdot \varphi
  \]
  for all~$y \in \Univ(\glie)$,~$\varphi \in M^*$.

  For every element~$x$ of~$\glie$ we have
  \[
    (\class{x} \cdot \varphi)(m)
    =
    ( S(\class{x}) \cdot \varphi )(m)
    =
    ( - \class{x}^{\,\op} \cdot \varphi )(m)
    =
    ( \varphi \cdot (- \class{x}) )(m)
    =
    \varphi( - \class{x} \cdot m )
    =
    - \varphi( \class{x} \cdot m )
  \]
  for all~$\varphi \in M^*$,~$m \in M$.
  Both constructed~\module{$\Univ(\glie)$} structures on~$M^*$ hence coincide.
\end{remark}



\subsection{Direct Sum of Lie~algebras}

\begin{recall}
  \label{homomorphism out of a tensor product}
  Let~$A$ and~$B$ be two~{\algebras{$\kf$}}.
  Then the inclusion maps
  \begin{alignat*}{2}
    \Iota_A
    &\colon
    A
    \to
    A \tensor B \,,
    &
    \quad
    a
    &\mapsto
    a \tensor 1 \,,
    \\
    \Iota_B
    &\colon
    B
    \to
    A \tensor B \,,
    &
    \quad
    b
    &\mapsto
    1 \tensor b
  \end{alignat*}
  are injective homomorphisms of algebras.
  We may therefore identify the algebras~$A$ and~$B$ with the associated subalgebras~$A \tensor 1$ and~$1 \tensor B$ of~$A \tensor B$.
  We note that~$A$ and~$B$ commute in~$A \tensor B$ because
  \[
    \Iota_A(a) \Iota_B(b)
    =
    (a \tensor 1) (b \tensor 1)
    =
    a \tensor b
    =
    (b \tensor 1) (a \tensor 1)
    =
    \Iota_B(b) \Iota_A(a)
  \]
  for all~$a \in A$,~$b \in B$.
  
  Let~$C$ be another~{\algebra{$\kf$}}.
  
  If~$\Phi$ is a homomorphism of algebras from~$A \otimes B$ to~$C$ then the restrictions~$\Phi_A$ and~$\Phi_B$ given by~$\Phi_A = \Phi \circ \Iota_A$ and~$\Phi_B = \Phi \circ \Iota_B$ are again homomorphisms of algebras.
  The images of~$\Phi_A$ and~$\Phi_B$ commute in~$C$ because~$A$ and~$B$ commute in~$A \tensor B$.
  More explicitely,
  \begin{align*}
    \Phi_A(a) \Phi_B(b)
    &=
    \Phi(a \tensor 1) \Phi(1 \tensor b)
    \\
    &=
    \Phi( (a \tensor 1) (1 \tensor b) )
    \\
    &=
    \Phi( a \tensor b )
    \\
    &=
    \Phi( (1 \tensor b) (a \tensor 1) )
    \\
    &=
    \Phi(1 \tensor b) \Phi(a \tensor 1)
    \\
    &=
    \Phi_B(b) \Phi_A(a)
  \end{align*}
  for all~$a \in A$,~$b \in B$.
  
  Suppose on the other hand that~$\Psi_A$ is a homomorphism of algebras from~$A$ to~$C$ and that~$\Psi_B$ is a homomorphism of algebras from~$B$ to~$C$.
  There exists a unique linear map~$\Psi$ from~$A \otimes B$ to~$C$ given by
  \[
    \Psi(a \otimes b)
    \mapsto
    \Psi_A(a) \Psi_:(b)
  \]
  for all~$a \in A$,~$b \in B$.
  Suppose that the images of~$\Psi_A$ and~$\Psi_B$ commute in~$C$.
  The linear map~$\Psi$ is then again an homomorphism of algebras because
  \begin{align*}
    \Psi(a_1 \tensor b_1) \Psi(a_2 \tensor b_2)
    &=
    \Psi_A(a_1) \Psi_B(b_1) \Psi_A(a_2) \Psi_B(b_2)
    \\
    &=
    \Psi_A(a_1) \Psi_A(a_2) \Psi_B(b_1) \Psi_B(b_2)
    \\
    &=
    \Psi_A(a_1 a_2) \Psi_B(b_1 b_2)
    \\
    &=
    \Psi( (a_1 a_2) \tensor (b_1 b_2) )
    \\
    &=
    \Psi( (a_1 \tensor b_1) (a_2 \tensor b_2) )
  \end{align*}
  for all~$a_1, a_2 \in A$,~$b_1, b_2 \in B$, as well as
  \[
    \Psi( 1_{A \otimes B} )
    =
    \Psi( 1_A \otimes 1_B )
    =
    \Psi_A( 1_A ) \Psi_B( 1_B )
    =
    1_C \cdot 1_C
    =
    1_C \,.
  \]
  
  These above two constructions are mutually inverse and hence result in a {\onetoonetext} correspondence
  \begin{align*}
    \left\{
      \begin{tabular}{@{}c@{}}
        algebra homomorphisms \\
        $\Phi \colon A \tensor B \to C$
      \end{tabular}
    \right\}
    &\onetoone
    \left\{
      (\Phi_A, \Phi_B)
    \suchthat*
      \begin{tabular}{@{}c@{}}
        algebra homomorphisms   \\
        $\Phi_A \colon A \to C$ \\
        $\Phi_B \colon B \to C$ \\
        whose images commute
      \end{tabular}
    \right\}  \,,
    \\
    \Phi
    &\mapsto
    (\Phi \circ \Iota_A, \Phi \circ \Iota_B)  \,,
    \\
    \biggl( a \tensor b \mapsto \Phi_A(a) \Phi_B(b) \biggr)
    &\mapsfrom
    (\Phi_A, \Phi_B)  \,.
  \end{align*}
\end{recall}


\begin{example}
  \label{explicit isomorphism for uea of direct sum}
  Let~$\glie$ and~$\hlie$ be two Lie~algebras.
  We show in the following that
  \[
    \Univ(\glie \oplus \hlie)
    \cong
    \Univ(\glie) \tensor \Univ(\hlie) \,.
  \]
  The isomorphism from~$\Univ(\glie \oplus \hlie)$ to~$\Univ(\glie) \tensor \Univ(\hlie)$ is given on the algebra generators~$\class{(x,y)}$ with~$(x,y)$ in~$\glie \oplus \hlie$ by
  \[
    \class{(x,y)}
    \mapsto
    \class{x} \tensor 1 + 1 \tensor \class{y} \,.
  \]
  The inverse isomorphism from~$\Univ(\glie) \tensor \Univ(\hlie)$ to~$\Univ(\glie \oplus \hlie)$ is given on the simple tensors~$\class{x} \tensor \class{y}$ with~$x$ in~$\Univ(\glie)$ and~$y$ in~$\Univ(\hlie)$ by
  \[
    \class{x} \tensor \class{y}
    \mapsto
    \Univ( \iota_1 )( \class{x} )
    \cdot
    \Univ( \iota_2 )( \class{y} ) \,.
  \]
  Here we denote by~$\iota_1$ is the canonical homomorphism of Lie~algebras from~$\glie$ to~$\glie \oplus \hlie$ (i.e. the inclusion into the first sammand) and similarly by~$\iota_2$ is the canonical homomorphism of Lie~algebras from~$\hlie$ to~$\glie \oplus \hlie$ (i.e. the inclusion into the second summand).

  We present two ways in which the above isomorphism(s) can be derived.
  \begin{itemize}
    \item
      It follows from \cref{homomorphism out of direct sum} and \cref{homomorphism out of a tensor product} that we get for every~\algebra{$\kf$}~$A$ bijections
      \begin{align*}
        {}&
        \left\{
          \begin{tabular}{@{}c@{}}
            algebra homomorphisms \\
            $\Phi \colon \Univ(\glie \oplus \hlie) \to A$
          \end{tabular}
        \right\}
        \\
        \cong{}&
        \left\{
          \begin{tabular}{@{}c@{}}
            Lie~algebra homomorphisms \\
            $\varphi \colon \glie \oplus \hlie \to A$
          \end{tabular}
        \right\}
        \\
        \cong{}&
        \left\{
          ( \varphi_1, \varphi_2 )
        \suchthat*
          \begin{tabular}{@{}c@{}}
            Lie~algebra homomorphisms \\
            $\varphi_1 \colon \glie \to A$ and~$\varphi_2 \colon \hlie \to A$ \\
            whose images commute
          \end{tabular}
        \right\}
        \\
        \cong{}&
        \left\{
          (\Phi_1, \Phi_2)
        \suchthat*
          \begin{tabular}{@{}c@{}}
            algebra homomorphisms               \\
            $\Phi_1 \colon \Univ(\glie) \to A$  \\
            $\Phi_2 \colon \Univ(\hlie) \to A$  \\
            whose images commute
          \end{tabular}
        \right\}
        \\
        \cong{}&
        \left\{
          \begin{tabular}{@{}c@{}}
             algebra homomorphims \\
             $\Phi \colon \Univ(\glie) \tensor \Univ(\hlie) \to A$
          \end{tabular}
        \right\} \,.
      \end{align*}
      The claimed isomorphism therefore follows from Yoneda’s lemma.
    \item
      We can construct the isomorphism(s) more explicitely, as follows.

      We note that for the induced homomorphisms of algebras
      \begin{align*}
        \Univ(\iota_1) &\colon \Univ(\glie) \to \Univ(\glie \oplus \hlie) \,, \\
        \Univ(\iota_2) &\colon \Univ(\hlie) \to \Univ(\glie \oplus \hlie)
      \end{align*}
      the images of~$\Univ(\iota_1)$ and~$\Univ(\iota_2)$ commute.
      Indeed, the algebra~$\Univ(\glie \oplus \hlie)$ is generated by the image of~$\glie \oplus \hlie$ in~$\Univ(\glie \oplus \hlie)$, and the images~$\iota_1(\glie)$ and~$\iota_2(\hlie)$ commute in~$\glie \oplus \hlie$.
      Thus
      \[
        \Univ(\iota_1)( \class{x} )
        \cdot
        \Univ(\iota_2)( \class{y} )
        =
        \class{(x,0)} \cdot \class{(0,y)}
        =
        \class{(0,y)} \cdot \class{(x,0)}
        =
        \Univ(\iota_2)( \class{y} )
        \cdot
        \Univ(\iota_2)( \class{x} ) \,,
      \]
      for all~$x \in \glie$,~$y \in \hlie$.
      We used for the middle equality that the elements~$(x,0)$ and~$(0,y)$ commute in~$\glie \oplus \hlie$ and hence also in~$\Univ(\glie \oplus \hlie)$.
      It follows from this observation that the two homomorphisms of algebras~$\Univ( \iota_1 )$ and~$\Univ( \iota_2 )$ induce a homomorphism of algebras
      \[
        \Phi
        \colon
        \Univ(\glie) \tensor \Univ(\hlie)
        \to
        \Univ(\glie \oplus \hlie) \,.
      \]
      This homomorphism is given on simple tensors by
      \[
        \Phi(t \tensor u)
        =
        \Univ(\iota_1)(t) \cdot \Univ(\iota_2)(u)
      \]
      for all~$t \in \Univ(\glie)$,~$u \in \Univ(\hlie)$.
      It holds in particular for all~$x \in \glie$,~$y \in \glie$ that
      \[
        \Phi(\class{x} \tensor \class{y})
        =
        \Univ(\iota_1)( \class{x} )
        \cdot
        \Univ(\iota_2)( \class{y} )
        =
        \class{ \iota_1(x) }
        \cdot
        \class{ \iota_2(y) }
        =
        \class{(x,0)} \cdot \class{(0,y)}  \,,
      \]
%      We observe that the map
%      \[
%        \psi'
%        \colon
%        \glie \times \hlie
%        \to
%        \Univ(\glie) \tensor \Univ(\hlie) \,,
%        \quad
%        (x,y)
%        \mapsto
%        \class{(x,0)} \tensor 1 + 1 \tensor \class{(0,y)} \,.
%      \]

      To construct the inverse~$\Psi$ of~$\Phi$ we observe that the equality
      \[
        \Psi\Bigl( \class{(x,0)} \Bigr)
        =
        \Psi\Bigl( \class{(x,0)} \cdot 1 \Bigr)
        =
        \Psi\Bigl( \Univ(\iota_1)( \class{x} ) \cdot \Univ(\iota_2)(1) \Bigr)
        =
        \Psi( \Phi( \class{x} \tensor 1 ) )
        =
        \class{x} \tensor 1
      \]
      has to hold for all~$x \in \glie$, and similarly
      
      \[
        \Psi\Bigl( \class{(0,y)} \Bigr)
        =
        1 \tensor \class{y}
      \]
      for all~$y \in \hlie$.
      It then follows that more generally
      \[
        \Psi\Bigl( \class{(x,y)} \Bigr)
        =
        \Psi\Bigl( \class{(x,0)} + \class{(0,y)} \Bigr)
        =
        \class{x} \tensor 1 + 1 \tensor \class{y}
      \]
      for all~$(x,y) \in \glie \oplus \hlie$.
      
      Motivated by these calculations we consider the map
      \[
        \psi
        \colon
        \glie \oplus \hlie
        \to
        \Univ(\glie) \tensor \Univ(\hlie) \,,
        \quad
        (x,y)
        \mapsto
        \class{x} \otimes 1 + 1 \otimes{y} \,.
      \]
      This map is a homomorphism of Lie~algebras because it is linear with
      \begin{align*}
        {}&
        [\psi((x_1, y_1)), \psi((x_2, y_2))]
        \\
        ={}&
        [
          \class{x_1} \tensor 1 + 1 \tensor \class{y_1} \,,
          \class{x_2} \tensor 1 + 1 \tensor \class{y_2}
        ]
        \\
        ={}&
          [\class{x_1} \tensor 1, \class{x_2} \tensor 1]
        + \underbrace{ [\class{x_1} \tensor 1, 1 \tensor \class{y_2}] }_{=0}
        + \underbrace{ [1 \tensor \class{y_1} \,, \class{x_2} \tensor 1] }_{=0}
        + [1 \tensor \class{y_1}, 1 \tensor \class{y_2}]
        \\
        ={}&
          [\class{x_1}, \class{x_2}] \tensor 1
        + 1 \tensor [\class{y_1}, \class{y_2}]
        \\
        ={}&
          \class{[x_1, x_2]} \tensor 1
        + 1 \tensor \class{[y_1, y_2]}  \,.
        \\
        ={}&
        \psi\bigl( ( [x_1, x_2], [y_1, y_2] ) \bigr)
        \\
        ={}&
        \psi\bigl( [(x_1, y_1), (x_2, y_2)] \bigr) \,.
      \end{align*}
      for all~$(x_1, y_1), (x_2, y_2) \in \glie \oplus \hlie$.
      It hence follows from the universal property of the {\ua}~$\Univ(\glie \oplus \hlie)$ that there exists a unique homomorphism of algebras~$\Psi$ from~$\Univ(\glie \oplus \hlie)$ to~$\Univ(\glie) \tensor \Univ(\hlie)$ that makes the triangular diagram
      \[
        \begin{tikzcd}[row sep = large]
          \glie \oplus \hlie
          \arrow{r}[above]{\psi}
          \arrow{d}[left]{\class{(-)}}
          &
          \Univ(\glie) \tensor \Univ(\hlie)
          \\
          \Univ(\glie \oplus \hlie)
          \arrow[dashed]{ur}[below right]{\Psi}
          &
          {}
        \end{tikzcd}
      \]
      commute.
      This homomorphism~$\psi$ is given by
      \[
        \Psi\Bigl( \class{(x,y)} \Bigr)
        =
        \class{x} \tensor 1 + 1 \tensor \class{y} \,.
      \]
      for all~$(x,y) \in \glie \oplus \hlie$.
      
      We now check that the two homomorphisms~$\Phi$ and~$\Psi$ are mutually inverse.
      We have on the one hand
      \begin{align*}
        \Phi\Bigl( \Psi\Bigl( \class{(x,y)} \Bigr) \Bigr)
        &=
        \Phi( \class{x} \tensor 1 + 1 \tensor \class{y} )
        \\
        &=
        \Phi( \class{x} \tensor 1 ) + \Phi( 1 \tensor \class{y} )
        \\
        &=
        \Univ(\iota_1)(\class{x}) \cdot \Univ(\iota_2)(1)
        + \Univ(\iota_1)(1) \cdot \Univ(\iota_2)(\class{y})
        \\
        &=
        \class{\iota_1(x)} \cdot 1
        + 1 \cdot \class{\iota_2(y)}
        \\
        &=
        \class{(x,0)} + \class{(0,y)}
        \\
        &=
        \class{(x,y)}
      \end{align*}
      for all~$(x,y) \in \glie \oplus \hlie$, and thus~$\Phi \circ \Psi = \id_{\Univ(\glie \oplus \hlie)}$.
      We have on the other hand
      \begin{align*}
        \Psi( \Phi( \class{x} \tensor \class{y} ) )
        &=
        \Psi( \Univ(\iota_1)(\class{x}) \cdot \Univ(\iota_2)(\class{y}) )
        \\
        &=
        \Psi\Bigl( \class{(x,0)} \class{(0,y)} \Bigr)
        \\
        &=
        \Psi\Bigl( \class{(x,0)} \Bigr)
        \Psi\Bigl( \class{(0,y)} \Bigr)
        \\
        &=
        ( \class{x} \tensor 1 + 1 \tensor 0 )
        \cdot ( 0 \tensor 1 + 1 \tensor \class{y} )
        \\
        &=
        (\class{x} \tensor 1)
        \cdot (1 \tensor \class{y})
        \\
        &=
        \class{x} \tensor \class{y}
      \end{align*}
      for all~$x \in \glie$ and~$y \in \hlie$, and thus~$\Psi \circ \Phi = \id_{\Univ(\glie) \otimes \Univ(\hlie)}$.
  \end{itemize}
\end{example}


\begin{construction}
  \label{construction of comultiplication}
  Let~$\glie$ be a Lie~algebra.
  We have a homomorphism of Lie~algebras
  \[
    \delta
    \colon
    \glie
    \to
    \glie \oplus \glie \,,
    \quad
    x
    \mapsto
    (x,x) \,.
  \]
  This homomorphism of Lie~algebras extends a homomorphism of algebras
  \[
    \Delta'
    \colon
    \Univ(\glie)
    \to
    \Univ(\glie \oplus \glie) \,.
  \]
  As seen in \cref{explicit isomorphism for uea of direct sum} we have an isomorphism of algebras from~$\Univ(\glie \oplus \glie)$ to~$\Univ(\glie) \otimes \Univ(\glie)$ given by~$\class{(x,y)} \mapsto \class{x} \otimes 1 + 1 \otimes \class{y}$ for all~$x, y \in \glie$.
  Under this isomorphism we can regard the homomorphism~$\Delta'$ as a homomorphism of algebras
  \[
    \Delta
    \colon
    \Univ(\glie)
    \to
    \Univ(\glie) \otimes \Univ(\glie) \,.
  \]
  This homomorphism is uniquely determined by
  \[
    \Delta( \class{x} )
    =
    \class{x} \otimes 1 + 1 \otimes \class{x}
  \]
  for all~$x \in \glie$.
\end{construction}


\begin{definition}
  Let~$\glie$ be a Lie~algebra.
  The algebra homomorphism~$\Delta$ from~$\Univ(\glie)$ to~$\Univ(\glie) \otimes \Univ(\glie)$ from \cref{construction of comultiplication} is the \defemph{comultiplication} of~$\glie$.
\end{definition}


\begin{remark}
  Let~$\glie$ be a Lie~algebra.
  For every two representations~$M$ and~$N$ of~$\glie$ the tensor product~$M \otimes_{\kf} N$ becomes again a representation of~$\glie$ via the aciton
  \[
    x \act (m \otimes n)
    =
    (x \act m) \otimes n + m \otimes (x \act n)
  \]
  for all~$x \in \glie$,~$m \in M$,~$n \in N$.
  In other words, for every two~\modules{$\Univ(\glie)$}~$M$ and~$N$ the tensor product~$M \otimes_{\kf} N$ becomes again a~\module{$\Univ(\glie)$}.

  This can be explained via the comultiplication~$\Delta$ of~$\glie$.
  Indeed, suppose that~$M$ and~$N$ are two~\modules{$\Univ(\glie)$}.
  These module structure correspond to algebra homomorphisms
  \[
    \Phi_M
    \colon
    \Univ(\glie)
    \to
    \End_{\kf}(M) \,,
    \quad
    \Phi_N
    \colon
    \Univ(\glie)
    \to
    \End_{\kf}(N) \,.
  \]
  We have now the the homomorphism of algebras
  \[
    \Delta
    \colon
    \Univ(\glie)
    \to
    \Univ(\glie) \otimes \Univ(\glie) \,,
  \]
  the homomorphism of algebras
  \[
    \Phi_M \otimes \Phi_N
    \colon
    \Univ(\glie) \otimes \Univ(\glie)
    \to
    \End_{\kf}(M) \otimes \End_{\kf}(N)
  \]
  and the homomorphism of algebras
  \[
    \Psi
    \colon
    \End_{\kf}(M) \otimes \End_{\kf}(N)
    \to
    \End_{\kf}(M \otimes_{\kf} N) \,,
    \quad
    f \otimes g
    \mapsto
    f \otimes g \,.
  \]
  The composite of the homomorphism is a homomorphism of algebras
  \[
    \Phi_{M \otimes N}
    \colon
    \Univ(\glie)
    \to
    \End_{\kf}(M \otimes N) \,.
  \]
  This homomorphism corresponds to a~\module{$\Univ(\glie)$} structure on~$M \otimes_{\kf} N$.
  This module structure is given for every element~$x$ of~$\glie$ by
  \begin{align*}
    \class{x} \cdot (m \otimes n)
    &=
    \Phi_{M \otimes N}( \class{x} )( m \otimes n )
    \\
    &=
    \Psi( (\Phi_M \otimes \Phi_N)( \Delta(\class{x}) ) )( m \otimes n)
    \\
    &=
    \Psi( (\Phi_M \otimes \Phi_N)( \class{x} \otimes 1 + 1 \otimes \class{x} ) )( m \otimes n)
    \\
    &=
    \Psi( \Phi_M(\class{x}) \otimes \Phi_N(1) + \Phi_M(1) \otimes \Phi_N(\class{x}) )( m \otimes n)
    \\
    &=
    \Psi( \Phi_M(\class{x}) \otimes \id_N + \id_M \otimes \Phi_N(\class{x}) )( m \otimes n)
    \\
    &=
    \Phi_M(\class{x})(m) \otimes \id_N(n) + \id_M(m) \otimes \Phi_N(\class{x})(n)
    \\
    &=
    (\class{x} \cdot m) \otimes n + m \otimes (\class{x} \cdot n)
  \end{align*}
  for all~$m \in M$,~$n \in N$.
  This is precisely the same module structure as above.
\end{remark}


\begin{remark}
  Let~$\glie$ be a Lie~algebra.
  The comultiplication~$\Delta$, counit~$\varepsilon$ and antipode~$S$ satisfy certain compatibility conditions.
  More precisely, the comultiplication~$\Delta$ satisfies the following \defemph{coassociativity} diagram.
  \[
    \begin{tikzcd}[sep = large]
      \Univ(\glie)
      \arrow{r}[above]{\Delta}
      \arrow{d}[left]{\Delta}
      &
      \Univ(\glie) \otimes \Univ(\glie)
      \arrow{d}[right]{\Delta \otimes \id}
      \\
      \Univ(\glie) \otimes \Univ(\glie)
      \arrow{r}[above]{\id \otimes \Delta}
      &
      \Univ(\glie) \otimes \Univ(\glie) \otimes \Univ(\glie)
    \end{tikzcd}
  \]
  The comultiplication~$\Delta$ and counit~$\varepsilon$ satisfy the folloing \defemph{counital} diagram.
  \[
    \begin{tikzcd}[row sep = large]
      \Univ(\glie) \otimes \Univ(\glie)
      \arrow{d}[left]{\id \otimes \varepsilon}
      &
      \Univ(\glie)
      \arrow{l}[above]{\Delta}
      \arrow{r}[above]{\Delta}
      \arrow[equal]{d}
      &
      \Univ(\glie) \otimes \Univ(\glie)
      \arrow{d}[right]{\varepsilon \otimes \id}
      \\
      \Univ(\glie) \otimes \kf
      \arrow{r}[above]{\cong}
      &
      \Univ(\glie)
      &
      \kf \otimes \Univ(\glie)
      \arrow{l}[above]{\cong}
    \end{tikzcd}
  \]
  The comultiplication~$\Delta$, counit~$\varepsilon$ and antipode~$S$ satisfy the following \defemph{antipode} diagram.
  \[
    \begin{tikzcd}[column sep = small, row sep = large]
      {}
      &
      \Univ(\glie) \otimes \Univ(\glie)
      \arrow{rr}{S \otimes \id}
      &
      {}
      &
      \Univ(\glie) \otimes \Univ(\glie)
      \arrow{dr}{\mathrm{mult}}
      &
      {}
      \\
      \Univ(\glie)
      \arrow{ur}[above left]{\Delta}
      \arrow{rr}[above]{\varepsilon}
      \arrow{dr}[below left]{\Delta}
      &
      {}
      &
      \kf
      \arrow{rr}[above]{\mathrm{incl}}
      &
      {}
      &
      \Univ(\glie)
      \\
      {}
      &
      \Univ(\glie) \otimes \Univ(\glie)
      \arrow{rr}[above]{\id \otimes S}
      &
      {}
      &
      \Univ(\glie) \otimes \Univ(\glie)
      \arrow{ur}[below right]{\mathrm{mult}}
      &
      {}
    \end{tikzcd}
  \]
  This means altogether that the algebra~$\Univ(\glie)$ together with the comultiplication~$\Delta$, counit~$\varepsilon$ and antipode~$S$ is a \defemph{Hopf algebra}.
\end{remark}




\subsection{Quotient Lie~Algebra}


\begin{example}
  Let~$\glie$ be a Lie~algebra, let~$I$ be an ideal of~$\glie$ and let~$\ideal{I}$ be the two-sided ideal of~$\Univ(\glie)$ generated by all elements of the form~$\class{x}$ with~$x$ in~$I$.
  Let~$\pi$ be the canonical quotient homomorphism from~$\glie$ to~$\glie/I$.
  Then the induced homomorphism of algebras~$\Univ(\pi)$ from~$\Univ(\glie)$ to~$\Univ(\glie/I)$ factors through~$\Univ(\glie) / \ideal{I}$, and induces an isomorphism
  \[
    \Univ(\glie/I)
    \cong
    \Univ(\glie) / \ideal{I} \,,
  \]
  which is is given by
  \[
    \class{ \class{x} }
    \mapsto
    \class{ \class{x} }
  \]
  for all~$x \in \glie$.

  Indeed, we have for every~{\algebra{$\kf$}}~$A$ bijections
  \begin{align*}
    {}&
    \{ \textstyle\text{algebra homomorphisms~$\Psi \colon \Univ(\glie/I) \to A$} \}
    \\
    \cong{}&
    \{ \textstyle\text{Lie~algebra homomorphisms~$\psi \colon \glie/I \to A$} \}
    \\
    \cong{}&
    \{ \textstyle\text{Lie~algebra homomorphisms~$\varphi \colon \glie \to A$ with~$\varphi(x) = 0$ for all~$x \in I$} \}
    \\
    \cong{}&
    \{ \textstyle\text{algebra homomorphisms~$\Phi \colon \Univ(\glie) \to A$ with~$\Phi(\class{x}) = 0$ for all~$x \in I$} \}
    \\
    ={}&
    \{ \textstyle\text{algebra homomorphisms~$\Phi \colon \Univ(\glie) \to A$ with~$\Phi(y) = 0$ for all~$y \in \ideal{I}$} \}
    \\
    \cong{}&
    \{ \textstyle\text{algebra homomorphisms~$\Psi \colon \Univ(\glie) / \ideal{I} \to A$} \} \,.
  \end{align*}
  These bijections are natural in~$A$, whence~$\Univ(\glie/I) \cong \Univ(\glie) / \ideal{I}$ by Yoneda’s lemma.
  
  More explicitely, the composite
  \[
    \glie
    \to
    \Univ(\glie)
    \to
    \Univ(\glie) / \ideal{I} \,,
    \quad
    x \mapsto
    \class{ \class{x} }
  \]
  is a homomorphisms of Lie~algebras.
  This homomorphism annihilates the Lie~ideal~$I$ of~$\glie$ and thus induces a homomorphism of Lie~algebras
  \[
    \glie/I
    \to
    \Univ(\glie) / \ideal{I} \,,
    \quad
    \class{x}
    \mapsto
    \class{ \class{x} } \,,
  \]
  which in turn induces a homomorphism of algebras
  \begin{alignat*}{3}
    \Phi
    &\colon
    \Univ(\glie/I)
    \to
    \Univ(\glie) / \ideal{I}  \,,
    &
    \quad
    \class{ \class{x} }
    &\mapsto
    \class{ \class{x} }
    &
    \qquad
    &\text{for all~$x \in \glie$.}
  \intertext{
  On the other hand, the quotient homomorphism~$\pi$ from~$\glie$ to~$\glie/I$ induces the homomorphism of algebras~$\Univ(\pi)$ from~$\Univ(\glie)$ to~$\Univ(\glie/I)$, given by
  }
    \Univ(\pi)
    &\colon
    \Univ(\glie)
    \to
    \Univ( \glie / I ) \,,
    &
    \quad
    \class{x}
    &\mapsto
    \class{ \class{x} }
    &
    \qquad
    &\text{for all~$x \in \glie$.}
  \intertext{
  This homomorphism annihilates all residue classes~$\class{x}$ with~$x$ in~$I$, and hence induces a homomorphism of algebras
  }
    \Psi
    &\colon
    \Univ(\glie) / \ideal{I}
    \to
    \Univ(\glie/I)  \,,
    &
    \quad
    \class{ \class{x} }
    &\mapsto
    \class{ \class{x} }
    &
    \qquad
    &\text{for all~$x \in \glie$.}
  \end{alignat*}
  We have thus constructed two mutually inverse algebra isomorphisms~$\Phi$ and~$\Psi$.
\end{example}


\begin{example}
  It follows from the previous example that for any Lie~algebra~$\glie$,
  \begin{align*}
    \Univ( \glie^{\ab} )
    &=
    \Univ( \glie/[\glie, \glie] )
    \\
    &\cong
    \Univ(\glie) / \ideal{[\glie, \glie]}
    \\
    &=
    \Univ(\glie)
    /
    \ideal[\Big]{ \class{[x,y]} \suchthat x, y \in \glie }
    \\
    &=
    \Univ(\glie)
    /
    \ideal{
      \class{x} \, \class{y} - \class{y} \, \class{x} 
    \suchthat
      x, y \in \glie
    } \,.
  \end{align*}
  The ideal~$\ideal{ \class{x} \, \class{y} - \class{y} \, \class{x} \suchthat x, y \in \glie }$ is the commutator ideal of~$\Univ(\glie)$ because the elements~$\class{x}$ with~$x$ in~$\glie$ form an algebra generating set of~$\Univ(\glie)$.
  The universal enveloping algebra of the abelianization of~$\glie$ is hence the abelianization (commutativization?) of the universal enveloping algebra of~$\glie$.
  
  This can also be seen from Yoneda’s lemma since we have for every commutative~{\algebra{$\kf$}}~$A$ bijections
  \begin{align*}
    \SwapAboveDisplaySkip
    {}&
    \{ \textstyle\text{algebra homomorphisms~$\Univ(\glie/[\glie,\glie]) \to A$} \}
    \\
    \cong{}&
    \{ \textstyle\text{Lie~algebra homomorphism~$\glie/[\glie, \glie] \to A$} \}
    \\
    \cong{}&
    \{ \textstyle\text{Lie~algebra homomorphisms~$\glie \to A$} \}
    \\
    \cong{}&
    \{ \textstyle\text{algebra homomorphisms~$\Univ(\glie) \to A$} \}
    \\
    \cong{}&
    \{ \textstyle\text{algebra homomorphisms~$\Univ(\glie)/I \to A$} \}
  \end{align*}
  which are natural in~$A$, where~$I$ denotes the commutator ideal of~$\Univ(\glie)$.
\end{example}





\subsection{Free Lie~Algebra}


\begin{definition}
  Let~$X$ be a set.
  A~\defemph{free~\liealgebra{$\kf$} on the set~$X$}\index{free Lie algebra}\index{Lie algebra!free} is a~\liealgebra{$\kf$}~$F(X)$ together with a set-theoretic map~$i$ from~$X$ to~$F(X)$ such that the following universal property holds:
  for every~\liealgebra{$\kf$}~$\glie$ and every map~$f$ from~$X$ to~$\glie$ there exists a unique homomorphism of Lie~algebras~$\varphi$ from~$F(X)$ to~$\glie$ that makes the following triangular diagram commute.
  \[
    \begin{tikzcd}
      X
      \arrow{dr}[above right]{f}
      \arrow{d}[left]{i}
      &
      {}
      \\
      F(X)
      \arrow[dashed]{r}[below]{\varphi}
      &
      \glie
    \end{tikzcd}
  \]
\end{definition}


\begin{remark}
  \leavevmode
  \begin{enumerate}
    \item
      A free Lie~algebra on a set~$X$ consists of a Lie~algebra~$X$ together with an element~$\class{x}$ of~$F(X)$ for every element~$x$ of~$X$, such that the following universal property holds:
      for every Lie~algebra~$\glie$ and every famile~$(y_x)_{x \in X}$ of elements of~$\glie$ there exists a unique homomorphism of Lie~algebras~$\varphi$ from~$F(X)$ to~$\glie$ such that~$\varphi( \class{x} ) = y_x$ for all~$x \in X$.
    \item
      The free Lie~algebra on a set~$X$ is unique up to unique isomorphism, in the following sense.

      If~$(F_1, i_1)$ and~$(F_2, i_2)$ are two free Lie~algebras on~$X$ then there exist unique homomorphisms of Lie~algebras~$\varphi$ from~$F_1$ to~$F_2$ and~$\psi$ from~$F_2$ to~$F_1$ that make the triangular diagrams
      \[
        \begin{tikzcd}[column sep = small]
          {}
          &
          X
          \arrow{dl}[above left]{\iota_1}
          \arrow{dr}[above right]{\iota_2}
          &
          {}
          \\
          F_1
          \arrow[dashed]{rr}[below]{\varphi}
          &
          {}
          &
          F_2
        \end{tikzcd}
        \qquad\text{and}\qquad
        \begin{tikzcd}[column sep = small]
          {}
          &
          X
          \arrow{dl}[above left]{\iota_2}
          \arrow{dr}[above right]{\iota_1}
          &
          {}
          \\
          F_2
          \arrow[dashed]{rr}[below]{\psi}
          &
          {}
          &
          F_1
        \end{tikzcd}
      \]
      commute.
      These homomorphisms~$\varphi$ and~$\psi$ are mutually inverse isomorphisms.

      Because of this uniqueness we will talk about \emph{the} free~\liealgebra{$\kf$} on~$X$.
    \item
      For every map~$f$ from a set~$X$ to a set~$Y$ there exists a unique homomorphism of Lie~algebras~$F(f)$ from~$F(X)$ to~$F(Y)$ that makes the square diagram
      \[
        \begin{tikzcd}
          X
          \arrow{r}[above]{f}
          \arrow{d}
          &
          Y
          \arrow{d}
          \\
          F(X)
          \arrow[dashed]{r}[below]{F(f)}
          &
          F(Y)
        \end{tikzcd}
      \]
      commute.
      It holds for every set~$X$ that~$F(\id_X) = \id_{F(X)}$, and it holds for all composable maps of sets~$f$ from~$X$ to~$Y$ and~$g$ from~$Y$ to~$Z$ that~$F(g \circ f) = F(g) \circ F(f)$.
      The assignment~$F$ is thus a functor from the category~$\cSet$ to the category~$\cLie{\kf}$.
      
      The universal property of the free Lie~algebra states that the functor~$F$ is left adjoint to the forgetful functor from~$\cLie{\kf}$ to~$\cSet$, which assigns to each Lie~algebra its underlying set.
  \end{enumerate}
\end{remark}


\begin{example}[Free Lie~algebras]
  \label{uea of free lie algebra}
  Let~$I$ be a set and let~$F(I)$ be the free~{\liealgebra{$\kf$}} on the set~$I$.
  We have for every~\algebra{$\kf$}~$A$ bijections
  \begin{align*}
    {}&
    \{ \textstyle\text{algebra homomorphisms~$\Univ(F(I)) \to A$} \}
    \\
    \cong{}&
    \{ \textstyle\text{Lie~algebra homomorphisms~$F(I) \to A$} \}
    \\
    \cong{}&
    \{ \textstyle\text{set-theoretic maps~$I \to A$} \}
    \\\
    \cong{}&
    \{ \textstyle\text{algebra homomorphisms~$\kf\gen{x_i \suchthat i \in I} \to A$} \} \,,
  \end{align*}
  and these bijections are natural in~$A$.
  We hence find by Yoneda’s~lemma that
  \[
    \Univ(F(I))
    \cong
    \kf\gen{x_i \suchthat i \in I} \,.
  \]
  More precisely, there exists for the composite
  \[
    I
    \to
    F(I)
    \to
    \Univ(F(I))
  \]
  a unique homomorphism of algebras~$\Phi$ from~$\kf\gen{x_i \suchthat i \in I}$ to~$\Univ(F(I))$ that makes the diagram
  \[
    \begin{tikzcd}[column sep = small]
        {}
      & I
        \arrow[bend right, out = -45, in=225]{ddl}[above left]{i \mapsto x_i}
        \arrow[bend left]{dr}
      & {}
      \\
        {}
      & {}
      & F(I)
        \arrow{d}
      \\
        \kf\gen{x_i \suchthat i \in I}
        \arrow[dashed]{rr}[below]{\Phi}
      & {}
      & \Univ(F(I))
    \end{tikzcd}
  \]
  commute.
  This homomorphism~$\Phi$ is an isomorphism.
\end{example}


\begin{remark}
  Lie~subalgebras of free Lie~algebras are again free by the Shirshov--Witt~theorem.
\end{remark}










