\chapter[The Universal Enveloping Algebra]{The Universal Enveloping \texorpdfstring{\\}{} Algebra}
\label{universal enveloping algebra}


\begin{convention}
	For this \lcnamecref{universal enveloping algebra} we fix the following conventions.
	
	The field~$\kf$ is arbitrary.
	We denote by~$\cAlg{\kf}$\glsadd{category algebras} the category of~\algebras{$\kf$}.
	If~$A$ is a~\algebra{$\kf$}, then by an~\defemph{\module{$A$}} we mean a left, unitial~\module{$A$}.
	The resulting category of~\modules{$A$}\index{category!of A-modules@of $A$-modules} is denoted by~$\cMod{A}$\glsadd{category modules}.
\end{convention}





%\section{Yoneda’s Lemma for Algebras}
%
%
%% \begin{fluff}
%%   If~$A$ and~$B$ are two~{\algebras{$\kf$}} and~$\varphi \colon A \to B$ is a homomorphism of~{\algebras{$\kf$}} then~$\varphi$ induces for every other~{\algebra{$\kf$}}~$C$ a map~$\varphi^*_C \colon \Hom_{\cAlg{\kf}}(B,C) \to \Hom_{\cAlg{\kf}}(A,C)$.
%%   These induced maps are compatible in the sense that for all~{\algebras{$\kf$}}~$C$ and~$D$ and every homomorphism of~{\algebras{$\kf$}}~$f \colon C \to D$ the following square diagram commutes:
%%   \[
%%     \begin{tikzcd}
%%       \Hom_{\cAlg{\kf}}(A,C)
%%       \arrow{r}[above]{\varphi^*_C}
%%       \arrow{d}[left]{f_*}
%%       &
%%       \Hom_{\cAlg{\kf}}(B,C)
%%       \arrow{d}[right]{f_*}
%%       \\
%%       \Hom_{\cAlg{\kf}}(A,D)
%%       \arrow{r}[below]{\varphi^*_D}
%%       &
%%       \Hom_{\cAlg{\kf}}(B,D)
%%     \end{tikzcd}
%%   \]
%%   If~$\varphi$ is an isomorphism then the induced map~$\varphi^*_C$ is for every~{\algebra{$\kf$}}~$C$ a bijection.
%%
%%   The following \lcnamecref{yoneda lemma very weak version}, which is a weak form of Yoneda’s lemma applied to the category of~{\algebras{$\kf$}}, gives a converse to this observation.
%% \end{fluff}
%
%
%
%\begin{proposition}[Yoneda’s lemma for~{\algebras{$\kf$}}]
%	\label{yoneda lemma very weak version}
%	Let~$A$ and~$B$ be two~{\algebras{$\kf$}}.
%	\begin{enumerate}
%		\item
%			Let~$\Phi$ be a homomorphism of~{\algebras{$\kf$}} from~$A$ to~$B$.
%			This homomorphism induces for every other~{\algebra{$\kf$}}~$T$ a map
%			\[
%				\Phi^*_T
%				\colon
%				\Hom_{\cAlg{\kf}}(B,T) \to \Hom_{\cAlg{\kf}}(A,T) \,,
%				\quad
%				\Psi
%				\mapsto
%				\Psi \circ \Phi \,.
%			\]
%			These induced maps are compatible in the sense that for any two~{\algebras{$\kf$}}~$T$ and~$U$ and every homomorphism of~{\algebras{$\kf$}}~$\Psi$ from~$T$ to~$U$ the following square diagram commutes.
%			\[
%				\begin{tikzcd}
%					\Hom_{\cAlg{\kf}}(B,T)
%					\arrow{r}[above]{\Phi^*_T}
%					\arrow{d}[left]{\Psi_*}
%					&
%					\Hom_{\cAlg{\kf}}(A,T)
%					\arrow{d}[right]{\Psi_*}
%					\\
%					\Hom_{\cAlg{\kf}}(B,U)
%					\arrow{r}[below]{\Phi^*_U}
%					&
%					\Hom_{\cAlg{\kf}}(A,U)
%				\end{tikzcd}
%			\]
%			If~$\Phi$ is an isomorphism, then the induced map~$\Phi^*_T$ is a bijection for every~{\algebra{$\kf$}}~$T$.
%		\item
%			\label{natural homomorphisms}
%			Let conversely~$(\eta_T)_T$ be a family of maps
%			\[
%				\eta_T
%				\colon
%				\Hom_{\cAlg{\kf}}(B,T)
%				\to
%				\Hom_{\cAlg{\kf}}(A,T)
%			\]
%			where~$T$ runs through the class of~{\algebras{$\kf$}}, and suppose that for any two~\algebras{$\kf$}~$T$ and~$U$ and every homomorphism of~\algebras{$\kf$}~$\Psi$ from~$T$ to~$U$ the square diagram
%			\begin{equation}
%				\label{natural square}
%				\begin{tikzcd}
%					\Hom_{\cAlg{\kf}}(B,T)
%					\arrow{r}[above]{\eta_T}
%					\arrow{d}[left]{\Psi_*}
%					&
%					\Hom_{\cAlg{\kf}}(A,T)
%					\arrow{d}[right]{\Psi_*}
%					\\
%					\Hom_{\cAlg{\kf}}(B,U)
%					\arrow{r}[below]{\eta_U}
%					&
%					\Hom_{\cAlg{\kf}}(A,U)
%				\end{tikzcd}
%			\end{equation}
%			commutes.
%			Then there exists an algebra homomorphism~$\Phi$ from~$A$ to~$B$ such that~$\eta_T = \Phi^*_T$ for every~{\algebra{$\kf$}}~$T$.
%			This homomorphism~$\Phi$ is unique and given by~$\Phi = \eta_B(\id_B)$.
%		\item
%			\label{onetoone correspondence morphisms and natural trans}
%			The above constructions give a {\onetoonetext} correspondence
%			\begin{align*}
%				\left\{
%					\begin{tabular}{c}
%						algebra homomorphism \\
%						$\Phi \colon A \to B$
%					\end{tabular}
%				\right\}
%				&\onetoone
%				\left\{
%					\begin{tabular}{c}
%						families~$(\eta_T)_T$ of maps \\
%						$\eta_T \colon \Hom_{\cAlg{\kf}}(B,T) \to \Hom_{\cAlg{\kf}}(A,T)$ \\
%						such that the square diagram~\eqref{natural square} commutes \\
%						for every algebra homomorphism~$f \colon T \to U$
%					\end{tabular}
%				\right\}  \,,
%			\\
%				\Phi
%				&\mapsto
%				(\Phi^*_T)_T \,,
%			\\
%				\eta_B(\id_B)
%				&\mapsfrom
%				(\eta_T)_T  \,.
%			\end{align*}
%		\item
%			A homomorphism of algebras~$\Phi$ from~$A$ to~$B$ is an isomorphism if and only if the map~$\Phi^*_T$ is bijective for every~\algebra{$\kf$}~$T$.
%			The {\onetoonetext} correspondence from part~\ref*{onetoone correspondence morphisms and natural trans} therefore restricts to a {\onetoonetext} correspondence.
%			\[
%				\left\{
%					\begin{tabular}{c}
%						algebra isomorphism \\
%						$\Phi \colon A \to B$
%					\end{tabular}
%				\right\}
%				\onetoone
%				\left\{
%					\begin{tabular}{c}
%						families~$(\eta_T)_T$ of bijections \\
%						$\eta_T \colon \Hom_{\cAlg{\kf}}(B,T) \to \Hom_{\cAlg{\kf}}(A,T)$ \\
%						such that the square diagram~\eqref{natural square} commutes \\
%						for every algebra homomorphism~$f \colon T \to U$
%					\end{tabular}
%				\right\}  \,.
%			\]
%	\end{enumerate}
%\end{proposition}
%
%
%\begin{proof}
%	\leavevmode
%	\begin{enumerate}
%		\item
%			The maps~$\eta_T$ are well-defined because the composite of two homomorphisms of algebras is again a homomorphism of algebras.
%			The square diagram~\eqref{natural square} commutes because
%			\[
%				\Psi_*( \Phi^*_T( \Theta ) )
%				=
%				\Psi_* (\Theta \circ \Phi)
%				=
%				\Psi \circ \Theta \circ \Phi
%				=
%				\Phi^*_U(\Psi \circ \Theta)
%				=
%				\Phi^*_U(\Psi_*(\Theta))
%			\]
%			for every~$\Theta \in \Hom_{\cAlg{\kf}}(B,T)$.
%		\item
%			Suppose first that there exists a homorphism of algebras~$\Phi$ from~$A$ to~$B$ with~$(\eta_T)_T = (\Phi^*_T)_T$.
%			Then
%			\[
%				\eta_B(\id_B)
%				=
%				\Phi^*_B(\id_B)
%				=
%				\id_B \circ \Phi
%				=
%				\Phi \,.
%			\]
%			This shows the uniqueness of~$\Phi$ as well as the claimed formula for~$\Phi$.
%
%			To show the existence of the desired homomorphism~$\Phi$ we set~$\Phi$ to be~$\eta_B(\id_B)$.
%			This is an element of~$\Hom_{\cAlg{\kf}}(A,B)$, and thus a homomorphism of algebras from~$A$ to~$B$.
%			If~$T$ is any~\algebra{$\kf$}, then it follows for every element~$\Psi$ of~$\Hom_{\cAlg{\kf}}(B,T)$ from the commutativity of the square diagram
%			\[
%				\begin{tikzcd}
%					\Hom_{\cAlg{\kf}}(B,B)
%					\arrow{r}[above]{\eta_B}
%					\arrow{d}[left]{\Psi_*}
%					&
%					\Hom_{\cAlg{\kf}}(A,B)
%					\arrow{d}[right]{\Psi_*}
%					\\
%					\Hom_{\cAlg{\kf}}(B,T)
%					\arrow{r}[below]{\eta_T}
%					&
%					\Hom_{\cAlg{\kf}}(A,T)
%				\end{tikzcd}
%			\]
%			that
%			\[
%				\eta_T(\Psi)
%				=
%				\eta_T(\Psi \circ \id_B)
%				=
%				\eta_T(\Psi_*(\id_B))
%				=
%				\Psi_*(\eta_B(\id_B))
%				=
%				\Psi_*(\Phi)
%				=
%				\Psi \circ \Phi \,.
%			\]
%			This shows that~$\eta_T = \Phi^*_T$, as claimed.
%		\item
%			Let~$\Phi$ be a homomorphism of algebras from~$A$ to~$B$.
%			Then
%			\[
%				\Phi^*_B(\id_B)
%				=
%				\id_B \circ \Phi
%				=
%				\Phi \,.
%			\]
%			Together with part~\ref*{onetoone correspondence morphisms and natural trans} this shows that the two constructions are mutually inverse.
%		\item
%			If the homomorphism~$\Phi$ is an isomorphism, then the map~$\Phi^*_T$ is bijective for every~{\algebra{$\kf$}} because the two maps~$\Phi^*_T$ and~$(\Phi^{-1})^*_T$ are mutually inverse.
%
%			Suppose on the other hand that the map~$\Phi^*_T$ is a bijection for every~{\algebra{$\kf$}}~$T$.
%			Then the family of maps~$( (\Phi^*_T)^{-1} )_T$ makes for every homomorphism of algebras~$\Psi$ from~$T$ to~$U$ the square diagram
%			\[
%				\begin{tikzcd}[column sep = large]
%					\Hom_{\cAlg{\kf}}(A,T)
%					\arrow{r}[above]{ (\Phi^*_T)^{-1} }
%					\arrow{d}[left]{\Psi_*}
%					&
%					\Hom_{\cAlg{\kf}}(B,T)
%					\arrow{d}[right]{\Psi_*}
%					\\
%					\Hom_{\cAlg{\kf}}(A,U)
%					\arrow{r}[below]{ (\Phi^*_U)^{-1} }
%					&
%					\Hom_{\cAlg{\kf}}(B,U)
%				\end{tikzcd}
%			\]
%			commute.
%			There hence exists by part~\ref*{natural homomorphisms} a (unique) homomorphism of algebras~$\Theta$ from~$B$ to~$A$ for which~$(\Phi^*_T)^{-1}$ equals~$\Theta^*_T$ for every~{\algebra{$\kf$}}~$T$.
%			We find for the composite~$\Theta \circ \Phi$ that
%			\[
%				(\Theta \circ \Phi)^*_T
%				=
%				\Phi^*_T \circ \Theta^*_T
%				=
%				\Phi^*_T \circ (\Phi^*_T)^{-1}
%				=
%				\id_{\Hom_{\cAlg{\kf}}(A,T)}
%				=
%				(\id_A)^*_T
%			\]
%			for every~{\algebra{$\kf$}}~$T$.
%%       It hence follows from the commutativity of the diagram
%%       \[
%%         \begin{tikzcd}
%%           \Hom_{\cAlg{\kf}}(A,T)
%%           \arrow[bend left]{rr}[above]{\varphi^*_T \circ \psi^*_T}
%%           \arrow{r}[above]{\psi^*_T}
%%           \arrow{d}[left]{f_*}
%%           &
%%           \Hom_{\cAlg{\kf}}(B,T)
%%           \arrow{r}[above]{\varphi^*_T}
%%           \arrow{d}[left]{f_*}
%%           &
%%           \Hom_{\cAlg{\kf}}(A,T)
%%           \arrow{d}[left]{f_*}
%%           \\
%%           \Hom_{\cAlg{\kf}}(A,U)
%%           \arrow{r}[above]{\psi^*_U}
%%           \arrow[bend right]{rr}[below]{\varphi^*_U \circ \psi^*_U}
%%           &
%%           \Hom_{\cAlg{\kf}}(B,U)
%%           \arrow{r}[above]{\varphi^*_U}
%%           &
%%           \Hom_{\cAlg{\kf}}(A,U)
%%         \end{tikzcd}
%%       \]
%%       that the outer square diagram
%%       \[
%%         \begin{tikzcd}[column sep = large]
%%           \Hom_{\cAlg{\kf}}(A,T)
%%           \arrow{r}[above]{(\psi \circ \varphi)^*_T}
%%           \arrow{d}[left]{f_*}
%%           &
%%           \Hom_{\cAlg{\kf}}(A,T)
%%           \arrow{d}[right]{f_*}
%%           \\
%%           \Hom_{\cAlg{\kf}}(A,U)
%%           \arrow{r}[below]{(\psi \circ \varphi)^*_U}
%%           &
%%           \Hom_{\cAlg{\kf}}(A,U)
%%         \end{tikzcd}
%%       \]
%%       commutes.
%%       If we replace~$\psi \circ \varphi$ in this diagram by the identity~$\id_A$ then the resulting square diagram
%%       \[
%%         \begin{tikzcd}[column sep = large]
%%           \Hom_{\cAlg{\kf}}(A,T)
%%           \arrow{r}[above]{(\id_A)^*_T}
%%           \arrow{d}[left]{f_*}
%%           &
%%           \Hom_{\cAlg{\kf}}(A,T)
%%           \arrow{d}[right]{f_*}
%%           \\
%%           \Hom_{\cAlg{\kf}}(A,U)
%%           \arrow{r}[below]{(\id_A)^*_U}
%%           &
%%           \Hom_{\cAlg{\kf}}(A,U)
%%         \end{tikzcd}
%%       \]
%%       also commutes.
%			Hence~$\Theta \circ \Phi = \id_A$ by the uniqueness in part~\ref*{natural homomorphisms}.
%			We find in the same way that also~$\Phi \circ \Theta = \id_B$.
%			This shows that~$\Phi$ and~$\Theta$ are mutually inverse isomorphisms.
%		\qedhere
%	\end{enumerate}
%\end{proof}
%
%
%\begin{definition}
%	A family~$(\eta_T)_T$ of maps~$\eta_T \colon \Hom_{\cAlg{\kf}}(B,T) \to \Hom_{\cAlg{\kf}}(A,T)$, where~$A$ and~$B$ are two~{\algebras{$\kf$}} and~$T$ ranges over all~{\algebras{$\kf$}}, is \defemph{natural}\index{natural} if the square diagram~\eqref{natural square} commutes for ever homomorphism of algebras~$\Psi$ from~$A$ to~$B$.
%\end{definition}
%
%
%\begin{remark}
%	\Cref{yoneda lemma very weak version} holds with the same proof for every kind of mathematical structure that has a suitable notion of homomorphisms between them, i.e.\ in every category.
%	It is then know as the \defemph{Yoneda lemma}\index{Yoneda lemma}, which is one of the most important statement in all of category theory (and mathematics).
%\end{remark}


\section{Tensor Algebra and Symmetric Algebra}


% \begin{example}[Monoid algebra]
%   In the following all monoids will be written multiplicaitvely unless otherwise mentioned.
%   The neutral element of a monoid~$M$ will be denoted by~$1$ or~$1_M$.
%   Given two monoids~$M$ and~$N$ a map~$f \colon M \to N$ is a homomorphism of monids if~$f(m \cdot m') = f(m) \cdot f(m')$ for all~$m, m' \in M$ and~$f(1_M) = 1_N$.
%   If~$M$ is any monoid then the identity~$\id_M$ is a homomorphism and if~$f \colon M \to N$ and~$g \colon N \to P$ are composable homomorphisms of monoids then their composition~$g \circ f \colon M \to P$ is again a homomorphism of monoids.
%   The resulting category of monoids is denoted by~$\cMon$.
%   
%   \begin{description}
%     \item[Construction:]
%       If~$M$ is a monoid then the monoid algebra~\gls*{monoid algebra} is the (free) vector space with basis~$M$ together with the unique bilinear extension~$\kf[M] \times \kf[M] \to \kf[M]$ of the multiplication~$M \times M \to M$ as its multiplication.
%   
%       This means that the elements of~$\kf[M]$ are formal {\linear{$\kf$}} combinations~$\sum_{m \in M} a_m m$ with~$a_m = 0$ for all but finitely many~$m \in M$.
%       The multiplication of two such elements is given by
%       \[
%         \left(
%           \sum_{m \in M} a_m m
%         \right)
%         \left(
%           \sum_{n \in M} b_n n
%         \right)
%         =
%         \sum_{m, n \in M} (a_m b_n) m n  \,.
%       \]
%       We identify every element~$m \in M$ with the corresponding element~$1 \cdot m \in \kf[M]$.
%       The product~$m \cdot n$ of two elements~$m, n \in M$ in~$\kf[M]$ is then the same as their product in~$M$.
%       The associativity of the multiplication of~$\kf[M]$ follows from the associativity of the multiplication of~$M$, and the neutral element of~$M$ is given by the multiplicative neutral element for~$\kf[M]$.
%       
%     \item[Universal Property:]
%       If~$A$ is any~{\algebra{$\kf$}} then~$(A, \cdot)$ is a multiplicative monoid, which we will denote by~$A^-$.
%       If~$M$ is any monoid and~$f \colon M \to A^-$ is a monoid hommorphism then~$f$ extends uniquely to an algebra homomorphism~$F \colon \kf[M] \to A$.
%       The algebra homomorphism~$F$ is given on elements by
%       \[
%         F\left( \sum_{m \in M} a_m m \right)
%         =
%         \sum_{m \in M} a_m f(m) \,.
%       \]
%       On the other hand every algebra homomorphism~$\kf[M] \to A$ restricts to a monoid homomorphism~$M \to A^-$.
%       This construction results in a {\onetoone} correspondence
%       \[
%         \{
%           \text{monoid homomorphisms~$M \to A^-$}
%         \}
%         \longonetoone
%         \{
%           \text{algebra homomorphisms~$\kf[M] \to A$}
%         \}  \,.
%       \]
%     
%     \item[Uniqueness:]
%       The monoid algebra~$\kf[M]$ together with the inclusion~$i \colon M \to \kf[M]$ is uniquely determined by its universal property up to isomorphism:
%   \end{description}
% \end{example}


% \begin{recall}[Free algebra]
%   Let~$I$ be any set.
%   The \defemph{noncommutative polynomial algebra}~$\kf\gen{X_i \suchthat i \in I}$\index{noncommutative polynomial algebra} has as a basis the set of all monomials
%   \[
%     X_{i_1} \dotsm X_{i_n}
%     \qquad
%     \text{with~$i_1, \dotsc, i_n$}
%   \]
%   and the multiplication is on these basis elements given by
%   \[
%     X_{i_1} \dotsm X_{i_n}
%     \cdot
%     X_{j_1} \dotsm X_{j_m}
%     =
%     X_{i_1} \dotsm X_{i_n} X_{j_1} \dotsm X_{j_m} \,.
%   \]
%   In contrast to the usual (commutative) polynomial algebra~$\kf[X_i \suchthat i \in I]$ the variables~$X_i$ are not required to commute with each other.
%   
%   We can alternatively construct~$\kf\gen{X_i \suchthat i \in I}$ as the monomial algebra of the free monoid on~$I$:
%   Let~$M$ be the set of all words in~$I$, i.e.\ the set of all finite sequences
%   \[
%     (i_1, \dotsc, i_n)
%     \qquad
%     \text{with~$i_1, \dotsc, i_n \in I$}  \,.
%   \]
%   Then~$M$ is a monoid with respect to concatenation of words given by
%   \[
%     (i_1, \dotsc, i_n) (j_1, \dotsc, j_m)
%     =
%     (i_1, \dotsc, i_n, j_1, \dotsc, j_m)
%   \]
%   for all words~$(i_1, \dotsc, i_n), (j_1, \dotsc, j_m) \in M$.
%   The neutral element of~$M$ is given by the empty word~$()$.
% \end{recall}





\subsection{Reviewing the Tensor Algebra}


\begin{recall}[Tensor algebra]
  Let~$V$ be a vector space.
  \begin{description}
    \item[Construction:]
      For all~$v_1, \dotsc, v_d \in V$ we denote the resulting simple tensor~$v_1 \tensor \dotsb \tensor v_d$ in~$V^{\tensor d}$ by~$(v_1, \dotsc, v_d)$.
      Observe that for~$d = 0$ the tensor power~$V^{\tensor d} = V^{\tensor 0}$ has as a basis the emtpy simple tensor~$()$.
      We will therefore identify the tensor power~$V^{\tensor 0}$ with the ground field~$\kf$, so that empty simple tensor~$()$ corresponds to~$1 \in \kf$.
      
      For all~$p, q \geq 0$ we define a multiplication
      \[
        \mu_{p,q}
        \colon
        V^{\tensor p} \times V^{\tensor q}
        \to
        V^{\tensor (p+q)} \,,
        \quad
        (x,y)
        \mapsto
        x y
      \]
      that is on simple tensors~$(v_1, \dotsc, v_p) \in V^{\tensor p}$ and~$(v_{p+1}, \dotsc, v_{p+q}) \in V^{\tensor q}$ given by
      \[
        (v_1, \dotsc, v_p) \cdot (v_{p+1}, \dotsc, v_{p+q})
        =
        (v_1, \dotsc, v_{p+q})  \,.
      \]
      Note that for~$p = 0$ or~$q = 0$ this multiplication is just scalar multiplication.  
      These multiplications fit together associatively in the sense that for all~$p, q, r \geq 0$ and simple tensors~$x \in V^{\tensor p}$,~$y \in V^{\tensor q}$ and~$z \in V^{\tensor r}$ the equality
      \[
        x \cdot (y \cdot z)
        =
        (x \cdot y) \cdot z
      \]
      holds.
      
      Let~$\Tensor(V) \defined \bigoplus_{d \geq 0} V^{\tensor d}$.
      We can fit together the multiplications~$\mu_{p,q}$ with~$p, q \geq 0$ to a single multiplication
      \[
        \mu
        \colon
        \Tensor(V) \times \Tensor(V)
        \to
        \Tensor(V)  \,,
        \quad
        (x,y)
        \mapsto
        xy 
      \]
      that is given on elements~$x, y \in \Tensor(V)$ with~$x = (x_d)_{d \geq 0}$ and~$y = (y_d)_{d \geq 0}$ by
      \[
        x y
        =
        \left(
          \sum_{p+q = d} x_p y_q
        \right)_{d \geq 0} \,.
      \]
      This multiplication is built precisely so that it follows from the bilinearity of the multiplications~$\mu_{p,q}$ that the multipliation~$\mu$ is again bilinear.
      It follows from the associativities of the multiplications~$\mu_{p,q}$ that the multiplication~$\mu$ is associative.
      We may identify the ground field~$\kf = V^{\tensor 0}$ with the corresponding direct summand in~$\Tensor(V)$ to regard~$\kf$ as a linear subspace of~$\Tensor(V)$.
      We have seen above that~$1 \in \kf$ is then unital for the multiplication of~$\Tensor(V)$.
      We have thus altogether constructed a~{\algebra{$\kf$}}~$\Tensor(V)$.
      
      We may identify~$V = V^{\tensor 1}$ with the corresponding direct summand of~$\Tensor(V)$ to regard~$V$ as a linear subspace of~$\Tensor(V)$.
      We then have for all~$v_1, \dotsc, v_n \in V$ that
      \[
        v_1 \dotsm v_n
        =
        (v_1) \dotsm (v_n)
        =
        (v_1, \dotsc, v_n)
        =
        v_1 \tensor \dotsb \tensor v_n  \,.
      \]
      It follows in particular that~$\Tensor(V)$ is then generated by~$V$ as an algebra.
      The algebra~\gls*{tensor algebra} is the \defemph{tensor algebra of~$V$}
      
      We will more generally identify for all~$d \geq 0$ the tensor power~$V^{\tensor d}$ with the corresponding summand in~$\Tensor(V)$.
      The tensor algebra~$\Tensor(V)$ hence consists of linear combinations simple tensors~$v_1 \tensor \dotsb \tensor v_n$.
    
    \item[Universal Property:]
      The tensor algebra~$\Tensor(V)$ can be though of as the \enquote{free~{\algebra{$\kf$}} on~$V$}, in at least two ways:
      \begin{itemize}
        \item(Informal)
          The tensor algebra~$\Tensor(V)$ arises from~$V$ by starting with the elements of~$V$ and adding to~$V$ all kinds of expressions that can be constructed from the elements of~$V$ by algebra operations.
          But it follows from the axioms that many of these expressions have to be the same, so that we only end up with expressions of a certain form.
          
          Let us be a bit more explicit:
          Suppose that a~{\algebra{$\kf$}}~$A$ contains~$V$ as a linear subspace.
          Then it also contains products of the form~$v_1 \dotsm v_n$ with~$v_i \in V$ and hence sums of such products, i.e.\ elements of the form
          \[
            \sum_{i=k}^r v_{i_1} \dotsm v_{i_{n_k}}
          \]
          with~$r \geq 0$ and~$v_{ij} \in V$.
          If we continue to combine elements of this form with algebra operations then we do not gain any new elements, since by the axioms of an algebra they must already be of the above form.
          
          But in an arbitrary~{\algebra{$\kf$}} it may happen that some of these expressions are equal even though this does not follow pureley from the axioms of a~{\algebra{$\kf$}}.
          Consider for example the polynomial ring~$A = \kf[x, y]$ and the linear subspace~$V = \gen{x, y}_{\kf}$.
          It follows from the axioms of a~{\algebra{$\kf$}} that the expressions~$x (x+y)$ and~$x^2 + xy$ are the same, but it does not follow just from the axioms that~$xy = yx$, even though this holds in~$A$.
          There are hence certain additional \emph{relations} between the elements~$x$ and~$y$ of~$V$ in the ambient {\algebra{$\kf$}}~$A$.
          
          In the tensor algebra~$\Tensor(V)$ this does not happen:
          Whenever two expressions~$x$ and~$y$ that are built from elements of~$V$ via algebra operations coincide, then this equality can be derived from the algebra axioms alone.
          Hence there exist no additional relations between the elements of~$V$ in~$\Tensor(V)$.
          The only required condition is that~$V$ is a linear subspace of~$\Tensor(V)$, i.e.\ that addition and scalar multiplication in~$V$ does coincide with the one coming from~$\Tensor(V)$.
          
          The tensor algebra~$\Tensor(V)$ is in this way the \enquote{freest} way of expanding~$V$ into a~{\algebra{$\kf$}}.
        \item(Formal)
          Let~$\iota \colon V \to \Tensor(V)$ be the inclusion map, which is~{\linear{$\kf$}}.
          Then if~$A$ is any other~{\algebra{$\kf$}} and~$f \colon V \to A$ any~{\linear{$\kf$}} map (which one can think of as somewhat of an inclusion, albeit not injective), then~$f$ extends uniquely to an algebra homomorphism~$f^+ \colon \Tensor(V) \to A$, i.e.\ there exists a unique algebra homomorphism~$f^+ \colon \Tensor(V) \to A$ that makes the triangular diagram
          \[
            \begin{tikzcd}
              V
              \arrow{r}[above]{f}
              \arrow{d}[left]{i}
              &
              A
              \\
              \Tensor(V)
              \arrow[dashed]{ur}[below right]{f^+}
              &
              {}
            \end{tikzcd}
          \]
          commute.
          The algebra homomorphism~$f^+$ is given by
          \[
            f^+(v_1 \tensor \dotsb \tensor v_d)
            =
            f(v_1) \dotsm f(v_d)
          \]
          for all~$d \geq 0$ and simple tensors~$v_1 \tensor \dotsb \tensor v_d \in V^{\tensor d}$.
          This construction results in a {\onetoone} correspondence
          \begin{align*}
            \{ \text{\linear{$\kf$} maps~$V \to A$} \}
            &\longonetoone
            \{ \text{algebra homomorphisms~$\Tensor(V) \to A$} \} \,,
            \\
            f
            &\longmapsto
            f^+ \,,
            \\
            \restrict{F}{V}
            &\longmapsfrom
            F \,.
          \end{align*}
          Hence~$(\Tensor(V), i)$ is the \enquote{universal way} of mapping the vector space~$V$ into a~{\algebra{$\kf$}}.
          
          This formal explanation relates to the previous informal explanation in the following way:
          If~$A$ is any~{\algebra{$\kf$}} that contains~$V$ as a linear subspace then the inclusion~$V \to A$ extend uniquely to an algebra homomorphism~$\Tensor(V) \to A$.
          Every relation between expressions built from the elements of~$V$ that holds in~$\Tensor(V)$ must then also hold in~$A$.
          Therefore the only relations that hold in~$\Tensor(V)$ between such expressions are the one that hold in \emph{every}~{\algebra{$\kf$}} containing~$V$.
      \end{itemize}
      
    \item[Uniqueness]
      The above universal property determines the tensor algebra~$\Tensor(V)$ together with the inclusion~$i \colon V \to \Tensor(V)$ uniquely up to unique isomorphism, in the following sense:
      Let~$A$ be another~{\algebra{$\kf$}} and let~$j \colon V \to T$ be a~{\linear{$\kf$}} map such that for every~{\algebra{$\kf$}}~$A$ and every~{\linear{$\kf$}} map~$f \colon V \to A$ there exists a unique algebra homomorphism~$F \colon T \to A$ that makes the triangular diagram
      \[
        \begin{tikzcd}
          V
          \arrow{r}[above]{f}
          \arrow{d}[left]{j}
          &
          A
          \\
          T
          \arrow{ur}[below right]{F}
          &
          {}
        \end{tikzcd}
      \]
      commute.
      
      Then there exist unique algebra homomorphisms~$f \colon A \to T$ and~$g \colon T \to A$ that make the triangular diagrams
      \[
        \begin{tikzcd}[column sep = small]
          {}
          &
          V
          \arrow{dl}[above left]{i}
          \arrow{dr}[above right]{j}
          &
          {}
          \\
          \Tensor(V)
          \arrow[dashed]{rr}[below]{f}
          &
          {}
          &
          T
        \end{tikzcd}
        \qquad\text{and}\qquad
        \begin{tikzcd}[column sep = small]
          {}
          &
          V
          \arrow{dl}[above left]{j}
          \arrow{dr}[above right]{i}
          &
          {}
          \\
          T
          \arrow[dashed]{rr}[below]{g}
          &
          {}
          &
          \Tensor(V)
        \end{tikzcd}
      \]
      commute.
      It then follows that the compositions~$g \circ f \colon \Tensor(V) \to \Tensor(V)$ and~$f \circ g \colon T \to T$ make the triangular diagrams
      \[
        \begin{tikzcd}[column sep = small]
          {}
          &
          V
          \arrow{dl}[above left]{i}
          \arrow{dr}[above right]{i}
          &
          {}
          \\
          \Tensor(V)
          \arrow[dashed]{rr}[below]{g \circ f}
          &
          {}
          &
          \Tensor(V)
        \end{tikzcd}
        \qquad\text{and}\qquad
        \begin{tikzcd}[column sep = small]
          {}
          &
          V
          \arrow{dl}[above left]{j}
          \arrow{dr}[above right]{j}
          &
          {}
          \\
          T
          \arrow[dashed]{rr}[below]{f \circ g}
          &
          {}
          &
          T
        \end{tikzcd}
      \]
      commute.
      The algebra homomorphisms~$g \circ f$ and~$f \circ g$ are unique with this propert by the universal properties of~$(\Tensor(V), i)$ and~$(T, j)$.
      But the identities~$\id_{\Tensor(V)}$ and~$\id_T$ also make these diagrams commute.
      We therefore find that~$g \circ f = \id_{\Tensor(V)}$ and~$f \circ g = \id_{\Tensor(V)}$, so that~$f$ and~$g$ are mutually inverse algebra isomorphisms.
    
    \item[Functoriality:]
      If~$f \colon V \to W$ is any~{\linear{$\kf$}} map then we can consider the following diagram:
      \[
        \begin{tikzcd}
          V
          \arrow{r}[above]{f}
          \arrow{d}
          &
          W
          \arrow{d}
          \\
          \Tensor(V)
          &
          \Tensor(W)
        \end{tikzcd}
      \]
      By applying the universal property of the tensor algebra~$\Tensor(V)$ to the composition~$V \to W \to \Tensor(W)$ it follows that there exists a unique algebra homomorphism~$f_* \colon \Tensor(V) \to \Tensor(W)$ that makes the square diagram
      \[
        \begin{tikzcd}
          V
          \arrow{r}[above]{f}
          \arrow{d}
          &
          W
          \arrow{d}
          \\
          \Tensor(V)
          \arrow[dashed]{r}[below]{f_*}
          &
          \Tensor(W)
        \end{tikzcd}
      \]
      commute.
      This induced algebra homorphism is functorial in the following sense:
      \begin{itemize}
        \item
          It holds that~$(\id_V)_* = \id_{\Tensor(V)}$.
          Indeed, the commutativity of the square 
          \[
            \begin{tikzcd}[column sep = large]
              V
              \arrow{r}[above]{f}
              \arrow{d}
              &
              V
              \arrow{d}
              \\
              \Tensor(V)
              \arrow[dashed]{r}[below]{(\id_V)_*}
              &
              \Tensor(V)
            \end{tikzcd}
          \]
          shows that the identity~$\id_{\Tensor(V)}$ satifies the defining property of the induced algebra homomorphism~$(\id_V)_*$.
        \item
          It holds for all linear maps~$f \colon U \to V$ and~$g \colon V \to W$ that~$(\spacing g \circ f)_* = g_* \circ f_*$.
          Indeed, it follows from the commutativity of the diagram
          \[
            \begin{tikzcd}
              U
              \arrow[dashed, bend left=45]{rr}[above]{g \circ f}
              \arrow{r}[above]{f}
              \arrow{d}
              &
              V
              \arrow{r}[above]{g}
              \arrow{d}
              &
              W
              \arrow{d}
              \\
              \Tensor(U)
              \arrow{r}[below]{f_*}
              \arrow[dashed, bend right=45]{rr}[below]{g_* \circ f_*}
              &
              \Tensor(V)
              \arrow{r}[below]{g_*}
              &
              \Tensor(W)
            \end{tikzcd}
          \]
          that the subdiagram
          \[
            \begin{tikzcd}[column sep = large]
              U
              \arrow{r}[above]{g \circ f}
              \arrow{d}
              &
              W
              \arrow{d}
              \\
              \Tensor(U)
              \arrow[dashed]{r}[below]{g_* \circ f_*}
              &
              \Tensor(W)
            \end{tikzcd}
          \]
          commutes.
          This shows that the composition~$g_* \circ f_*$ satisfies the defining property of the induced algebra homomorphism~$(\spacing g \circ f)_*$.
      \end{itemize}
      
      This shows that the assignment~$V \mapsto \Tensor(V)$ extends to a (covariant) functor~$\Tensor \colon \cVect{\kf} \to \cAlg{\kf}$.
      The universal property of the tensor algebra states that the functor~$\Tensor$ is left adjoint to the forgetful functor~$\cAlg{\kf} \to \cVect{\kf}$ that assigns to each~{\algebra{$\kf$}} its underlying~{\vectorspace{$\kf$}}.
    
    \item[Description via a basis:]
      If a basis~$(v_i)_{i \in I}$ of~$V$ is choosen then every tensor power~$V^{\tensor d}$ inherits a basis that is given by all simple tensors
      \[
        v_{i_1} \tensor \dotsb \tensor v_{i_d}
      \]
      with~$i_1, \dotsc, i_d \in I$.
      It follows that the tensor power has as a basis of all such simple tensors with~$d \geq 0$ and~$i_{i_1, \dotsc, i_d} \in I$.
      The product of two such basis vectors is again a basis vector.
      So we may think about the basis vectors as finite words~$i_1 \dotsm i_d$ in the alphabet~$I$, and as the multiplication of two basis vectors as the concatenation of the corresponding words.
      
      If we think about the basis vector~$v_i$ of~$V$ as a formal variable~$X_i$ then we see that the tensor algebra~$\Tensor(V)$ is isomorphic to the noncommutative polynomial ring~$\kf\gen{X_i \suchthat i \in I}$.
      This noncommutative polynomial ring is also the free~{\algebra{$\kf$}} on the generators~$X_i$ with~$i \in I$, while~$V$ is the free~{\vectorspace{$\kf$}} on the letters~$i \in I$.
      This gives another explanation for why~$\Tensor(V)$ is the free~{\algebra{$\kf$}} on the vector space~$V$.
      More exicitely, we have the following commutative diagram of forgetful functors:
      \[
        \begin{tikzcd}
          \cVect{\kf}
          \arrow{d}
          &
          \cAlg{\kf}
          \arrow{l}
          \arrow{dl}
          \\
          \cSet
          &
          {}
        \end{tikzcd}
      \]
      It then follows that the resulting diagram of left adjoint functors
      \[
        \begin{tikzcd}
          \cVect{\kf}
          \arrow{r}[above]{\Tensor}
          &
          \cAlg{\kf}
          \\
          \cSet
          \arrow{u}[left]{F}
          \arrow{ur}[below right]{\kf\gen{X_i \suchthat i \in (-)}}
          &
          {}
        \end{tikzcd}
      \]
      commutes up to natural isomorphism.
      Hence~$\Tensor(V) \cong \Tensor(F(I)) \cong \kf\gen{X_i \suchthat i \in I}$.
  \end{description}
\end{recall}





\subsection{Reviewing the Symmetric Algebra}


\begin{recall}[Symmetric power]
  Let~$V$ be a vector space and let~$d \geq 0$.
  The~{\howmanyth{$d$}} \defemph{symmetric power}\index{symmetric!power}~$\Symm^d(V)$ is the quotient vector space of the tensor power~$\Symm^d(V)$ by the~{\linear{$\kf$}} subspace~$U_d$ that is generated by all all differences
  \[
      v_1 \tensor \dotsb \tensor v_d
    - v_{\sigma(1)} \tensor \dotsb \tensor v_{\sigma(d)}
  \]
  where~$v_1 \tensor \dotsb \tensor v_d \in V^{\tensor d}$ is a simple tensor and~$\sigma \in S_n$ is a permuation.
  Hence
  \begin{align*}
    \Symm^d(V)
    &=
    V^{\tensor d} / U_d
    \\
    &=
    V^{\tensor d}
    /
    \gen{
        v_1 \tensor \dotsb \tensor v_d
      - v_{\sigma(1)} \tensor \dotsb \tensor v_{\sigma(d)} 
    \suchthat
      v_1, \dotsc, v_n \in V,
      \sigma \in S_n
    }_{\kf} \,.
  \end{align*}
  Observe that~$\Symm^0(V) = V^{\tensor 0} = \kf$ because~$U_0 = 0$.
  For~$v_1, \dotsc, v_n \in V$ we denote the residue class of the simple tensor~$v_1 \tensor \dotsb \tensor v_d$ in~$\Symm^d(V)$ by~$v_1 \dotsm v_d$, and call this a \defemph{symmetric simple tensor}\index{symmetric!simple tensor}.
  
  We have by construction of~$\Symm^d(V)$ that
  \[
    v_1 \dotsm v_d
    =
    v_{\sigma(1)} \dotsm v_{\sigma(d)}
  \]
  for all simple symmetric tensors~$v_1 \dotsm v_n \in \Symm^d(V)$ and permutations~$\sigma \in S_d$, and for~$d \geq 1$ the symmetric power~$\Symm^d(V)$ is universal with this property in the following sense:
  The map
  \[
    V^{\times d}
    \to
    \Symm^d(V)  \,,
    \quad
    (v_1, \dotsc, v_d)
    \mapsto
    v_1 \dotsm v_d
  \]
  is symmetric and multilinear, and if~$f \colon V^{\times d} \to W$ is any other symmetric multilinear map into any vector space~$W$ then there exists a unique linear map~$g \colon \Symm^d(V) \to W$ that makes the triangular diagram
  \[
    \begin{tikzcd}
      V^{\times d}
      \arrow{d}
      \arrow{dr}[above right]{f}
      &
      {}
      \\
      \Symm^d(V)
      \arrow[dashed]{r}[below]{g}
      &
      W
    \end{tikzcd}
  \]
  commute.
  A linear map~$\Symm^d(V) \to W$ is in this sense the same as a symmetric bilinear map~$V^{\times d} \to W$.
  
  If~$(v_i)_{i \in I}$ is a basis of~$V$ such that~$(I, \leq)$ is a linearly ordered set then the ordered monomials
  \[
    v_{i_1} \dotsm v_{i_d}
    \qquad
    \text{with~$i_1 \leq \dotsb \leq i_d$}
  \]
  form a basis of the symmetric power~$\Symm^d(V)$.
  If~$V$ is of finite dimension~$n$ then it follows that
  \[
    \dim \Symm^d(V)
    =
    \binom{n+d-1}{d}  \,.
  \]
\end{recall}


\begin{recall}[Symmetric algebra]
  Let~$V$ be a vector space.
  Just as the tensor algebra~$\Tensor(V)$ is the free~{\algebra{$\kf$}} on~$V$ and can be constructed by using the tensor powers~$V^{\tensor d}$ we can use the symmetric powers~$\Symm^d(V)$ to construct the \defemph{symmetric algebra}\index{symmetric algebra}~\gls*{symmetric algebra}.
  The argumentation is analogous to that for the tensor algebra, so we will skip some of the details this time.
  
  \begin{description}
    \item[Construction:]
      For all~$v_1, \dotsc, v_d \in V$ we denote the corresponding simple symmetric tensor in~$\Symm^d(V)$ by~$v_1 \dotsm v_d$.
      We can define on~$\Symm(V) \defined \bigoplus_{d \geq 0} \Symm^d(V)$ a multiplication such that
      \[
        (v_1 \dotsm v_p) \cdot (v_{p+1} \dotsm v_{p+q})
        =
        v_1 \dotsm v_p v_{p+1} \dotsm v_{p+q}
      \]
      for all~$p, q \geq 0$ and all simple symmetric tensors~$v_1 \dotsm v_p \in \Symm^{\spacing p}(V)$ and~$v_{p+1}, \dotsc, v_{p+q} \in \Symm^q(V)$.
      By identifying~$\Symm^0(V)$ with the ground field~$\kf$ this makes~$\Symm(V)$ into an associative~{\algebra{$\kf$}}.
      This is already a commutative~{\algebra{$\kf$}} because
      \begin{align*}
        (v_1 \dotsm v_p) \cdot (v_{p+1} \dotsm v_{p+q})
        &=
        v_1 \dotsm v_p v_{p+1} \dotsm v_{p+q}
        \\
        &=
        v_{p+1} \dotsm v_{p+q} v_1 \dotsm v_p
        \\
        &=
        (v_{p+1} \dotsm v_{p+q}) \cdot (v_1 \dotsm v_p)
      \end{align*}
      for all~$p, q \geq 0$ and all simple symmetric tensors~$v_1 \dotsm v_p \in \Symm^{\spacing p}(V)$ and~$v_{p+1}, \dotsc, v_{p+q} \in \Symm^q(V)$. 
      We can identify~$V = \Symm^1(V)$ with the corresponding direct summand of~$\Symm(V)$, and more generally every symmetric power~$\Symm^d(V)$ with the corresponding direct summand of~$\Symm(V)$.
      The algebra~$\Symm(V)$ thus consists of linear combinations of simple symmetric tensors.
      
    \item[Universal property:]
      The symmetric algebra~$\Symm(V)$ is the \enquote{free commutative~{\algebra{$\kf$}}} on the vector space~$V$ in the following sense:
      If~$i \colon V \to \Symm(V)$ is the inclusion then there exists for every~{\algebra{$\kf$}}~$A$ and every linear map~$f \colon V \to A$ a unique algebra homomorphism~$f^+ \colon \Symm(V) \to A$ that makes the triangular diagram
      \[
        \begin{tikzcd}
          V
          \arrow{r}[above]{f}
          \arrow{d}[left]{i}
          &
          A
          \\
          \Symm(V)
          \arrow{ur}[below right]{f^+}
          &
          {}
        \end{tikzcd}
      \]
      commute.
      The algebra homomorphism~$f^+$ is given by
      \[
        f^+(v_1 \dotsm v_d)
        =
        f(v_1) \dotsm f(v_d)
      \]
      for all~$d \geq 0$ and simple symmetric tensors~$v_1, \dotsc, v_d \in V$.
      This construction results in a {\onetoone} correspondence
      \begin{align*}
        \{ \text{\linear{$\kf$} maps~$V \to A$} \}
        &\longonetoone
        \{ \text{algebra homomorphisms~$\Symm(V) \to A$} \} \,,
        \\
        f
        &\longmapsto
        f^+ \,,
        \\
        \restrict{F}{V}
        &\longmapsfrom
        F \,.
      \end{align*}
      
      It follows that a relations between elements of~$V$ holds in the symmetric algebra~$\Symm(V)$ if and only if it holds in every commutative algebra that contains~$V$.
      
    \item[Uniqueness]
      If~$S$ is a commutative~{\algebra{$\kf$}} and~$j \colon V \to S$ is a~{\linear{$\kf$}} such that~$(S, j)$ satisfies the same universal property as the symmetric algebra~$(\Symm(V), i)$  then there exists unique algebra homomorphisms~$f \colon \Symm(V) \to S$ and~$g \colon S \to \Symm(V)$ that make the triangular diagrams
      \[
        \begin{tikzcd}[column sep = small]
          {}
          &
          V
          \arrow{dl}[above left]{i}
          \arrow{dr}[above right]{j}
          &
          {}
          \\
          \Symm(V)
          \arrow[dashed]{rr}[below]{f}
          &
          {}
          &
          S
        \end{tikzcd}
        \qquad\text{and}\qquad
        \begin{tikzcd}[column sep = small]
          {}
          &
          V
          \arrow{dl}[above left]{j}
          \arrow{dr}[above right]{i}
          &
          {}
          \\
          S
          \arrow[dashed]{rr}[below]{g}
          &
          {}
          &
          \Symm(V)
        \end{tikzcd}
      \]
      commute.
      Then~$f$ and~$g$ are mutually inverse algebra isomorphisms.
      
    \item[Functoriality]
      For every linear map~$f \colon V \to W$ there exists a unique induced algebra homomorphism~$f_* \colon \Symm(V) \to \Symm(W)$ that makes the square diagram
      \[
        \begin{tikzcd}
          V
          \arrow{r}[above]{f}
          \arrow{d}
          &
          W
          \arrow{d}
          \\
          \Symm(V)
          \arrow[dashed]{r}[below]{f_*}
          &
          \Symm(W)
        \end{tikzcd}
      \]
      commmute.
      It holds that~$(\id_V)_* = \id_{\Symm(V)}$ and it holds for all composable~{\linear{$\kf$}} maps~$f \colon U \to V$ and~$g \colon V \to W$ that~$(\spacing g \circ f)_* = g_* \circ f_*$.
      This construction promotes the assignment~$V \mapsto \Symm(V)$ to a (covariant) functor~$\Symm \colon \cVect{\kf} \to \cCAlg{\kf}$, where~$\cCAlg{\kf}$ denotes the category of commutative~{\algebras{$\kf$}}.
      
    \item[Description via a basis]
      If~$(v_i)_{i \in I}$ is a basis of$~V$ where~$(I, \leq)$ is a linearly ordered set then the symmetric power~$\Symm^d(V)$ inherits a basis that is given by all simple symmetric tensors
      \[
        v_{i_1} \dotsm v_{i_d}
        \qquad
        \text{where~$i_1 \leq \dotsb \leq i_d$} \,.
      \]
      It follows that the symmetric algebra~$\Symm(V)$ has as a basis all simple symmetric tensors~$v_{i_1} \dotsm v_{i_d}$ with~$d \geq 0$ and~$i_1, \dotsc, i_d \in I$ with~$i_1 \leq \dotsb \leq i_d$.
      This basis may also be written as
      \[
        v_{i_1}^{\nu_1} \dotsm v_{i_r}^{\nu_r}
      \]
      with~$r \geq 0$,~$i_1, \dotsc, i_r \in I$ such that~$i_1 < \dotsb < i_r$ and~$\nu_1, \dotsc, \nu_r \geq 0$ (which is connected to the above description via~$d = \nu_1 + \dotsb + \nu_r$).
      
      We see from this description that the symmetric algebra~$\Symm(V)$ is isomorphic to the commutative polynomial ring~$\kf[X_i \suchthat i \in I]$, which is the free commutative~{\algebra{$\kf$}} on the generators~$i \in I$
      This can again be explained by considering the commutative diagram of forgetful functors
      \[
        \begin{tikzcd}
          \cVect{\kf}
          \arrow{d}
          &
          \cCAlg{\kf}
          \arrow{l}
          \arrow{dl}
          \\
          \cSet
          &
          {}
        \end{tikzcd}
      \]
      from which we see that the resulting diagram of left adjoints
      \[
        \begin{tikzcd}
          \cVect{\kf}
          \arrow{r}[above]{\Symm}
          &
          \cCAlg{\kf}
          \\
          \cSet
          \arrow{u}[left]{F}
          \arrow{ur}[below right]{\kf[X_i \suchthat i \in (-)]}
          &
          {}
        \end{tikzcd}
      \]
      commutes up to natural isomorphism.
      
    \item[Contruction via the tensor algbra]
      The symmetric algebra~$\Symm(V)$ can also be constructed as a quotient of the tensor algebra~$\Tensor(V)$.
      We give multiple ways how to see and think about this.
      Let in the following~$i \colon V \to \Tensor(V)$ and~$j \colon V \to \Symm(V)$ denote the 
      \begin{itemize}
        \item
          Let~$I$ be the commutator ideal of~$\Tensor(V)$, i.e.\ the two-sided ideal generated by all commutators
          \[
            x \tensor y - y \tensor x
          \]
          with~$x, y \in \Tensor(V)$.
          Let~$\pi \colon \Tensor(V) \to \Tensor(V)/I$ be the canonical projection.
          Then the quotient~$\Tensor(V)/I$ is commutative, and hence there exists by the universal property of the symmetric algebra a unique algebra homomorphism~$f \colon \Symm(V) \to \Tensor/I$ that makes the diagram
          \[
            \begin{tikzcd}
              {}
              &
              V
              \arrow[bend right]{ddl}[above left]{i}
              \arrow[bend left]{dr}[above right]{j}
              &
              {}
              \\
              {}
              &
              {}
              &
              \Tensor(V)
              \arrow{d}[right]{\pi}
              \\
              \Symm(V)
              \arrow[dashed]{rr}[above]{f}
              &
              {}
              &
              \Tensor(V)/I
            \end{tikzcd}
          \]
          commute.
          The homomorphism~$f$ is on the genareting set~$V$ of~$\Symm(V)$ given by~$f(v) = \class{v}$.
          On the other hand we get from the universal property of the tensor algebra~$\Tensor(V)$ a unique algebra homomorphism~$\tilde{g} \colon \Tensor(V) \to \Symm(V)$ that makes the diagram
          \[
            \begin{tikzcd}
              {}
              &
              V
              \arrow[bend right]{dl}[above left]{j}
              \arrow[bend left]{ddr}[above right]{i}
              &
              {}
              \\
              \Tensor(V)
              \arrow[bend left, dashed]{drr}[above right]{\tilde{g}}
              \arrow{d}[left]{\pi}
              &
              {}
              &
              {}
              \\
              \Tensor(V)/I
              &
              {}
              &
              \Symm(V)
            \end{tikzcd}
          \]
          commute.
          The commutator~$I$ is contained in the kernel of~$\tilde{g}$ because the algebra~$\Symm(V)$ is commutative.
          Hence there exists a unique algebra homomorphism~$g \colon \Tensor(V)/I \to \Symm(V)$ that makes the diagram
          \[
            \begin{tikzcd}
              {}
              &
              V
              \arrow[bend right]{dl}[above left]{j}
              \arrow[bend left]{ddr}[above right]{i}
              &
              {}
              \\
              \Tensor(V)
              \arrow[bend left]{drr}[above right]{\tilde{g}}
              \arrow{d}[left]{\pi}
              &
              {}
              &
              {}
              \\
              \Tensor(V)/I
              \arrow[dashed]{rr}[below]{g}
              &
              {}
              &
              \Symm(V)
            \end{tikzcd}
          \]
          commute.
          The algebra homomorphism~$g$ is given on the generators~$\class{v}$ with~$v \in V$ of~$\tensor(V)/I$ given by~$g(\class{v}) = v$.
          
          It follows from the explicit descriptions of~$f$ and~$g$ on generators that their are mutually inverse algebra isomorphisms.
          Thus~$\Symm(V) \cong \Tensor(V)/I$ via the isomorphism~$f$.
          
          Observe also that the commutator ideal~$I$ is already generated by the commutators~$v \tensor w - w \tensor v$ with~$v, w \in V$.
          Indeed, the ideal~$J$ generated by these elements is contained in~$I$.
          But on the other hand the quotient~$\Tensor(V)/J$ is already commutative because it is generated by the residue classes~$\class{v}$ with~$v \in V$, all of which commute with each other.
          The commutator ideal~$I$ is therefore contained in the kernel of the canonical projection~$\Tensor(V) \to \Tensor(V)/J$, i.e.\ it is containted in~$J$.
          
        \item
          The above argumentatio is not surprising if we remember that~$\Tensor(V)$ is the universal~{\algebra{$\kf$}} on~$V$ and that quotiening out the commutator ideal is the universal way of making an algebra commutative.
          The quotient~$\Tensor(V)/I$ therefore ought to be the universal commutative~{\algebra{$\kf$}}.
          
          This motivation can be formalized by observing that the diagram of forgetful functors
          \[
            \begin{tikzcd}
              \cAlg{\kf}
              \arrow{d}
              &
              \cCAlg{\kf}
              \arrow{l}
              \arrow{dl}
              \\
              \cVect{\kf}
              &
              {}
            \end{tikzcd}
          \]
          commutes.
          It follows that the resulting diagram of left adjoints
          \[
            \begin{tikzcd}
              \cAlg{\kf}
              \arrow{r}[above]{C}
              &
              \cCAlg{\kf}
              \\
              \cVect{\kf}
              \arrow{u}[left]{\Tensor}
              \arrow{ur}[below right]{\Symm}
              &
              {}
            \end{tikzcd}
          \]
          commutes up to natural isomorphism.
          The adjoint~$C$ of the forgetful functor~$\cCAlg{\kf} \to \cAlg{\kf}$ is given by quotiening out the commutator ideal, and hence~$\Symm(V) \cong \Tensor(V)/I$ as before.
        \item
          The above argumentation be also expressed by observing that for every commutative~{\algebra{$\kf$}}~$A$ there exist natural bijections
          \begin{align*}
            {}&
            \{ \text{algebra homomorphisms~$\Symm(V) \to A$} \}
            \\
            \cong{}&
            \{ \text{{\linear{$\kf$}} maps~$V \to A$} \}
            \\
            \cong{}&
            \{ \text{algebra homomorphisms~$\Tensor(V) \to A$} \}
            \\
            \cong{}&
            \{ \text{algebra homomorphisms~$\Tensor(V)/I \to A$} \} \,,
          \end{align*}
          where the last bijection uses that the algebra~$A$ is commutative and therefore every algebra homomorphism~$\Tensor(V) \to A$ contains the commutator ideal~$I$ in its kernel.
          It now follows from Yoneda’s~lemma that~$\Symm(V) \cong \Tensor(V)/I$.
      \end{itemize}
  \end{description}
\end{recall}


\begin{remark}  % TODO: Fix glossary and bigwedge
  One can similarly construct the \emph{exterior algebra}~$\gls*{exterior algebra} = \bigoplus_{d \geq 0} \Exterior^d(V)$ of a vector space~$V$ by replacing the use of the tensor powers~$V^{\tensor d}$ or symmetric powers~$\Symm^d(V)$ by the exterior powers~$\Exterior^d(V)$.
  For any other~{\algebra{$\kf$}}~$A$ an algebra homomorphism~$F \colon \Exterior(V) \to A$ is then the same as a~{\linear{$\kf$}}~$f \colon V \to A$ with~$f(v)^2 = 0$ for every~$v \in V$.
% TODO: Do we have to worry about char(k) = 2?
  It thus follows from a similar argumentation as for the symmetric algebra that~$\Exterior(V) \cong \Tensor(V)/I$ for the two-sided ideal~$I$ in~$\Tensor(V)$ generated by all~$v \tensor v$ with~$v \in V$.
  
  
  If~$V$ is finite dimensional then the exterior algebra~$\Exterior(V)$ is again finite dimensional, namely with~$\dim \Exterior(V) = 2^{\dim V}$.
  This is different to both the tensor algebra~$\Tensor(V)$ and symmetric algebra~$\Symm(V)$, which are infinite dimensional whenever~$V \neq 0$.
\end{remark}





\section{Universal Enveloping Algebra}





\subsection{Definition}


\begin{definition}
  Let~$\glie$ be a~\liealgebra{$\kf$}.
  A \defemph{universal enveloping algebra}\index{universal enveloping algebra} of~$\glie$ is a~\algebra{$\kf$}~$\Univ(\glie)$\glsadd{universal enveloping algebra} together with a homomorphism of Lie~algebras~$\iota$ from~$\glie$ to~$\Univ(\glie)$ such that the following universal property holds:
  for every~{\algebra{$\kf$}}~$A$ and every homomorphism of Lie~algebras~$\varphi$ from~$\glie$ to~$A$ there exists a unique homomorphism of~\algebras{$\kf$}~$\Phi$  from~$\Univ(\glie)$ to~$A$ that makes the triangular diagram
  \[
    \begin{tikzcd}
      \glie
      \arrow{r}[above]{\phi}
      \arrow{d}[left]{\iota}
      &
      A
      \\
      \Univ(\glie)
      \arrow[dashed]{ur}[below right]{\Phi}
      &
      {}
    \end{tikzcd}
  \]
  commute, i.e.\ such that~$\varphi = \Phi \circ \iota$.
\end{definition}


\begin{remark}[Uniqueness of universal enveloping algebras]
  \label{uniqueness of universal enveloping algebras}
  Let~$\glie$ be a Lie algebra and suppose that~$(\Univ(\glie)_1, \iota_1)$ and~$(\Univ(\glie)_2, \iota_2)$ are two~{\uas} of~$\glie$.
  Then there exist unique algebra homomorphisms~$\Phi$ from~$\Univ(\glie)_1$ to~$\Univ(\glie)_2$ and~$\Psi$ from~$\Univ(\glie)_2$ to~$\Univ(\glie)_1$ that make the triangular diagrams
  \[
    \begin{tikzcd}[column sep = small]
      {}
      &
      \glie
      \arrow{dl}[above left]{\iota_1}
      \arrow{dr}[above right]{\iota_2}
      &
      {}
      \\
      \Univ(\glie)_1
      \arrow[dashed]{rr}[below]{\Phi}
      &
      {}
      &
      \Univ(\glie)_2
    \end{tikzcd}
    \qquad\text{and}\qquad
    \begin{tikzcd}[column sep = small]
      {}
      &
      \glie
      \arrow{dl}[above left]{\iota_2}
      \arrow{dr}[above right]{\iota_1}
      &
      {}
      \\
      \Univ(\glie)_2
      \arrow[dashed]{rr}[below]{\Psi}
      &
      {}
      &
      \Univ(\glie)_1
    \end{tikzcd}
  \]
  commute.
  It follows that the composites~$\Psi \circ \Phi$ and~$\Phi \circ \Psi$ make the triangle diagrams
  \[
    \begin{tikzcd}[column sep = small]
      {}
      &
      \glie
      \arrow{dl}[above left]{\iota_1}
      \arrow{dr}[above right]{\iota_1}
      &
      {}
      \\
      \Univ(\glie)_1
      \arrow[dashed]{rr}[below]{\Psi \circ \Phi}
      &
      {}
      &
      \Univ(\glie)_1
    \end{tikzcd}
    \qquad\text{and}\qquad
    \begin{tikzcd}[column sep = small]
      {}
      &
      \glie
      \arrow{dl}[above left]{\iota_2}
      \arrow{dr}[above right]{\iota_2}
      &
      {}
      \\
      \Univ(\glie)_2
      \arrow[dashed]{rr}[below]{\Phi \circ \Psi}
      &
      {}
      &
      \Univ(\glie)_2
    \end{tikzcd}
  \]
  commute.

  The algebra homomorphisms~$\Phi \circ \Psi$ and~$\Psi \circ \Phi$ are unique with this property by the universal properties of the {\uas}~$(\Univ(\glie)_1, \iota_1)$ and~$(\Univ(\glie)_2, \iota_2)$.
  But the identities~$\id_{\Univ(\glie)_1}$ and~$\id_{\Univ(\glie)_2}$ also makes these diagrams commute.
  We thus find that the composite~$\Psi \circ \Phi$ equals~$\id_{\Univ(\glie)_1}$ and the composite~$\Phi \circ \Psi$ equals~$\id_{\Univ(\glie)_2}$.
  The homomorphisms~$\Phi$ and~$\Psi$ are therefore mutually inverse isomorphisms.
  
  This shows that a {\ua} of~$\glie$ is unique up to unique isomorphism.
  We will therefore talk about \emph{the} {\ua} of~$\glie$.
  We will often also surpress the algebra homorphism~$\iota$ from~$\glie$ to~$\Univ(\glie)$ from our notation.
\end{remark}


% \begin{remark}
%   One can also formulate the above argument is a more categorical way:
%   Consider the category~$\catC$ where
%   \begin{itemize}
%     \item
%       objects of~$\catC$ is a pairs~$(A, i)$ consisting of a~{\algebra{$\kf$}}~$A$ and a Lie~algebra homomorphism~$i \colon \glie \to A$,
%     \item
%       a morphism~$\phi \colon (A, i) \to (B, j)$ is an algebra homomorphism~$\phi \colon A \to B$ that makes the triangular diagram
%       \[
%         \begin{tikzcd}[column sep = small]
%         {}
%         &
%         \glie
%         \arrow{dl}[above left]{i}
%         \arrow{dr}[above right]{j}
%         &
%         {}
%         \\
%         A
%         \arrow[dashed]{rr}[below]{\phi}
%         &
%         {}
%         &
%         B
%       \end{tikzcd}
%     \]
%       commute, and
%     \item
%       the composition of two morphisms is just their usual set-theoretic composition.
%   \end{itemize}
%   A {\ua} of~$\glie$ is nothing but an inital object in this category~$\catC$.
%   The argumentation from \cref{uniqueness of universal enveloping algebras} is then the usual argument for the uniqueness of inital objects up to unique isomorphism.
% \end{remark}



\subsection{Construction}


\begin{fluff}
  Let~$\glie$ be a Lie~algebra.
  We will in the following show that the {\ua} of~$\glie$ exists.
  For this we will first conclude from the universal property of~$\Univ(\glie)$ that we should be able to construct~$\Univ(\glie)$ as a certain quotient algebra of the tensor algebra~$\Tensor(\glie)$.
  We then show that this quotient does indeed have the correct universal property.

  Suppose that~$\glie$ admits a universal enveloping algebra~$\Univ(\glie)$ and let~$\iota$ be the canonical homomorphism of Lie~algebras from~$\glie$ to~$\Univ(\glie)$.
  We first observe that the algebra~$\Univ(\glie)$ is generated by the image of~$\iota$.

  Indeed, let~$U$ be the subalgebra of~$\Univ(\glie)$ which is generated by the image of~$\iota$, and let~$\iota'$ be the restriction of~$\iota$ to a homomorphism of Lie~algebras from~$\glie$ to~$U$.
  For every~{\algebra{$\kf$}}~$A$ and every Lie~algebra homomorphism~$\varphi$ from~$\glie$ to~$A$ the induced homomorphism of algebras~$\Phi$ from~$\Univ(\glie)$ to~$A$ restricts to an homomorphism of algebras~$\Phi'$ from~$U$ to~$A$.
  This homomorphism~$\Phi'$ makes the triangular diagram
  \[
    \begin{tikzcd}
      \glie
      \arrow{r}[above]{\varphi}
      \arrow{d}[left]{\iota'}
      &
      A
      \\
      U
      \arrow[dashed]{ur}[below right]{\Phi'}
      &
      {}
    \end{tikzcd}
  \]
  commute.
  The homomorphism~$\Phi'$ is unique with this property because the algebra~$U$ is generated by the image of~$\iota'$, and the composite~$\Phi \circ \iota'$ equals the fixed homomorphism of Lie~algebras~$\varphi$.
  This shows that the algebra~$U$ together with the homomorphism of Lie~algebras~$\iota'$ is again a {\ua} for~$\glie$.
  
  It follows from the uniqueness of the {\ua} of~$\glie$, as discussed in \cref{uniqueness of universal enveloping algebras}, that there exists a unique homomorphism of algebras~$\Iota$ from~$U$ to~$\Univ(\glie)$ that makes the triangular diagram
  \[
    \begin{tikzcd}[column sep = small]
      {}
      &
      \glie
      \arrow{dl}[above left]{\iota'}
      \arrow{dr}[above right]{\iota}
      &
      {}
      \\
      U
      \arrow[dashed]{rr}[below]{\Iota}
      &
      {}
      &
      \Univ(\glie)
    \end{tikzcd}
  \]
  commute, and that this homomorphism is already an isomorphism.
  The homomorphism~$\Iota$ is the inclusion map from~$U$ to~$\Univ(\glie)$ because this is a homomorphism of algebras from~$U$ to~$\Univ(\glie)$ which makes the above triangular diagram commute.
  We have thus found that the inclusion map from~$U$ to~$\Univ(\glie)$ is an isomorphism of algebras, whence~$U$ equals~$\Univ(\glie)$.

  We now apply the universal property of the tensor algebra~$\Tensor(\glie)$ to the linear map~$\iota$.
  We find that there exists a unique homomorphism of algebras~$\Phi'$ from~$\Tensor(\glie)$ to~$\Univ(\glie)$ that makes the triangular diagram
  \[
    \begin{tikzcd}[column sep = small]
      {}
      &
      \glie
      \arrow{dl}
      \arrow{dr}[above right]{\iota}
      &
      {}
      \\
      \Tensor(\glie)
      \arrow[dashed]{rr}[below]{\Phi'}
      &
      {}
      &
      \Univ(\glie)
    \end{tikzcd}
  \]
  commute.
  The homomorphism~$\Phi'$ is surjective because~$\Univ(\glie)$ is generated by the image of~$\iota$ as an algebra.
  It follows that~$\Phi'$ induces an isomorphism of algebras
  \[
    \Psi
    \colon
    \Tensor(\glie) / I
    \to
    \Univ(\glie)
  \]
  where the ideal~$I$ is the kernel of~$\Phi'$.
  This isomorphism makes the resulting diagram
   \[
    \begin{tikzcd}[column sep = small]
      {}
      &
      \glie
      \arrow[bend right]{dl}
      \arrow[bend left]{ddr}[above right]{\iota}
      &
      {}
      \\
      \Tensor(\glie)
      \arrow[bend left]{drr}[below left]{\Phi}
      \arrow{d}
      &
      {}
      &
      {}
      \\
      \Tensor(\glie)/I
      \arrow[dashed]{rr}[below]{\Psi}
      &
      {}
      &
      \Univ(\glie)
    \end{tikzcd}
  \]
  commute.
 
  Let~$A$ be another~\algebra{$\kf$}.
  Every linear map~$g$ from~$\glie$ to~$A$ factors through a homomorphism of algebras~$\Psi'$ from~$\Tensor(\glie)$ to~$A$.
  It follows from the above isomorphism~$\Phi$ between~$\Tensor(\glie) / I$ and~$\Univ(\glie)$ that the homomorphism~$\Psi'$ factors trough a homorphism from~$\Tensor(\glie) / I$ to~$A$ if and only if the linear map~$g$ is a homomorphism of Lie~algebras.

  That~$g$ is a homomorphism of Lie~algebras means that
  \[
    g(x) g(y) - g(y) g(x) - g([x,y]) = 0
  \]
  for all~$x, y \in \glie$.
  This is equivalent to the condition
  \[
    \Psi(x) \Psi(y) - \Psi(y) \Psi(x) - \Psi([x,y]_{\glie})
    =
    0
  \]
  for all~$x, y \in \glie$, and further äquivalent to the condition
  \[
    \Psi( xy - yx - [x,y]_{\glie} )
    =
    0
  \]
  for all~$x, y \in \glie$.

  We have now seen that an algebra homomorphism~$\Psi$ from~$\Tensor(\glie)$ to some~\algebra{$\kf$}~$A$ factors trough the quotient~$\Tensor(\glie)/I$ if and only if~$\Psi$ annihilates all those elements of~$\Tensor(\glie)$ that are of the form~$xy - yx - [x,y]_{\glie}$ with~$x$,~$y$ in~$\glie$.
  This means that the ideal~$I$ needs to be generated by those elements.
  
  We have now altogether seen that the universal enveloping algebra~$\Univ(\glie)$ needs to be constructable as the quotient of the tensor algebra~$\Tensor(\glie)$ by the ideal~$I$ which is generated by all those elements of the form~$x y - y x - [x,y]_{\glie}$ with~$x$,~$y$ in~$\glie$.
  We will conversely show in the following \lcnamecref{existence of uea} that this construction will indeed give us the universal enveloping algebra.
\end{fluff}


\begin{proposition}[Existence of the universal enveloping algebra]
  \label{existence of uea}
  Let~$\glie$ be a Lie~algebra.
  Let~$\Tensor(\glie)$ be the tensor algebra of the underlying vector space of~$\glie$ and let~$I$ the two-sided ideal of~$\Tensor(\glie)$ generated by all the elements $x y - y x - [x,y]_{\glie}$ with~$x$,~$y$ in~$\glie$.
  The quotient algebra~$U \defined T(\glie)/I$ together with the~{\linear{$\kf$}} map
  \[
    \iota
    \colon
    \glie
    \to
    \Univ(\glie) \,,
    \quad
    x
    \mapsto
    \class{x}
  \]
  is a {\ua} for~$\glie$.
\end{proposition}


\begin{proof}
  The map~$\iota$ is~{\linear{$\kf$}} and it compatible with the Lie brackets because
  \[
    [\iota(x), \iota(y)]
    =
    [\class{x}, \class{y}]
    =
    \class{x} \, \class{y} - \class{y} \, \class{x}
    =
    \class{x y - y x}
    =
    \class{[x,y]_{\glie}}
    =
    \iota([x,y]_{\glie}) \,.
  \]
  for all~$x, y \in \glie$.
  Given any~\algebra{$\kf$}~$A$ and Lie algebra homomorphism~$\varphi$ from~$\glie$ to~$A$ there exists a unique homorphism of~\algebras{$\kf$}~$\Phi'$ from~$\Tensor(\glie)$ to~$A$ that makes the triangular diagram
  \[
    \begin{tikzcd}
      \glie
      \arrow{r}[above]{\varphi}
      \arrow{d}
      &
      A
      \\
      \Tensor(V)
      \arrow[dashed]{ur}[below right]{\Phi'}
      &
      {}
    \end{tikzcd}
  \]
  commute.
  The homomorphism~$\Phi'$ is given by~$\Phi'(x) = \varphi(x)$ for all~$x \in \glie$.
  It follows that
  \begin{align*}
    \Phi'(x y - y x)
    &=
    \Phi'(x) \Phi'(y) - \Phi'(y) \Phi'(x)
    \\
    &=
    \varphi(x) \varphi(y) - \varphi(y) \varphi(x)
    \\
    &=
    [ \varphi(x), \varphi(y) ]
    \\
    &=
    \varphi( [x,y]_{\glie} )
    \\
    &=
    \Phi'( [x,y]_{\glie} )
  \end{align*}
  for all~$x, y \in \glie$.
  The ideal~$I$ is therefore contained in the kernel of the homomorphism~$\Phi'$.
  It follows that there exists a unique homomorphism of algebras~$\Phi$ from~$U$ to~$A$ that makes the diagram
  \[
    \begin{tikzcd}
      \glie
      \arrow{r}[above]{\varphi}
      \arrow{d}
      &
      A
      \\
      \Tensor(\glie)
      \arrow[bend right= 20]{ur}[above left]{\Phi'}
      \arrow{d}[left]{\pi}
      &
      {}
      \\
      U
      \arrow[dashed, bend right = 30]{uur}[below right]{\Phi}
      &
      {}
    \end{tikzcd}
  \]
  commute, where~$\Pi$ denotes the canonical projection from~$\Tensor(V)$ to~$\Tensor(V)/I$.
  We may add the homomorphism of Lie~algebras~$\iota$ to this diagram.
  We then arrive at the following commutative diagram.
  \[
    \begin{tikzcd}
      \glie
      \arrow{r}[above]{\varphi}
      \arrow{d}
      \arrow[bend right = 55]{dd}[left]{\iota}
      &
      A
      \\
      \Tensor(\glie)
      \arrow[bend right= 20]{ur}[above left]{\Phi'}
      \arrow{d}[left]{\pi}
      &
      {}
      \\
      U
      \arrow[bend right = 30]{uur}[below right]{\Phi}
      &
      {}
    \end{tikzcd}
  \]
  We have in particular the following commutative subdiagram.
  \[
    \begin{tikzcd}
      \glie
      \arrow{r}[above]{\varphi}
      \arrow{d}[left]{i}
      &
      A
      \\
      U
      \arrow{ur}[below right]{\Phi}
      &
      {}
    \end{tikzcd}
  \]
  We have thus shown that every homomorphism of Lie~algebras~$\varphi$ from~$\glie$ to~$A$ extends to a homomorphism of algebra~$\Phi$ from~$U$ to~$A$ .
  The algebra~$U$ is generated by the image of~$\iota$ whence the homomorphism of algebras~$\Phi$ is unique with this property.
\end{proof}


\begin{remark}
  The above proof may be summarized by observing that we have bijections
  \begin{align*}
    {}&
    \{ \textstyle\text{algebra homomorphisms~$\Phi \colon \Tensor(\glie)/I \to A$} \}
    \\
    \cong{}&
    \{ \text{algebra homomorphisms~$\Phi' \colon \Tensor(\glie) \to A$ with~$\Phi'(I) = 0$} \}
    \\
    \cong{}&
    \left\{
      \begin{tabular}{@{}c@{}}
        algebra homomorphisms~$\Phi' \colon \Tensor(\glie) \to A$ with  \\
        $\Phi'(x y - y x - [x,y]_{\glie}) = 0$ for all~$x, y \in \glie$
      \end{tabular}
    \right\}
    \\
    \cong{}&
    \left\{
      \begin{tabular}{@{}c@{}}
        algebra homomorphisms~$\Phi' \colon \Tensor(\glie) \to A$ with  \\
        $\Phi'(x) \Phi'(y) - \Phi'(y) \Phi'(x) - \Phi'([x,y]_{\glie}) = 0$ for all~$x, y \in \glie$
      \end{tabular}
    \right\}
    \\
    \cong{}&
    \left\{
      \begin{tabular}{@{}c@{}}
        algebra homomorphisms~$\Phi' \colon \Tensor(\glie) \to A$ with  \\
        $\Phi'(x) \Phi'(y) - \Phi'(y) \Phi'(x) = \Phi'([x,y]_{\glie})$ for all~$x, y \in \glie$
      \end{tabular}
    \right\}
    \\
    \cong{}&
    \left\{
      \begin{tabular}{@{}c@{}}
        {\linear{$\kf$}} maps~$\varphi \colon \glie \to A$ with  \\
        $\varphi(x) \varphi(y) - \varphi(y) \varphi(x) = \varphi([x,y])$ for all~$x, y \in \glie$
      \end{tabular}
    \right\}
    \\
    \cong{}&
    \left\{
      \begin{tabular}{@{}c@{}}
        {\linear{$\kf$}} maps~$\varphi \colon \glie \to A$ with  \\
        $[\varphi(x), \varphi(y)] = \varphi([x,y])$ for all~$x, y \in \glie$ 
      \end{tabular}
    \right\}
    \\
    ={}&
    \{ \textstyle\text{Lie~algebra homomorphisms~$\varphi \colon \glie \to A$} \} \,,
  \end{align*}
  and that these bijections are natural in~$A$.
  This shows that the~{\algebra{$\kf$}}~$\Tensor(\glie)/I$ represents the right kind of functor.
  We can also see that the identity of~$\Tensor(\glie)/I$ corresponds under the above bijections (for~$A = \Tensor(\glie)/I$) to the map~$\iota$ from~$\glie$ to~$\Tensor(\glie)/I$.
\end{remark}



\subsection{Properties}

\subsubsection{Anti-Homomorphisms}

\begin{proposition}
  Let~$\glie$ be a Lie~algebra, let~$\iota$ be the canonical homomorphism of Lie~algebras from~$\glie$ to~$\Univ(\glie)$ and let~$A$ be a~\algebra{$\kf$}.
  We have a well-defined {\onetoonetext} correspondence given by
  \begin{align*}
    \SwapAboveDisplaySkip
    \left\{
      \begin{tabular}{@{}c@{}}
        anti-homomorphisms \\
        of Lie~algebras
        $\varphi \colon \glie \to A$
      \end{tabular}
    \right\}
    &\onetoone
    \left\{
      \begin{tabular}{@{}c@{}}
        anti-homomorphisms \\
        of algebras
        $\Phi \colon \Univ(\glie) \to A$
      \end{tabular}
    \right\} \,,
    \\
    \Phi \circ \iota
    &\mapsfrom
    \Phi \,.
  \end{align*}
\end{proposition}

\begin{proof}
  We have {\onetoonetext} correspondence given by
  \begin{align*}
    \left\{
      \begin{tabular}{@{}c@{}}
        homomorphisms \\
        of Lie~algebras
        $\varphi \colon \glie \to A^{\op}$
      \end{tabular}
    \right\}
    &\onetoone
    \left\{
      \begin{tabular}{@{}c@{}}
        homomorphisms \\
        of algebras
        $\Phi \colon \Univ(\glie) \to A^{\op}$
      \end{tabular}
    \right\} \,,
    \\
    \Phi \circ \iota
    &\mapsfrom
    \Phi \,.
  \end{align*}
  An anti-homomorphism of Lie~algebras from~$\glie$ to~$A$ is the same as a homomorphim of Lie~algebras from~$\glie$ to~$A^{\op}$, and an anti-homomorphism of algebras from~$\Univ(\glie)$ to~$A$ is the same as a homomorphism of algebras from~$\Univ(\glie)$ to~$A^{\op}$.
\end{proof}

\subsubsection{Representations and Modules}

\begin{proposition}
  \label{representations are modules}
  Let~$M$ be a~{\vectorspace{$\kf$}} and let~$\glie$ be a Lie~algebra.
  Let~$\Univ(\glie)$ be the universal enveloping algebra of~$\glie$ and let~$\iota$ be the canonical homorphism of Lie~algebras from~$\glie$ to~$\Univ(\glie)$.
  \begin{enumerate}
    \item
      For every homomorphism of Lie~algebras~$\rho$ from~$\glie$ to~$\gllie(M)$ let~$\widehat{\rho}$ denote the corresponding homomorphism of algebras from~$\Univ(\glie)$ to~$\End_{\kf}(M)$.
      Then the assignments
      \begin{align*}
        \left\{
        \begin{tabular}{@{}c@{}}
          representations \\
          $\rho \colon \glie \to \gllie(M)$
        \end{tabular}
        \right\}
        &\onetoone
        \left\{
        \begin{tabular}{@{}c@{}}
          $\Univ(\glie)$-module structures \\
          $\Rho \colon \Univ(\glie) \to \End_{\kf}(M)$
        \end{tabular}
        \right\}  \,,
        \\
        \rho
        &\mapsto
        \widehat{\rho} \,,
        \\
        \Rho \circ \iota
        &\mapsfrom
        \Rho  \,,
      \end{align*}
      constitute a {\onetoonetext} correspondence.
    \item
      For every anti-homomorphism of Lie~algebras~$\rho$ from~$\glie$ to~$\gllie(M)$ let~$\widehat{\rho}$ denote the corresponding anti-homomorphism of algebras from~$\Univ(\glie)$ to~$\End_{\kf}(M)$.
      Then the assignments
      \begin{align*}
        \left\{
        \begin{tabular}{@{}c@{}}
          right representations \\
          $\rho \colon \glie \to \gllie(M)$
        \end{tabular}
        \right\}
        &\onetoone
        \left\{
        \begin{tabular}{@{}c@{}}
          right $\Univ(\glie)$-module structures \\
          $\Rho \colon \Univ(\glie) \to \End_{\kf}(M)$
        \end{tabular}
        \right\}  \,,
        \\
        \rho
        &\mapsto
        \widehat{\rho} \,,
        \\
        \Rho \circ \iota
        &\mapsfrom
        \Rho  \,,
      \end{align*}
      constitute a {\onetoonetext} correspondence.
  \end{enumerate}
\end{proposition}

\begin{fluff}
  Let~$\glie$ be a Lie~algebra.
  If~$M$ is a left~\module{$\Univ(\glie)$} then the corresponding act of~$\glie$ on~$M$ is given by
  \[
    x \act m
    =
    \class{x} \cdot m
  \]
  for all~$x \in \glie$,~$m \in M$.
  Similarly, if~$M$ is a right~\module{$\Univ(\glie)$} then the corresponding right action of~$\glie$ on~$M$ is given by
  \[
    m \act x
    =
    m \cdot \class{x}
  \]
  for all~$x \in \glie$,~$m \in M$.
\end{fluff}

\subsubsection{Functoriality}

\begin{remark}
  Let~$\glie$ be a Lie~algebra with universal enveloping algebra~$\Univ(\glie)$.
  \Cref{representations are modules} shows that representations of~$\glie$ are the same as~{\modules{$\Univ(\glie)$}}.
  We get from this correspondence an isomorphism of categories between~$\cRep{\glie}$ and~$\cMod{\Univ(\glie)}$.
\end{remark}


\begin{lemma}[Functoriality of the universal enveloping algebra]
  \label{functoriality of universal enveloping algebra}
  Let~$\glie$,~$\hlie$ and~$\klie$ be Lie~algebras.
  \begin{enumerate}
    \item
      For every homomorphism of Lie~algebras~$\varphi$ from~$\glie$ to~$\hlie$ there exists a unique homomorphism of algebras~$\Univ(\varphi)$ from~$\Univ(\glie)$ to~$\Univ(\hlie)$ that makes the following square diagram commute.
      \[
        \begin{tikzcd}[column sep = large]
          \glie
          \arrow{r}[above]{\varphi}
          \arrow{d}[left]{\iota_{\glie}}
          &
          \hlie
          \arrow{d}[right]{\iota_{\hlie}}
          \\
          \Univ(\glie)
          \arrow[dashed]{r}[below]{\Univ(\varphi)}
          &
          \Univ(\hlie)
        \end{tikzcd}
      \]
    \item
      It holds that~$\Univ(\glie) = \id_{\Univ(\glie)}$.
    \item
      It holds for all composable homomorphisms of Lie~algebras~$\varphi$ from~$\glie$ to~$\hlie$ and~$\psi$ from~$\hlie$ to~$\klie$ that
      \[
        \Univ( \psi \circ \varphi )
        =
        \Univ( \psi ) \circ \Univ( \varphi ) \,.
      \]
  \end{enumerate}
\end{lemma}


\begin{proof}
  \leavevmode
  \begin{enumerate}
    \item
      The composite~$\iota_{\hlie} \circ \varphi$ is a homomorphism of Lie~algebras from~$\glie$ to~$\Univ(\hlie)$.
      By the universal property of the universal enveloping algebra~$\Univ(\glie)$ there exists a unique homomorphism of algebras~$\Univ(\varphi)$ from~$\Univ(\glie)$ to~$\Univ(\hlie)$ with~$\Univ(\varphi) \circ \iota_{\glie} = \iota_{\hlie} \circ \varphi$.
    \item
      The square diagram
      \[
        \begin{tikzcd}[column sep = huge]
          \glie
          \arrow{r}[above]{\id_{\glie}}
          \arrow{d}
          &
          \glie
          \arrow{d}
          \\
          \Univ(\glie)
          \arrow[dashed]{r}[below]{\id_{\Univ(\glie)}}
          &
          \Univ(\glie)
        \end{tikzcd}
      \]
      commutes, which shows that the identity homomorphism~$\id_{\Univ(\glie)}$ satisfies the defining property of the induced algebra homomorphism~$\Univ( \id_{\glie} )$.
    \item
      We have the following commutative diagram:
      \[
        \begin{tikzcd}[column sep = large]
          \glie
          \arrow[dashed, bend left = 40]{rr}[above]{\psi \circ \varphi}
          \arrow{r}[above]{\varphi}
          \arrow{d}
          &
          \hlie
          \arrow{r}[above]{\psi}
          \arrow{d}
          &
          \klie
          \arrow{d}
          \\
          \Univ(\glie)
          \arrow{r}[below]{\Univ(\varphi)}
          \arrow[dashed, bend right = 40]{rr}[below]{\Univ(\psi) \circ \Univ(\varphi)}
          &
          \Univ(\hlie)
          \arrow{r}[below]{\Univ(\psi)}
          &
          \Univ(\klie)
        \end{tikzcd}
      \]
      The commutativity of the outer square diagram
      \[
        \begin{tikzcd}[column sep = huge]
          \glie
          \arrow{r}[above]{\psi \circ \varphi}
          \arrow{d}
          &
          \klie
          \arrow{d}
          \\
          \Univ(\glie)
          \arrow[dashed]{r}[below]{\Univ(\psi) \circ \Univ(\varphi)}
          &
          \Univ(\klie)
        \end{tikzcd}
      \]
      shows that the composite~$\Univ(\psi) \circ \Univ(\varphi)$ satisfies the defining property of the induced algebra homomorphism~$\Univ(\psi \circ \varphi)$.
    \qedhere
  \end{enumerate}
\end{proof}


\begin{remark}
  \Cref{functoriality of universal enveloping algebra} shows that the assignment~$\glie \mapsto \Univ(\glie)$ of a Lie~algebra~$\glie$ to its universal eveloping algebra~$\Univ(\glie)$ can be extended to a (covariant) functor~$\Univ$ from~$\cLie{\kf}$ to~$\cAlg{\kf}$.
  The universal property of the {\ua} states that the functor~$\Univ$ is left adjoint to the forgetful functor from~$\cAlg{\kf}$ to~$\cLie{\kf}$, which assigns to each~{\algebra{$\kf$}} its underlying Lie~algebra.
\end{remark}

\subsubsection{Derivations}

\begin{proposition}
  \label{extending derivation to universal enveloping algebra}
  Let~$\glie$ be a Lie~algebra.
  Every derivation of~$\glie$ extends uniquely to an extension of~$\Univ(\glie)$.
  More explicitely, there exists for every Lie~algebra derivation~$\delta$ of~$\glie$ a unique algebra derivation~$\Delta$ of~$\Univ(\glie)$ such that the following square diagram commutes.
  \[
    \begin{tikzcd}
      \glie
      \arrow{r}[above]{\delta}
      \arrow{d}
      &
      \glie
      \arrow{d}
      \\
      \Univ(\glie)
      \arrow[dashed]{r}[above]{\Delta}
      &
      \Univ(\glie) \,.
    \end{tikzcd}
  \]
\end{proposition}


\begin{lemma}
  \label{translating between derivations and homomorphisms}
  \leavevmode
  \begin{enumerate}
    \item
      Let~$A$ be a~\algebra{$\kf$} and let~$B$ be the~\algebra{$\kf$}
      \[
        B
        \defined
        \begin{pmatrix}
          A & A \\
          0 & A
        \end{pmatrix} \,.
      \]
      A map~$\Delta$ from~$A$ to~$A$ is a derivation of~$A$ if and only if the map
      \[
        \Phi
        \colon
        A
        \to
        B \,,
        \quad
        a
        \mapsto
        \begin{pmatrix}
          a & \Delta(a) \\
          0 & a
        \end{pmatrix}
      \]
      is a homomorphism of algebras.
    \item
      Let~$\glie$ be a Lie~algebra and let
      \[
        B
        \defined
        \begin{pmatrix}
          \Univ(\glie)  & \Univ(\glie)  \\
          0             & \Univ(\glie)
        \end{pmatrix} \,.
      \]
      A map~$\delta$ from~$\glie$ to~$B$ is a derivation of~$\glie$ if and only if the map
      \[
        \varphi
        \colon
        \glie
        \to
        B \,,
        \quad
        x
        \mapsto
        \begin{pmatrix}
          x & x \\
          0 & x
        \end{pmatrix}
      \]
      is a homomorphism of Lie~algebras
  \end{enumerate}
\end{lemma}


\begin{proof}
  \leavevmode
  \begin{enumerate}
    \item
      The map~$\Phi$ is linear if and only if the map~$\Delta$ is linear.
      We have
      \[
        \Phi(a) \cdot \Phi(b)
        =
        \begin{pmatrix}
          a & \Delta(a) \\
          0 & a
        \end{pmatrix}
        \begin{pmatrix}
          b & \Delta(b) \\
          0 & b
        \end{pmatrix}
        =
        \begin{pmatrix}
          ab  & \Delta(a) b + a \Delta(b)
          0   & ab
        \end{pmatrix}
      \]
      for all~$a, b \in A$.
      The map~$\Phi$ is thus multiplicaitve if and only if~$\Delta(ab) = \Delta(a) b + a \Delta(b)$ for all~$a, b \in A$.
      If~$\Delta$ is a derivation then we also have~$\Delta(1) = 0$ and thus~$\Phi(1) = 1$.
      This shows altogether that~$\Delta$ is a derivation if and only if~$\Phi$ is a homomorphism of algebras.
    \item
      We have
      \begin{align*}
        [ \varphi(x), \varphi(y) ]
        &=
        \Biggl[
          \begin{pmatrix}
            x & \delta(x) \\
            0 & x
          \end{pmatrix},
          \begin{pmatrix}
            y & \delta(y) \\
            0 & y
          \end{pmatrix}
        \Biggr]
        \\
        &=
        \begin{pmatrix}
          x & \delta(x) \\
          0 & x
        \end{pmatrix}
        \begin{pmatrix}
          y & \delta(y) \\
          0 & y
        \end{pmatrix}
        -
        \begin{pmatrix}
          y & \delta(y) \\
          0 & y
        \end{pmatrix}
        \begin{pmatrix}
          x & \delta(x) \\
          0 & x
        \end{pmatrix}
        \\
        &=
        \begin{pmatrix}
          xy  & x \delta(y) + \delta(x) y \\
          0   & xy
        \end{pmatrix}
        -
        \begin{pmatrix}
          yx  & y \delta(x) + \delta(y) x \\
          0   & yx
        \end{pmatrix}
        \\
        &=
        \begin{pmatrix}
          xy - yx & \delta(x) y - y \delta(x) + x \delta(y) - \delta(y) x \\
          0       & xy - yx
        \end{pmatrix}
        \\
        &=
        \begin{pmatrix}
          [x,y] & [\delta(x), y] + [x, \delta(y)] \\
          0     & [x,y]
        \end{pmatrix}
      \end{align*}
      for all~$x, y \in \glie$.
      It follows that the map~$\varphi$ is a homomorphism of Lie~algebras if and only if~$\delta([x,y]) = [\delta(x), y] + [x, \delta(y)]$ for all~$x, y \in \glie$, i.e. if and only if the map~$\delta$ is a derivation of~$\glie$.
    \qedhere
  \end{enumerate}
\end{proof}


\begin{proof}[Proof of \cref{extending derivation to universal enveloping algebra}]
  The uniqueness of~$\Delta$ follows from \cref{dervation is uniquely determined by algebra generators} because~$\glie$ generates the algebra~$\Univ(\glie)$.

  Let~$B$ be the~\algebra{$\kf$} given by
  \[
    B
    \defined
    \begin{pmatrix}
      \Univ(\glie)  & \Univ(\glie) \\
      0             & \Univ(\glie)
    \end{pmatrix} \,.
  \]
  It follows from \cref{translating between derivations and homomorphisms} that the map
  \[
    \varphi
    \colon
    \glie
    \to
    A \,,
    \quad
    x
    \mapsto
    \begin{pmatrix}
      x & \delta(x) \\
      0 & x
    \end{pmatrix}
  \]
  is a homomorphism of Lie~algebras because~$\delta$ is a derivation of~$\glie$.
  It follows from the universal property of the universal enveloping algebra~$\Univ(\glie)$ that the homomorphism of Lie~algebras~$\varphi$ extends uniquely to a homomorphism of algebras~$\Phi$ from~$\Univ(\glie)$ to~$B$.
  This homomorphism~$\Phi$ is of the form
  \[
    \Phi(x)
    =
    \begin{pmatrix}
      \Phi_1(x) & \Delta(x) \\
      0         & \Phi_2(x)
    \end{pmatrix}
  \]
  for all~$x \in \Univ(\glie)$ for some unique linear mape
  \[
    \Phi_1, \Phi_2, \Delta
    \colon
    \Univ(\glie)
    \to
    \Univ(\glie) \,.
  \]
  The maps~$\Phi_1$ and~$\Phi_2$ are homomorphisms of algebras because~$\Phi$ is a homomorphism of algebras.
  They satisfy the equalities~$\Phi_1(x) = \varphi(x) = x$ and~$\Phi_2(x) = \varphi(x) = x$ for all~$x \in \glie$.
  It follows that~$\Phi_1(x) = x$ and~~$\Phi_2(x) = x$ for all~$x \in \Univ(\glie)$ because~$\glie$ generates~$\Univ(\glie)$ as an algebra.
  The homomorphism~$\Phi$ is thus of the form
  \[
    \Phi(x)
    =
    \begin{pmatrix}
      x & \Delta(x) \\
      0 & x
    \end{pmatrix}
  \]
  for all~$x \in \glie$.
  It follows from \cref{translating between derivations and homomorphisms} that the linear map~$\Delta$ is an algebra derivation of~$\Univ(\glie)$.
  This darivation satisfies the equality~$\Delta(x) = \delta(x)$ for all~$x \in \glie$ because~$\Phi$ is an extension of~$\varphi$.
  In other words,~$\Delta$ is an extension of~$\delta$.
\end{proof}



\section{Examples}


% TODO: UEA of semidirect product.


\begin{convention}
  Let~$\glie$ be a Lie~algebra.
  For every element~$x$ of~$\glie$ the image of~$x$ of~$\Univ(\glie)$ is denoted by~$\class{x}$.
\end{convention}


\subsection{Abelian Lie~Algebras}


\begin{examples}
  Let~$\glie$ be an abelian Lie~algebra.
  It follows from the explicit construction of the universal enveloping algebra~$\Univ(\glie)$ that
  \[
    \Univ(\glie)
    \cong
    \Tensor(\glie)/(x \tensor y - y \tensor x \suchthat x, y \in \glie)
    \cong
    \Symm(\glie)
  \]
  with the canonical homomorphism of Lie~algebras from~$\glie$ to~$\Univ(\glie)$ corresponding to the inclusion map from~$\glie$ to~$\Symm(\glie)$.

  This can also be seen more abstractly, as follows.
  
  We observe that if~$V$ is any~\vectorspace{$\kf$} and~$A$ is any~\algebra{$\kf$} then a linear map~$f$ from~$V$ to~$A$ extends to a homomorphism of algebras from~$\Symm(V)$ to~$A$ (necessarily uniquely) if and only if the image of~$f$ is contained in a commutative subalgebra of~$A$, if and only if the image of~$f$ is commutative in~$A$.
  It follows from this observation that we have for any~{\algebra{$\kf$}} bijections
  \begin{align*}
    {}&
    \{ \textstyle \text{Lie~algebra homomorphisms~$\glie \to A$} \}
    \\
    \cong{}&
    \{ \textstyle \text{{\linear{$\kf$}} maps~$\glie \to A$ with commutative image} \}
    \\
    \cong{}&
    \{ \textstyle \text{algebra homomorphisms~$\Symm(\glie) \to A$} \} \,.
  \end{align*}
  These bijections are natural in~$A$.
  This shows that the symmetric algebra~$\Symm(\glie)$ together with the inclusion from~$\glie$ to~$\Symm(\glie)$ satisfies the universal property of the universal enveloping algebra of~$\glie$.
\end{examples}


\begin{example}
  We find for~$\glie = 0$ that~$\Univ(\glie) = \kf$.
\end{example}


\begin{definition}
  \leavevmode
  \begin{enumerate}
    \item
      Let~$A$ be an algebra.
      An \defemph{augumentation} of~$A$ is a homomorphism of algebras~$\varepsilon$ from~$A$ to~$\kf$.
    \item
      An \defemph{augumented algebra} is a~\algebra{$\kf$}~$A$ together with an augumentation of~$A$.
    \item
      Let~$(A, \varepsilon)$ be an augumented algebra.
      The kerrnel of~$\varepsilon$ is the \defemph{augumentation ideal} of~$A$.
  \end{enumerate}
\end{definition}


\begin{remark}
  Augumented algebras are always nonzero.
\end{remark}


\begin{proposition}
  \label{decomposition for augumented algebra}
  Let~$(A, \varepsilon) $ be an augumented algebra.
  Then~$A = \kf \oplus \ker(\varepsilon)$ as vector spaces.
\end{proposition}


\begin{proof}
  Let~$\eta$ be the inclusion from~$\kf$ to~$A$.
  The composito~$\varepsilon \circ \eta$ is the identity on~$\kf$ whence the composite~$\eta \circ \varepsilon$ is idempotent.
  It follows that
  \[
    A
    =
    \im(\eta \circ \varepsilon)
    \oplus \ker(\eta \circ \varepsilon) \,.
  \]
  It follows from the surjectivity of~$\varepsilon$ that~$\im(\eta \circ \varepsilon) = \im(\eta) = \kf$ and from the injectivity of~$\eta$ that~$\ker(\eta \circ \varepsilon) = \ker(\varepsilon)$.
\end{proof}


\begin{construction}
  \label{construction of counit}
  Let~$\glie$ be a Lie~algebra.
  We have a unique homomorphism of Lie~algebras from~$\glie$ to~$0$.
  This homomorphism of Lie~algebras induces a homomorphism of algebras
  \[
    \varepsilon
    \colon
    \Univ(\glie)
    \to
    \kf \,.
  \]
  This homomorphism of algebras is uniquely determined by the condition~$\varepsilon( \class{x} ) = 0$ for all~$x \in \glie$ because~$\Univ(\glie)$ is generated as an algebra by the image of~$\glie$ of~$\Univ(\glie)$.
\end{construction}


\begin{definition}
  Let~$\glie$ be a Lie~algebra.
  The homomorphism of algebras~$\varepsilon$ from~$\Univ(\glie)$ to~$\kf$ from \cref{construction of counit} is the \defemph{counit} of~$\glie$.
  Its makes~$\Univ(\glie)$ into an augumented algebra.
\end{definition}


\begin{remark}
  Let~$\glie$ be a Lie~algebra.
  We can regard~$\kf$ as the trivial representation of~$\glie$, and thus as a~\module{$\Univ(\glie)$}.
  This~\module{$\Univ(\glie)$} structure can also be explained with help of the counit~$\varepsilon$, by regarding~$\varepsilon$ as a homomorphism of algebras from~$\Univ(\glie)$ to~$\End_{\kf}(\kf) = \kf$.
\end{remark}


\begin{proposition}
  \label{augumentation ideal is spanned by monomials}
  The augumentation ideal of~$\Univ(\glie)$ is spanned as a vector space by all the monomials~$\class{x_1} \dotsm \class{x_n}$ with~$n \geq 1$,~$x_1, \dotsc, x_n \in \glie$.
\end{proposition}


\begin{proof}
  Let~$\varepsilon$ be the counit of~$\Univ(\glie)$, let~$U$ be the linear subspace of~$\Univ(\glie)$ spanned by all monomials~$\class{x_1} \dotsm \class{x_n}$ with~$n \geq 0$,~$x_1, \dotsc, x_n \in \glie$.
  We have
  \[
    \varepsilon( \class{x_1} \dotsm \class{x_n} )
    =
    \varepsilon( \class{x_1} ) \dotsm \varepsilon( \class{x_n} )
    =
    0 \dotsm 0
    =
    0
  \]
  for each such monomial, whence the linear space~$U$ is contained in~$\ker(\varepsilon)$.
  We know on the other hand that~$\Univ(\glie)$ is generated by~$\class{\glie}$ as an algebra.
  This means that the monomials
  \[
    \class{x_1}  \dotsm \class{x_n}
    \qquad
    \text{with~$n \geq 0$,~$x_1, \dotsc, x_n \in \glie$}
  \]
  span the algebra~$\Univ(\glie)$ as a vector space.
  We thus have~$\Univ(\glie) = \kf + U$.
  Together with the decomposition~$\Univ(\glie) = \kf \oplus \ker(\varepsilon)$ and the inclusion~$U \subseteq \ker(\varepsilon)$ we find that~$U = \ker(\varepsilon)$.
\end{proof}



\subsection{Opposite Lie~Algebra}

\begin{example}
  \label{uea of opposite by first principles}
  Let~$\glie$ be a Lie~algebra.
  We have for every~{\algebra{$\kf$}}~$A$ bijections
  \begin{align*}
    {}&
    \{ \text{algebra homomorphisms~$\Univ(\glie^\op) \to A$   } \}
    \\
    \cong{}&
    \{ \text{Lie~algebra homomorphisms~$\glie^\op \to A$} \}
    \\
    ={}&
    \{ \text{Lie~algebra homomorphisms~$\glie \to A^\op$} \}
    \\
    \cong{}&
    \{ \text{algebra homomorphisms~$\Univ(\glie) \to A^\op$} \}
    \\
    ={}&
    \{ \text{algebra homomorphisms~$\Univ(\glie)^\op \to A$} \}
  \end{align*}
  that are natural in~$A$.
  (We used implicitely that taking the underlying Lie~algebra of a~{\algebra{$\kf$}} commutes with taking opposites.)
  It follows from Yoneda’s~lemma that~$\Univ(\glie^\op) \cong \Univ(\glie)^\op$.
  The canonical homomorphism of Lie~algebras from~$\glie^{\op}$ to~$\Univ(\glie^{\op})$ corresponds to the homomorphism from~$\glie^{\op}$ to~$\Univ(\glie)^{\op}$ given by~$\class{x^{\op}} \mapsto \class{x}^{\op}$ for all~$x \in \glie$.

  We can also derive the above isomorphism in a more explicit way, as we will now explain.

  The map from~$\glie$ to~$\glie^{\op}$ given by$x \mapsto x^{\op}$ for all~$x \in \glie$ is an anti-isomorphism of Lie~algebras.
  Similarly, the map from~$\Univ(\glie^{\op})$ to~$\Univ(\glie^{\op})^{\op}$ given by~$y \mapsto y^{\op}$ for all~$y \in \Univ(\glie^{\op})$ is an anti-isomorphism of algebras, and thus an anti-isomorphism of Lie~algebras. 
  The composite
  \[
    \glie
    \xto{x \mapsto x^{\op}}
    \glie^{\op}
    \to
    \Univ(\glie^{\op})
    \xto{y \mapsto y^{\op}}
    \Univ(\glie^{\op})^{\op}
  \]
  is therefore a homomorphism of Lie~algebras.
  This composite hence induces a homomorphism of algebras~$\Phi$ from~$\Univ(\glie)$ to~$\Univ(\glie^{\op})^{\op}$.
  This is the unique algebra homomorphism that makes the square diagram
  \[
    \begin{tikzcd}
      \glie
      \arrow{r}[above]{x \mapsto x^{\op}}
      \arrow{d}
      &
      \glie^{\op}
      \arrow{d}
      \\
      \Univ(\glie)
      \arrow[dashed]{r}[below]{\varphi}
      &
      \Univ(\glie^{\op})^{\op}
    \end{tikzcd}
  \]
  commute.
  We can regard~$\Phi$ is an algebra homomorphism from~$\Univ(\glie)^{\op}$ to~$\Univ(\glie^{\op})$.
  This homomorphism~$\Phi$ is uniquely determined by the identity
  \[
    \Phi\Bigl( \class{x}^{\,\op} \Bigr)
    =
    \class{ x^{\op} } 
  \]
  for all~$x \in \glie$.

  By switching the roles of~$\glie$ and~$\glie^{\op}$ we also have a homomorphism of algebras
  \[
    \Psi''
    \colon
    \Univ(\glie^{\op})^{\op}
    \to
    \Univ( (\glie^{\op})^{\op} )
  \]
  which is given by
  \[
    \Psi''\Bigl( \class{y}^{\,\op} \Bigr)
    =
    \class{ y^{\op} }
  \]
  for all~$y \in \glie^{\op}$
  We have~$(\glie^{\op})^{\op}$ whence~$\Psi$ is a homomorphism of algebras~$\Phi'$ from~$\Univ(\glie^{\op})^{\op}$ to~$\Univ(\glie)$, that is given by
  \[
    \Psi'\biggl( \class{ x^{\op} }^{\,\op} \biggr)
    =
    \class{ (x^{\op})^{\op} }
    =
    \class{ x }
  \]
  for all~$x \in \glie$.
  We can now regard~$\Psi'$ as a homomorphism of algebras~$\Phi$ from~$\Univ(\glie^{\op})$ to~$\Univ(\glie)^{\op}$, that is given by
  \[
    \Psi\Bigl( \class{ x }^{\,\op} \Bigr)
    =
    \class{ x }^{\op}
  \]
  for all~$x \in \glie$.

  The composite~$\Phi \circ \Psi$ as a homomorphism of algebras from~$\Univ( \glie^{\op} )$ to~$\Univ( \glie^{\op} )$, given by
  \[
    (\Phi \circ \Psi)\Bigl( \class{ x^{\op} } \Bigr)
    =
    \Phi\Bigl( \Psi\Bigl( \class{ x^{\op} }  \Bigr) \Bigr)
    =
    \Phi\Bigl( \class{ x }^{\,\op} \Bigr)
    =
    \class{ x^{\op} }
  \]
  for all~$x \in \glie$, and thus
  \[
    (\Phi \circ \Psi)\bigl( \class{y} \bigr)
    =
    \class{ y }
  \]
  for all~$y \in \glie^{\op}$.
  This shows that the composite~$\Phi \circ \Psi$ is the identity of~$\Univ( \glie^{\op} )$.

  The composite~$\Psi \circ \Phi$ is a homomorphism of algebras from~$\Univ( \glie )^{\op}$ to~$\Univ( \glie )^{\op}$, given by
  \[
    (\Psi \circ \Phi)\Bigl( \class{x}^{\,\op} \Bigr)
    =
    \Psi\Bigl( \Phi\Bigl( \class{x}^{\,\op} \Bigr) \Bigr)
    =
    \Psi\Bigl( \class{ x^{\op} } \Bigr)
    =
    \class{x}^{\op} 
  \]
  for all~$x \in \glie$.
  The algebra~$\Univ(\glie)$ is generated by the elements~$\class{x}$ with~$x$ in~$\glie$, whence the opposite algebra~$\Univ(\glie)^{\op}$ is generated by the elements~$\class{x}^{\,\op}$ with~$\glie$.
  It therefore follows from the above calculation that the composite~$\Psi \circ \Phi$ is the identity of~$\Univ( \glie )^{\op}$.

  We have altogether shows that the two homomorphisms of algebras~$\Phi$ and~$\Psi$ are mutually inverse isomorphisms.
\end{example}


\begin{construction}
  \label{construction of antipode}
  Let~$\glie$ be a Lie~algebra.
  We have an isomorphism of Lie~algebras
  \[
    \glie
    \to
    \glie^{\op} \,,
    \quad
    x
    \mapsto
    - x^{\op} \,.
  \]
  This isomorphism of Lie~algebras induces an isomorphism of algebras
  \[
    S'
    \colon
    \Univ( \glie )
    \to
    \Univ( \glie^{\op} )
  \]
  which is uniquely determined by
  \[
    S'( \class{x} )
    =
    - \class{ x^{\op} }
  \]
  for all~$x \in \glie$.
  Under the above isomorphism~$\Univ( \glie^{\op} ) \cong \Univ(\glie)^{\op}$ we can regard~$S'$ as an isomorphism of algebras
  \[
    S
    \colon
    \Univ( \glie )
    \to
    \Univ( \glie )^{\op}
  \]
  which is uniquely determined by
  \[
    S( \class{x} )
    =
    - \class{x}^{\,\op}
  \]
  for all~$x \in \glie$.
\end{construction}


\begin{definition}
  Let~$\glie$ be a Lie~algebra.
  The isomorphism of algebras~$S$ from~$\Univ(\glie)$ to~$\Univ(\glie)^{\op}$ from \cref{construction of antipode} is the \defemph{antipode} of~$\Univ(\glie)$.
\end{definition}


\begin{remark}
  Let~$\glie$ be a Lie~algebra.
  For every representation~$M$ of~$\glie$ its dual~$M^*$ becomes again a representation of~$\glie$ via the action
  \[
    (x \act \varphi)(m)
    =
    - \varphi(m)
  \]
  for all~$x \in \glie$,~$\varphi \in M^*$,~$m \in M$.
  In other words, for every~\module{$\Univ(\glie)$}~$M$ its dual~$M^*$ becomes again a~\module{$\Univ(\glie)$}.

  This can also be explained via the antipode.
  Let~$M$ be a~\module{$\Univ(\glie)$}.
  The the dual~$M^*$ becomes a right~\module{$\Univ(\glie)$} via the multiplication
  \[
    (\varphi \cdot y)(m)
    =
    \varphi(ym)
  \]
  for all~$y \in \Univ(\glie)$,~$\varphi \in M^*$,~$m \in M$.
  This right~\module{$\Univ(\glie)$} structure on~$M^*$ corresponds to a left~\module{$\Univ(\glie)^{\op}$} structure on~$M^*$ given by
  \[
    y^{\op} \cdot \varphi
    =
    \varphi \cdot y
  \]
  for all~$y \in \Univ(\glie)$,~$\varphi \in M^*$.
  By using the isomorphism of algebras~$S$ from~$\Univ(\glie)$ to~$\Univ(\glie)^{\op}$ we can pull back this~\module{$\Univ(\glie)^{\op}$} structure to a~\module{$\Univ(\glie)$} structure given by
  \[
    y \cdot \varphi
    =
    S(y) \cdot \varphi
  \]
  for all~$y \in \Univ(\glie)$,~$\varphi \in M^*$.

  For every element~$x$ of~$\glie$ we have
  \[
    (\class{x} \cdot \varphi)(m)
    =
    ( S(\class{x}) \cdot \varphi )(m)
    =
    ( - \class{x}^{\,\op} \cdot \varphi )(m)
    =
    ( \varphi \cdot (- \class{x}) )(m)
    =
    \varphi( - \class{x} \cdot m )
    =
    - \varphi( \class{x} \cdot m )
  \]
  for all~$\varphi \in M^*$,~$m \in M$.
  Both constructed~\module{$\Univ(\glie)$} structures on~$M^*$ hence coincide.
\end{remark}



\subsection{Direct Sum of Lie~algebras}

\begin{recall}
  \label{homomorphism out of a tensor product}
  Let~$A$ and~$B$ be two~{\algebras{$\kf$}}.
  Then the inclusion maps
  \begin{alignat*}{2}
    \Iota_A
    &\colon
    A
    \to
    A \tensor B \,,
    &
    \quad
    a
    &\mapsto
    a \tensor 1 \,,
    \\
    \Iota_B
    &\colon
    B
    \to
    A \tensor B \,,
    &
    \quad
    b
    &\mapsto
    1 \tensor b
  \end{alignat*}
  are injective homomorphisms of algebras.
  We may therefore identify the algebras~$A$ and~$B$ with the associated subalgebras~$A \tensor 1$ and~$1 \tensor B$ of~$A \tensor B$.
  We note that~$A$ and~$B$ commute in~$A \tensor B$ because
  \[
    \Iota_A(a) \Iota_B(b)
    =
    (a \tensor 1) (b \tensor 1)
    =
    a \tensor b
    =
    (b \tensor 1) (a \tensor 1)
    =
    \Iota_B(b) \Iota_A(a)
  \]
  for all~$a \in A$,~$b \in B$.
  
  Let~$C$ be another~{\algebra{$\kf$}}.
  
  If~$\Phi$ is a homomorphism of algebras from~$A \otimes B$ to~$C$ then the restrictions~$\Phi_A$ and~$\Phi_B$ given by~$\Phi_A = \Phi \circ \Iota_A$ and~$\Phi_B = \Phi \circ \Iota_B$ are again homomorphisms of algebras.
  The images of~$\Phi_A$ and~$\Phi_B$ commute in~$C$ because~$A$ and~$B$ commute in~$A \tensor B$.
  More explicitely,
  \begin{align*}
    \Phi_A(a) \Phi_B(b)
    &=
    \Phi(a \tensor 1) \Phi(1 \tensor b)
    \\
    &=
    \Phi( (a \tensor 1) (1 \tensor b) )
    \\
    &=
    \Phi( a \tensor b )
    \\
    &=
    \Phi( (1 \tensor b) (a \tensor 1) )
    \\
    &=
    \Phi(1 \tensor b) \Phi(a \tensor 1)
    \\
    &=
    \Phi_B(b) \Phi_A(a)
  \end{align*}
  for all~$a \in A$,~$b \in B$.
  
  Suppose on the other hand that~$\Psi_A$ is a homomorphism of algebras from~$A$ to~$C$ and that~$\Psi_B$ is a homomorphism of algebras from~$B$ to~$C$.
  There exists a unique linear map~$\Psi$ from~$A \otimes B$ to~$C$ given by
  \[
    \Psi(a \otimes b)
    \mapsto
    \Psi_A(a) \Psi_:(b)
  \]
  for all~$a \in A$,~$b \in B$.
  Suppose that the images of~$\Psi_A$ and~$\Psi_B$ commute in~$C$.
  The linear map~$\Psi$ is then again an homomorphism of algebras because
  \begin{align*}
    \Psi(a_1 \tensor b_1) \Psi(a_2 \tensor b_2)
    &=
    \Psi_A(a_1) \Psi_B(b_1) \Psi_A(a_2) \Psi_B(b_2)
    \\
    &=
    \Psi_A(a_1) \Psi_A(a_2) \Psi_B(b_1) \Psi_B(b_2)
    \\
    &=
    \Psi_A(a_1 a_2) \Psi_B(b_1 b_2)
    \\
    &=
    \Psi( (a_1 a_2) \tensor (b_1 b_2) )
    \\
    &=
    \Psi( (a_1 \tensor b_1) (a_2 \tensor b_2) )
  \end{align*}
  for all~$a_1, a_2 \in A$,~$b_1, b_2 \in B$, as well as
  \[
    \Psi( 1_{A \otimes B} )
    =
    \Psi( 1_A \otimes 1_B )
    =
    \Psi_A( 1_A ) \Psi_B( 1_B )
    =
    1_C \cdot 1_C
    =
    1_C \,.
  \]
  
  These above two constructions are mutually inverse and hence result in a {\onetoonetext} correspondence
  \begin{align*}
    \left\{
      \begin{tabular}{@{}c@{}}
        algebra homomorphisms \\
        $\Phi \colon A \tensor B \to C$
      \end{tabular}
    \right\}
    &\onetoone
    \left\{
      (\Phi_A, \Phi_B)
    \suchthat*
      \begin{tabular}{@{}c@{}}
        algebra homomorphisms   \\
        $\Phi_A \colon A \to C$ \\
        $\Phi_B \colon B \to C$ \\
        whose images commute
      \end{tabular}
    \right\}  \,,
    \\
    \Phi
    &\mapsto
    (\Phi \circ \Iota_A, \Phi \circ \Iota_B)  \,,
    \\
    \biggl( a \tensor b \mapsto \Phi_A(a) \Phi_B(b) \biggr)
    &\mapsfrom
    (\Phi_A, \Phi_B)  \,.
  \end{align*}
\end{recall}


\begin{example}
  \label{explicit isomorphism for uea of direct sum}
  Let~$\glie$ and~$\hlie$ be two Lie~algebras.
  We show in the following that
  \[
    \Univ(\glie \oplus \hlie)
    \cong
    \Univ(\glie) \tensor \Univ(\hlie) \,.
  \]
  The isomorphism from~$\Univ(\glie \oplus \hlie)$ to~$\Univ(\glie) \tensor \Univ(\hlie)$ is given on the algebra generators~$\class{(x,y)}$ with~$(x,y)$ in~$\glie \oplus \hlie$ by
  \[
    \class{(x,y)}
    \mapsto
    \class{x} \tensor 1 + 1 \tensor \class{y} \,.
  \]
  The inverse isomorphism from~$\Univ(\glie) \tensor \Univ(\hlie)$ to~$\Univ(\glie \oplus \hlie)$ is given on the simple tensors~$\class{x} \tensor \class{y}$ with~$x$ in~$\Univ(\glie)$ and~$y$ in~$\Univ(\hlie)$ by
  \[
    \class{x} \tensor \class{y}
    \mapsto
    \Univ( \iota_1 )( \class{x} )
    \cdot
    \Univ( \iota_2 )( \class{y} ) \,.
  \]
  Here we denote by~$\iota_1$ is the canonical homomorphism of Lie~algebras from~$\glie$ to~$\glie \oplus \hlie$ (i.e. the inclusion into the first sammand) and similarly by~$\iota_2$ is the canonical homomorphism of Lie~algebras from~$\hlie$ to~$\glie \oplus \hlie$ (i.e. the inclusion into the second summand).

  We present two ways in which the above isomorphism(s) can be derived.
  \begin{itemize}
    \item
      It follows from \cref{homomorphism out of direct sum} and \cref{homomorphism out of a tensor product} that we get for every~\algebra{$\kf$}~$A$ bijections
      \begin{align*}
        {}&
        \left\{
          \begin{tabular}{@{}c@{}}
            algebra homomorphisms \\
            $\Phi \colon \Univ(\glie \oplus \hlie) \to A$
          \end{tabular}
        \right\}
        \\
        \cong{}&
        \left\{
          \begin{tabular}{@{}c@{}}
            Lie~algebra homomorphisms \\
            $\varphi \colon \glie \oplus \hlie \to A$
          \end{tabular}
        \right\}
        \\
        \cong{}&
        \left\{
          ( \varphi_1, \varphi_2 )
        \suchthat*
          \begin{tabular}{@{}c@{}}
            Lie~algebra homomorphisms \\
            $\varphi_1 \colon \glie \to A$ and~$\varphi_2 \colon \hlie \to A$ \\
            whose images commute
          \end{tabular}
        \right\}
        \\
        \cong{}&
        \left\{
          (\Phi_1, \Phi_2)
        \suchthat*
          \begin{tabular}{@{}c@{}}
            algebra homomorphisms               \\
            $\Phi_1 \colon \Univ(\glie) \to A$  \\
            $\Phi_2 \colon \Univ(\hlie) \to A$  \\
            whose images commute
          \end{tabular}
        \right\}
        \\
        \cong{}&
        \left\{
          \begin{tabular}{@{}c@{}}
             algebra homomorphims \\
             $\Phi \colon \Univ(\glie) \tensor \Univ(\hlie) \to A$
          \end{tabular}
        \right\} \,.
      \end{align*}
      The claimed isomorphism therefore follows from Yoneda’s lemma.
    \item
      We can construct the isomorphism(s) more explicitely, as follows.

      We note that for the induced homomorphisms of algebras
      \begin{align*}
        \Univ(\iota_1) &\colon \Univ(\glie) \to \Univ(\glie \oplus \hlie) \,, \\
        \Univ(\iota_2) &\colon \Univ(\hlie) \to \Univ(\glie \oplus \hlie)
      \end{align*}
      the images of~$\Univ(\iota_1)$ and~$\Univ(\iota_2)$ commute.
      Indeed, the algebra~$\Univ(\glie \oplus \hlie)$ is generated by the image of~$\glie \oplus \hlie$ in~$\Univ(\glie \oplus \hlie)$, and the images~$\iota_1(\glie)$ and~$\iota_2(\hlie)$ commute in~$\glie \oplus \hlie$.
      Thus
      \[
        \Univ(\iota_1)( \class{x} )
        \cdot
        \Univ(\iota_2)( \class{y} )
        =
        \class{(x,0)} \cdot \class{(0,y)}
        =
        \class{(0,y)} \cdot \class{(x,0)}
        =
        \Univ(\iota_2)( \class{y} )
        \cdot
        \Univ(\iota_2)( \class{x} ) \,,
      \]
      for all~$x \in \glie$,~$y \in \hlie$.
      We used for the middle equality that the elements~$(x,0)$ and~$(0,y)$ commute in~$\glie \oplus \hlie$ and hence also in~$\Univ(\glie \oplus \hlie)$.
      It follows from this observation that the two homomorphisms of algebras~$\Univ( \iota_1 )$ and~$\Univ( \iota_2 )$ induce a homomorphism of algebras
      \[
        \Phi
        \colon
        \Univ(\glie) \tensor \Univ(\hlie)
        \to
        \Univ(\glie \oplus \hlie) \,.
      \]
      This homomorphism is given on simple tensors by
      \[
        \Phi(t \tensor u)
        =
        \Univ(\iota_1)(t) \cdot \Univ(\iota_2)(u)
      \]
      for all~$t \in \Univ(\glie)$,~$u \in \Univ(\hlie)$.
      It holds in particular for all~$x \in \glie$,~$y \in \glie$ that
      \[
        \Phi(\class{x} \tensor \class{y})
        =
        \Univ(\iota_1)( \class{x} )
        \cdot
        \Univ(\iota_2)( \class{y} )
        =
        \class{ \iota_1(x) }
        \cdot
        \class{ \iota_2(y) }
        =
        \class{(x,0)} \cdot \class{(0,y)}  \,,
      \]
%      We observe that the map
%      \[
%        \psi'
%        \colon
%        \glie \times \hlie
%        \to
%        \Univ(\glie) \tensor \Univ(\hlie) \,,
%        \quad
%        (x,y)
%        \mapsto
%        \class{(x,0)} \tensor 1 + 1 \tensor \class{(0,y)} \,.
%      \]

      To construct the inverse~$\Psi$ of~$\Phi$ we observe that the equality
      \[
        \Psi\Bigl( \class{(x,0)} \Bigr)
        =
        \Psi\Bigl( \class{(x,0)} \cdot 1 \Bigr)
        =
        \Psi\Bigl( \Univ(\iota_1)( \class{x} ) \cdot \Univ(\iota_2)(1) \Bigr)
        =
        \Psi( \Phi( \class{x} \tensor 1 ) )
        =
        \class{x} \tensor 1
      \]
      has to hold for all~$x \in \glie$, and similarly
      
      \[
        \Psi\Bigl( \class{(0,y)} \Bigr)
        =
        1 \tensor \class{y}
      \]
      for all~$y \in \hlie$.
      It then follows that more generally
      \[
        \Psi\Bigl( \class{(x,y)} \Bigr)
        =
        \Psi\Bigl( \class{(x,0)} + \class{(0,y)} \Bigr)
        =
        \class{x} \tensor 1 + 1 \tensor \class{y}
      \]
      for all~$(x,y) \in \glie \oplus \hlie$.
      
      Motivated by these calculations we consider the map
      \[
        \psi
        \colon
        \glie \oplus \hlie
        \to
        \Univ(\glie) \tensor \Univ(\hlie) \,,
        \quad
        (x,y)
        \mapsto
        \class{x} \otimes 1 + 1 \otimes{y} \,.
      \]
      This map is a homomorphism of Lie~algebras because it is linear with
      \begin{align*}
        {}&
        [\psi((x_1, y_1)), \psi((x_2, y_2))]
        \\
        ={}&
        [
          \class{x_1} \tensor 1 + 1 \tensor \class{y_1} \,,
          \class{x_2} \tensor 1 + 1 \tensor \class{y_2}
        ]
        \\
        ={}&
          [\class{x_1} \tensor 1, \class{x_2} \tensor 1]
        + \underbrace{ [\class{x_1} \tensor 1, 1 \tensor \class{y_2}] }_{=0}
        + \underbrace{ [1 \tensor \class{y_1} \,, \class{x_2} \tensor 1] }_{=0}
        + [1 \tensor \class{y_1}, 1 \tensor \class{y_2}]
        \\
        ={}&
          [\class{x_1}, \class{x_2}] \tensor 1
        + 1 \tensor [\class{y_1}, \class{y_2}]
        \\
        ={}&
          \class{[x_1, x_2]} \tensor 1
        + 1 \tensor \class{[y_1, y_2]}  \,.
        \\
        ={}&
        \psi\bigl( ( [x_1, x_2], [y_1, y_2] ) \bigr)
        \\
        ={}&
        \psi\bigl( [(x_1, y_1), (x_2, y_2)] \bigr) \,.
      \end{align*}
      for all~$(x_1, y_1), (x_2, y_2) \in \glie \oplus \hlie$.
      It hence follows from the universal property of the {\ua}~$\Univ(\glie \oplus \hlie)$ that there exists a unique homomorphism of algebras~$\Psi$ from~$\Univ(\glie \oplus \hlie)$ to~$\Univ(\glie) \tensor \Univ(\hlie)$ that makes the triangular diagram
      \[
        \begin{tikzcd}[row sep = large]
          \glie \oplus \hlie
          \arrow{r}[above]{\psi}
          \arrow{d}[left]{\class{(-)}}
          &
          \Univ(\glie) \tensor \Univ(\hlie)
          \\
          \Univ(\glie \oplus \hlie)
          \arrow[dashed]{ur}[below right]{\Psi}
          &
          {}
        \end{tikzcd}
      \]
      commute.
      This homomorphism~$\psi$ is given by
      \[
        \Psi\Bigl( \class{(x,y)} \Bigr)
        =
        \class{x} \tensor 1 + 1 \tensor \class{y} \,.
      \]
      for all~$(x,y) \in \glie \oplus \hlie$.
      
      We now check that the two homomorphisms~$\Phi$ and~$\Psi$ are mutually inverse.
      We have on the one hand
      \begin{align*}
        \Phi\Bigl( \Psi\Bigl( \class{(x,y)} \Bigr) \Bigr)
        &=
        \Phi( \class{x} \tensor 1 + 1 \tensor \class{y} )
        \\
        &=
        \Phi( \class{x} \tensor 1 ) + \Phi( 1 \tensor \class{y} )
        \\
        &=
        \Univ(\iota_1)(\class{x}) \cdot \Univ(\iota_2)(1)
        + \Univ(\iota_1)(1) \cdot \Univ(\iota_2)(\class{y})
        \\
        &=
        \class{\iota_1(x)} \cdot 1
        + 1 \cdot \class{\iota_2(y)}
        \\
        &=
        \class{(x,0)} + \class{(0,y)}
        \\
        &=
        \class{(x,y)}
      \end{align*}
      for all~$(x,y) \in \glie \oplus \hlie$, and thus~$\Phi \circ \Psi = \id_{\Univ(\glie \oplus \hlie)}$.
      We have on the other hand
      \begin{align*}
        \Psi( \Phi( \class{x} \tensor \class{y} ) )
        &=
        \Psi( \Univ(\iota_1)(\class{x}) \cdot \Univ(\iota_2)(\class{y}) )
        \\
        &=
        \Psi\Bigl( \class{(x,0)} \class{(0,y)} \Bigr)
        \\
        &=
        \Psi\Bigl( \class{(x,0)} \Bigr)
        \Psi\Bigl( \class{(0,y)} \Bigr)
        \\
        &=
        ( \class{x} \tensor 1 + 1 \tensor 0 )
        \cdot ( 0 \tensor 1 + 1 \tensor \class{y} )
        \\
        &=
        (\class{x} \tensor 1)
        \cdot (1 \tensor \class{y})
        \\
        &=
        \class{x} \tensor \class{y}
      \end{align*}
      for all~$x \in \glie$ and~$y \in \hlie$, and thus~$\Psi \circ \Phi = \id_{\Univ(\glie) \otimes \Univ(\hlie)}$.
  \end{itemize}
\end{example}


\begin{construction}
  \label{construction of comultiplication}
  Let~$\glie$ be a Lie~algebra.
  We have a homomorphism of Lie~algebras
  \[
    \delta
    \colon
    \glie
    \to
    \glie \oplus \glie \,,
    \quad
    x
    \mapsto
    (x,x) \,.
  \]
  This homomorphism of Lie~algebras extends a homomorphism of algebras
  \[
    \Delta'
    \colon
    \Univ(\glie)
    \to
    \Univ(\glie \oplus \glie) \,.
  \]
  As seen in \cref{explicit isomorphism for uea of direct sum} we have an isomorphism of algebras from~$\Univ(\glie \oplus \glie)$ to~$\Univ(\glie) \otimes \Univ(\glie)$ given by~$\class{(x,y)} \mapsto \class{x} \otimes 1 + 1 \otimes \class{y}$ for all~$x, y \in \glie$.
  Under this isomorphism we can regard the homomorphism~$\Delta'$ as a homomorphism of algebras
  \[
    \Delta
    \colon
    \Univ(\glie)
    \to
    \Univ(\glie) \otimes \Univ(\glie) \,.
  \]
  This homomorphism is uniquely determined by
  \[
    \Delta( \class{x} )
    =
    \class{x} \otimes 1 + 1 \otimes \class{x}
  \]
  for all~$x \in \glie$.
\end{construction}


\begin{definition}
  Let~$\glie$ be a Lie~algebra.
  The algebra homomorphism~$\Delta$ from~$\Univ(\glie)$ to~$\Univ(\glie) \otimes \Univ(\glie)$ from \cref{construction of comultiplication} is the \defemph{comultiplication} of~$\glie$.
\end{definition}


\begin{remark}
  Let~$\glie$ be a Lie~algebra.
  For every two representations~$M$ and~$N$ of~$\glie$ the tensor product~$M \otimes_{\kf} N$ becomes again a representation of~$\glie$ via the aciton
  \[
    x \act (m \otimes n)
    =
    (x \act m) \otimes n + m \otimes (x \act n)
  \]
  for all~$x \in \glie$,~$m \in M$,~$n \in N$.
  In other words, for every two~\modules{$\Univ(\glie)$}~$M$ and~$N$ the tensor product~$M \otimes_{\kf} N$ becomes again a~\module{$\Univ(\glie)$}.

  This can be explained via the comultiplication~$\Delta$ of~$\glie$.
  Indeed, suppose that~$M$ and~$N$ are two~\modules{$\Univ(\glie)$}.
  These module structure correspond to algebra homomorphisms
  \[
    \Phi_M
    \colon
    \Univ(\glie)
    \to
    \End_{\kf}(M) \,,
    \quad
    \Phi_N
    \colon
    \Univ(\glie)
    \to
    \End_{\kf}(N) \,.
  \]
  We have now the the homomorphism of algebras
  \[
    \Delta
    \colon
    \Univ(\glie)
    \to
    \Univ(\glie) \otimes \Univ(\glie) \,,
  \]
  the homomorphism of algebras
  \[
    \Phi_M \otimes \Phi_N
    \colon
    \Univ(\glie) \otimes \Univ(\glie)
    \to
    \End_{\kf}(M) \otimes \End_{\kf}(N)
  \]
  and the homomorphism of algebras
  \[
    \Psi
    \colon
    \End_{\kf}(M) \otimes \End_{\kf}(N)
    \to
    \End_{\kf}(M \otimes_{\kf} N) \,,
    \quad
    f \otimes g
    \mapsto
    f \otimes g \,.
  \]
  The composite of the homomorphism is a homomorphism of algebras
  \[
    \Phi_{M \otimes N}
    \colon
    \Univ(\glie)
    \to
    \End_{\kf}(M \otimes N) \,.
  \]
  This homomorphism corresponds to a~\module{$\Univ(\glie)$} structure on~$M \otimes_{\kf} N$.
  This module structure is given for every element~$x$ of~$\glie$ by
  \begin{align*}
    \class{x} \cdot (m \otimes n)
    &=
    \Phi_{M \otimes N}( \class{x} )( m \otimes n )
    \\
    &=
    \Psi( (\Phi_M \otimes \Phi_N)( \Delta(\class{x}) ) )( m \otimes n)
    \\
    &=
    \Psi( (\Phi_M \otimes \Phi_N)( \class{x} \otimes 1 + 1 \otimes \class{x} ) )( m \otimes n)
    \\
    &=
    \Psi( \Phi_M(\class{x}) \otimes \Phi_N(1) + \Phi_M(1) \otimes \Phi_N(\class{x}) )( m \otimes n)
    \\
    &=
    \Psi( \Phi_M(\class{x}) \otimes \id_N + \id_M \otimes \Phi_N(\class{x}) )( m \otimes n)
    \\
    &=
    \Phi_M(\class{x})(m) \otimes \id_N(n) + \id_M(m) \otimes \Phi_N(\class{x})(n)
    \\
    &=
    (\class{x} \cdot m) \otimes n + m \otimes (\class{x} \cdot n)
  \end{align*}
  for all~$m \in M$,~$n \in N$.
  This is precisely the same module structure as above.
\end{remark}


\begin{remark}
  Let~$\glie$ be a Lie~algebra.
  The comultiplication~$\Delta$, counit~$\varepsilon$ and antipode~$S$ satisfy certain compatibility conditions.
  More precisely, the comultiplication~$\Delta$ satisfies the following \defemph{coassociativity} diagram.
  \[
    \begin{tikzcd}[sep = large]
      \Univ(\glie)
      \arrow{r}[above]{\Delta}
      \arrow{d}[left]{\Delta}
      &
      \Univ(\glie) \otimes \Univ(\glie)
      \arrow{d}[right]{\Delta \otimes \id}
      \\
      \Univ(\glie) \otimes \Univ(\glie)
      \arrow{r}[above]{\id \otimes \Delta}
      &
      \Univ(\glie) \otimes \Univ(\glie) \otimes \Univ(\glie)
    \end{tikzcd}
  \]
  The comultiplication~$\Delta$ and counit~$\varepsilon$ satisfy the folloing \defemph{counital} diagram.
  \[
    \begin{tikzcd}[row sep = large]
      \Univ(\glie) \otimes \Univ(\glie)
      \arrow{d}[left]{\id \otimes \varepsilon}
      &
      \Univ(\glie)
      \arrow{l}[above]{\Delta}
      \arrow{r}[above]{\Delta}
      \arrow[equal]{d}
      &
      \Univ(\glie) \otimes \Univ(\glie)
      \arrow{d}[right]{\varepsilon \otimes \id}
      \\
      \Univ(\glie) \otimes \kf
      \arrow{r}[above]{\cong}
      &
      \Univ(\glie)
      &
      \kf \otimes \Univ(\glie)
      \arrow{l}[above]{\cong}
    \end{tikzcd}
  \]
  The comultiplication~$\Delta$, counit~$\varepsilon$ and antipode~$S$ satisfy the following \defemph{antipode} diagram.
  \[
    \begin{tikzcd}[column sep = small, row sep = large]
      {}
      &
      \Univ(\glie) \otimes \Univ(\glie)
      \arrow{rr}{S \otimes \id}
      &
      {}
      &
      \Univ(\glie) \otimes \Univ(\glie)
      \arrow{dr}{\mathrm{mult}}
      &
      {}
      \\
      \Univ(\glie)
      \arrow{ur}[above left]{\Delta}
      \arrow{rr}[above]{\varepsilon}
      \arrow{dr}[below left]{\Delta}
      &
      {}
      &
      \kf
      \arrow{rr}[above]{\mathrm{incl}}
      &
      {}
      &
      \Univ(\glie)
      \\
      {}
      &
      \Univ(\glie) \otimes \Univ(\glie)
      \arrow{rr}[above]{\id \otimes S}
      &
      {}
      &
      \Univ(\glie) \otimes \Univ(\glie)
      \arrow{ur}[below right]{\mathrm{mult}}
      &
      {}
    \end{tikzcd}
  \]
  This means altogether that the algebra~$\Univ(\glie)$ together with the comultiplication~$\Delta$, counit~$\varepsilon$ and antipode~$S$ is a \defemph{Hopf algebra}.
\end{remark}




\subsection{Quotient Lie~Algebra}


\begin{example}
  Let~$\glie$ be a Lie~algebra, let~$I$ be an ideal of~$\glie$ and let~$\ideal{I}$ be the two-sided ideal of~$\Univ(\glie)$ generated by all elements of the form~$\class{x}$ with~$x$ in~$I$.
  Let~$\pi$ be the canonical quotient homomorphism from~$\glie$ to~$\glie/I$.
  Then the induced homomorphism of algebras~$\Univ(\pi)$ from~$\Univ(\glie)$ to~$\Univ(\glie/I)$ factors through~$\Univ(\glie) / \ideal{I}$, and induces an isomorphism
  \[
    \Univ(\glie/I)
    \cong
    \Univ(\glie) / \ideal{I} \,,
  \]
  which is is given by
  \[
    \class{ \class{x} }
    \mapsto
    \class{ \class{x} }
  \]
  for all~$x \in \glie$.

  Indeed, we have for every~{\algebra{$\kf$}}~$A$ bijections
  \begin{align*}
    {}&
    \{ \textstyle\text{algebra homomorphisms~$\Psi \colon \Univ(\glie/I) \to A$} \}
    \\
    \cong{}&
    \{ \textstyle\text{Lie~algebra homomorphisms~$\psi \colon \glie/I \to A$} \}
    \\
    \cong{}&
    \{ \textstyle\text{Lie~algebra homomorphisms~$\varphi \colon \glie \to A$ with~$\varphi(x) = 0$ for all~$x \in I$} \}
    \\
    \cong{}&
    \{ \textstyle\text{algebra homomorphisms~$\Phi \colon \Univ(\glie) \to A$ with~$\Phi(\class{x}) = 0$ for all~$x \in I$} \}
    \\
    ={}&
    \{ \textstyle\text{algebra homomorphisms~$\Phi \colon \Univ(\glie) \to A$ with~$\Phi(y) = 0$ for all~$y \in \ideal{I}$} \}
    \\
    \cong{}&
    \{ \textstyle\text{algebra homomorphisms~$\Psi \colon \Univ(\glie) / \ideal{I} \to A$} \} \,.
  \end{align*}
  These bijections are natural in~$A$, whence~$\Univ(\glie/I) \cong \Univ(\glie) / \ideal{I}$ by Yoneda’s lemma.
  
  More explicitely, the composite
  \[
    \glie
    \to
    \Univ(\glie)
    \to
    \Univ(\glie) / \ideal{I} \,,
    \quad
    x \mapsto
    \class{ \class{x} }
  \]
  is a homomorphisms of Lie~algebras.
  This homomorphism annihilates the Lie~ideal~$I$ of~$\glie$ and thus induces a homomorphism of Lie~algebras
  \[
    \glie/I
    \to
    \Univ(\glie) / \ideal{I} \,,
    \quad
    \class{x}
    \mapsto
    \class{ \class{x} } \,,
  \]
  which in turn induces a homomorphism of algebras
  \begin{alignat*}{3}
    \Phi
    &\colon
    \Univ(\glie/I)
    \to
    \Univ(\glie) / \ideal{I}  \,,
    &
    \quad
    \class{ \class{x} }
    &\mapsto
    \class{ \class{x} }
    &
    \qquad
    &\text{for all~$x \in \glie$.}
  \intertext{
  On the other hand, the quotient homomorphism~$\pi$ from~$\glie$ to~$\glie/I$ induces the homomorphism of algebras~$\Univ(\pi)$ from~$\Univ(\glie)$ to~$\Univ(\glie/I)$, given by
  }
    \Univ(\pi)
    &\colon
    \Univ(\glie)
    \to
    \Univ( \glie / I ) \,,
    &
    \quad
    \class{x}
    &\mapsto
    \class{ \class{x} }
    &
    \qquad
    &\text{for all~$x \in \glie$.}
  \intertext{
  This homomorphism annihilates all residue classes~$\class{x}$ with~$x$ in~$I$, and hence induces a homomorphism of algebras
  }
    \Psi
    &\colon
    \Univ(\glie) / \ideal{I}
    \to
    \Univ(\glie/I)  \,,
    &
    \quad
    \class{ \class{x} }
    &\mapsto
    \class{ \class{x} }
    &
    \qquad
    &\text{for all~$x \in \glie$.}
  \end{alignat*}
  We have thus constructed two mutually inverse algebra isomorphisms~$\Phi$ and~$\Psi$.
\end{example}


\begin{example}
  It follows from the previous example that for any Lie~algebra~$\glie$,
  \begin{align*}
    \Univ( \glie^{\ab} )
    &=
    \Univ( \glie/[\glie, \glie] )
    \\
    &\cong
    \Univ(\glie) / \ideal{[\glie, \glie]}
    \\
    &=
    \Univ(\glie)
    /
    \ideal[\Big]{ \class{[x,y]} \suchthat x, y \in \glie }
    \\
    &=
    \Univ(\glie)
    /
    \ideal{
      \class{x} \, \class{y} - \class{y} \, \class{x} 
    \suchthat
      x, y \in \glie
    } \,.
  \end{align*}
  The ideal~$\ideal{ \class{x} \, \class{y} - \class{y} \, \class{x} \suchthat x, y \in \glie }$ is the commutator ideal of~$\Univ(\glie)$ because the elements~$\class{x}$ with~$x$ in~$\glie$ form an algebra generating set of~$\Univ(\glie)$.
  The universal enveloping algebra of the abelianization of~$\glie$ is hence the abelianization (commutativization?) of the universal enveloping algebra of~$\glie$.
  
  This can also be seen from Yoneda’s lemma since we have for every commutative~{\algebra{$\kf$}}~$A$ bijections
  \begin{align*}
    \SwapAboveDisplaySkip
    {}&
    \{ \textstyle\text{algebra homomorphisms~$\Univ(\glie/[\glie,\glie]) \to A$} \}
    \\
    \cong{}&
    \{ \textstyle\text{Lie~algebra homomorphism~$\glie/[\glie, \glie] \to A$} \}
    \\
    \cong{}&
    \{ \textstyle\text{Lie~algebra homomorphisms~$\glie \to A$} \}
    \\
    \cong{}&
    \{ \textstyle\text{algebra homomorphisms~$\Univ(\glie) \to A$} \}
    \\
    \cong{}&
    \{ \textstyle\text{algebra homomorphisms~$\Univ(\glie)/I \to A$} \}
  \end{align*}
  which are natural in~$A$, where~$I$ denotes the commutator ideal of~$\Univ(\glie)$.
\end{example}





\subsection{Free Lie~Algebra}


\begin{definition}
  Let~$X$ be a set.
  A~\defemph{free~\liealgebra{$\kf$} on the set~$X$}\index{free Lie algebra}\index{Lie algebra!free} is a~\liealgebra{$\kf$}~$F(X)$ together with a set-theoretic map~$i$ from~$X$ to~$F(X)$ such that the following universal property holds:
  for every~\liealgebra{$\kf$}~$\glie$ and every map~$f$ from~$X$ to~$\glie$ there exists a unique homomorphism of Lie~algebras~$\varphi$ from~$F(X)$ to~$\glie$ that makes the following triangular diagram commute.
  \[
    \begin{tikzcd}
      X
      \arrow{dr}[above right]{f}
      \arrow{d}[left]{i}
      &
      {}
      \\
      F(X)
      \arrow[dashed]{r}[below]{\varphi}
      &
      \glie
    \end{tikzcd}
  \]
\end{definition}


\begin{remark}
  \leavevmode
  \begin{enumerate}
    \item
      A free Lie~algebra on a set~$X$ consists of a Lie~algebra~$X$ together with an element~$\class{x}$ of~$F(X)$ for every element~$x$ of~$X$, such that the following universal property holds:
      for every Lie~algebra~$\glie$ and every famile~$(y_x)_{x \in X}$ of elements of~$\glie$ there exists a unique homomorphism of Lie~algebras~$\varphi$ from~$F(X)$ to~$\glie$ such that~$\varphi( \class{x} ) = y_x$ for all~$x \in X$.
    \item
      The free Lie~algebra on a set~$X$ is unique up to unique isomorphism, in the following sense.

      If~$(F_1, i_1)$ and~$(F_2, i_2)$ are two free Lie~algebras on~$X$ then there exist unique homomorphisms of Lie~algebras~$\varphi$ from~$F_1$ to~$F_2$ and~$\psi$ from~$F_2$ to~$F_1$ that make the triangular diagrams
      \[
        \begin{tikzcd}[column sep = small]
          {}
          &
          X
          \arrow{dl}[above left]{\iota_1}
          \arrow{dr}[above right]{\iota_2}
          &
          {}
          \\
          F_1
          \arrow[dashed]{rr}[below]{\varphi}
          &
          {}
          &
          F_2
        \end{tikzcd}
        \qquad\text{and}\qquad
        \begin{tikzcd}[column sep = small]
          {}
          &
          X
          \arrow{dl}[above left]{\iota_2}
          \arrow{dr}[above right]{\iota_1}
          &
          {}
          \\
          F_2
          \arrow[dashed]{rr}[below]{\psi}
          &
          {}
          &
          F_1
        \end{tikzcd}
      \]
      commute.
      These homomorphisms~$\varphi$ and~$\psi$ are mutually inverse isomorphisms.

      Because of this uniqueness we will talk about \emph{the} free~\liealgebra{$\kf$} on~$X$.
    \item
      For every map~$f$ from a set~$X$ to a set~$Y$ there exists a unique homomorphism of Lie~algebras~$F(f)$ from~$F(X)$ to~$F(Y)$ that makes the square diagram
      \[
        \begin{tikzcd}
          X
          \arrow{r}[above]{f}
          \arrow{d}
          &
          Y
          \arrow{d}
          \\
          F(X)
          \arrow[dashed]{r}[below]{F(f)}
          &
          F(Y)
        \end{tikzcd}
      \]
      commute.
      It holds for every set~$X$ that~$F(\id_X) = \id_{F(X)}$, and it holds for all composable maps of sets~$f$ from~$X$ to~$Y$ and~$g$ from~$Y$ to~$Z$ that~$F(g \circ f) = F(g) \circ F(f)$.
      The assignment~$F$ is thus a functor from the category~$\cSet$ to the category~$\cLie{\kf}$.
      
      The universal property of the free Lie~algebra states that the functor~$F$ is left adjoint to the forgetful functor from~$\cLie{\kf}$ to~$\cSet$, which assigns to each Lie~algebra its underlying set.
  \end{enumerate}
\end{remark}


\begin{example}[Free Lie~algebras]
  \label{uea of free lie algebra}
  Let~$I$ be a set and let~$F(I)$ be the free~{\liealgebra{$\kf$}} on the set~$I$.
  We have for every~\algebra{$\kf$}~$A$ bijections
  \begin{align*}
    {}&
    \{ \textstyle\text{algebra homomorphisms~$\Univ(F(I)) \to A$} \}
    \\
    \cong{}&
    \{ \textstyle\text{Lie~algebra homomorphisms~$F(I) \to A$} \}
    \\
    \cong{}&
    \{ \textstyle\text{set-theoretic maps~$I \to A$} \}
    \\\
    \cong{}&
    \{ \textstyle\text{algebra homomorphisms~$\kf\gen{x_i \suchthat i \in I} \to A$} \} \,,
  \end{align*}
  and these bijections are natural in~$A$.
  We hence find by Yoneda’s~lemma that
  \[
    \Univ(F(I))
    \cong
    \kf\gen{x_i \suchthat i \in I} \,.
  \]
  More precisely, there exists for the composite
  \[
    I
    \to
    F(I)
    \to
    \Univ(F(I))
  \]
  a unique homomorphism of algebras~$\Phi$ from~$\kf\gen{x_i \suchthat i \in I}$ to~$\Univ(F(I))$ that makes the diagram
  \[
    \begin{tikzcd}[column sep = small]
        {}
      & I
        \arrow[bend right, out = -45, in=225]{ddl}[above left]{i \mapsto x_i}
        \arrow[bend left]{dr}
      & {}
      \\
        {}
      & {}
      & F(I)
        \arrow{d}
      \\
        \kf\gen{x_i \suchthat i \in I}
        \arrow[dashed]{rr}[below]{\Phi}
      & {}
      & \Univ(F(I))
    \end{tikzcd}
  \]
  commute.
  This homomorphism~$\Phi$ is an isomorphism.
\end{example}


\begin{remark}
  Lie~subalgebras of free Lie~algebras are again free by the Shirshov--Witt~theorem.
\end{remark}








% TODO: Hopf algebra structure
%       Explain this also via hopf objects (applying U to a Hopf object in Lie(k) gives a Hopf algebra)





